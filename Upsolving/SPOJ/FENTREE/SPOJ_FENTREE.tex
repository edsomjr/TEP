%\section{SPOJ -- Fenwick Trees}

\begin{frame}[fragile]{Problema}

Mr. Fenwick has an array $a$ with many integers, and his children love to do operations on the
array with their father. The operations can be a query or an update.

For each query the children say two indices $l$ and $r$, and their father answers back with the 
sum of the elements from indices $l$ to $r$ (both included).

When there is an update, the children say an index $i$ and a value $x$, and Fenwick will add $x$ to
$a_i$ (so the new value of $a_i$  is $a_i + x$).

Because indexing the array from zero is too obscure for children, all indices start from 1.
Fenwick is now too busy to play games, so he needs your help with a program that plays with his
children for him, and he gave you an input/output specification.

\end{frame}

\begin{frame}[fragile]{Entrada e saída}

\textbf{Input}

The first line of the input contains $N$ ($1\leq N\leq 10^6$) . The second line contains $N$ 
integers $a_i$ ($-10^9\leq a_i\leq 10^9$), the initial values of the array. The third line 
contains $Q$ $(1\leq Q\leq 3\times 10^5$), the number of operations that will be made. Each of the 
next $Q$ lines contains an operation.  Query operations are of the form \lq\lq q $l\ r$\rq\rq 
$(1\leq l\leq r\leq N)$, while update operations are of the form \lq\lq u $i\ x$\rq\rq\ 
($1\leq i\leq N, -10^9\leq x\leq 10^9$).

\textbf{Output}

You have to print the answer for every query in a different
line, in the same order of the input.

\end{frame}

\begin{frame}[fragile]{Exemplo de entradas e saídas}

\begin{minipage}[t]{0.5\textwidth}
\textbf{Sample Input}
\begin{verbatim}
10
3 2 4 0 42 33 -1 -2 4 4
6
q 3 5
q 1 10
u 5 -2
q 3 5
u 6 7
q 4 7
\end{verbatim}
\end{minipage}
\begin{minipage}[t]{0.45\textwidth}
\textbf{Sample Output}
\begin{verbatim}
46
89
44
79
\end{verbatim}
\end{minipage}
\end{frame}

\begin{frame}[fragile]{Solução}

    \begin{itemize}
        \item A solução \textit{naive} consiste em percorrer cada intervalo a cada consulta,
            de modo que a complexidade seria igual a $O(QN)$, onde $Q$ é o número de 
            \textit{queries} do tipo q

        \item Como $Q\leq 3\times 10^5$ e $N\leq 10^6$, esta solução levaria ao TLE

        \item O uso de uma árvore de Fenwick permite responder cada uma das \textit{queries}
            com complexidade $O(\log N)$

        \item A construção da árvore tem complexidade $O(N\log N)$, de modo que a solução teria
            complexidade $O((N + Q)\log N)$

        \item É preciso tomar cuidado com possíveis \textit{overflows}, usando o tipo
            \code{cpp}{long long} para armazenar as informações dos nós da árvore
   \end{itemize}

\end{frame}

\begin{frame}[fragile]{Solução AC com complexidade $O((Q + N)\log N)$}
    \inputsnippet{cpp}{1}{20}{codes/FENTREE.cpp}
\end{frame}

\begin{frame}[fragile]{Solução AC com complexidade $O((Q + N)\log N)$}
    \inputsnippet{cpp}{22}{39}{codes/FENTREE.cpp}
\end{frame}

\begin{frame}[fragile]{Solução AC com complexidade $O((Q + N)\log N)$}
    \inputsnippet{cpp}{41}{59}{codes/FENTREE.cpp}
\end{frame}

\begin{frame}[fragile]{Solução AC com complexidade $O((Q + N)\log N)$}
    \inputsnippet{cpp}{61}{82}{codes/FENTREE.cpp}
\end{frame}
