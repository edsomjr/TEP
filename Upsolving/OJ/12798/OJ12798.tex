%\section{OJ 12798 -- Handball}

\begin{frame}[fragile]{Problema}

Frustrated and disappointed with the results of its football team, the Super Brazilian Club (SBC)
decided to invest in the handball team. In order to better rate the players, the coaches would like to
analyse their regularity. Specifically, they are interested in knowing how many players scored goals in
all matches.

As the data volume is very big, they would like to have a computer program to do this counting.

\end{frame}

\begin{frame}[fragile]{Entrada e saída}

\textbf{Input}

The input contains several test cases. The first line of a test case contains two integers $N$ and 
$M$ $(1\leq N\leq 100\ \mbox{and}\ 1\leq M\leq 100)$ indicating, respectively, the number of players and the number of matches. Each one of the next $N$ lines describes the performance of one player: 
the $i$-th line contains $M$ integers $X_j$ $(0\leq X_j\leq 100, \mbox{for}\ 1\leq j\leq M)$, 
giving the number of goals that the $i$-th player scored in each match.

\textbf{Output}

For each test case in the input your program must output one line, containing one integer, the number
of players that scored goals in all matches!

\end{frame}


\begin{frame}[fragile]{Exemplo de entradas e saídas}

\begin{scriptsize}
\begin{minipage}[t]{0.6\textwidth}
\textbf{Sample Input}
\begin{verbatim}
5 3
0 0 0
1 0 5
0 0 0
0 1 2
1 1 0
12 5
4 4 2 3 7
0 0 0 1 0
7 4 7 0 6
1 2 3 3 2
0 0 0 0 0
4 0 9 10 10
0 1 0 0 0
1 2 0 2 3
10 10 10 1 0
0 3 3 3 4
10 10 0 10 10
1 1 2 0 9
\end{verbatim}
\end{minipage}
\begin{minipage}[t]{0.35\textwidth}
\textbf{Sample Output}
\begin{verbatim}
0
2
\end{verbatim}
\end{minipage}
\end{scriptsize}

\end{frame}

\begin{frame}[fragile]{Solução com complexidade $O(NM\log N)$}

    \begin{itemize}
        \item O problema consiste em determinar quantos jogadores fizeram gols em todas as
            partidas

        \item Uma maneira de se manter este registro é inicializar uma árvore de Fenwinck 
            onde todos os elementos da sequência $a_k$ são iguais a um

        \item Assim, para cada jogador, caso ele não marque nenhum gol em uma determinada 
            partida, basta fazer $a_i = 0$ atrás da soma de $-1$ na posição $i$

        \item Este processo deve ser feito, no máximo, uma vez por jogador

        \item Ao final, a resposta será $RSQ(1, N)$

   \end{itemize}

\end{frame}

\begin{frame}[fragile]{Solução com complexidade $O(NM\log N)$}
    \inputsnippet{cpp}{1}{20}{codes/12798.cpp}
\end{frame}

\begin{frame}[fragile]{Solução com complexidade $O(NM\log N)$}
    \inputsnippet{cpp}{22}{38}{codes/12798.cpp}
\end{frame}

\begin{frame}[fragile]{Solução com complexidade $O(NM\log N)$}
    \inputsnippet{cpp}{40}{59}{codes/12798.cpp}
\end{frame}

\begin{frame}[fragile]{Solução com complexidade $O(NM\log N)$}
    \inputsnippet{cpp}{61}{79}{codes/12798.cpp}
\end{frame}

\begin{frame}[fragile]{Solução $O(NM)$}

    \begin{itemize}
        \item Observe que o problema é estático: o número de gols marcados por cada jogador
            não muda dinamicamente, e a pergunta só é feita uma vez, ao final
            
        \item Assim, não é necessário o uso de uma árvore de Fenwick

        \item O registro de cada jogador pode ser feito em um vetor de \textit{bits}, onde
            1 significa que ele marcou gols em todos os jogos

        \item A implementação fica semelhante à anterior, porém com acesso aos elementos em
            $O(1)$

        \item O método \code{c}{count()} do \code{c}{bitset} da STL pode ser utilizada para
            totalizar o número de jogadores que marcaram gols em todos os jogos
    \end{itemize}

\end{frame}

\begin{frame}[fragile]{Solução com complexidade $O(NM)$}
    \inputsnippet{cpp}{1}{19}{codes/12798_linear.cpp}
\end{frame}

\begin{frame}[fragile]{Solução com complexidade $O(NM)$}
    \inputsnippet{cpp}{21}{41}{codes/12798_linear.cpp}
\end{frame}
