\begin{frame}[fragile]{OJ 10527 -- Persistent Numbers}

{\it
The multiplicative persistence of a number is defined by Neil Sloane (Neil J.A. Sloane in `The
Persistence of a Number' published in Journal of Recreational Mathematics 6, 1973, pp. 97-98., 1973)
as the number of steps to reach a one-digit number when repeatedly multiplying the digits. Example:
\[
679 \to 378 \to 168 \to 48 \to 32 \to 6.
\]

That is, the persistence of 679 is 5. The persistence of a single digit number is 0. At the time of
this writing it is known that there are numbers with the persistence of 11. It is not known whether
there are numbers with the persistence of 12 but it is known that if they exists then the smallest of
them would have more than 3000 digits.

The problem that you are to solve here is: what is the smallest number such that the first step of
computing its persistence results in the given number?
}
\end{frame}

\begin{frame}[fragile]{Entrada e saída}

\textbf{Input}

{\it For each test case there is a single line of input containing a decimal number with up to 1000 digits. A
line containing \texttt{-1} follows the last test case.}

\vspace{0.3in}


\textbf{Output}

{\it For each test case you are to output one line containing one integer number satisfying the condition
stated above or a statement saying that there is no such number in the format shown below.}

\end{frame}

\begin{frame}[fragile]{Exemplo de entrada e saída}

\begin{minipage}[t]{0.45\textwidth}
\textbf{Entrada}
\begin{verbatim}
0
1
4
7
18
49
51
768
-1
\end{verbatim}
\end{minipage}
\begin{minipage}[t]{0.5\textwidth}
\textbf{Saída}
\begin{verbatim}
10
11
14
17
29
77
There is no such number.
2688
\end{verbatim}
\end{minipage}

\end{frame}


\begin{frame}[fragile]{Solução $O(\log n)$}

    \begin{itemize}
        \item Se $n$ é o produto dos dígitos de $x$, então a fatoração prima de $n$ só pode
            conter primos cuja representação decimal só tem um dígito, a saber: 2, 3, 5 e 7

        \item Assim, se a fatoração de $n$ tem qualquer outro primo o problema não
            terá solução

        \item Nos demais casos, a fatoração prima de $n$ seria uma solução, embora nem sempre
            seja a mínima

        \item Para minimizar a solução, é preciso agrupar os fatores primos em dígitos compostos

        \item Antes de fazer esta redução, tratemos primeiro de um caso especial
    \end{itemize}

\end{frame}


\begin{frame}[fragile]{Solução $O(\log n)$}

    \begin{itemize}
        \item Mesmo não estando explícito no texto do problema, espera-se que $x$ tenha, no 
            mínimo, dois dígitos, conforme se observa nos exemplos de entrada e saída

        \item Assim, se $n$ tiver um único digo, a solução mínima seria o número $10 + n$

        \item Nos demais casos, para minimizar $x$ agruparemos os fatores primos nos compostos
            9, 8, 6 e 4, nesta ordem, de forma gulosa

        \item Feito este agrupamento, $x$ é formado por estes agrupamentos, ordenados do
            menor para o maior
    \end{itemize}

\end{frame}

\begin{frame}[fragile]{Solução $O(\log n)$}
    \inputsnippet{py3}{5}{17}{codes/10527.py}
\end{frame}

\begin{frame}[fragile]{Solução $O(\log n)$}
    \inputsnippet{py3}{20}{35}{codes/10527.py}
\end{frame}

\begin{frame}[fragile]{Solução $O(\log n)$}
    \inputsnippet{py3}{36}{51}{codes/10527.py}
\end{frame}

\begin{frame}[fragile]{Solução $O(\log n)$}
    \inputsnippet{py3}{54}{60}{codes/10527.py}
\end{frame}
