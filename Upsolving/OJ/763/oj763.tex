\begin{frame}[fragile]{OJ 763 -- Fibinary Numbers}

{\it
The standard interpretation of the binary number $1010$ is $8 + 2 = 10$. An alternate way to view the
sequence ``\texttt{1010}'' is to use Fibonacci numbers as bases instead of powers of two. For this problem, the
terms of the Fibonacci sequence are:
\[
1, 2, 3, 5, 8, 13, 21, \ldots
\]

Where each term is the sum of the two preceding terms (note that there is only one $1$ in the sequence
as defined here). Using this scheme, the sequence ``\texttt{1010}'' could be interpreted as $1\times 5+0\times 3+1\times 2+0\times 1 = 7$.
This representation is called a Fibinary number.

Note that there is not always a unique Fibinary representation of every number. For example the
number 10 could be represented as either $8 + 2$ $(10010)$ or as $5 + 3 + 2$ $(1110)$. To make the Fibinary
representations unique, larger Fibonacci terms must always be used whenever possible (i.e. disallow
$2$ adjacent $1$’s). Applying this rule to the number $10$, means that 10 would be represented as $8+2$
$(10010)$.

Write a program that takes two valid Fibinary numbers and prints the sum in Fibinary form.
}

\end{frame}

\begin{frame}[fragile]{Entrada e saída}

\textbf{Input}

{\it The input file contains several test cases with a blank line between two consecutive.

Each test case consists in two lines with Fibinary numbers. These numbers will have at most $100$ digits.}

\vspace{0.3in}


\textbf{Output}

{\it For each test case, print the sum of the two input numbers in Fibinary form.

It must be a blank line between two consecutive outputs.}

\end{frame}

\begin{frame}[fragile]{Exemplo de entrada e saída}

\begin{minipage}[t]{0.45\textwidth}
\textbf{Entrada}
\begin{verbatim}
10010
1

10000
1000

10000
10000
\end{verbatim}
\end{minipage}
\begin{minipage}[t]{0.5\textwidth}
\textbf{Saída}
\begin{verbatim}
10100


100000


100100
\end{verbatim}
\end{minipage}

\end{frame}


\begin{frame}[fragile]{Solução em $O(N)$}

    \begin{itemize}
        \item Primeiramente é preciso observar que não é possível somar diretamente os números em
            base de Fibonacci

        \item Por exemplo, em base de Fibonacci o número 5 é representado por \texttt{1000} e a
            soma $5 + 5 = 10$ teria representação \code{cpp}{10010}

        \item Veja que ao somar os dois dígitos \texttt{1} correspondentes, o dígito que o ocupa
            a segunda posição da representação, o qual já teria sido processado, foi modificado

        \item Assim, a solução consiste em converter $a$ e $b$ para a base decimal, obter a soma
            $c = a + b$ e converter $c$ para a base de Fibonacci

        \item Se $N = \max\{|a|, |b|\}$, então esta solução tem complexidade $O(N)$
    \end{itemize}

\end{frame}

\begin{frame}[fragile]{Solução em $O(N)$}
    \inputsnippet{py}{1}{16}{codes/763.py}
\end{frame}

\begin{frame}[fragile]{Solução em $O(N)$}
    \inputsnippet{py}{17}{33}{codes/763.py}
\end{frame}

\begin{frame}[fragile]{Solução em $O(N)$}
    \inputsnippet{py}{34}{50}{codes/763.py}
\end{frame}
