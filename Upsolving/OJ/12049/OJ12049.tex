%\section{OJ 12049 -- Just Prune The List}

\begin{frame}[fragile]{Problema}

You are given two list of integers. You can remove any number of elements from any of them. You
have to ensure that after removing some elements both of the list will contain same elements, but not
necessarily in same order. For example consider the following two lists
\begin{table}[h]
    \centering
    \begin{tabular}{|l|ccccc|}
    \hline
    List \# 1 & 1 & 2 & 3 & 2 & 1 \\
    List \# 2 & 1 & 2 & 5 & 2 & 3 \\
    \hline
    \end{tabular}
\end{table}

After removing 1 from first list and 5 from second list, both lists will contain same elements. We
will find the following lists after removing two elements.
\begin{table}[h]
    \centering
    \begin{tabular}{|l|cccc|}
    \hline
    List \# 1 & 1 & 2 & 3 & 2 \\
    List \# 2 & 1 & 2 & 2 & 3 \\
    \hline
    \end{tabular}
\end{table}

What is the minimum number of elements to be removed to obtain two list having same elements?
\end{frame}

\begin{frame}[fragile]{Entrada e saída}

\textbf{Input}

The first line of the input file contains an integer $T$ $(T\leq 100)$ which denotes the total 
number of test cases. The description of each test case is given below:

First line will contain two integers $N$ $(1 \leq N \leq 10000)$, the number of element in the 
first list and $M$ $(1\leq M\leq 10000)$, the number of element in the second list. The next line 
will contain $N$ integers
representing the first list followed by another line having $M$ elements representing the second list. Each integers in the input is 32 bit signed integer.

\textbf{Output}

For each test case output a single line containing the number of elements to be removed. See sample
output for exact format.
\end{frame}


\begin{frame}[fragile]{Exemplo de entradas e saídas}

\begin{minipage}[t]{0.6\textwidth}
\textbf{Sample Input}
\begin{verbatim}
1
5 5
1 2 3 2 1
1 2 5 2 3
\end{verbatim}
\end{minipage}
\begin{minipage}[t]{0.35\textwidth}
\textbf{Sample Output}
\begin{verbatim}
2
\end{verbatim}
\end{minipage}
\end{frame}

\begin{frame}[fragile]{Solução com complexidade $O(N\log N)$}

    \begin{itemize}
        \item O problema consiste em determinar quantos elementos, em ambas listas, não
            pertencem a interseção de ambas

        \item Observe que as listas podem conter elementos repetidos: armazená-los em um
            \texttt{set} levaria a perda de informações importantes

        \item Uma solução é armazenar ambas listas nos \code{c}{multisets} $s$ e $r$ e, para
            cada elemento de $r$, verificar se ele está ou não em $s$

        \item Se estiver, uma cópia dele deve ser removida de $s$

        \item Caso contrário, é um elemento a ser eliminado, acrescentando uma unidade à resposta

        \item Ao final do processo, cada elemento restante em $s$ acrescenta uma unidade ao
            resultado final

        \item A STL do C++ pode ser utilizada: a função \code{c}{set_intersection()} computa
            a interseção entre dois contêiners ordenados

        \item Neste caso a resposta passa a ser a soma $N + M$, subtraída do dobro da interseção
   \end{itemize}

\end{frame}

\begin{frame}[fragile]{Solução com complexidade $O(N\log N)$}
    \inputsnippet{cpp}{1}{20}{codes/12049.cpp}
\end{frame}

\begin{frame}[fragile]{Solução com complexidade $O(N\log N)$}
    \inputsnippet{cpp}{22}{42}{codes/12049.cpp}
\end{frame}
