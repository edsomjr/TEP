\begin{frame}[fragile]{OJ 10127 -- Ones}

{\it
Given any integer $0 \leq n \leq 10000$ not divisible by $2$
or $5$, some multiple of $n$ is a number which in decimal
notation is a sequence of $1$’s. How many digits are in
the smallest such a multiple of $n$?
}

\end{frame}

\begin{frame}[fragile]{Entrada e saída}

\textbf{Input}

{\it A file of integers at one integer per line.}

\vspace{0.3in}


\textbf{Output}

{\it Each output line gives the smallest integer $x > 0$ such that $$p = \sum_{i = 0}^{x-1} 1 \times 10^i = a \times b,$$
where $a$ is the corresponding input integer, and $b$ is an integer greater than zero.}

\end{frame}

\begin{frame}[fragile]{Exemplo de entrada e saída}

\begin{minipage}[t]{0.45\textwidth}
\textbf{Entrada}
\begin{verbatim}
3
7
9901
\end{verbatim}
\end{minipage}
\begin{minipage}[t]{0.5\textwidth}
\textbf{Saída}
\begin{verbatim}
3
6
12
\end{verbatim}
\end{minipage}

\end{frame}



\begin{frame}[fragile]{Solução com complexidade $O(\log n)$}

    \begin{itemize}
        \item A solução do problema consiste na construção iterativa do valor de $x$

        \item Inicialmente $x = 1$

        \item Enquanto $\Mod{x}{n} > 0$, $x$ deve ir para o próximo número $x'$ cujos dígitos são
            todos iguais a 1

        \item Temos que $x' = 10x + 1$

        \item A cada atualização a resposta, que deve ser iniciada em 1, deve ser incrementada

        \item A complexidade da solução depende do número de dígitos de $x$, o qual será sempre
            menor ou igual a $n$

    \end{itemize}

\end{frame}

\begin{frame}[fragile]{Solução com complexidade $O(\log n)$}
    \inputsnippet{cpp}{5}{16}{codes/10127.cpp}
\end{frame}
