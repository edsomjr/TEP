\begin{frame}[fragile]{OJ 10219 -- Find the ways!}

{\it
An American, a Frenchman and an Englishwoman had been to Dhaka, the capital of Bangladesh. They
went sight-seeing in a taxi. The three tourists were talking about the sites in the city. The American
was very proud of tall buildings in New York. He boasted to his friends, ``Do you know that the Empire
State Building was built in three months?''.

``Really?'' replied the Frenchman. ``The Eiffel Tower in Paris was built in only one month!''
(However, The truth is, the construction of the Tower began in January 1887. Forty Engineers and
designers under Eiffel’s direction worked for two years. The tower was completed in March 1889.).

``How interesting!'' said the Englishwoman. ``Buckingham Palace in London was built in only two
weeks!!''.
}

\end{frame}

\begin{frame}[fragile]{OJ 10219 -- Find the ways!}

{\it
At that moment the taxi passed a big slum (However, in Bangladesh we call it ``Bostii''). ``What
was that? When it was built?'' The Englishwomen asked the driver who was a Bangladeshi.

``I don’t know!'', answered the driver. ``It wasn’t there yesterday!''.

However in Bangladesh, illegal establishment of slums is a big time problem. Government is trying
to destroy these slums and remove the peoples living there to a far place, formally in a planned village
outside the city. But they can’t find any ways, how to destroy all these slums!

Now, can you imagine yourself as a slum destroyer? In how many ways you can destroy k slums out
of n slums! Suppose there are 10 slums and you are given the permission of destroying 5 slums, surly
you can do it in 252 ways, which is only a 3 digit number, Your task is to find out the digits in ways
you can destroy the slums!
}

\end{frame}

\begin{frame}[fragile]{Entrada e saída}

\textbf{Input}

{\it The input file will contain one or more test cases.

Each test case consists of one line containing two integers $n$ $(n \geq 1)$ and $k$ $(1 \leq k \leq n)$.}

\vspace{0.3in}


\textbf{Output}

{\it For each test case, print one line containing the required number. This number will always fit into an
integer, i.e. it will be less than $2^{31} - 1$.}

\end{frame}

\begin{frame}[fragile]{Exemplo de entrada e saída}

\begin{minipage}[t]{0.45\textwidth}
\textbf{Entrada}
\begin{verbatim}
20 5
100 10
200 15
\end{verbatim}
\end{minipage}
\begin{minipage}[t]{0.5\textwidth}
\textbf{Saída}
\begin{verbatim}
5
14
23
\end{verbatim}
\end{minipage}

\end{frame}



\begin{frame}[fragile]{Solução em $O(n)$}

    \begin{itemize}
        \item O número de dígitos $D$ de um inteiro $x$ em uma base $b > 1$ é dado por
        $$
            D = \lfloor 1 + \log_b x\rfloor
        $$

        \item O coeficiente binomial pode ser escrito como
        $$
            \binom{n}{k} = \frac{n!}{(n - k)!k!} = \frac{n\times (n - 1)\times \ldots \times
                (n - k + 1)}{k\times (k - 1)\times \ldots 2 \times 1}
        $$

        \item Combinando ambas expressões, com $m = n - k + 1$, obtemos
        $$
            D = \left\lfloor 1 + \sum_{i = m}^n \log_{10} i - \sum_{i = 1}^k \log_{10} i \right\rfloor
        $$
    \end{itemize}

\end{frame}

\begin{frame}[fragile]{Solução em $O(n)$}
    \inputsnippet{cpp}{5}{18}{codes/10219.cpp}
\end{frame}
