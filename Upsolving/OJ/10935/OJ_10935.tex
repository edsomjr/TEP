%\section{OJ 10935 -- Throwing Cards Away I}

\begin{frame}[fragile]{Problema}

Given is an ordered deck of $n$ cards numbered 1 to
$n$ with card 1 at the top and card $n$ at the bottom.
The following operation is performed as long as there
are at least two cards in the deck:

\begin{center}
Throw away the top card and move the
card that is now on the top of the deck to
the bottom of the deck.
\end{center}

Your task is to find the sequence of discarded
cards and the last, remaining card.

\end{frame}

\begin{frame}[fragile]{Entrada e saída}

\textbf{Input}

Each line of input (except the last) contains a number
$n\leq 50$. The last line contains ‘\texttt{0}’ and this line should
not be processed.

\textbf{Output}

For each number from the input produce two lines of
output. The first line presents the sequence of discarded cards, the second line reports the last remaining card. No line will have leading or trailing spaces.
See the sample for the expected format.
\end{frame}

\begin{frame}[fragile]{Exemplo de entradas e saídas}

\begin{footnotesize}
\textbf{Sample Input}
\begin{verbatim}
7
19
10
6
0
\end{verbatim}

\textbf{Sample Output}
\begin{verbatim}
Discarded cards: 1, 3, 5, 7, 4, 2
Remaining card: 6
Discarded cards: 1, 3, 5, 7, 9, 11, 13, 15, 17, 19, 4, 8, 12, 16, 2, 10, 18, 14
Remaining card: 6
Discarded cards: 1, 3, 5, 7, 9, 2, 6, 10, 8
Remaining card: 4
Discarded cards: 1, 3, 5, 2, 6
Remaining card: 4
\end{verbatim}
\end{footnotesize}

\end{frame}

\begin{frame}[fragile]{Solução com complexidade $O(N)$}

    \begin{itemize}
        \item O processo descrito no problema pode ser simulado através do uso de uma fila

        \item Os elementos removidos podem ser armazenados ou em uma outra fila ou em um
            vetor

        \item Como, a cada ciclo, o tamanho da fila diminui em uma unidade, o algoritmo tem
            complexidade $O(N)$

        \item A saída deve ser formatada com cuidado: no caso $n = 1$ não há elementos a serem
            removidos
   \end{itemize}

\end{frame}

\begin{frame}[fragile]{Solução com complexidade $O(N)$}
    \inputsnippet{cpp}{1}{20}{codes/10935.cpp}
\end{frame}

\begin{frame}[fragile]{Solução com complexidade $O(N)$}
    \inputsnippet{cpp}{21}{40}{codes/10935.cpp}
\end{frame}

\begin{frame}[fragile]{Solução com complexidade $O(N)$}
    \inputsnippet{cpp}{42}{62}{codes/10935.cpp}
\end{frame}
