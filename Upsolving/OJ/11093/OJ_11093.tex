%\section{OJ 11093 -- Just Finish it Up}

\begin{frame}[fragile]{Problema}

Along a circular track, there are $N$ gas stations, which are numbered clockwise from 1 up to $N$. 
At station $i$, there are $p_i$ gallons of petrol available. To race from station $i$ to its 
clockwise neighbor one need $q_i$ gallons of petrol. Consider a race where a car will start the 
race with an empty fuel tank. Your task is to find whether the car can complete the race from any 
of the stations or not. If it can then mention the smallest possible station $i$ from which the 
lap can be completed.

\end{frame}
 
\begin{frame}[fragile]{Entrada e saída}

\textbf{Input}

First line of the input contains one integer $T$ the number of test cases. Each test case will 
start with a line containing one integer $N$, which denotes the number of gas stations. In the 
next few lines contain $2N$ integers. First $N$ integers denote the values of $p_i$s 
(petrol available at station $i$), subsequent $N$ integers denote the value of $q_i$s 
(amount of petrol needed to go to the next station in the clockwise direction).

\textbf{Output}

For each test case, output the case number in the format \lq\lq \texttt{Case $c$:}” , 
where $c$ is the case number starting from 1. Then display whether it is possible to complete a 
lap by a car with an empty tank or not. If it is not possible to complete the lap then display 
\lq\lq \texttt{Not possible}”. If possible, then display \lq\lq 
\texttt{Possible from station $X$}”, where $X$ is the first possible station from which the car 
can complete the lap.

\end{frame}


\begin{frame}[fragile]{Exemplo de entradas e saídas}

\textbf{Constraints}

\begin{itemize}
    \item $T < 25$
    \item $N < 100001$
\end{itemize}

\vspace{0.2in}

\begin{minipage}[t]{0.5\textwidth}
\textbf{Sample Input}
\begin{verbatim}
2
5
1 1 1 1 1
1 1 2 1 1
7
1 1 1 10 1 1 1
2 2 2 2 2 2 2
\end{verbatim}
\end{minipage}
\begin{minipage}[t]{0.45\textwidth}
\textbf{Sample Output}
\begin{verbatim}
Case 1: Not possible
Case 2: Possible from station 4
\end{verbatim}
\end{minipage}
\end{frame}

\begin{frame}[fragile]{Solução com complexidade $O(N^2)$}

    \begin{itemize}
        \item A solução do problema com complexidade $O(N^2)$ consiste em simular, a partir 
            de todos os postos possíveis, uma corrida, verificando se é possível 
            completar uma volta ou não

        \item Importante: uma corrida equivale a uma volta completa, isto é,
            partir de um posto $i$ e retornar a $i$

        \item A cada estação, é preciso ver se o valor de $q_i$ é menor ou igual ao combustível
            disponível

        \item Se for, esta quantia é subtraída, e o combustível disponível no novo posto deve
            ser adicionado ao total

        \item Ao iniciar a corrida em $i$, o combustível inicial é igual a $p_i$

        \item Esta solução deveria levar ao TLE, dado que $N \leq 10^5$, mas os casos de
            teste são fracos e levam ao AC
   \end{itemize}

\end{frame}

\begin{frame}[fragile]{Solução com complexidade $O(N)$}

    \begin{itemize}
        \item Contudo, é possível resolver este problema com complexidade $O(N)$

        \item Para tal, é preciso verificar uma propriedade da solução 

        \item Suponha que a corrida comece no posto $i$ e que, no posto $k$, verifique-se que
            não é possível chegar ao posto $k + 1$

        \item Na solução quadrática, a simulação recomeçaria no posto $i + 1$

        \item Contudo, é possível começar direto no posto $k + 1$, saltando todos os intermediários

        \item Eis a prova: se é possível chegar em $i + 1$ a partir de $i$, começar a prova em
            $i + 1$ significa ter ou a mesma quantidade de combustível resultante de se 
            começar em $i$ (se $p_i = q_i$), ou ter menos combustível (se $p_i > q_i$)

        \item Assim, se partir de $i$ significa não alcançar $k + 1$, partir de $i + 1$ tem mesmo
            significado

        \item Portanto, cada posto será visitado, no máximo, duas vezes, levando a uma solução
            linear
    \end{itemize}

\end{frame}

\begin{frame}[fragile]{Solução com complexidade $O(N)$}
    \inputsnippet{cpp}{1}{18}{codes/11093.cpp}
\end{frame}

\begin{frame}[fragile]{Solução com complexidade $O(N)$}
    \inputsnippet{cpp}{20}{38}{codes/11093.cpp}
\end{frame}

\begin{frame}[fragile]{Solução com complexidade $O(N)$}
    \inputsnippet{cpp}{40}{58}{codes/11093.cpp}
\end{frame}

\begin{frame}[fragile]{Solução com complexidade $O(N)$}
    \inputsnippet{cpp}{60}{79}{codes/11093.cpp}
\end{frame}

