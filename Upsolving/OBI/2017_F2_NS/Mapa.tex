\section{Mapa}

\begin{frame}[fragile]{Problema}

Harry ganhou um mapa mágico no qual ele pode visualizar o trajeto realizado por seus amigos. Ele
agora precisa de sua colaboração para, com a ajuda do mapa, determinar onde Hermione se encontra.

O mapa tem $L$ linhas e $C$ colunas de caracteres, que podem ser `\texttt{.}' (ponto), a letra 
`\texttt{o}' (minúscula) ou a letra `\texttt{H}' (maiúscula). A posição inicial de Hermione no mapa
é indicada pela letra `\texttt{o}', que aparece exatamente uma vez no mapa. A letra `\texttt{H}'
indica uma posição em que Hermione pode ter passado -- o mapa é impreciso, e nem toda letra 
`\texttt{H}' no mapa representa realmente uma posição pela qual Hermione passou. Mas todas as
posições pelas quais Hermione passou são representadas pela letra `\texttt{H}' no mapa.

\end{frame}


\begin{frame}[fragile]{Problema}

A partir da posição inicial de Hermione, Harry sabe determinar a posição atual de sua amiga, apesar
da imprecisão do mapa, porque eles combinaram que Hermione somente se moveria de forma que seu
movimento apareceria no mapa como estritamente horizontal ou estritamente vertical (nunca diagonal).
Além disso, Hermione combinou que não se moveria de forma a deixar que Harry tivesse dúvidas sobre
seu caminho (por exemplo, Hermione não passa duas vezes pela mesma posição). Considere o mapa
abaixo, com 6 linhas e 7 colunas:

\end{frame}

\begin{frame}[fragile]{Problema}

    \begin{table}[!ht]
        \centering
        \begin{tabular}{cccccccc}
            & 1 & 2 & 3 & 4 & 5 & 6 & 7 \\
            1 & . & . & . & \textbf{H} & \textbf{H} & \textbf{H} & . \\
            2 & H & H & H & . & . & . & \textbf{H} \\
            3 & H & . & H & H & H & . & . \\
            4 & H & . & . & . & H & H & . \\
            5 & H & . & o & . & . & . & . \\
            6 & H & H & H & . & . & \textbf{H} & \textbf{H} \\
        \end{tabular}
    \end{table}

\end{frame}


\begin{frame}[fragile]{Problema}

A posição inicial de Hermione no mapa é (5,3), e sua posição atual é (4,6). As posições marcadas em
negrito (`\textbf{H}') são erros no mapa.

Dado um mapa e a posição inicial de Herminone, você deve escrever um programa para determinar a
posição atual de Herminone.

\end{frame}

\begin{frame}[fragile]{Entrada e saída}

\textbf{Entrada}

A primeira linha contém dois números inteiros $L$ e $C$, indicando respectivamente o número de
linhas e o número de colunas. Cada uma das seguintes $L$ linhas contém $C$ caracteres.

\vspace{0.1in}

\textbf{Saída}

Seu programa deve produzir uma única linha na saída, contendo dois números inteiros: o número da
linha e o número da coluna da posição atual de Hermione.

\end{frame}

\begin{frame}[fragile]{Entrada e saída}

\textbf{Restrições}

\begin{itemize}
    \item $2 \leq L, C \leq 100$
    \item Apenas os caracteres `\texttt{.}', `\texttt{o}' e `\texttt{H}' aparecem no mapa.
    \item A letra `\texttt{o}' aparece exatamente uma vez no mapa.
    \item A letra `\texttt{H}' aparece ao menos uma vez no mapa.
    \item O caminho de Hermione está totalmente contido no mapa.
    \item Na posição da letra `\texttt{o}' no mapa, há apenas uma letra `\texttt{H}' como vizinho
        imediato na vertical ou horizontal.
    \item Na posição atual de Hermione no mapa, há apenas uma letra `\texttt{H}' como vizinho
        imediato na vertical ou horizontal.
    \item Em cada uma das posições intermediárias do caminho de Hermione, há exatamente duas letras
        `\texttt{H}' como vizinhas imediatas na vertical ou horizontal. 
\end{itemize}

\end{frame}

\begin{frame}[fragile]{Entrada e saída}

\textbf{Informações sobre a pontuação}

\begin{itemize}
    \item Em um conjunto de casos de teste somando 20 pontos, $N\leq 8$
\end{itemize}
\end{frame}

\begin{frame}[fragile]{Exemplo de entradas e saídas}

\begin{minipage}[t]{0.45\textwidth}
\textbf{Entrada}
\begin{verbatim}
3 4
HHHH
H...
o.HH
\end{verbatim}
\end{minipage}
\begin{minipage}[t]{0.5\textwidth}
\textbf{Saída}
\begin{verbatim}
1 4
\end{verbatim}
\end{minipage}
\end{frame}

\begin{frame}[fragile]{Exemplo de entradas e saídas}

\begin{minipage}[t]{0.45\textwidth}
\textbf{Entrada}
\begin{verbatim}
6 7
...HHH.
HHH....
H.HHH..
H...HH.
H.o....
HHH.HH.
\end{verbatim}
\end{minipage}
\begin{minipage}[t]{0.5\textwidth}
\textbf{Saída}
\begin{verbatim}
4 6
\end{verbatim}
\end{minipage}
\end{frame}


\begin{frame}[fragile]{Solução}

    \begin{itemize}
        \item O conjunto de restrições dadas para o problema simplifica a solução

        \item Isto porque as restrições delimitam um caminho único entre a posição inicial de Harry
            e a posição final de Hermione

        \item Assim, a partir da posição inicial, basta consultar os quatro vizinhos (norte, sul,
            leste e oeste) a procura de um `\texttt{H}'

        \item De acordo as restrições, haverá apenas um vizinho com tal marcação

        \item Daí, basta se mover para esta posição e apagar o `\texttt{H}', de modo que ele não
            seja mais levado em consideração
    \end{itemize}

\end{frame}

\begin{frame}[fragile]{Solução $O(LC)$}
    \inputsnippet{cpp}{1}{19}{codes/mapa.cpp}
\end{frame}

\begin{frame}[fragile]{Solução $O(LC)$}
    \inputsnippet{cpp}{20}{40}{codes/mapa.cpp}
\end{frame}

\begin{frame}[fragile]{Solução $O(LC)$}
    \inputsnippet{cpp}{42}{62}{codes/mapa.cpp}
\end{frame}
