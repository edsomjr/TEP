\begin{frame}[fragile]{Codeforces 630C -- Lucky Numbers}

{\it
The numbers of all offices in the new building of the Tax Office of IT City will have lucky numbers.

Lucky number is a number that consists of digits $7$ and $8$ only. Find the maximum number of offices in the new building of the Tax Office given that a door-plate can hold a number not longer than $n$ digits.
}

\end{frame}

\begin{frame}[fragile]{Entrada e saída}

\textbf{Input}

{\it
The only line of input contains one integer $n$ $(1 \leq n \leq 55)$ -- the maximum length of a number that a door-plate can hold.}

\vspace{0.2in}

\textbf{Output}

{\it
Output one integer -- the maximum number of offices, than can have unique lucky numbers not longer than $n$ digits.
}

\end{frame}

\begin{frame}[fragile]{Exemplos de entrada e saída}

\begin{minipage}[t]{0.45\textwidth}
\textbf{Entrada}
\begin{verbatim}
2
\end{verbatim}
\end{minipage}
\begin{minipage}[t]{0.5\textwidth}
\textbf{Saída}
\begin{verbatim}
6
\end{verbatim}
\end{minipage}

\end{frame}

\begin{frame}[fragile]{Solução com complexidade $O(1)$}

    \begin{itemize}
        \item Para um $M$ fixo, há $2^M$ números distintos que podem ser formados usando
            os dígitos 7 e 8, pois, para cada posição há duas escolhas: 7 ou 8

        \item Assim, a resposta $S$ do problema é dada por
        $$
            S = \sum_{i = 1}^N 2^i = 2^1 + 2^2 + \ldots + 2^N
        $$

        \item Observe que
        $$
            S + 1 = 1 + 2^1 + 2^2 + \ldots + 2^N = 2^{N + 1} - 1
        $$

        \item Assim $S = 2^{N + 1} - 2$ e esta expressão pode ser computada em $O(1)$ por meio
            de um deslocamento binário
    \end{itemize}

\end{frame}

\begin{frame}[fragile]{Solução com complexidade $O(1)$}
    \inputsnippet{cpp}{1}{17}{codes/630C.cpp}
\end{frame}
