\begin{frame}[fragile]{Codeforces 1033D -- Divisors}

\it{
You are given $n$ integers $a_1, a_2, \ldots, a_n$. Each of $a_i$ has between 3 and 5 divisors. Consider $a = \prod a_i$ -- 
the product of all input integers. Find the number of divisors of $a$.
As this number may be very large, print it modulo prime number 998244353.
}

\end{frame}

\begin{frame}[fragile]{Entrada e saída}

\textbf{Input}

{\it
The first line contains a single integer $n$ $(1\leq n\leq 500)$ -- the number of numbers.

Each of the next $n$ lines contains an integer $a_i$ $(1\leq a_i\leq 2\times 10^{18})$.  It is guaranteed that the number of divisors of each $a_i$ is between 3
 and 5.
}

\vspace{0.2in}

\textbf{Output}

{\it
Print a single integer $d$ -- the number of divisors of the product $a_1\cdot a_2\cdot \ldots\cdot  a_n$ modulo 998244353.
}

\end{frame}

\begin{frame}[fragile]{Exemplos de entrada e saída}

\begin{minipage}[t]{0.45\textwidth}
\textbf{Entrada}
\begin{verbatim}
3
9
15
143

1
7400840699802997
\end{verbatim}
\end{minipage}
\begin{minipage}[t]{0.5\textwidth}
\textbf{Saída}
\begin{verbatim}
32




4
\end{verbatim}
\end{minipage}

\end{frame}


\begin{frame}[fragile]{Solução $O(n^2)$}

    \begin{itemize}
        \item A restrição do número de divisores de um elemento $a_i$ implica em apenas 4 cenários
            distintos, onde $p$ e $q$ são primos distintos
        \begin{enumerate}
            \item $a_i = p^2$
            \item $a_i = p^3$
            \item $a_i = pq$
            \item $a_i = p^4$
        \end{enumerate}

        \item No primeiro caso, $\tau(p^2) = 2 + 1 = 3$, ou seja, $a_i$ tem 3 divisores

        \item Nos casos 2 e 3 temos 4 divisores, pois $\tau(p^3) = 3 + 1 = 4$ e 
            $\tau(pq) = (1 + 1)(1 + 1) = 4$

        \item No último caso $\tau(p^4) = 5$

        \item A solução portanto, depende da identificação destes casos e do devido tratamento
            dado a eles
    \end{itemize}

\end{frame}

\begin{frame}[fragile]{Solução $O(n^2)$}

    \begin{itemize}
        \item É possível determinar se um número $n$ é um quadrado perfeito por meio de uma
            rotina baseada em busca binária com complexidade $O(\log n)$

        \item Esta rotina pode identificar os casos 1 e 4

        \item Uma rotina semelhante identifica se $n$ é um cubo perfeito também em $O(\log n)$, e
            pode identificar o caso 2

        \item Nos casos 1, 2 e 4 os primos identificados pelas rotinas acima devem ser acumulados
            em um histograma que contém a fatoração do produto de todos os termos, de acordo com
            o expoente em questão
        
    \end{itemize}

\end{frame}

\begin{frame}[fragile]{Solução $O(n^2)$}

    \begin{itemize}
        \item O caso 3 é o mais difícil e que merece mais atenção e cuidado

        \item Uma vez que $a_i \leq 2\times 10^{18}$, a fatoração de tais termos não pode ser
            feita por meio de um algoritmo \textit{naive}

        \item Uma forma de obter esta fatoração é computar o maior divisor comum $d$ entre todos os
            pares de números da forma $a = pq$

        \item Caso $d$ seja um divisor próprio destes números e $d$ não for uma chave do 
            histograma da fatoração, ele deve ser registrado no histograma, inicialmente 
            associado ao expoente zero
    \end{itemize}

\end{frame}

\begin{frame}[fragile]{Solução $O(n^2)$}

    \begin{itemize}
        \item Após este processamento, as chaves primas do histograma podem ser usadas numa 
            tentativa de fatoração destes números

        \item Caso um número possa ser fatorado, os dois fatores devem atualizar o histograma

        \item Se o número não for fatorado, ele não compartilha primos com os demais número,
            exceto possivelmente com cópias idênticas de si mesmo

        \item Assim, tais números devem ser guardados em um segundo histograma
    \end{itemize}

\end{frame}

\begin{frame}[fragile]{Solução $O(n^2)$}

    \begin{itemize}
        \item Finalizado todos estes passos, a resposta pode ser computada a partir da entrada
            de ambos histogramas

        \item A resposta inicialmente é igual a 1

        \item Para todo par $(p, k)$ do primeiro histograma, a resposta deve ser atualizada
            por meio de seu produto por $(k + 1)$

        \item Para todo par $(x, c)$ do segundo histograma, a atualização deve ser por meio do
            produto por $(c + 1)^2$

        \item Ou seja, para cada conjunto de $c$ repetições do número $x = p_jq_j$, a fatoração
            do produto dos $a_i$ conterá os fatores $p_j^c$ e $q_j^c$, e cada um contribui com
            um fator $(c + 1)$ no cálculo do número de divisores deste produto
    \end{itemize}

\end{frame}
\begin{frame}[fragile]{Solução $O(n^2)$}
    \inputsnippet{cpp}{8}{24}{codes/1033D.cpp}
\end{frame}

\begin{frame}[fragile]{Solução $O(n^2)$}
    \inputsnippet{cpp}{26}{42}{codes/1033D.cpp}
\end{frame}

\begin{frame}[fragile]{Solução $O(n^2)$}
    \inputsnippet{cpp}{44}{59}{codes/1033D.cpp}
\end{frame}

\begin{frame}[fragile]{Solução $O(n^2)$}
    \inputsnippet{cpp}{60}{76}{codes/1033D.cpp}
\end{frame}

\begin{frame}[fragile]{Solução $O(n^2)$}
    \inputsnippet{cpp}{78}{93}{codes/1033D.cpp}
\end{frame}

\begin{frame}[fragile]{Solução $O(n^2)$}
    \inputsnippet{cpp}{95}{104}{codes/1033D.cpp}
\end{frame}
