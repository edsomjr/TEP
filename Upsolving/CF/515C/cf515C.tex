\begin{frame}[fragile]{Codeforces 515C -- Drazil and Factorial}

{\it
Drazil is playing a math game with Varda.

Let's define $F(x)$ or positive integer $x$ as a product of factorials of its digits. For example, $F(135) = 1! \times 3!\times 5! = 720$.

First, they choose a decimal number $a$ consisting of $n$ digits that contains at least one digit larger than 1. This number may possibly start with leading zeroes. Then they should find maximum positive number $x$ satisfying following two conditions:

\begin{enumerate}
    \item $x$ doesn't contain neither digit 0 nor digit 1.
    \item $F(x) = F(a)$.
\end{enumerate}

Help friends find such number.
}

\end{frame}

\begin{frame}[fragile]{Entrada e saída}

\textbf{Input}

{\it
The first line contains an integer $n$ $(1 \leq n \leq 15)$ -- the number of digits in $a$.

The second line contains $n$ digits of $a$. There is at least one digit in $a$ that is larger than 1. Number $a$ may possibly contain leading zeroes.
}

\vspace{0.2in}

\textbf{Output}

{\it
Output a maximum possible integer satisfying the conditions above. There should be no zeroes and ones in this number decimal representation.
}

\end{frame}

\begin{frame}[fragile]{Exemplos de entrada e saída}

\begin{minipage}[t]{0.45\textwidth}
\textbf{Entrada}
\begin{verbatim}
4
1234


3
555
\end{verbatim}
\end{minipage}
\begin{minipage}[t]{0.5\textwidth}
\textbf{Saída}
\begin{verbatim}
33222



555
\end{verbatim}
\end{minipage}

\end{frame}



\begin{frame}[fragile]{Solução $O(n\log n)$}

    \begin{itemize}
        \item Para determinar o valor de $x$ é preciso, inicialmente, determinar a fatoração
            prima de $F(a)$

        \item Como $F(a)$ é o produto do fatorial de cada dígito de $a$, esta fatoração conterá,
            no máximo, 4 primos distintos: 2, 3, 5 e 7

        \item Esta fatoração será composta pelo produto das fatorações de cada dígito de $a$

        \item Uma vez que há apenas 10 dígitos decimais e alguns deles podem se repetir em
            $a$, podemos usar um histograma para evitar o cálculo de uma mesma fatoração
            repetidas vezes
    \end{itemize}

\end{frame}

\begin{frame}[fragile]{Solução $O(n\log n)$}

    \begin{itemize}
        \item Observe que o menor fatorial que contém o primo $p$ em sua fatoração é $p!$

        \item Como desejamos o maior $x$ possível, podemos escolher, gulosamente, o maior
            dentre os fatoriais $2!, 3!, 5!$ e $7!$ que ainda pode ser formado com os fatores
            disponíveis

        \item A cada fatorial escolhido é preciso atualizar a lista dos fatores disponíveis

        \item Eventualmente o resultado pode exceder os limites de um \code{cpp}{long long},
            então utilize uma string para armazenar o resultado, evitando assim o \textit{overflow}
    \end{itemize}

\end{frame}

\begin{frame}[fragile]{Solução $O(n\log n)$}
    \inputsnippet{cpp}{18}{26}{codes/515C.cpp}
\end{frame}

\begin{frame}[fragile]{Solução $O(n\log n)$}
    \inputsnippet{cpp}{28}{39}{codes/515C.cpp}
\end{frame}

\begin{frame}[fragile]{Solução $O(n\log n)$}
    \inputsnippet{cpp}{41}{56}{codes/515C.cpp}
\end{frame}

\begin{frame}[fragile]{Solução $O(n\log n)$}
    \inputsnippet{cpp}{57}{69}{codes/515C.cpp}
\end{frame}
