\begin{frame}[fragile]{Problema}

{\it
Iahub and Iahubina went to a date at a luxury restaurant. Everything went fine until paying for the food. Instead of money, the waiter wants Iahub to write a Hungry sequence consisting of $n$ integers.

A sequence $a_1, a_2, \ldots, a_n$, consisting of $n$ integers, is Hungry if and only if:

\begin{itemize}
    \item Its elements are in increasing order. That is an inequality $a_i < a_j$ holds for any two indices $i, j$ $(i < j)$.
    \item For any two indices $i$ and $j$ $(i < j)$, $a_j$ must not be divisible by $a_i$.
\end{itemize}

Iahub is in trouble, so he asks you for help. Find a Hungry sequence with $n$ elements.
}
\end{frame}

\begin{frame}[fragile]{Entrada e saída}

\textbf{Input}

{\it
The input contains a single integer: $n$ $(1 \leq n \leq 10^5)$.
}

\vspace{0.2in}

\textbf{Output}

{\it
Output a line that contains n space-separated integers $a_1$ $a_2, \ldots, a_n$ $(1 \leq a_i \leq 10^7)$, representing a possible Hungry sequence. Note, that each 
$a_i$ must not be greater than $10000000$ $(10^7)$ and less than $1$.

If there are multiple solutions you can output any one.
}

\end{frame}

\begin{frame}[fragile]{Exemplos de entrada e saída}

\begin{minipage}[t]{0.45\textwidth}
\textbf{Entrada}
\begin{verbatim}
3


5
\end{verbatim}
\end{minipage}
\begin{minipage}[t]{0.5\textwidth}
\textbf{Saída}
\begin{verbatim}
2 9 15


11 14 20 27 31
\end{verbatim}
\end{minipage}

\end{frame}

\begin{frame}[fragile]{Solução com complexidade $O(M\log \log M)$}

    \begin{itemize}
        \item Suponha que você deseje iniciar uma sequência com estas características em $a_1 = k$

        \item Devido ao segundo critério, nenhum dos elementos subjacentes da sequência pode ser 
            múltiplo de $k$

        \item Ou seja, incluir $k$ na sequência ``criva'' todos seus múltiplos

        \item Desta maneira, iniciando com $a_1 = 2$ (pois 1 divide qualquer número) e aplicando
            o crivo de Erastótenes, os candidatos a demais elementos são todos primos

        \item Como $\pi(10^7) = 664579 > 10^5$, basta imprimir na saída os $N$ primeiros primos
    \end{itemize}

\end{frame}

\begin{frame}[fragile]{Solução com complexidade $O(M\log \log M)$}
    \inputsnippet{cpp}{1}{21}{codes/327B.cpp}
\end{frame}
