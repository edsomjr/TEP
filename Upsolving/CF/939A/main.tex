\documentclass{whiteboard}
\begin{document}
\begin{frame}[plain,t]
 \bbcover{Codeforces Round \#464 (Div. 2)}{Problem A -- Love Triangle}{Prof. Edson Alves}{Faculdade UnB Gama}
\end{frame}

\begin{frame}[plain,t]
\vspace*{\fill}
 \bbenglish{ As you could know there are no male planes nor female planes. However, each plane on Earth likes some other plane. There are $n$ planes on Earth, numbered from $1$ to $n$, and the plane with number $i$ likes the plane with number $f_i$, where $1\leq f_i\leq n$ and $f_i\neq i$.}

 \vspace{0.2in}

 \bbenglish{We call a love triangle a situation in which plane $A$ likes plane $B$, plane $B$ likes plane $C$ and plane $C$ likes plane $A$. Find out if there is any love triangle on Earth.}
\vspace*{\fill}
\end{frame}

\begin{frame}[plain,t]
\vspace*{\fill}
 \bbtext{Como você deve saber, não há aviões machos ou aviões fêmeas. Contudo, cada avião na Terra gosta de algum outro avião. Há $n$ aviões na Terra, numerados de $1$ a $n$, e o avião número $i$ gosta do avião número $f_i$, onde $1\leq f_i\leq n$ e $f_i\neq i$.}

 \vspace{0.2in}

 \bbtext{Denominaremos um triângulo amoroso uma situação na qual o avião $A$ gosta do avião $B$, o avião $B$ gosta do avião $C$ e o avião $C$ gosta do avião $A$. Determine se há algum triângulo amoroso na Terra.}
\vspace*{\fill}
\end{frame}

\begin{frame}[plain,t]
\vspace*{\fill}
 \bbbold{Input}

 \vspace{0.1in}

 \bbenglish{The first line contains a single integer $n$ $(2\leq n\leq 5000)$ -- the number of planes.}

 \vspace{0.1in}

 \bbenglish{The second line contains $n$ integers $f_1, f_2, \ldots, f_n$ $(1\leq f_i\leq n, f_i\neq i)$, meaning that the $i$-th plane likes the $f_i$-th.}

 \vspace{0.2in}

 \bbbold{Output}

 \vspace{0.1in}

 \bbenglish{Output «\texttt{YES}» if there is a love triangle consisting of planes on Earth. Otherwise, output «\texttt{NO}».}
\vspace*{\fill}
\end{frame}

\begin{frame}[plain,t]
\vspace*{\fill}
 \bbbold{Entrada}

 \vspace{0.1in}

 \bbtext{A primeira linha contém um único inteiro $n$ $(2\leq n\leq 5000)$ -- o número de aviões.}

 \vspace{0.1in}

 \bbtext{A segunda linha contém $n$ inteiros $f_1, f_2, \ldots, f_n$ $(1\leq f_i\leq n, f_i\neq i)$, os quais indicam que o $i$-ésimo avião gosta do $f_i$-ésimo avião.}

 \vspace{0.2in}

 \bbbold{Saída}

 \vspace{0.1in}

 \bbtext{Imprima «\texttt{YES}» se há um triângulo amoroso entre os aviões na Terra. Caso contrário, imprima «\texttt{NO}».}
\vspace*{\fill}
\end{frame}

\begin{frame}[plain,t]
\begin{tikzpicture}
\node[draw,opacity=0] at (0, 0) {x};
\node[draw,opacity=0] at (14, 8) {x};
 \node[anchor=west] at (0, 7) { \bbbold{Exemplo de entrada e saída} };
\end{tikzpicture}
\end{frame}

\begin{frame}[plain,t]
\begin{tikzpicture}
\node[draw,opacity=0] at (0, 0) {x};
\node[draw,opacity=0] at (14, 8) {x};
 \node[anchor=west] at (0, 7) { \bbbold{Exemplo de entrada e saída} };
 \node at (1, 6) { \bbtext{5} };
\end{tikzpicture}
\end{frame}

\begin{frame}[plain,t]
\begin{tikzpicture}
\node[draw,opacity=0] at (0, 0) {x};
\node[draw,opacity=0] at (14, 8) {x};
 \node[anchor=west] at (0, 7) { \bbbold{Exemplo de entrada e saída} };
 \node at (1, 6) { \bbtext{5} };
 \node at (1, 5) { \footnotesize \bbcomment{\# de aviões} };
 \draw[->,color=BBViolet] (1, 5.2) -- (1, 5.8);
\end{tikzpicture}
\end{frame}

\begin{frame}[plain,t]
\begin{tikzpicture}
\node[draw,opacity=0] at (0, 0) {x};
\node[draw,opacity=0] at (14, 8) {x};
 \node[anchor=west] at (0, 7) { \bbbold{Exemplo de entrada e saída} };
 \node at (1, 6) { \bbtext{5} };
 \node at (1, 5) { \bbtext{2} };
 \node at (1.5, 5) { \bbtext{4} };
 \node at (2, 5) { \bbtext{5} };
 \node at (2.5, 5) { \bbtext{1} };
 \node at (3, 5) { \bbtext{3} };
\end{tikzpicture}
\end{frame}

\begin{frame}[plain,t]
\begin{tikzpicture}
\node[draw,opacity=0] at (0, 0) {x};
\node[draw,opacity=0] at (14, 8) {x};
 \node[anchor=west] at (0, 7) { \bbbold{Exemplo de entrada e saída} };
 \node at (1, 6) { \bbtext{5} };
 \node at (1, 5) { \bbtext{2} };
 \node at (1.5, 5) { \bbtext{4} };
 \node at (2, 5) { \bbtext{5} };
 \node at (2.5, 5) { \bbtext{1} };
 \node at (3, 5) { \bbtext{3} };
 \node at (1, 4) { $f_1$ };
 \node at (1.5, 4) { $f_2$ };
 \node at (2, 4) { $f_3$ };
 \node at (2.5, 4) { $f_4$ };
 \node at (3, 4) { $f_5$ };
 \draw[->,color=BBViolet] (1, 4.3) to (1, 4.7);
 \draw[->,color=BBViolet] (1.5, 4.3) to (1.5, 4.7);
 \draw[->,color=BBViolet] (2, 4.3) to (2, 4.7);
 \draw[->,color=BBViolet] (2.5, 4.3) to (2.5, 4.7);
 \draw[->,color=BBViolet] (3, 4.3) to (3, 4.7);
\end{tikzpicture}
\end{frame}

\begin{frame}[plain,t]
\begin{tikzpicture}
\node[draw,opacity=0] at (0, 0) {x};
\node[draw,opacity=0] at (14, 8) {x};
 \node[anchor=west] at (0, 7) { \bbbold{Exemplo de entrada e saída} };
 \node at (1, 6) { \bbtext{5} };
 \node at (1, 5) { \bbtext{2} };
 \node at (1.5, 5) { \bbtext{4} };
 \node at (2, 5) { \bbtext{5} };
 \node at (2.5, 5) { \bbtext{1} };
 \node at (3, 5) { \bbtext{3} };
 \node[draw,circle,very thick] (A1) at (6, 4) { \bbtext{1} };
 \node[draw,circle,very thick] (A2) at (9, 7) { \bbtext{2} };
 \node[draw,circle,very thick] (A3) at (12, 4) { \bbtext{3} };
 \node[draw,circle,very thick] (A4) at (11, 1) { \bbtext{4} };
 \node[draw,circle,very thick] (A5) at (7, 1) { \bbtext{5} };
\end{tikzpicture}
\end{frame}

\begin{frame}[plain,t]
\begin{tikzpicture}
\node[draw,opacity=0] at (0, 0) {x};
\node[draw,opacity=0] at (14, 8) {x};
 \node[anchor=west] at (0, 7) { \bbbold{Exemplo de entrada e saída} };
 \node at (1, 6) { \bbtext{5} };
 \node at (1, 5) { \bbtext{2} };
 \node at (1.5, 5) { \bbtext{4} };
 \node at (2, 5) { \bbtext{5} };
 \node at (2.5, 5) { \bbtext{1} };
 \node at (3, 5) { \bbtext{3} };
 \node[draw,circle,very thick] (A1) at (6, 4) { \bbtext{1} };
 \node[draw,circle,very thick] (A2) at (9, 7) { \bbtext{2} };
 \node[draw,circle,very thick] (A3) at (12, 4) { \bbtext{3} };
 \node[draw,circle,very thick] (A4) at (11, 1) { \bbtext{4} };
 \node[draw,circle,very thick] (A5) at (7, 1) { \bbtext{5} };
 \draw[-latex,very thick] (A1) to (A2);
\end{tikzpicture}
\end{frame}

\begin{frame}[plain,t]
\begin{tikzpicture}
\node[draw,opacity=0] at (0, 0) {x};
\node[draw,opacity=0] at (14, 8) {x};
 \node[anchor=west] at (0, 7) { \bbbold{Exemplo de entrada e saída} };
 \node at (1, 6) { \bbtext{5} };
 \node at (1, 5) { \bbtext{2} };
 \node at (1.5, 5) { \bbtext{4} };
 \node at (2, 5) { \bbtext{5} };
 \node at (2.5, 5) { \bbtext{1} };
 \node at (3, 5) { \bbtext{3} };
 \node[draw,circle,very thick] (A1) at (6, 4) { \bbtext{1} };
 \node[draw,circle,very thick] (A2) at (9, 7) { \bbtext{2} };
 \node[draw,circle,very thick] (A3) at (12, 4) { \bbtext{3} };
 \node[draw,circle,very thick] (A4) at (11, 1) { \bbtext{4} };
 \node[draw,circle,very thick] (A5) at (7, 1) { \bbtext{5} };
 \draw[-latex,very thick] (A1) to (A2);
 \draw[-latex,very thick] (A2) to (A4);
\end{tikzpicture}
\end{frame}

\begin{frame}[plain,t]
\begin{tikzpicture}
\node[draw,opacity=0] at (0, 0) {x};
\node[draw,opacity=0] at (14, 8) {x};
 \node[anchor=west] at (0, 7) { \bbbold{Exemplo de entrada e saída} };
 \node at (1, 6) { \bbtext{5} };
 \node at (1, 5) { \bbtext{2} };
 \node at (1.5, 5) { \bbtext{4} };
 \node at (2, 5) { \bbtext{5} };
 \node at (2.5, 5) { \bbtext{1} };
 \node at (3, 5) { \bbtext{3} };
 \node[draw,circle,very thick] (A1) at (6, 4) { \bbtext{1} };
 \node[draw,circle,very thick] (A2) at (9, 7) { \bbtext{2} };
 \node[draw,circle,very thick] (A3) at (12, 4) { \bbtext{3} };
 \node[draw,circle,very thick] (A4) at (11, 1) { \bbtext{4} };
 \node[draw,circle,very thick] (A5) at (7, 1) { \bbtext{5} };
 \draw[-latex,very thick] (A1) to (A2);
 \draw[-latex,very thick] (A2) to (A4);
 \draw[-latex,very thick] (A3) [bend left] to (A5);
\end{tikzpicture}
\end{frame}

\begin{frame}[plain,t]
\begin{tikzpicture}
\node[draw,opacity=0] at (0, 0) {x};
\node[draw,opacity=0] at (14, 8) {x};
 \node[anchor=west] at (0, 7) { \bbbold{Exemplo de entrada e saída} };
 \node at (1, 6) { \bbtext{5} };
 \node at (1, 5) { \bbtext{2} };
 \node at (1.5, 5) { \bbtext{4} };
 \node at (2, 5) { \bbtext{5} };
 \node at (2.5, 5) { \bbtext{1} };
 \node at (3, 5) { \bbtext{3} };
 \node[draw,circle,very thick] (A1) at (6, 4) { \bbtext{1} };
 \node[draw,circle,very thick] (A2) at (9, 7) { \bbtext{2} };
 \node[draw,circle,very thick] (A3) at (12, 4) { \bbtext{3} };
 \node[draw,circle,very thick] (A4) at (11, 1) { \bbtext{4} };
 \node[draw,circle,very thick] (A5) at (7, 1) { \bbtext{5} };
 \draw[-latex,very thick] (A3) [bend left] to (A5);
 \draw[-latex,very thick,color=BBOrange] (A1) to (A2);
 \draw[-latex,very thick,color=BBOrange] (A2) to (A4);
 \draw[-latex,very thick,color=BBOrange] (A4) to (A1);
 \draw[-latex,very thick] (3.5, 5) to (4, 5);
 \node[anchor=west] at (4, 5) { \bbinfo{\tt YES} };
\end{tikzpicture}
\end{frame}

\begin{frame}[plain,t]
\begin{tikzpicture}
\node[draw,opacity=0] at (0, 0) {x};
\node[draw,opacity=0] at (14, 8) {x};
 \node[anchor=west] at (0, 7) { \bbbold{Exemplo de entrada e saída} };
 \node at (1, 6) { \bbtext{5} };
 \node at (1, 5) { \bbtext{2} };
 \node at (1.5, 5) { \bbtext{4} };
 \node at (2, 5) { \bbtext{5} };
 \node at (2.5, 5) { \bbtext{1} };
 \node at (3, 5) { \bbtext{3} };
 \node[draw,circle,very thick] (A1) at (6, 4) { \bbtext{1} };
 \node[draw,circle,very thick] (A2) at (9, 7) { \bbtext{2} };
 \node[draw,circle,very thick] (A3) at (12, 4) { \bbtext{3} };
 \node[draw,circle,very thick] (A4) at (11, 1) { \bbtext{4} };
 \node[draw,circle,very thick] (A5) at (7, 1) { \bbtext{5} };
 \draw[-latex,very thick] (A3) [bend left] to (A5);
 \draw[-latex,very thick,color=BBOrange] (A1) to (A2);
 \draw[-latex,very thick,color=BBOrange] (A2) to (A4);
 \draw[-latex,very thick,color=BBOrange] (A4) to (A1);
 \draw[-latex,very thick] (3.5, 5) to (4, 5);
 \node[anchor=west] at (4, 5) { \bbinfo{\tt YES} };
 \draw[-latex,very thick] (A5) [bend left] to (A3);
\end{tikzpicture}
\end{frame}

\begin{frame}[plain,t]
\begin{tikzpicture}
\node[draw,opacity=0] at (0, 0) {x};
\node[draw,opacity=0] at (14, 8) {x};
 \node[anchor=west] at (0, 7) { \bbbold{Exemplo de entrada e saída} };
 \node at (1, 6) { \bbtext{5} };
 \node at (1, 5) { \bbtext{5} };
 \node at (1.5, 5) { \bbtext{5} };
 \node at (2, 5) { \bbtext{5} };
 \node at (2.5, 5) { \bbtext{5} };
 \node at (3, 5) { \bbtext{1} };
\end{tikzpicture}
\end{frame}

\begin{frame}[plain,t]
\begin{tikzpicture}
\node[draw,opacity=0] at (0, 0) {x};
\node[draw,opacity=0] at (14, 8) {x};
 \node[anchor=west] at (0, 7) { \bbbold{Exemplo de entrada e saída} };
 \node at (1, 6) { \bbtext{5} };
 \node at (1, 5) { \bbtext{5} };
 \node at (1.5, 5) { \bbtext{5} };
 \node at (2, 5) { \bbtext{5} };
 \node at (2.5, 5) { \bbtext{5} };
 \node at (3, 5) { \bbtext{1} };
 \node[draw,circle,very thick] (A1) at (6, 4) { \bbtext{1} };
 \node[draw,circle,very thick] (A2) at (9, 7) { \bbtext{2} };
 \node[draw,circle,very thick] (A3) at (12, 4) { \bbtext{3} };
 \node[draw,circle,very thick] (A4) at (11, 1) { \bbtext{4} };
 \node[draw,circle,very thick] (A5) at (7, 1) { \bbtext{5} };
\end{tikzpicture}
\end{frame}

\begin{frame}[plain,t]
\begin{tikzpicture}
\node[draw,opacity=0] at (0, 0) {x};
\node[draw,opacity=0] at (14, 8) {x};
 \node[anchor=west] at (0, 7) { \bbbold{Exemplo de entrada e saída} };
 \node at (1, 6) { \bbtext{5} };
 \node at (1, 5) { \bbtext{5} };
 \node at (1.5, 5) { \bbtext{5} };
 \node at (2, 5) { \bbtext{5} };
 \node at (2.5, 5) { \bbtext{5} };
 \node at (3, 5) { \bbtext{1} };
 \node[draw,circle,very thick] (A1) at (6, 4) { \bbtext{1} };
 \node[draw,circle,very thick] (A2) at (9, 7) { \bbtext{2} };
 \node[draw,circle,very thick] (A3) at (12, 4) { \bbtext{3} };
 \node[draw,circle,very thick] (A4) at (11, 1) { \bbtext{4} };
 \node[draw,circle,very thick] (A5) at (7, 1) { \bbtext{5} };
 \draw[-latex,very thick] (A1) [bend left] to (A5);
 \draw[-latex,very thick] (A2) to (A5);
 \draw[-latex,very thick] (A3) to (A5);
 \draw[-latex,very thick] (A4) to (A5);
 \draw[-latex,very thick] (A5) [bend left] to (A1);
\end{tikzpicture}
\end{frame}

\begin{frame}[plain,t]
\begin{tikzpicture}
\node[draw,opacity=0] at (0, 0) {x};
\node[draw,opacity=0] at (14, 8) {x};
 \node[anchor=west] at (0, 7) { \bbbold{Exemplo de entrada e saída} };
 \node at (1, 6) { \bbtext{5} };
 \node at (1, 5) { \bbtext{5} };
 \node at (1.5, 5) { \bbtext{5} };
 \node at (2, 5) { \bbtext{5} };
 \node at (2.5, 5) { \bbtext{5} };
 \node at (3, 5) { \bbtext{1} };
 \node[draw,circle,very thick] (A1) at (6, 4) { \bbtext{1} };
 \node[draw,circle,very thick] (A2) at (9, 7) { \bbtext{2} };
 \node[draw,circle,very thick] (A3) at (12, 4) { \bbtext{3} };
 \node[draw,circle,very thick] (A4) at (11, 1) { \bbtext{4} };
 \node[draw,circle,very thick] (A5) at (7, 1) { \bbtext{5} };
 \draw[-latex,very thick] (A1) [bend left] to (A5);
 \draw[-latex,very thick] (A2) to (A5);
 \draw[-latex,very thick] (A3) to (A5);
 \draw[-latex,very thick] (A4) to (A5);
 \draw[-latex,very thick] (A5) [bend left] to (A1);
 \draw[-latex,very thick] (3.5, 5) to (4, 5);
 \node[anchor=west] at (4, 5) { \bbinfo{\tt NO} };
\end{tikzpicture}
\end{frame}

\begin{frame}[plain,t]
\begin{tikzpicture}
\node[draw,opacity=0] at (0, 0) {x};
\node[draw,opacity=0] at (14, 8) {x};
 \node[anchor=west] at (0, 7) { \Large \bbbold{Solução} };
\end{tikzpicture}
\end{frame}

\begin{frame}[plain,t]
\begin{tikzpicture}
\node[draw,opacity=0] at (0, 0) {x};
\node[draw,opacity=0] at (14, 8) {x};
 \node[anchor=west] at (0, 7) { \Large \bbbold{Solução} };
 \node[draw,circle,very thick] (A) at (2, 5) { \bbtext{A} };
\end{tikzpicture}
\end{frame}

\begin{frame}[plain,t]
\begin{tikzpicture}
\node[draw,opacity=0] at (0, 0) {x};
\node[draw,opacity=0] at (14, 8) {x};
 \node[anchor=west] at (0, 7) { \Large \bbbold{Solução} };
 \node[draw,circle,very thick] (A) at (2, 5) { \bbtext{A} };
 \node[draw,circle,very thick] (B) at (5, 5) { \bbtext{B} };
 \draw[-latex,very thick] (A) to node[anchor=south] { $f(A)$ } (B);
\end{tikzpicture}
\end{frame}

\begin{frame}[plain,t]
\begin{tikzpicture}
\node[draw,opacity=0] at (0, 0) {x};
\node[draw,opacity=0] at (14, 8) {x};
 \node[anchor=west] at (0, 7) { \Large \bbbold{Solução} };
 \node[draw,circle,very thick] (A) at (2, 5) { \bbtext{A} };
 \node[draw,circle,very thick] (B) at (5, 5) { \bbtext{B} };
 \draw[-latex,very thick] (A) to node[anchor=south] { $f(A)$ } (B);
 \node[draw,circle,very thick] (C) at (8, 5) { \bbtext{C} };
 \draw[-latex,very thick] (B) to node[anchor=south] { $f(B)$ } (C);
\end{tikzpicture}
\end{frame}

\begin{frame}[plain,t]
\begin{tikzpicture}
\node[draw,opacity=0] at (0, 0) {x};
\node[draw,opacity=0] at (14, 8) {x};
 \node[anchor=west] at (0, 7) { \Large \bbbold{Solução} };
 \node[draw,circle,very thick] (A) at (2, 5) { \bbtext{A} };
 \node[draw,circle,very thick] (B) at (5, 5) { \bbtext{B} };
 \draw[-latex,very thick] (A) to node[anchor=south] { $f(A)$ } (B);
 \node[draw,circle,very thick] (C) at (8, 5) { \bbtext{C} };
 \draw[-latex,very thick] (B) to node[anchor=south] { $f(B)$ } (C);
 \node[draw,circle,very thick] (X) at (11, 5) { \bbtext{X} };
 \draw[-latex,very thick] (C) to node[anchor=south] { $f(C)$ } (X);
\end{tikzpicture}
\end{frame}

\begin{frame}[plain,t]
\begin{tikzpicture}
\node[draw,opacity=0] at (0, 0) {x};
\node[draw,opacity=0] at (14, 8) {x};
 \node[anchor=west] at (0, 7) { \Large \bbbold{Solução} };
 \node[draw,circle,very thick] (A) at (2, 5) { \bbtext{A} };
 \node[draw,circle,very thick] (B) at (5, 5) { \bbtext{B} };
 \draw[-latex,very thick] (A) to node[anchor=south] { $f(A)$ } (B);
 \node[draw,circle,very thick] (C) at (8, 5) { \bbtext{C} };
 \draw[-latex,very thick] (B) to node[anchor=south] { $f(B)$ } (C);
 \node[draw,circle,very thick] (X) at (11, 5) { \bbtext{X} };
 \draw[-latex,very thick] (C) to node[anchor=south] { $f(C)$ } (X);
 \draw[-latex,very thick] (B) to [bend right] node[anchor=north] { $f(f(A))$ } (C);
\end{tikzpicture}
\end{frame}

\begin{frame}[plain,t]
\begin{tikzpicture}
\node[draw,opacity=0] at (0, 0) {x};
\node[draw,opacity=0] at (14, 8) {x};
 \node[anchor=west] at (0, 7) { \Large \bbbold{Solução} };
 \node[draw,circle,very thick] (A) at (2, 5) { \bbtext{A} };
 \node[draw,circle,very thick] (B) at (5, 5) { \bbtext{B} };
 \draw[-latex,very thick] (A) to node[anchor=south] { $f(A)$ } (B);
 \node[draw,circle,very thick] (C) at (8, 5) { \bbtext{C} };
 \draw[-latex,very thick] (B) to node[anchor=south] { $f(B)$ } (C);
 \node[draw,circle,very thick] (X) at (11, 5) { \bbtext{X} };
 \draw[-latex,very thick] (C) to node[anchor=south] { $f(C)$ } (X);
 \draw[-latex,very thick] (B) to [bend right] node[anchor=north] { $f(f(A))$ } (C);
 \draw[-latex,very thick] (C) to [bend right] node[anchor=north] { $f(f(f(A)))$ } (X);
\end{tikzpicture}
\end{frame}

\begin{frame}[plain,t]
\begin{tikzpicture}
\node[draw,opacity=0] at (0, 0) {x};
\node[draw,opacity=0] at (14, 8) {x};
 \node[anchor=west] at (0, 7) { \Large \bbbold{Solução} };
 \node[draw,circle,very thick] (A) at (2, 5) { \bbtext{A} };
 \node[draw,circle,very thick] (B) at (5, 5) { \bbtext{B} };
 \draw[-latex,very thick] (A) to node[anchor=south] { $f(A)$ } (B);
 \node[draw,circle,very thick] (C) at (8, 5) { \bbtext{C} };
 \draw[-latex,very thick] (B) to node[anchor=south] { $f(B)$ } (C);
 \node[draw,circle,very thick] (X) at (11, 5) { \bbtext{X} };
 \draw[-latex,very thick] (C) to node[anchor=south] { $f(C)$ } (X);
 \draw[-latex,very thick] (B) to [bend right] node[anchor=north] { $f(f(A))$ } (C);
 \draw[-latex,very thick] (C) to [bend right] node[anchor=north] { $f(f(f(A)))$ } (X);
 \node at (6.5, 2) { \Large { $A = f(f(f(A)))$ \bbtext{?} } };
\end{tikzpicture}
\end{frame}

\begin{frame}[plain,t]
 \inputsnippet{cpp}{1}{18}{codes/939A.cpp}
\end{frame}

\begin{frame}[plain,t]
 \inputsnippet{cpp}{19}{36}{codes/939A.cpp}
\end{frame}

\end{document}
