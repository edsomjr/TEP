\section{C -- Be Together}

\begin{frame}[fragile]{Problema}

Evi has $N$ integers $a_1, a_2, \ldots, a_N$. His objective is to have $N$ equal integers by
transforming some of them.

He may transform each integer at most once. Transforming an integer $x$ into another integer $y$
costs him $(x - y)^2$ dollars. Even if $a_i = a_j$ ($i\neq j$), he has to pay the cost separately
for transforming each of them (See Sample 2).

Find the minimum total cost to achieve his objective.

\end{frame}

\begin{frame}[fragile]{Entrada e saída}

\textbf{Constraints}

\begin{itemize}
    \item $1\leq N \leq 100$
    \item $-100\leq a_i \leq 100$
\end{itemize}

\vspace{0.1in}

\textbf{Input}

Input is given from Standard Input in the following format:
\begin{atcoderio}{llll}
$N$ \\
$a_1$ & $a_2$ & \ldots & $a_N$ \\
\end{atcoderio}

\textbf{Output}

Print the minimum total cost to achieve Evi's objective.

\end{frame}

\begin{frame}[fragile]{Exemplo de entradas e saídas}

\begin{minipage}[t]{0.45\textwidth}
\textbf{Entrada}
\begin{verbatim}
2
4 8


3
1 1 3


3
4 2 5
\end{verbatim}
\end{minipage}
\begin{minipage}[t]{0.5\textwidth}
\textbf{Saída}
\begin{verbatim}
8



3



5
\end{verbatim}
\end{minipage}
\end{frame}

\begin{frame}[fragile]{Solução}

    \begin{itemize}
        \item Considere, sem perda de generalidade, que $a_1$ seja o menor dentre todos os números,
            e que $a_N$ seja o maior dentre eles

        \item Seja $C(x)$ o custo para transformar todos os elementos $a_1, a_2, \ldots, a_N$

        \item Observe que $C(x) > C(a_1)$, se $x < a_1$

        \item Isto porque, estando $x$ à esquerda de todos os números, se aproximar de $a_1$ 
            reduz todos os custos individuais e, portanto, o custo total da transformação
        
        \item De forma análoga, $C(x) > C(a_N)$, se $x > a_N$

    \end{itemize}

\end{frame}


\begin{frame}[fragile]{Solução}

    \begin{itemize}
        \item Logo, o valor de $x$ que minimiza o custo $C(x)$ deve pertencer ao intervalo
            $[a_1, a_N]$

        \item Como os limites do problema são relativamente pequenos, computar $C(x)$ para todos
            os $x$ neste intervalo e registrar o menor valor de $C(x)$ teria complexidade $O(M)$ e
            veredito AC, onde 
            \[
                M = \max\{ |a_1|, |a_2|, \ldots, |a_N| \}
            \]

        \item Contudo, este problema pode ser resolvido em $O(1)$

        \item Temos que
        \[
            C(x) = (x - a_1)^2 + (x - a_2)^2 + \ldots + (x - a_N)^2
        \]
    \end{itemize}

\end{frame}


\begin{frame}[fragile]{Solução}

    \begin{itemize}
        \item Derivando em relação a $x$ obtemos
        \[
            C'(x) = 2(x - a_1) + 2(x - a_2) + \ldots + 2(x - a_N)
        \]

        \item A função $C(x)$ terá um ponto crítico em $y$ se $C'(y) = 0$

        \item Esta igualdade nos dá
        \[
            y = \frac{a_1 + a_2 + \ldots + a_N}{N}
        \]

        \item Assim, $C(x)$ será minimizada quando $x$ for a média aritmética dos números
            $a_1, a_2, \ldots, a_N$

        \item Como os números devem ser todos inteiros após a transformação, o valor da média
            deve ser arredondado para o cálculo de $C(y)$, que será a resposta do problema
    \end{itemize}

\end{frame}
\begin{frame}[fragile]{Solução $O(N)$}
    \inputsnippet{cpp}{1}{15}{codes/C.cpp}
\end{frame}

\begin{frame}[fragile]{Solução $O(N)$}
    \inputsnippet{cpp}{16}{36}{codes/C.cpp}
\end{frame}
