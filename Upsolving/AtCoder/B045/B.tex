\section{B -- Card Game for Three (ABC Edit)}

\begin{frame}[fragile]{Problema}

Alice, Bob and Charlie are playing \textit{Card Game for Three}, as below:

\begin{itemize}
    \item At first, each of the three players has a deck consisting of some number of cards. Each
        card has a letter \texttt{\textcolor{red}{'a'}}, \texttt{\textcolor{red}{'b'}} or
        \texttt{\textcolor{red}{'c'}} written on it. The orders of the cards in the decks cannot be
        rearranged.

    \item The players take turns. Alice goes first.

    \item If the current player's deck contains at least one card, discard the top card in the
        deck. Then, the player whose name begins with the letter on the discarded card, takes the
        next turn. (For example, if the card says \texttt{\textcolor{red}{'a'}}, Alice takes the
        next turn.)

    \item If the current player's deck is empty, the game ends and the current player wins the
        game. 
\end{itemize}

\end{frame}

\begin{frame}[fragile]{Problema}

You are given the initial decks of the players. More specifically, you are given three strings 
$S_A$, $S_B$ and $S_C$. The $i$-th ($1\leq i\leq |S_A|$) letter in $S_A$ is the letter on the 
$i$-th card in Alice's initial deck. $S_B$ and $S_C$ describes Bob's and Charlie's initial decks in
the same way.

Determine the winner of the game.

\end{frame}

\begin{frame}[fragile]{Entrada e saída}

\textbf{Constraints}

\begin{itemize}
    \item $1\leq |S_A| \leq 100$
    \item $1\leq |S_B| \leq 100$
    \item $1\leq |S_C| \leq 100$
    \item Each letter in $S_A, S_B, S_C$ is \texttt{\textcolor{red}{'a'}},
        \texttt{\textcolor{red}{'b'}} or \texttt{\textcolor{red}{'c'}}.
\end{itemize}

\end{frame}

\begin{frame}[fragile]{Entrada e saída}

\textbf{Input}

Input is given from Standard Input in the following format:
\begin{atcoderio}{ll}
$S_A$\\
$S_B$\\
$S_C$\\
\end{atcoderio}

\textbf{Output}

If Alice will win, print \texttt{\textcolor{red}{'A'}}. If Bob will win, print
\texttt{\textcolor{red}{'B'}}. If Charlie will win, print \texttt{\textcolor{red}{'C'}}.

\end{frame}

\begin{frame}[fragile]{Exemplo de entradas e saídas}

\begin{minipage}[t]{0.45\textwidth}
\textbf{Entrada}
\begin{verbatim}
aca
accc
ca


abcb
aacb
bccc
\end{verbatim}
\end{minipage}
\begin{minipage}[t]{0.5\textwidth}
\textbf{Saída}
\begin{verbatim}
A




C
\end{verbatim}
\end{minipage}
\end{frame}

\begin{frame}[fragile]{Solução}

    \begin{itemize}
        \item A solução do problema consiste em simular o comportamento descrito do editor

        \item Seja $N$ o tamanho de uma string $S$

        \item A remoção de um caractere do início de uma string não faz parte da API do C++, e 
            mesmo que seja implementada ela terá complexidade $O(N)$, por conta do deslocamento
            dos caracteres

        \item Já o método \code{cpp}{pop_back()} permite a exclusão do último caractere em $O(1)$

        \item Assim, se as strings forem invertidas (o método \code{cpp}{reverse()} pode ser
            utilizado para este fim), a simulação fica simplificada e mais eficiente

        \item Portanto a solução tem complexidade $O(M)$, onde
        \[
            M = \max\{ S_A, S_B, S_C \}
        \]
    \end{itemize}

\end{frame}

\begin{frame}[fragile]{Solução $O(M)$}
    \inputsnippet{cpp}{1}{21}{codes/B.cpp}
\end{frame}

\begin{frame}[fragile]{Solução $O(M)$}
    \inputsnippet{cpp}{22}{42}{codes/B.cpp}
\end{frame}
