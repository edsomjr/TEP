\section{A -- Iroha and Haiku}

\begin{frame}[fragile]{Problema}

Iroha loves Haiku. Haiku is a short form of Japanese poetry. A Haiku consists of three phrases with 
5, 7 and 5 syllables, in this order.

To create a Haiku, Iroha has come up with three different phrases. These phrases have $A$, $B$ and 
$C$ syllables, respectively. Determine whether she can construct a Haiku by using each of the
phrases once, in some order.

\end{frame}

\begin{frame}[fragile]{Entrada e saída}

\textbf{Constraints}

\begin{itemize}
    \item $1\leq A, B, C\leq 10$
\end{itemize}

\vspace{0.1in}

\textbf{Input}

Input is given from Standard Input in the following format:
\begin{atcoderio}{lll}
$A$ & $B$ & $C$ \\
\end{atcoderio}

\textbf{Output}

If it is possible to construct a Haiku by using each of the phrases once, print
\texttt{\textcolor{red}{'YES'}} YES (case-sensitive). Otherwise, print 
\texttt{\textcolor{red}{'NO'}}.

\end{frame}

\begin{frame}[fragile]{Exemplo de entradas e saídas}

\begin{minipage}[t]{0.45\textwidth}
\textbf{Entrada}
\begin{verbatim}
5 5 7


7 7 5
\end{verbatim}
\end{minipage}
\begin{minipage}[t]{0.5\textwidth}
\textbf{Saída}
\begin{verbatim}
YES


NO
\end{verbatim}
\end{minipage}
\end{frame}

\begin{frame}[fragile]{Solução}

    \begin{itemize}
        \item O problema consiste em saber se os números $A, B$ e $C$ consistem na sequência
            5, 7, 5, em alguma ordem

        \item Há um total de $3! = 6$ permutações possíveis dos valores de $A$, $B$ e $C$

        \item Contudo, como o 5 se repete na sequência, basta verificar apenas três destas
            seis permutações

        \item Outra solução possível é armazenar os valores de $A, B$ e $C$ em um vetor, ordená-lo
            e, em seguida, compará-lo com a sequência 5, 5, 7

        \item Qualquer uma das duas abordagens resolve o problema em $O(1)$
    \end{itemize}

\end{frame}

\begin{frame}[fragile]{Solução $O(1)$}
    \inputsnippet{cpp}{1}{21}{codes/A.cpp}
\end{frame}
