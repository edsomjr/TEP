\section{D -- An Invisible Hand}

\begin{frame}[fragile]{Problema}

There are $N$ towns located in a line, conveniently numbered 1 through $N$. Takahashi the 
merchant is going on a travel from town 1 to town $N$, buying and selling apples.

Takahashi will begin the travel at town 1, with no apple in his possession. The actions that 
can be performed during the travel are as follows:

\begin{itemize}
    \item \textit{Move}: When at town $i$ $(i < N)$, move to town $i + 1$.

    \item \textit{Merchandise}: Buy or sell an arbitrary number of apples at the current town. 
        Here, it is assumed that one apple can always be bought and sold for $A_i$ yen (the 
        currency of Japan) at town $i$ $(1\leq i\leq N)$), where $A_i$ are distinct integers. 
        Also, you can assume that he has an infinite supply of money.
\end{itemize}

For some reason, there is a constraint on merchandising apple during the travel: the sum of the 
number of apples bought and the number of apples sold during the whole travel, must be at most 
$T$. (Note that a single apple can be counted in both.)

\end{frame}


\begin{frame}[fragile]{Problema}

During the travel, Takahashi will perform actions so that the profit of the travel is maximized.
Here, the profit of the travel is the amount of money that is gained by selling apples, minus 
the amount of money that is spent on buying apples. Note that we are not interested in apples 
in his possession at the end of the travel.

Aoki, a business rival of Takahashi, wants to trouble Takahashi by manipulating the market 
price of apples. Prior to the beginning of Takahashi's travel, Aoki can change $A_i$ into 
another arbitrary non-negative integer $A'_i$ for any town $i$, any number of times. The cost 
of performing this operation is $|A_i - A'_i|$. After performing this operation, different 
towns may have equal values of $A_i$.

Aoki's objective is to decrease Takahashi's expected profit by at least 1 yen. Find the minimum 
total cost to achieve it. You may assume that Takahashi's expected profit is initially at least 
1 yen.

\end{frame}

\begin{frame}[fragile]{Entrada e saída}

\textbf{Constraints}

\begin{itemize}
    \item $1\leq N\leq 10^5$
    \item $1\leq A_i \leq 10^9$ $(1\leq i\leq N)$
    \item $A_i$ are distinct.
    \item $2\leq T\leq 10^9$
    \item In the initial state, Takahashi's expected profit is at least 1 yen.
\end{itemize}

\end{frame}

\begin{frame}[fragile]{Entrada e saída}

\textbf{Input}

Input is given from Standard Input in the following format:
\begin{atcoderio}{llll}
$N$ & $T$ \\
$A_1$ & $A_2$ & \ldots & $A_N$ \\
\end{atcoderio}

\textbf{Output}

Print the minimum total cost to decrease Takahashi's expected profit by at least 1 yen.

\end{frame}

\begin{frame}[fragile]{Exemplo de entradas e saídas}

\begin{minipage}[t]{0.55\textwidth}
\textbf{Entrada}
\begin{verbatim}
3 2
100 50 200


5 8
50 30 40 10 20


10 100
7 10 4 5 9 3 6 8 2 1
\end{verbatim}
\end{minipage}
\begin{minipage}[t]{0.4\textwidth}
\textbf{Saída}
\begin{verbatim}
1



2



2
\end{verbatim}
\end{minipage}
\end{frame}

\begin{frame}[fragile]{Solução}

    \begin{itemize}
        \item Dados os limites do problema, não é possível propôr soluções que avaliem todas
            as possíveis interações, mesmo que seja por meio de programação dinâmica

        \item Assim, para se chegar a solução é preciso simplificar ao máximo as opções a 
            serem consideradas

        \item Primeiramente, observe que se maças forem compradas na $i$-ésima cidade, elas
            devem ser vendidas na cidade $j$ ($i < j \leq N$) com maior $A_j$

        \item Assim, para cada cidade é possível computar o lucro máximo $L_i$ de se adquirir
            $T/2$ maças em $i$ e vendê-las pelo maior lucro possível

        \item Outra importante observação é que é ótimo comprar $T/2$ maças em uma mesma cidade
            e vendê-las todas em ou única outra cidade

    \end{itemize}

\end{frame}


\begin{frame}[fragile]{Solução}

    \begin{itemize}
        \item Seja $L = \max\{ L_1, L_2, \ldots, L_N \}$

        \item Para reduzir o lucro em um mais ienes basta reduzir o valor do $A_j$ associado
            ao maior lucro

        \item Porém, caso exista mais do que um valor $L_i$ que seja igual a $L$, a redução de
            um $A_j$ apenas não é suficiente

        \item Logo para cada $L_i = L$ devemos reduzir 1 iene no valor $A_j$ associado

        \item Portanto a solução é a quantidade de tais $L_i$

        \item Usando uma fila com prioridades e computando cada $L_i$ em $O(1)$, a solução
            tem complexidade $O(N\log N)$
    \end{itemize}

\end{frame}

\begin{frame}[fragile]{Solução $O(N\log N)$}
    \inputsnippet{cpp}{1}{21}{codes/D.cpp}
\end{frame}

\begin{frame}[fragile]{Solução $O(N\log N)$}
    \inputsnippet{cpp}{22}{42}{codes/D.cpp}
\end{frame}
