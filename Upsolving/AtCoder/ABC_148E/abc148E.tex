\begin{frame}[fragile]{AtCoder Beginner Contest 148E -- Double Factorial}

For an integer $n$ not less than 0, let us define $f(n)$ as follows:
\begin{itemize}
    \item $f(n)=1\ (\mbox{if}\ n<2)$
    \item $f(n)=nf(n-2)\ (\mbox{if}\ n\leq 2)$
\end{itemize}

Given is an integer $N$. Find the number of trailing zeros in the decimal notation of $f(N)$.

\end{frame}

\begin{frame}[fragile]{Entrada e saída}

\textbf{Constraints}

\begin{itemize}
    \item $0 \leq N\leq 10^{18}$
\end{itemize}

\vspace{0.1in}

\textbf{Input}

Input is given from Standard Input in the following format:
\begin{atcoderio}{lll}
$N$ \\
\end{atcoderio}

\textbf{Output}


Print the number of trailing zeros in the decimal notation of $f(N)$.

\end{frame}

\begin{frame}[fragile]{Exemplos de entradas e saídas}

\begin{minipage}[t]{0.45\textwidth}
\textbf{Entrada}
\begin{verbatim}
12


5


1000000000000000000
\end{verbatim}
\end{minipage}
\begin{minipage}[t]{0.5\textwidth}
\textbf{Saída}
\begin{verbatim}
1


0


124999999999999995
\end{verbatim}
\end{minipage}
\end{frame}


\begin{frame}[fragile]{Solução com complexidade $O(\log N)$}

    \begin{itemize}
        \item Observe que a função $f(n)$ é uma variante do fatorial, que computa o produto dos
            pares ou dos ímpares menores ou iguais a $n$, dependendo da paridade de $n$

        \item Se $n$ for ímpar, então $f(n)$ será também ímpar, e portanto não terá nenhum zero
            à direita

        \item Se for um positivo $n$ par, então $f(n)$ pode ser reescrita como
        $$
            f(n) = 2^{n/2}\times \left(\frac{n}{2}\right)!
        $$

        \item Neste caso, $f(n) = 2^r5^sm$, onde $(2, m) = 1 = (5, m)$, $s = E(n/2, 5)$ e $r = n/2 + E(n/2, 2)$

    \end{itemize}

\end{frame}

\begin{frame}[fragile]{Solução com complexidade $O(\log N)$}

    \begin{itemize}
        \item A representação decimal de $f(n)$ terá um zero à direita para cada par de fatores 2 e 
            5

        \item Assim, a solução $S$ do problema será dada por $S = \min(r, s)$

        \item Como $s\leq r$, pois $E(n/2, 2)\geq E(n/2, 5)$, então $S$ de fato corresponde a
            $E(n/2, 5)$

        \item Esta solução tem complexidade $O(\log n)$
    \end{itemize}

\end{frame}

\begin{frame}[fragile]{Solução com complexidade $O(\log N)$}
    \inputsnippet{cpp}{6}{22}{codes/B148E.cpp}
\end{frame}
