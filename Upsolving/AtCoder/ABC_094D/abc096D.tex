\begin{frame}[fragile]{Problema}

Print a sequence $a_1, a_2, \ldots, a_N$ whose length is $N$ that satisfies the following conditions:
\begin{itemize}
    \item $a_i (1\leq a_i\leq N)$ is a prime number at most $55 555$.
    \item The values of $a_i, a_2, \ldots, a_N$ are all different.
    \item In every choice of five different integers from $a_1, a_2, \ldots, a_N$, the sum of those integers is a composite number.
\end{itemize}
 
If there are multiple such sequences, printing any of them is accepted.

\vspace{0.2in}

\textbf{Notes}

An integer $N$ not less than $2$ is called a prime number if it cannot be divided evenly by any integers except $1$ and $N$, and called a composite number otherwise.

\end{frame}

\begin{frame}[fragile]{Entrada e saída}

\textbf{Constraints}

\begin{itemize}
    \item $N$ is an integer between $5$ and $55$ (inclusive).
\end{itemize}

\vspace{0.1in}

\textbf{Input}

Input is given from Standard Input in the following format:
\begin{atcoderio}{lll}
$N$ \\
\end{atcoderio}

\textbf{Output}

Print $N$ numbers $a_1, a_2, \ldots, a_N$ in a line, with spaces in between.

\end{frame}

\begin{frame}[fragile]{Exemplos de entradas e saídas}

\begin{minipage}[t]{0.45\textwidth}
\textbf{Entrada}
\begin{verbatim}
5


6


8
\end{verbatim}
\end{minipage}
\begin{minipage}[t]{0.5\textwidth}
\textbf{Saída}
\begin{verbatim}
3 5 7 11 31


2 3 5 7 11 13


2 5 7 13 19 37 67 79
\end{verbatim}
\end{minipage}
\end{frame}


\begin{frame}[fragile]{Solução com complexidade $O(M\log \log M)$}

    \begin{itemize}
        \item A solução consiste em identificar um subconjuntos de primos que atendam a primeira
            e a terceira condições

        \item Os $\pi(55555) = 5637$ primos podem se gerados pelo crivo de Erastótenes

        \item Devem ser selecionados dentre os primos aqueles cujo resto da divisão por 5 seja
            $k$, com $k > 0$

        \item Qualquer $k$ no intervalo $[1, 4]$ gera uma lista de mais de $1.400$ primos

        \item Assim, após o filtro todos os elementos selecionados serão da forma $5m + k$, e 
            daí
        $$
            (5m_1 + k) + (5m_2 + k) + \ldots + (5m_5 + k) = 5(m_1 + m_2 + \ldots + m_5 + k)
        $$

        \item Ou seja, a soma de quaisquer $N$ elementos dentre os filtrados é 
            divisível por 5, e portanto é um número composto

        \item A complexidade será igual a complexidade do crivo usado para gerar os primos até
            $M$, onde $M = 55555$
    \end{itemize}

\end{frame}

\begin{frame}[fragile]{Solução com complexidade $O(M\log \log M)$}
    \inputsnippet{cpp}{1}{21}{codes/abc096D.cpp}
\end{frame}
