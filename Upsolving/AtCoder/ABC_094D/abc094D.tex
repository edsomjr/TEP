\begin{frame}[fragile]{AtCoder Beginner Contest 094D -- Binomial Coefficients}

Let $\mathrm{comb}(n,r)$ be the number of ways to choose $r$ objects from among $n$ objects, disregarding order. From $n$ non-negative integers 
$a_1, a_2, \ldots, a_n$, select two numbers $a_i > a_j$ so that $\mathrm{comb}(a_i, a_j)$
 is maximized. If there are multiple pairs that maximize the value, any of them is accepted.

\end{frame}

\begin{frame}[fragile]{Entrada e saída}

\textbf{Constraints}

\begin{itemize}
    \item $2\leq n \leq 10^5$
    \item $0\leq a_i\leq 10^9$
    \item $a_1, a_2, \ldots, a_n$ are pairwise distinct.
    \item All values in input are integers.
\end{itemize}

\vspace{0.1in}

\textbf{Input}

Input is given from Standard Input in the following format:
\begin{atcoderio}{lll}
$n$ \\
$a_1\ \ a_2\ \ \ldots\ \ a_n$
\end{atcoderio}

\textbf{Output}

Print $a_i$ and $a_j$ that you selected, with a space in between.

\end{frame}

\begin{frame}[fragile]{Exemplos de entradas e saídas}

\begin{minipage}[t]{0.45\textwidth}
\textbf{Entrada}
\begin{verbatim}
5
6 9 4 2 11


2
100 0
\end{verbatim}
\end{minipage}
\begin{minipage}[t]{0.5\textwidth}
\textbf{Saída}
\begin{verbatim}
11 6



100 0
\end{verbatim}
\end{minipage}
\end{frame}


\begin{frame}[fragile]{Solução com complexidade $O(N)$}

    \begin{itemize}
        \item Este problema pode ser resolvido mediante duas importantes observações

        \item A primeira delas é que, para um inteiro não-negativo $m$ fixo,
        $$
            \binom{i}{m} < \binom{j}{m}
        $$
        para $i < j$, $m \leq i, j$

        \item Isto significa que, para uma coluna fixa, quanto maior a linha do Triângulo de
            Pascal, maior o valor do coeficiente binomial correspondente

        \item A segunda observação é que, para uma linha $n$ fixa, os coeficientes formam uma
            sequência crescente até o coeficiente central e segue numa sequência decrescente
            até o último coeficiente
    \end{itemize}

\end{frame}

\begin{frame}[fragile]{Solução com complexidade $O(N\log N)$}

    \begin{itemize}
        \item Assim, o coeficiente $\binom{n}{\lfloor n/2\rfloor}$ é o maior dentre todos de uma
            mesma linha e, quanto mais próximo deste centro, maior o coeficiente

        \item Assim, se os valores da sequência deles forem ordenados, o maior deles será o $a_i$
            procurado

        \item Para determinar o $a_j$, é preciso avaliar os $N - 1$ termos restantes, em relação
            à sua distância ao centro: o mais próximo deles é o $a_j$ desejado

        \item Esta solução tem complexidade $O(N\log N)$
    \end{itemize}

\end{frame}

\begin{frame}[fragile]{Solução com complexidade $O(N\log N)$}
    \inputsnippet{cpp}{5}{21}{codes/B094D.cpp}
\end{frame}
