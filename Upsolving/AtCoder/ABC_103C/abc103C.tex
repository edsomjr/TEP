\section{AtCoder Beginner Contest 103 -- Problem C: \texttt{/\char`\\/\char`\\/\char`\\/}}

\begin{frame}[fragile]{Problema}

A sequence $a_1, a_2, \ldots, a_n$ is said to be \texttt{/\char`\\/\char`\\/\char`\\/} when the 
following conditions are satisfied:

\begin{itemize}
    \item For each $i = 1, 2, \ldots, n - 2, a_i = a_{i + 2}$.
    \item Exactly two different numbers appear in the sequence.
\end{itemize}
You are given a sequence $v_1, v_2, \ldots, v_n$ whose length is even. We would like to make this 
sequence \texttt{/\char`\\/\char`\\/\char`\\/} by replacing some of its elements. Find the minimum 
number of elements that needs to be replaced.

\end{frame}

\begin{frame}[fragile]{Entrada e saída}

\textbf{Constraints}

\begin{itemize}
    \item $2\leq n\leq 10^5$
    \item $n$ is even.
    \item $1\leq v_i\leq 10^5$
    \item $v_i$ is an integer.
\end{itemize}

\textbf{Input}

Input is given from Standard Input in the following format:
\begin{align*}
&n\\
&v_1\ v_2\ \ldots\ v_n
\end{align*}

\textbf{Output}

Print the minimum number of elements that needs to be replaced.

\end{frame}

\begin{frame}[fragile]{Exemplo de entradas e saídas}
\begin{minipage}[t]{0.5\textwidth}
\textbf{Exemplo de Entrada}
\begin{verbatim}
4
3 1 3 2

6
105 119 105 119 105 119

4
1 1 1 1
\end{verbatim}
\end{minipage}
\begin{minipage}[t]{0.45\textwidth}
\textbf{Exemplo de Saída}
\begin{verbatim}
1


0


2
\end{verbatim}
\end{minipage}
\end{frame}

\begin{frame}[fragile]{Solução}

    \begin{itemize}
        \item A solução consiste em diversas etapas

        \item A primeira delas é construir o histograma dos elementos que estão nos índices
            ímpares e dos elementos que estão nos índices pares

        \item Aqui há um \textit{corner case}: se todos os elementos são iguais, é preciso
            inserir um valor sentinela no histograma, com número de ocorrências iguais a zero

        \item A etapa seguinte é computar, para cada elemento distinto do histograma, o custo de 
            tornar todos os valores da subsequência que ele pertence iguais a ele

        \item Estes custos devem ser armazenados em um vetor de pares, onde o primeiro elemento
            é o custo e o segundo é o número que preencherá todas as posições
    \end{itemize}

\end{frame}

\begin{frame}[fragile]{Solução}

    \begin{itemize}
        \item Feito isso para ambas subsequências, estes pares devem ser ordenados, do menor para
            o maior custo

        \item Por fim, se os elementos de menores custos de ambas subsequências forem distintos,
            a resposta será a soma destes custos

        \item Se forem iguais, é preciso ver o que é mais barato: trocar o segundo mais barato
            da sequência de índices ímpares e o mais barato da sequência dos pares ou o 
            contrário

        \item Esta solução tem complexidade $O(N\log N)$, devido à construção do histograma e
            da ordenação dos custos
    \end{itemize}

\end{frame}


\begin{frame}[fragile]{Solução AC com complexidade $O(N\log N)$}
    \inputsnippet{cpp}{1}{18}{codes/B111C.cpp}
\end{frame}

\begin{frame}[fragile]{Solução AC com complexidade $O(N\log N)$}
    \inputsnippet{cpp}{20}{37}{codes/B111C.cpp}
\end{frame}

\begin{frame}[fragile]{Solução AC com complexidade $O(N\log N)$}
    \inputsnippet{cpp}{39}{61}{codes/B111C.cpp}
\end{frame}
