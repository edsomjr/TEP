\section{C -- 1D Reversi}

\begin{frame}[fragile]{Problema}

Two foxes Jiro and Saburo are playing a game called 1D Reversi. This game is played on a board, 
using black and white stones. On the board, stones are placed in a row, and each player places 
a new stone to either end of the row. Similarly to the original game of Reversi, when a white 
stone is placed, all black stones between the new white stone and another white stone, turn 
into white stones, and vice versa.

In the middle of a game, something came up and Saburo has to leave the game. The state of the 
board at this point is described by a string $S$. There are $|S|$ (the length of $S$) stones on 
the board, and each character in $S$ represents the color of the $i$-th $(1\leq i\leq |S|)$
stone from the left. If the $i$-th character in $S$ is `\texttt{\textcolor{red}{B}}', it means
that the color of the corresponding stone on the board is black. Similarly, if the $i$-th 
character in $S$ is `\texttt{\textcolor{red}{W}}', it means that the color of the corresponding 
stone is white.

Jiro wants all stones on the board to be of the same color. For this purpose, he will place new 
stones on the board according to the rules. Find the minimum number of new stones that he needs 
to place.

\end{frame}

\begin{frame}[fragile]{Entrada e saída}

\textbf{Constraints}

\begin{itemize}
    \item $1\leq |S|\leq 10^5$
    \item Each character in $S$ is `\texttt{\textcolor{red}{B}}' or `\texttt{\textcolor{red}{W}}'.
\end{itemize}

\textbf{Input}

Input is given from Standard Input in the following format:
\begin{atcoderio}{llll}
$S$ \\
\end{atcoderio}

\vspace{0.1in}

\textbf{Output}

Print the minimum number of new stones that Jiro needs to place for his purpose.

\end{frame}

\begin{frame}[fragile]{Exemplo de entradas e saídas}

\begin{minipage}[t]{0.45\textwidth}
\textbf{Entrada}
\begin{verbatim}
BBBWW


WWWWWW


WBWBWBWBWB
\end{verbatim}
\end{minipage}
\begin{minipage}[t]{0.5\textwidth}
\textbf{Saída}
\begin{verbatim}
1


0


9
\end{verbatim}
\end{minipage}
\end{frame}

\begin{frame}[fragile]{Solução}

    \begin{itemize}
        \item Embora não seja óbvio à primeira vista, transformar todas as peças em pretas
            ou em brancas exige o mesmo número de operações

        \item A estratégia de solução é gulosa: enquanto houver duas cores entre as peças,
            coloque no início da fila uma peça de cor oposta à da primeira peça

        \item Isto vai acontecer $K - 1$ vezes, onde $K$ é o número de blocos contíguos de
            peças da mesma cor

        \item Assim, a solução consiste em determinar $K$, o que pode ser feito por meio da
            técnica dos dois ponteiros

        \item Assim a solução tem complexidade $O(N)$
    
    \end{itemize}

\end{frame}

\begin{frame}[fragile]{Solução $O(N)$}
    \inputsnippet{cpp}{1}{21}{codes/C.cpp}
\end{frame}

\begin{frame}[fragile]{Solução $O(N)$}
    \inputsnippet{cpp}{22}{42}{codes/C.cpp}
\end{frame}
