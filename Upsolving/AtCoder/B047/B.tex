\section{B -- Snuke's Coloring 2 (ABC Edit)}

\begin{frame}[fragile]{Problema}

There is a rectangle in the $xy$-plane, with its lower left corner at $(0, 0)$ and its upper 
right corner at $(W, H)$. Each of its sides is parallel to the $x$-axis or $y$-axis. Initially, 
the whole region within the rectangle is painted white.

Snuke plotted $N$ points into the rectangle. The coordinate of the $i$-th $(1\leq i\leq N)$ 
point was $(x_i, y_i)$.

Then, he created an integer sequence $a$ of length $N$, and for each $1\leq i\leq N$, he
painted some region within the rectangle black, as follows:

\begin{itemize}

    \item If $a_i = 1$, he painted the region satisfying $x < x_i$ within the rectangle.
    \item If $a_i = 2$, he painted the region satisfying $x > x_i$ within the rectangle.
    \item If $a_i = 3$, he painted the region satisfying $y < y_i$ within the rectangle.
    \item If $a_i = 4$, he painted the region satisfying $y > y_i$ within the rectangle.

\end{itemize}

Find the area of the white region within the rectangle after he finished painting.

\end{frame}

\begin{frame}[fragile]{Entrada e saída}

\textbf{Constraints}

\begin{itemize}
    \item $1\leq W, H\leq 100$
    \item $1\leq N\leq 100$
    \item $0\leq x_i\leq W (1\leq i\leq N)$
    \item $0\leq y_i\leq H (1\leq i\leq N)$
    \item $W, H, x_i$ and $y_i$ are integers.
    \item $a_i$ $(1\leq i\leq N)$ is $1, 2, 3$ or $4$.
\end{itemize}

\end{frame}

\begin{frame}[fragile]{Entrada e saída}

\textbf{Input}

Input is given from Standard Input in the following format:
\begin{atcoderio}{lll}
$W$ & $H$ & $N$ \\
$x_1$ & $y_1$ & $a_1$ \\
$x_2$ & $y_2$ & $a_2$ \\
$\ldots$ \\
$x_N$ & $y_N$ & $a_N$ \\
\end{atcoderio}

\textbf{Output}

Print the area of the white region within the rectangle after Snuke finished painting.

\end{frame}


\begin{frame}[fragile]{Exemplo de entradas e saídas}

\begin{minipage}[t]{0.45\textwidth}
\textbf{Entrada}
\begin{verbatim}
5 4 2
2 1 1
3 3 4

5 4 3
2 1 1
3 3 4
1 4 2

10 10 5
1 6 1
4 1 3
6 9 4
9 4 2
3 1 3
\end{verbatim}
\end{minipage}
\begin{minipage}[t]{0.5\textwidth}
\textbf{Saída}
\begin{verbatim}
9



0




64
\end{verbatim}
\end{minipage}
\end{frame}

\begin{frame}[fragile]{Solução}

    \begin{itemize}
        \item Observe que, a cada ação, a região branca ou se mantém, ou se reduz em uma
            de suas dimensões

        \item Assuma que a região branca seja delimita no eixo-$x$ pelo intervalo $[x_L, x_R]$
            e no eixo-$y$ pelo intervalo $[y_L, y_R]$

        \item Para $a_i = 1$, o valor de $x_L$ será dado por $x_L = \max(x_L, x_i)$

        \item De forma semelhante, para $a_i = 2$ vale que $x_R = \min(x_R, x_i)$

        \item O mesmo vale para $a_i = 3$ e $a_i = 4$, que afetam as coordenadas $y$ de maneira
            similar

        \item A solução tem complexidade $O(N)$, uma vez que cada pintura pode ser feita em
            $O(1)$
    \end{itemize}

\end{frame}

\begin{frame}[fragile]{Solução $O(N)$}
    \inputsnippet{cpp}{1}{21}{codes/B.cpp}
\end{frame}

\begin{frame}[fragile]{Solução $O(N)$}
    \inputsnippet{cpp}{22}{42}{codes/B.cpp}
\end{frame}
