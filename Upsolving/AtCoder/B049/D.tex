\section{D -- Connectivity}

\begin{frame}[fragile]{Problema}

There are $N$ cities. There are also $K$ roads and $L$ railways, extending between the cities. 
The $i$-th road bidirectionally connects the $p_i$-th and $q_i$-th cities, and the $i$-th 
railway bidirectionally connects the $r_i$-th and $s_i$-th cities. No two roads connect the 
same pair of cities. Similarly, no two railways connect the same pair of cities.

We will say city $A$ and $B$ are connected by roads if city $B$ is reachable from city $A$ by
traversing some number of roads. Here, any city is considered to be connected to itself by 
roads. We will also define connectivity by railways similarly.

For each city, find the number of the cities connected to that city by both roads and railways.

\end{frame}

\begin{frame}[fragile]{Entrada e saída}

\textbf{Constraints}

\begin{itemize}
    \item $2\leq N \leq 2\times 10^5$
    \item $1\leq K, L\leq 10^5$
    \item $1\leq p_i, q_i, r_i, s_i\leq N$
    \item $p_i < q_i$
    \item $r_i < s_i$
    \item When $i\neq j$, $(p_i, q_i)\neq (p_j, q_j)$
    \item When $i\neq j$, $(r_i, s_i)\neq (r_j, s_j)$
\end{itemize}

\end{frame}

\begin{frame}[fragile]{Entrada e saída}

\textbf{Input}

Input is given from Standard Input in the following format:
\begin{atcoderio}{llll}
$N$ & $K$ & $L$ \\
$p_1$ & $q_1$ \\
$\vdots$ \\
$p_K$ & $q_K$ \\
$r_1$ & $s_1$ \\
$\vdots$ \\
$r_L$ & $s_L$ \\
\end{atcoderio}

\textbf{Output}

Print $N$ integers. The $i$-th of them should represent the number of the cities connected to 
the $i$-th city by both roads and railways.

\end{frame}

\begin{frame}[fragile]{Exemplo de entradas e saídas}

\begin{minipage}[t]{0.55\textwidth}
\textbf{Entrada}
\begin{verbatim}
4 3 1
1 2
2 3
3 4
2 3


4 2 2
1 2
2 3
1 4
2 3
\end{verbatim}
\end{minipage}
\begin{minipage}[t]{0.4\textwidth}
\textbf{Saída}
\begin{verbatim}
1 2 2 1






1 2 2 1
\end{verbatim}
\end{minipage}
\end{frame}

\begin{frame}[fragile]{Exemplo de entradas e saídas}

\begin{minipage}[t]{0.55\textwidth}
\textbf{Entrada}
\begin{verbatim}
7 4 4
1 2
2 3
2 5
6 7
3 5
4 5
3 4
6 7
\end{verbatim}
\end{minipage}
\begin{minipage}[t]{0.4\textwidth}
\textbf{Saída}
\begin{verbatim}
1 1 2 1 2 2 2
\end{verbatim}
\end{minipage}
\end{frame}

\begin{frame}[fragile]{Solução}

    \begin{itemize}
        \item Este é um problema interessante por combinar dois problemas que, isoladamente,
            seriam de fácil solução, mas que combinados formam um problema mais sofisticado

        \item As conexões via estradas podem ser determinadas em $O(N + K)$ por meio de uma
            travessia em profundidade (BFS) ou em largura (BFS)

        \item De forma análoga, as conexões por linhas de trem podem ser determinadas em
            $O(N + L)$

       \item Um conjunto $T_ide $ cidades que estão conectadas por estradas corresponde a um
            componente conectado do grafo

        \item Da mesma forma, um conjuto $R_j$ de cidades conectadas por linhas de trem 
            também é um outro componente conectado, segundo este outro critério

        \item O problema consiste em determinar as interseções entre os conjuntos $T_i$ e
            $R_j$
    \end{itemize}

\end{frame}

\begin{frame}[fragile]{Solução}

    \begin{itemize}
        \item Comparar diretamente cada $T_i$ com todos $R_j$ leva a uma solução quadrática
            em $N$ e a um veredito $TLE$

        \item Uma maneira de se determinar a solução de forma mais eficiente é, na primeira
            travessia, identificar todos os vértices em um mesmo componente $T_i$ por meio
            de um identificador único (um inteiro, por exemplo)

        \item Na segunda travessia deve se manter um dicionário que associe todos os elementos
            de um componente $R_j$ que tenham o mesmo identificador único

        \item Ao final da deteceção de um componente $R_j$, a resposta para cada vértice $u$
            guardado no dicionário será o número de vértices que tem identificador igual a
            $u$

        \item Esta solução tem complexidade $O((N + K + L)\log N)$
    \end{itemize}

\end{frame}

\begin{frame}[fragile]{Solução $O((N + K + L)\log N)$}
    \inputsnippet{cpp}{1}{21}{codes/D.cpp}
\end{frame}

\begin{frame}[fragile]{Solução $O((N + K + L)\log N)$}
    \inputsnippet{cpp}{22}{42}{codes/D.cpp}
\end{frame}

\begin{frame}[fragile]{Solução $O((N + K + L)\log N)$}
    \inputsnippet{cpp}{43}{61}{codes/D.cpp}
\end{frame}

\begin{frame}[fragile]{Solução $O((N + K + L)\log N)$}
    \inputsnippet{cpp}{62}{82}{codes/D.cpp}
\end{frame}

\begin{frame}[fragile]{Solução $O((N + K + L)\log N)$}
    \inputsnippet{cpp}{83}{103}{codes/D.cpp}
\end{frame}
