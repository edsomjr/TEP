\section{D -- Unbalance}

\begin{frame}[fragile]{Problema}

Given a string $t$, we will call it unbalanced if and only if the length of $t$ is at least 2, and
more than half of the letters in $t$ are the same. For example, both
\texttt{\textcolor{red}{'voodoo'}} and \texttt{\textcolor{red}{'melee'}} are unbalanced, while
neither \texttt{\textcolor{red}{'noon'}} nor \texttt{\textcolor{red}{'a'}} is.

You are given a string $s$ consisting of lowercase letters. Determine if there exists a
(contiguous) substring of $s$ that is unbalanced. If the answer is positive, show a position where
such a substring occurs in $s$.

\end{frame}

\begin{frame}[fragile]{Entrada e saída}

\textbf{Constraints}

\begin{itemize}
    \item $2\leq |s| \leq 10^5$
    \item $s$ consists of lowercase letters.
\end{itemize}

\vspace{0.1in}

\textbf{Partial Score}

\begin{itemize}
    \item 200 points will be awarded for passing the test set satisfying $2 \leq N\leq 100$.
\end{itemize}

\end{frame}

\begin{frame}[fragile]{Entrada e saída}

\textbf{Input}

Input is given from Standard Input in the following format:
\begin{atcoderio}{llll}
$s$ \\
\end{atcoderio}

\textbf{Output}

If there exists no unbalanced substring of $s$, print \texttt{\textcolor{red}{'-1 -1'}}.

If there exists an unbalanced substring of $s$, let one such substring be $s_as_{a+1}\ldots s_b$
($1\leq a < b \leq |s|$) , and print {\textcolor{red}{\texttt{'}$a$ $b$\texttt{'}}}. If there exists more than one
such substring, any of them will be accepted.

\end{frame}

\begin{frame}[fragile]{Exemplo de entradas e saídas}

\begin{minipage}[t]{0.55\textwidth}
\textbf{Entrada}
\begin{verbatim}
needed


atcoder
\end{verbatim}
\end{minipage}
\begin{minipage}[t]{0.4\textwidth}
\textbf{Saída}
\begin{verbatim}
2 5


-1 -1
\end{verbatim}
\end{minipage}
\end{frame}

\begin{frame}[fragile]{Solução}

    \begin{itemize}
        \item Seja $N$ o tamanho da string $s$

        \item Há $O(N^2)$ substrings de $s$, e a verificação de cada substring pode ser feita em
            $O(N)$, o que levaria a uma solução $O(N^3)$, que receberia veredito TLE, pois
            $N \leq 10^5$

        \item É preciso, portanto, encontrar uma forma eficiente de se processar as substrings
            e também de reduzir o número de substrings a serem verificadas
    \end{itemize}

\end{frame}


\begin{frame}[fragile]{Solução}

    \begin{itemize}
        \item Como qualquer string desbalaceada pode ser dada como resposta do problema, o 
            princípio da casa dos pombos pode reduzir drasticamente o número de substrings a
            serem verificadas

        \item Uma string desbalanceada tem mais da metade de seus caracteres repetidos

        \item Se $N$ é par e $s$ é desbalanceada, ela possui ao menos um par de caracteres 
            consecutivos iguais

        \item Como uma string com $N = 2$ com dois caracteres iguais é desbalanceada, é suficiente
            indicar estes caracteres ao invés da string inteira
    \end{itemize}

\end{frame}

\begin{frame}[fragile]{Solução}

    \begin{itemize}
        \item Se $N$ é ímpar, é possível que $s$ seja desbalanceada e que não tenha dois caracteres
            consecutivos iguais

        \item Isto só ocorre se o número de repetições é igual a $(N + 1)/2$ e o caractere repetido
            aparece em todos os índices ímpares de $s$:
        \[
            s = cs_2cs_4\ldots s_{N - 1}c
        \]

        \item Porém, se $N = 3$ e $s$ é da forma $cs_2c$, ela também será desbalanceada e poderá
            ser utilizada como resposta ao invés de $s$

        \item Portanto, basta checar apenas estes dois casos, gerando uma solução $O(N)$

    \end{itemize}

\end{frame}

\begin{frame}[fragile]{Solução $O(|s|)$}
    \inputsnippet{cpp}{1}{21}{codes/D.cpp}
\end{frame}

\begin{frame}[fragile]{Solução $O(|s|)$}
    \inputsnippet{cpp}{22}{42}{codes/D.cpp}
\end{frame}
