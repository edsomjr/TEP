\section{A -- Children and Candies (ABC Edit)}

\begin{frame}[fragile]{Problema}

There are $N$ children in AtCoder Kindergarten. Mr. Evi will arrange the children in a line, then
give 1 candy to the first child in the line, 2 candies to the second child, ..., $N$ candies to the 
$N$-th child. How many candies will be necessary in total?

\end{frame}

\begin{frame}[fragile]{Entrada e saída}

\textbf{Constraints}

\begin{itemize}
    \item $1\leq N\leq 100$
\end{itemize}

\vspace{0.1in}

\textbf{Input}

Input is given from Standard Input in the following format:
\begin{atcoderio}{lll}
$N$ \\
\end{atcoderio}

\textbf{Output}

Print the necessary number of candies in total.

\end{frame}

\begin{frame}[fragile]{Exemplo de entradas e saídas}

\begin{minipage}[t]{0.45\textwidth}
\textbf{Entrada}
\begin{verbatim}
3


10


1
\end{verbatim}
\end{minipage}
\begin{minipage}[t]{0.5\textwidth}
\textbf{Saída}
\begin{verbatim}
6


55


1
\end{verbatim}
\end{minipage}
\end{frame}

\begin{frame}[fragile]{Solução}

    \begin{itemize}
        \item O problema consiste em determina a soma
        \[
            S(N) = \sum_{i = 1}^N = 1 + 2 + \ldots + N
        \]
            
        \item Como o máximo valor de $N$ é relativamente pequeno ($N\leq 100$), é possível 
            computar esta soma por meio de um laço, o que resulta em uma solução $O(N)$

        \item A soma $S(N)$, porém, pode ser computada por meio da expressão
        \[
            S(N) = \frac{N(N + 1)}{2}
        \]

        \item Esta expressão pode ser obtida por meio da soma dos termos de uma progressão
            aritmética de $N$ termos, com $a_1 = 1$ e $a_N = N$

        \item Usando esta expressão, a complexidade da solução é reduzida para $O(1)$
    \end{itemize}

\end{frame}

\begin{frame}[fragile]{Solução $O(1)$}
    \inputsnippet{cpp}{1}{21}{codes/A.cpp}
\end{frame}
