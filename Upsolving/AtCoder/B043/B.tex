\section{B -- Unhappy Hacking (ABC Edit)}

\begin{frame}[fragile]{Problema}

Sig has built his own keyboard. Designed for ultimate simplicity, this keyboard only has 3 keys on
it: the \texttt{\textcolor{red}{'0'}} key, the \texttt{\textcolor{red}{'1'}} key and the backspace
key.

To begin with, he is using a plain text editor with this keyboard. This editor always displays one
string (possibly empty). Just after the editor is launched, this string is empty. When each key on
the keyboard is pressed, the following changes occur to the string:

\end{frame}

\begin{frame}[fragile]{Entrada e saída}

\begin{itemize}
    \item The \texttt{\textcolor{red}{'0'}} key: a letter \texttt{\textcolor{red}{'0'}} will be
        inserted to the right of the string.
    \item The \texttt{\textcolor{red}{'1'}} key: a letter \texttt{\textcolor{red}{'1'}} will be
        inserted to the right of the string.
    \item The backspace key: if the string is empty, nothing happens. Otherwise, the rightmost
        letter of the string is deleted.
\end{itemize}

Sig has launched the editor, and pressed these keys several times. You are given a string $s$,
which is a record of his keystrokes in order. In this string, the letter
\texttt{\textcolor{red}{'0'}} stands for the \texttt{\textcolor{red}{'0'}} key, the letter
\texttt{\textcolor{red}{'1'}} stands for the \texttt{\textcolor{red}{'1'}} key and the letter
\texttt{\textcolor{red}{'B'}} stands for the backspace key. What string is displayed in the editor
now?

\end{frame}

\begin{frame}[fragile]{Entrada e saída}

\textbf{Constraints}

\begin{itemize}
    \item $1\leq |s|\leq 10$ ($|s|$ denotes the length of $s$)
    \item $s$ consists of the letters \texttt{\textcolor{red}{'0'}}, \texttt{\textcolor{red}{'1'}}
        e \texttt{\textcolor{red}{'B'}}.
    \item The correct answer is not and empty string.
\end{itemize}

\textbf{Input}

Input is given from Standard Input in the following format:
\begin{atcoderio}{ll}
$s$\\
\end{atcoderio}

\textbf{Output}

Print the string displayed in the editor in the end.

\end{frame}

\begin{frame}[fragile]{Exemplo de entradas e saídas}

\begin{minipage}[t]{0.45\textwidth}
\textbf{Entrada}
\begin{verbatim}
01B0


0BB1
\end{verbatim}
\end{minipage}
\begin{minipage}[t]{0.5\textwidth}
\textbf{Saída}
\begin{verbatim}
00


1
\end{verbatim}
\end{minipage}
\end{frame}

\begin{frame}[fragile]{Solução}

    \begin{itemize}
        \item A solução do problema consiste em simular o comportamento descrito do editor

        \item Isto pode ser feito, em C++, por meio de uma variável do tipo string

        \item No caso dos caracteres \texttt{\textcolor{red}{'0'}} e \texttt{\textcolor{red}{'1'}},
            o acréscimo à direita corresponde ao método \code{cpp}{push_back()}, o qual tem
            complexidade $O(1)$

        \item Os caracteres \texttt{\textcolor{red}{'B'}} correspondem ao método 
            \code{cpp}{pop_back()}, que também tem complexidade $O(1)$

        \item É preciso atentar ao fato de que o método \code{cpp}{push_back()} não deve ser 
            invocado caso a string esteja vazia

        \item Assim, cada um dos caracteres é processado em $O(1)$, de modo que a solução tem
            complexidade $O(N)$
    \end{itemize}

\end{frame}

\begin{frame}[fragile]{Solução $O(N)$}
    \inputsnippet{cpp}{1}{19}{codes/B.cpp}
\end{frame}

\begin{frame}[fragile]{Solução $O(N)$}
    \inputsnippet{cpp}{20}{40}{codes/B.cpp}
\end{frame}
