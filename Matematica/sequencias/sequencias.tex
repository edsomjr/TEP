\section*{Sequências}

\begin{frame}[fragile]{Definição de sequência e de subsequência}

    \metroset{block=fill}
    \begin{block}{Definição}
        Uma \textbf{sequência} $a_n$ é uma função cujo domínio é um subconjunto $A$ dos números naturais.
    \end{block}

    \vspace{0.2in}

    \metroset{block=fill}
    \begin{block}{Definição}
        Uma \textbf{subsequência} $b_n$ de $a_n: A \to X$ é uma sequência $b_n: B\subset A \to X$ tal que, para quaisquer índices $i < j$, existem índices
        $r < s$ tais que $b_i = a_r$ e $b_j = a_s$.
    \end{block}

\end{frame}

\begin{frame}[fragile]{Monotonicidade}

    \metroset{block=fill}
    \begin{block}{Definição}
        Uma sequência $a_n$ é \textbf{monotamente crescente}, ou não-decrescente, se $a_j \geq a_i$ para todos $i, j\in A$, com $i < j$.

        Uma sequência $a_n$ é \textbf{monotamente decrescente}, ou não-crescente, se $a_j\leq a_i$ para todos $i, j\in A$, com $i < j$.
    \end{block}

\end{frame}

\begin{frame}[fragile]{Sequência aritmética}

    \metroset{block=fill}
    \begin{block}{Definição}
        Uma \textbf{sequência} (ou progressão) \textbf{aritmética} é uma sequência cuja diferença entre dois termos consecutivos é constante. Esta diferença recebe o nome de \textbf{razão} da progressão aritmética.
    \end{block}

\end{frame}

\begin{frame}[fragile]{Termo geral da progressão aritmética}

    \metroset{block=fill}
    \begin{block}{Proposição}
        O $k$-ésimo termo de uma progressão aritmética $a_n$ de razão $r$ é dado por
        $$
            a_k = a_1 + (k - 1)r,
        $$
        onde $a_1$ é o primeiro termo da sequência.

        De modo geral,
        $$
            a_k = a_m + (k - m)r,
        $$
        onde $a_m$ é o $m$-ésimo termo.
    \end{block}

\end{frame}

\begin{frame}[fragile]{Sequência geométrica}

    \metroset{block=fill}
    \begin{block}{Definição}
        Uma \textbf{sequência} (ou progressão) \textbf{geométrica} é uma sequência cuja quociente entre dois termos consecutivos é constante. Este quociente recebe o nome de \textbf{razão} da progressão geométrica.
    \end{block}

\end{frame}

\begin{frame}[fragile]{Termo geral da progressão geométrica}

    \metroset{block=fill}
    \begin{block}{Proposição}
        O $k$-ésimo termo da progressão geométrica $a_n$ de razão $q$ é dado por
        $$
            a_k = a_1q^{k - 1},
        $$
        onde $a_1$ é o primeiro termo da progressão.
    \end{block}

\end{frame}

\begin{frame}[fragile]{Séries}

    \metroset{block=fill}
    \begin{block}{Definição}
        O $k$-ésimo termo da série $S_n$ é determinado pela soma dos primeiros $k$ termos de uma sequência $a_n$, isto é
        $$
            S_k = \sum_{i = 1}^k a_i = a_1 + a_2 + \ldots + a_k
        $$
    \end{block}

\end{frame}

\begin{frame}[fragile]{Série da progressão aritmética}

    \metroset{block=fill}
    \begin{block}{Proposição}
        O $k$-ésimo termo da série definida pela progressão aritmética $a_n$ de razão $r$ é dado por
        $$
            S_k = \frac{k(a_1 + a_k)}{2}
        $$
    \end{block}

    \vspace{0.2in}

Esta expressão pode ser deduzida através da soma das expressões
$$
\begin{array}{rl}
    S_k\! \! \! \! &= a_1 + (a_1 + r) + (a_1 + 2r) + \ldots + (a_1 + (k - 1)r) \\
    S_k\! \! \! \! &= (a_k - (k - 1)r) + (a_k - (k - 2)r) + \ldots + (a_k - r) + a_k
\end{array}
$$
\end{frame}

\begin{frame}[fragile]{Série da progressão geométrica}

    \metroset{block=fill}
    \begin{block}{Proposição}
        O $k$-ésimo termo da série definida pela progressão geométrica $a_n$ de razão $q$ é dado por
        $$
            S_k = \frac{a_1(1 - q^k)}{1 - q}
        $$
    \end{block}

    \vspace{0.2in}

Esta expressão pode ser deduzida através da diferença das expressões
$$
\begin{array}{rl}
    S_k\! \! \! \! &= a_1 + a_1q + a_1q^2 + \ldots + a_1q^{k - 1} \\
    qS_k\! \! \! \! &= a_1q + a_1q^2 + a_1q^3 + \ldots + a_1q^k \\
\end{array}
$$
\end{frame}

\begin{frame}[fragile]{Soma da progressão geométrica infinita}

Se $a_n$ é uma progressão geométrica de razão $|q| < 1$, então a série $S_n$ converge para o limite $S$ quando $n$ tende ao infinito:
\begin{align*}
    S &= \sum_{i = 1}^\infty a_i \\
      &= \lim_{n\to \infty} S_n \\
      & = \lim_{n\to \infty} \frac{a_1(1 - q^n)}{1 - q}\\
      & = \frac{a_1}{1 - q}
\end{align*}

\end{frame}

\begin{frame}[fragile]{Séries notáveis}

    \begin{enumerate}
        \item Soma dos $n$ primeiros naturais:
$$
    S_n = \sum_{i = 1}^n i = 1 + 2 + 3 + \ldots + n = \frac{n(n + 1)}{2}
$$

        \item Soma dos quadrados dos $n$ primeiros naturais:
$$
    S_n = \sum_{i = 1}^n i^2 = 1^2 + 2^2 + 3^2 + \ldots + n^2 = \frac{n(n + 1)(2n + 1)}{6}
$$

    \end{enumerate}

\end{frame}

\begin{frame}[fragile]{Séries notáveis}

    \begin{enumerate}
        \setcounter{enumi}{2}
        \item Soma dos cubos dos $n$ primeiros naturais:
$$
    S_n = \sum_{i = 1}^n i^3 = 1^3 + 2^3 + 3^3 + \ldots + n^3 = \left[\frac{n(n + 1)}{2}\right]^2
$$

        \item Soma dos $n$ primeiros ímpares:
$$
    S_n = \sum_{i = 1}^n  2i - 1 = 1 + 3 + 5 + \ldots + (2n - 1) = n^2
$$
    \end{enumerate}

\end{frame}

\begin{frame}[fragile]{Série de Newton}

    \metroset{block=fill}
    \begin{block}{Definição}
        A série de Newton é formada pelos termos da equação de diferenças finitas de Newton. Ela consiste em uma versão discreta da série de Taylor:
        $$
            f(x) = \sum_{k = 0}^\infty \frac{\Delta^k[f](a)}{k!}(x - a)_k,
        $$
        onde
        $$
            \Delta^k[f](a) = \Delta(\Delta^{k - 1}[f](a)), \ \ \ \ \Delta^1[f](a) = \Delta[f](a) = f(a + 1) - f(a)
        $$
        e
        $$
            x_k = x(x - 1)(x - 2)\ldots (x - k + 1)
        $$
    \end{block}

\end{frame}

\begin{frame}[fragile]{Representação de sequências arbitrárias por meio de polinômios}

A série de Newton pode ser utilizada para obter um polinômio $p(x)$ que gera uma sequência finita $a_n$ qualquer. Por exemplo, seja $a_n = 3, 7, 13, 21, 31$. O quadro abaixo computa as diferenças finitas para esta sequência.

\begin{table}[h]
    \centering

    \begin{tabular}{ccccc}
    \toprule
        $x$ & $f = \Delta^0$ & $\Delta^1$ & $\Delta^2$ & $\Delta^3$ \\
    \midrule
         1 & \textbf{3} \\
         2 & 7 & \textbf{4}\\
         3 & 13 & 6 & \textbf{2} \\
         4 & 21 & 8 & 2 & \textbf{0}\\
         5 & 31 & 10 & 2 & 0 \\
    \bottomrule
    \end{tabular}
\end{table}

\end{frame}

\begin{frame}[fragile]{Exemplo de representação de sequências arbitrárias por meio de polinômios}

Conforme pode ser observado, $\Delta^k = 0$ para todo $k > 2$. Isto significa que a sequência $a_n$ pode ser representada por um polinômio de grau 2. Este polinômio pode ser obtido por meio da substituição dos termos $\Delta$ da fórmula apresentada (em negrito na tabela):
\begin{align*}
    f(x) &= \Delta^0 \cdot 1 + \Delta^1 \cdot \dfrac{(x - 1)_1}{1!} + \Delta^2\cdot \dfrac{(x - 1)_2}{2!} \\
    &= 3\cdot 1 + 4(x - 1) + 2\cdot \dfrac{(x - 1)(x - 2)}{2} \\
    &= x^2 + x  + 1 \\
\end{align*}

\end{frame}
