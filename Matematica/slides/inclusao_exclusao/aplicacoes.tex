\section*{Aplicações do Princípio da Inclusão/Exclusão}

\begin{frame}[fragile]{Permutações caóticas}

    \metroset{block=fill}
    \begin{block}{Definição}
        Uma permutação de $N$ elementos é dita caótica se nenhum dos $N$ elementos ocupa, na 
        permutação, a mesma posição que ocupava na posicionamento original, isto é, o elemento $i$ não
        ocupa a $i$-ésima posição.
    \end{block}

\end{frame}

\begin{frame}[fragile]{Contando permutações caóticas}

    Podemos contar o total de permutações caóticas $D(N)$ (D de \textit{derangement}, em inglês) da
seguinte forma: seja $A_k$ o conjunto das permutações nas quais o elemento $k$ ocupa a $k$-ésima
posição. Assim
\[
        D(N) = P(N) - \sum_{i} | A_i | + \sum_{i<j}| A_i \cap  A_j | - \sum_{i<j<k}| A_i \cap  A_j \cap  A_k | 
             + ... + (-1)^N | A_1 \cap  A_2 \cap  ... \cap  A_N |
\]
onde $P(N)$ é o total de permutações de $N$ elementos. 

Observe que $|A_i| = (N - 1)!$ (pois o elemento $i$ está fixo), $|A_i \cap A_j| = (N - 2)!$ (dois elementos fixos), e assim por diante.
Logo
\[
        D(N) = N! - \binom{N}{1}(N - 1)! + \binom{N}{2}(N - 2)! - ... + (-1)^N \binom{N}{N}
\]
o que nos dá
\[
        D(N) = N!\left(1 - \frac{1}{1!} + \frac{1}{2!} - \frac{1}{3!} + ... + (-1)^N\frac{1}{N!}\right)
\]

\end{frame}
