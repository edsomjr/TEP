\begin{frame}[fragile]{Codeforces 630C -- Lucky Numbers}

    \textbf{Versão resumida do problema}: Determine quantos números distintos de até $N$ dígitos
        podem ser formados utilizando apenas os dígitos 7 e 8.

    \vspace{0.1in}
    \textbf{Restrição}: $1\leq N\leq 55$

\end{frame}

\begin{frame}[fragile]{Solução com complexidade $O(1)$}

    \begin{itemize}
        \item Para um $M$ fixo, há $2^M$ números distintos que podem ser formados usando
            os dígitos 7 e 8, pois, para cada posição há duas escolhas: 7 ou 8

        \item Assim, a resposta $S$ do problema é dada por
        $$
            S = \sum_{i = 1}^N 2^i = 2^1 + 2^2 + \ldots + 2^N
        $$

        \item Observe que
        $$
            S + 1 = 1 + 2^1 + 2^2 + \ldots + 2^N = 2^{N + 1} - 1
        $$

        \item Assim $S = 2^{N + 1} - 2$ e esta expressão pode ser computada em $O(1)$ por meio
            de um deslocamento binário
    \end{itemize}

\end{frame}

\begin{frame}[fragile]{Solução com complexidade $O(1)$}
    \inputsnippet{cpp}{1}{17}{codes/630C.cpp}
\end{frame}
