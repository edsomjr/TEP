\begin{frame}[fragile]{AtCoder Beginner Contest 094D -- Binomial Coefficients}

    \textbf{Versão resumida do problema}: dada $N$ inteiros não-negativos $a_1, a_2, \ldots, a_N$,
        determine o par $a_i > a_j$ que maximiza o coeficiente binomial $\binom{a_i}{a_j}$.

    \vspace{0.1in}
    \textbf{Restrições}:
    \begin{itemize}
        \item $2\leq N\leq 10^5$
        \item $0\leq a_i\leq 10^9$
    \end{itemize}
\end{frame}

\begin{frame}[fragile]{Solução com complexidade $O(N)$}

    \begin{itemize}
        \item Este problema pode ser resolvido mediante duas importantes observações

        \item A primeira delas é que, para um inteiro não-negativo $m$ fixo,
        $$
            \binom{i}{m} < \binom{j}{m}
        $$
        para $i < j$, $m \leq i, j$

        \item Isto significa que, para uma coluna fixa, quanto maior a linha do Triângulo de
            Pascal, maior o valor do coeficiente binomial correspondente

        \item A segunda observação é que, para uma linha $n$ fixa, os coeficientes formam uma
            sequência crescente até o coeficiente central e segue numa sequência decrescente
            até o último coeficiente
    \end{itemize}

\end{frame}

\begin{frame}[fragile]{Solução com complexidade $O(N\log N)$}

    \begin{itemize}
        \item Assim, o coeficiente $\binom{n}{\lfloor n/2\rfloor}$ é o maior dentre todos de uma
            mesma linha e, quanto mais próximo deste centro, maior o coeficiente

        \item Assim, se os valores da sequência deles forem ordenados, o maior deles será o $a_i$
            procurado

        \item Para determinar o $a_j$, é preciso avaliar os $N - 1$ termos restantes, em relação
            à sua distância ao centro: o mais próximo deles é o $a_j$ desejado

        \item Esta solução tem complexidade $O(N\log N)$
    \end{itemize}

\end{frame}

\begin{frame}[fragile]{Solução com complexidade $O(N\log N)$}
    \inputsnippet{cpp}{5}{21}{codes/B094D.cpp}
\end{frame}
