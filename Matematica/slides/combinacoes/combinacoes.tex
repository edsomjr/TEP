\section{Combinações}

\begin{frame}[fragile]{Definição}

    \metroset{block=fill}
    \begin{block}{Definição de combinação}
        Seja $A$ um conjunto com $n$ elementos distintos e $p$ um inteiro não negativo tal que 
        $p \leq n$.  Uma \textbf{combinação} deste $n$ elementos, tomados $p$ a $p$, consiste em uma 
        escolha de $p$ elementos distintos dentre os $n$ possíveis, onde cada combinação difere das
        demais pela qualidade dos elementos, mas não pela ordem.

        \textbf{Notação}: $C(n, p)$
    \end{block}

    \vspace{0.1in}

    Por exemplo, se $A = \{1, 2, 3, 4\}$ e $p = 2$, há $6$ combinações distintas, a saber: 
    $$
        12, 13, 14, 23, 24, 34
    $$
\end{frame}

\begin{frame}[fragile]{Cálculo de $C(n, p)$}

    \begin{itemize}
        \item Se $p < 0$, então $C(n, p)$ = 0

        \item Nos demais casos, $C(n,p)$ pode ser computado a partir de $A(n,p)$: basta contar,
            como apenas um, todos os arranjos que diferem apenas pela ordem de seus elementos

        \item Como $p$ elementos distintos geram $p!$ arranjos distintos, segue que
        $$
            C(n, p) = \frac{A(n, p)}{p!} = \frac{n!}{(n - p)!p!} = \binom{n}{p}
        $$
    \end{itemize}
\end{frame}

\begin{frame}[fragile]{Caracterização das combinações}

    \begin{itemize}
        \item Assim como feito com as permutações e com os arranjos, as combinações também podem
            ser caracterizadas por meio de uma analogia com um sorteio de bolas

        \item Neste sentido, uma combinação $C(n, p)$ corresponderia a retira de $p$ bolas dentre
            as $n$ bolas distintas contidas em uma caixa, sem reposição, onde a ordem das bolas
            não é relevante

        \item Assim, as retiradas $123, 321$ e $213$, por exemplo, seriam todas consideradas uma
            mesma combinação, uma vez que a qualidade das bolas é a mesma, embora tenha sido
            retiradas em ordens distintas
    \end{itemize}

\end{frame}
