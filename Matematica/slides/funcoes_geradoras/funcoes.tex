\section*{Funções Geradoras}

\begin{frame}[fragile]{Motivação}

Carlos quer comprar 6 camisetas de um mesmo modelo e há 3 opções de cores a disposição: amarelo,
branco, e celeste. De quantas maneiras ele pode adquirir as 6 camisetas, se ele deseja comprar ao
menos uma de cada cor, e no máximo 3 brancas, 2 amarelas e 2 celestes?

Sejam $x_a, x_b$ e $x_c$ as quantidades de camisetas de cada cor que serão adquiridas. O problema se
trata de encontrar todas as soluções da equação
\[
        x_a + x_b + x_c = 6
\]
com $1 \leq x_a \leq 3, 1 \leq x_b, x_c \leq 2$.

Observe que, se tivéssemos apenas as restrições inferiores,
o problema seria o problema das soluções de equações lineares com coeficientes unitários, com
$x_i > 0$, já apresentado anteriormente, cuja solução é $\binom{6 - 1}{3 - 1} = \binom{5}{2} = 10$.
Como tratar, porém, as restrições do lado direito das desigualdades que envolvem as variáveis
$x_i$?

\end{frame}

\begin{frame}[fragile]{Solução usando o Princípio da Inclusão/Exclusão}

Defina os conjuntos
\[
\begin{array}{l}
        A = \{\  \mbox{solução de}\ x_a + x_b + x_c = 6\ \mbox{com}\ x_a > 3\ \}\\
        B = \{\  \mbox{solução de}\ x_a + x_b + x_c = 6\ \mbox{com}\ x_b > 2\ \}\\
        C = \{\  \mbox{solução de}\ x_a + x_b + x_c = 6\ \mbox{com}\ x_c > 2\ \}
\end{array}
\]

O número de soluções $S$ do problema com as restrições apresentadas seria igual a
\[
        S = \binom{5}{2} - | A | - | B | - | C | 
          + | A \cap B | + | A \cap C | + | B \cap C | - | A \cap B \cap C |
\]

Resolvendo cada problema usando mudança de variáveis teríamos
\[
        S = 10 - 1 - 3 - 3 + 0 + 0 + 0 - 0 = 3
\]

A título de curiosidade, as soluções seriam
\[
\begin{array}{l}
        x_a = 2, x_b = 2, x_c = 2\\
        x_a = 3, x_b = 1, x_c = 2\\
        x_a = 3, x_b = 2, x_c = 1
\end{array}
\]


\end{frame}

\begin{frame}[fragile]{Solução utilizando multiplicação polinomial}

Esta abordagem, porém, é demasiadamente trabalhosa, uma vez que passamos de um problema para 8
deles, antes de se determinar a solução. E o número de problemas sobe exponencialmente a 
medida que o número de variáveis aumenta.

Uma abordagem diferente seria usar polinômios para representar as possíveis escolhas para
cada variável. Para as camisetas amarelas, usaríamos o polinômio
\[
        ax + a^2x^2 + a^3x^3
\]

onde o grau do termo $x$ ou da constante $a$ indicam o número de camisetas amarelas. Para as
demais cores teríamos
\[
        bx + b^2x^2
\]
e
\[
        cx + c^2x^2
\]
\end{frame}

\begin{frame}[fragile]{Solução utilizando multiplicação polinomial}

O produto deste polinômios seria, de acordo com a propriedade distributiva,
\[
\begin{split}
        (ax + a^2x^2 + a^3x^3)(bx + b^2x^2)(cx + c^2x^2) & =(abc)x^3 + (abc^2 + ab^2c + a^2bc)x^4 \\
    &+ (ab^2c^2 + a^2bc^2 + a^2b^2c + a^3bc)x^5\\
    &+ (a^2b^2c^2 + a^3bc^2+ a^3b^2c)x^6 + (a^3b^2c^2)x^7
\end{split}
\]

Veja que o coeficiente do termo $x^6$ nos fornece justamente as três soluções desejadas, onde
os expoentes das constantes $a, b, c$ indicam o número de itens da referida cor que foram
escolhidos. Veja que se fizéssemos $a = b = c = 1$, teríamos
\[ 
        (x + x^2 + x^3)(x + x^2)(x + x^2) = x^3 + 3x^4 + 4x^5 + 3x^6 + x^7
\]

O coeficiente de $x^5$ agora informa a quantidade de maneiras, mas não discrimina as maneiras em
si. De fato, os coeficientes de cada um dos termos do polinômio são as soluções do problema para
3, 4, 5, 6 e 7 camisas no total. O problema não tem solução para 2 ou menos ou 8 ou mais camisas.
Por conta desta propriedade, o polinômio $x³ + 3x^4 + 4x^5 + 3x^6 + x^7$ é chamado função 
geradora $f(x)$ do problema.

\end{frame}

\begin{frame}[fragile]{Série de potências}

    \metroset{block=fill}
    \begin{block}{Definição}
        Uma série de potências $S(x)$ é definida por uma sequência de coeficientes ($a_i$) de modo que
        \[
        S(x) = \sum_{i=0}^\infty a_ix^i
        \]

    \end{block}

    \vspace{0.2in}

Este definição permite observar que qualquer polinômio $p(x)$ é uma série de potência.

\end{frame}

\begin{frame}[fragile]{Funções geradoras}

    \metroset{block=fill}
    \begin{block}{Definição}
        Considere um problema de combinatório $P(i)$ que dependa de um parâmetro $i$. Se a série de 
potências $S(x)$ é tal que a sequência de seus coeficientes ($a_i$) são as soluções do problema
para o valor $i$, dizemos que $S(x)$ é a função geradora $f(x)$ do problema $P(i)$.
    \end{block}

\end{frame}

\begin{frame}[fragile]{Exemplos de funções geradoras}

    \begin{itemize}
        \item Por exemplo, considere o problema de se escolher $i$ elementos distintos dentre $n$ elementos,
        todos distintos. A função geradora deste problema é a função
        \[
        f(x) = \binom{n}{0} + \binom{n}{1}x + \binom{n}{2}x^2 + \ldots + \binom{n}{n}x^n
        \]

        \item Na prática, desejamos encontrar para $f(x)$ a expressão o mais sucinta possível. Das propriedades dos números binomiais temos que
        \[
        f(x) = (1 + x)^n
        \]

        \item Uma função geradora fundamental é a função associada a sequência de uns $(1, 1, 1, ....)$:
        \[
        g(x) = 1 + x + x^2 + \ldots = \frac{1}{1 - x}
        \]

        \item A soma infinita é igual a $1/(1 - x)$ nos casos em que $|x| < 1$. Porém, no estudo de funções
geradoras, não se atribui valores à variável $x$, de modo que não há preocupação com questões
de convergência. Tais séries são denominadas séries formais.
    \end{itemize}

\end{frame}
