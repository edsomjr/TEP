\section{Números de Fibonacci}

\begin{frame}[fragile]{Definição}

    \metroset{block=fill}
    \begin{block}{Definição dos números de Fibonacci}
        O $n$-ésimo número de Fibonacci $F(n)$ é definido pela recorrência
$$
\begin{array}{l}
        F(0) = 0\\
        F(1) = 1\\
        F(n) = F(n - 1) + F(n - 2),\ \ \ \ n \geq 2
\end{array}
$$
    \end{block}

    \vspace{0.1in}

    Os primeiros termos da sequência de Fibonacci são:
$$
    0, 1, 1, 2, 3, 5, 8, 13, 21, 34, 55, 89, 144, 233, 377, 610, 987, 1597, 2584, 4181, \ldots
$$
\end{frame}

\begin{frame}[fragile]{Limites práticos dos números de Fibonacci}

    \begin{itemize}
        \item Os números de Fibonacci crescem rapidamente, de modo que o número de termos que podem 
            ser computados em tipos inteiros de C/C++ é bastante restrito

        \item Para variáveis de $32$-\textit{bits} é possível calcular o valor exato de $F(n)$ para 
            $n\leq 46$ (a saber, $F(46) = 1836311903$)

        \item Para variáveis de $64$-\textit{bits}, os valores serão exatos para $n\leq 92$ 
            (observe que $F(92) = 7540113804746346429$)

        \item Para valores de $n$ superiores a $92$, é necessário ou trabalhar com aritmética 
            estendida ou com aritmética modular
    \end{itemize}

\end{frame}

\begin{frame}[fragile]{Implementação recursiva dos números de Fibonacci}

    \begin{itemize}
        \inputsyntax{cpp}{codes/fib.cpp}

        \item A implementação acima tem como vantagem a simplicidade, uma vez que corresponde à
            definição apresentada

        \item Contudo a complexidade assintótica é $O(2^n)$
    \end{itemize}

\end{frame}

\begin{frame}[fragile]{Implementação iterativa em Python}

    \begin{itemize}
        \inputsyntax{py3}{codes/iter_fib.py}

        \item Esta versão tem complexidade $O(n)$ 

        \item A linguagem Python implementa nativamente com aritmética estendida, de modo que esta
            função pode computar $F(n)$ para $n > 92$
    \end{itemize}

\end{frame}

\begin{frame}[fragile]{Implementação usando programação dinâmica}
    \inputsnippet{py}{1}{21}{codes/fib_dp.py}
\end{frame}

\begin{frame}[fragile]{Equações de diferenças lineares}

    \begin{itemize}
        \item Os números de Fibonacci podem ser definidos por meio de um sistema de equações de
            diferenças lineares

        \item Seja $u(n)$ um vetor cujas duas componentes são os números de Fibonacci $F(n + 1)$ e 
            $F(n)$

        \item Assim, vale que
$$
        u(n + 1) = Au(n),
$$
onde
$$
A = \begin{bmatrix}
    1 & 1 \\
    1 & 0 \\
\end{bmatrix}
$$
    \end{itemize}

\end{frame}

\begin{frame}[fragile]{Equações de diferenças lineares}

    \begin{itemize}
        \item Observe que $u(1) = Au(0), u(2) = Au(1) = A^2u(0)$, etc, e assim por diante

        \item De fato,
$$
        u(n) = A^nu(0)
$$

        \item Usando exponenciação rápida para computar $A^n$, é possível determinar $F(n)$ em
            $O(\log n)$ 

        \item Veja que $F(n)$ ocupará as posições da diagonal secundária de $A^n$
    \end{itemize}

\end{frame}

\begin{frame}[fragile]{Cálculo de $F(n)$ em $O(\log n)$}
    \inputsnippet{cpp}{5}{21}{codes/fast_fib.cpp}
\end{frame}

\begin{frame}[fragile]{Cálculo de $F(n)$ em $O(\log n)$}
    \inputsnippet{cpp}{23}{37}{codes/fast_fib.cpp}
\end{frame}

\begin{frame}[fragile]{Propriedades da sequência de Fibonacci}

    \begin{itemize}
        \item A sequência de Fibonacci tem várias propriedades interessantes

        \item A razão entre dois termos consecutivos da série tende à razão áurea, isto é,
$$
    \lim_{n\to \infty} \frac{F(n + 1)}{F(n)} = \frac{1 + \sqrt{5}}{2}
$$ 

        \item A soma dos $n$ primeiros termos da sequência pode ser computada por meio de uma soma
            telescópica e é igual a 
$$
    \sum_{i = 1}^n F(i) = F(n + 2) - 1
$$
    \end{itemize}

\end{frame}

\begin{frame}[fragile]{Propriedades da sequência de Fibonacci}

    \begin{itemize}
        \item A soma dos quadrados dos $n$ primeiros termos da sequência é igual a 
$$
    \sum_{i = 1}^n F(i)^2 = F(n)F(n+1)
$$

        \item Para qualquer $m > 1$ fixo, a sequência dos restos $r(n, m)$ é cíclica, onde
$$
r(n, m) = F(n)\!\!\!\! \mod m
$$

        \item O período de $r(n, m)$ é denominado Período de Pisano $\pi(m)$
    \end{itemize}

\end{frame}

\begin{frame}[fragile]{Período de Pisano}

    \begin{itemize}
        \item Alguns valores comuns:
        \begin{itemize}
            \item $\pi(2) = 3$
            \item $\pi(10) = 60$
            \item $\pi(100) = 300$
            \item $\pi(10^k) = 15 \times 10^{k - 1}, k \geq 3$
        \end{itemize}

        \item Exceto para o caso $m = 2$, o período de Pisano é sempre par
    \end{itemize}

\end{frame}
