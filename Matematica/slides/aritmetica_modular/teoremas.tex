\section{Teorema de Fermat, de Euler e de Wilson}

\begin{frame}[fragile]{Teorema de Fermat}

    \metroset{block=fill}
    \begin{block}{Pequeno Teorema de Fermat}
        Sejam $a$ um inteiro e $p$ um número primo. Então
        $$
            a^p\equiv\ \Mod{a}{p}
        $$
    \end{block}

\end{frame}

\begin{frame}[fragile]{Fermat e inversos multiplicativos módulo $p$}

    \begin{itemize}
        \item Se $a$ não é um múltiplo de $p$, então $(a, p) = 1$, de modo que existe o inverso
            multiplicativo de $a$ módulo $p$

        \item Neste caso, ao multiplicar ambos lados da congruência por $a^{-1}$, temos que
        $$
            a^{p - 1}\equiv\ \Mod{1}{p}
        $$

        \item O lado esquerdo pode ser reescrito como
        $$
            a\times a^{p - 2}\equiv\ \Mod{1}{p}
        $$

        \item Assim, se $(a, p) = 1$, então $a^{-1}\equiv\ \Mod{a^{p - 2}}{p}$
    \end{itemize}

\end{frame}

\begin{frame}[fragile]{Inversos multiplicativos módulo $p$ em C++ com complexidade $O(\log n)$}
    \inputsnippet{cpp}{15}{31}{codes/addmul.cpp}
\end{frame}

\begin{frame}[fragile]{Teorema de Euler}

    \metroset{block=fill}
    \begin{block}{Teorema de Euler}
        Seja $a$ e $m$ inteiros positivos tais que $m > 1$. Então
        $$
            a^{\varphi(n)}\equiv\ \Mod{1}{m}
        $$
    \end{block}

\end{frame}

\begin{frame}[fragile]{Euler e inversos multiplicativos módulo $m$}

    \begin{itemize}
        \item Nos casos em que $(a, m) = 1$, o Pequeno Teorema de Fermat é um caso especial
            do Teorema de Euler

        \item Nos casos de módulos que não são primos, o Teorema de Euler fornece uma alternativa
            para o cálculo do inverso multiplicativo módulo $m$

        \item Em termos de código, ambos teoremas fornecem maneiras efetivas de computar os 
            inversos modulares, com complexidade $O(\log n)$

        \item Além disso, a implementação é fácil de lembrar e de codificação, enquanto que 
            o algoritmo de Euclides estendido não é tão simples de lembrar e ainda pode gerar
            números negativos
    \end{itemize}

\end{frame}

\begin{frame}[fragile]{Implementação do inverso multiplicativo em $O(\log n)$}
    \inputsnippet{cpp}{33}{37}{codes/addmul.cpp}
\end{frame}

\begin{frame}[fragile]{Teorema de Wilson}

    \metroset{block=fill}
    \begin{block}{Teorema de Wilson}
        Se $p$ é um número primo, então
        $$
            (p - 1)!\equiv\ \Mod{-1}{p}
        $$
    \end{block}

\end{frame}

\begin{frame}[fragile]{Problemas sugeridos}

    \begin{enumerate}
        \item \href{https://atcoder.jp/contests/abc088/tasks/abc088_a}{AtCoder Beginner Contest 088 -- Problem A: Infinite Coins}

        \item \href{https://codeforces.com/problemset/problem/450/B}{Codeforces 450B -- Jzzhu and Sequences}

        \item \href{https://onlinejudge.org/index.php?option=com_onlinejudge&Itemid=8&category=24&page=show_problem&problem=310}{OJ 374 -- Big Mod} 

        \item \href{https://onlinejudge.org/index.php?option=com_onlinejudge&Itemid=8&category=24&page=show_problem&problem=1068}{OJ 10127 -- Ones}

        \item \href{https://onlinejudge.org/index.php?option=com_onlinejudge&Itemid=8&category=24&page=show_problem&problem=1153}{OJ 10212 -- The Last Non-zero Digit}

        \item \href{https://onlinejudge.org/index.php?option=com_onlinejudge&Itemid=8&category=24&page=show_problem&problem=1970}{OJ 11029 -- Leading and Trailing}
    \end{enumerate}

\end{frame}
