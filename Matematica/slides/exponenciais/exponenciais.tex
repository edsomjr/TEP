\section*{Exponenciação}

\begin{frame}[fragile]{Exponenciação nos naturais}

    \metroset{block=fill}
    \begin{block}{Definição}
        Sejam $a, n$ dois números naturais. A exponenciação $a^n$ (lê-se ``$a$ elevado a $n$'') é definida pela relação de recorrência, onde 
        \begin{enumerate}[(i)]
            \item $a^1 = a$, e
            \item $a^n = a \times a^{n - 1}$,
        \end{enumerate}

        onde $a$ é denominada \textbf{base} e $n$ é denominado \textbf{expoente}.
    \end{block}

    \vspace{0.2in}

    Em termos mais simples, a exponenciação nos naturais é uma multiplicação repetida: basta multiplicar $a$ por ele mesmo $n$ vezes. 

\end{frame}

\begin{frame}[fragile]{Propriedades da exponenciação}

    \begin{itemize}
        \item Como a multiplicação nos naturais é associativa, vale que
$$
a^{n + m} = a^{n} \times a^{m}
$$

        \item Também são decorrentes da multiplicação nos naturais as propriedades
$$
(a^{n})^{m} = a^{nm}
$$
e
$$
(ab)^n = a^n \times b^n
$$
    \end{itemize}

\end{frame}

\begin{frame}[fragile]{Expoente zero}

    \begin{itemize}
        \item Na exponenciação nos naturais é definido que, para qualquer $a$ natural, $a^0 = 1$ 

        \item De fato, esta definição é consistente com a exponenciação nos inteiros e nos demais conjuntos numéricos, como se verá a seguir

        \item $0^0$ é uma indeterminação (para qualquer natural $n$, $0^n = 0$)

        \item A exponenciação nos naturais é ensinada no ensino fundamental e médio, e serve para observar e aprender as propriedades fundamentais da exponenciação

        \item Porém é útil, na prática, conhecer as definições de exponenciação para outros conjuntos numéricos
    \end{itemize}

\end{frame}

\begin{frame}[fragile]{Expoentes inteiros}

    \metroset{block=fill}
    \begin{block}{Definição}
        Sejam $a, n$ dois números inteiros, com $a > 0$. Vale que

        \begin{enumerate}[(i)]
            \item $a^1 = a$, e 
            \item $a^{n - 1} = a^n / a$
        \end{enumerate}
    \end{block}

    \vspace{0.3in}

    A partir da definição acima, observe que:

        \begin{enumerate}[(a)]
            \item $a^0 = a^1 / a = 1$
            \item $a^{-1} = a^0 / a  = 1 / a$
            \item $a^{-n} = (a^{-1})^n = 1 / a^n$
        \end{enumerate}

\end{frame}

\begin{frame}[fragile]{Expoentes inteiros}

    \begin{itemize}
        \item As propriedades da exponenciação nos naturais permanecem todas verdadeiras para a exponenciação nos inteiros

        \item A reescrita da relação de recorrência permite expoentes negativos

        \item Esta recorrência justifica a notação $a^{-1}$ para o inverso multiplicativo de $a$, uma vez que 
$$
a^{-1} \times a = \left(\frac{1}{a}\right) \times a = 1
$$
    \end{itemize}

\end{frame}

\begin{frame}[fragile]{Raízes $n$-ésimas}

    \begin{itemize}
        \item Sejam $a, n$ dois números inteiros, com $a > 0$. Qual seria o significado de $a^{1/n}$? 

        \item Segundo as propriedades já descritas, seria um número $x$ tal que $x^n = a$

        \item Cada solução desta equação recebe o nome de raiz $n$-ésima de $a$
    \end{itemize}

\end{frame}

\begin{frame}[fragile]{Exponenciação nos racionais}

    \metroset{block=fill}
    \begin{block}{Definição}
Sejam $n, m$ números inteiros com $m$ diferente de zero e $a$ um número racional positivo. Então
$$
a^{n/m} = (a^{1/m})^n,
$$
onde $a^{1/m}$ é uma raiz $m$-ésima de $a$.
    \end{block}

\end{frame}

\begin{frame}[fragile]{Bases negativas}

    \begin{itemize}
        \item A definição de exponenciação nos racionais pode ser estendida para bases negativas, desde que o radical (o fator $1/m$ do expoente) seja ímpar

        \item Isto porque não há soluções, nos racionais, para $x^n = -1$ quando $n$ e par

        \item Por exemplo, $x^3 = -1$ tem solução nos racionais, mas $x^2 = -1$ não

        \item Bases negativas, em geral, podem violar propriedades da exponenciação 

        \item Por exemplo, calcule $((-2)^{3/4})^{4/3}$ usando e não usando as propriedades e veja o resultado!

        \item Tais exemplos justificam a restrição comum às bases positivas
    \end{itemize}

\end{frame}

\begin{frame}[fragile]{Exponenciação rápida}

    \begin{itemize}
        \item A implementação direta da definição de exponenciação nos naturais leva a uma rotina com complexidade $O(n)$

        \item Contudo, é possível implementar um algoritmo $O(\log n)$ para computar $a^n$, por meio da divisão e conquista, denominado exponenciação rápida

        \item Para tal, basta observar que, se $n$ é par, então
$$
    a^n = a^{n/2}\times a^{n/2}
$$

        \item Se $n$ é impar, vale que
$$
    a^n = a\times a^{\lfloor n/2\rfloor}\times a^{\lfloor n/2\rfloor}
$$

    \end{itemize}

\end{frame}

\begin{frame}[fragile]{Implementação recursiva da exponenciação rápida em C++}
    \inputsnippet{cpp}{5}{13}{codes/fast_exp.cpp}
\end{frame}

\begin{frame}[fragile]{Implementação iterativa da exponenciação rápida em C++}
    \inputsnippet{cpp}{15}{29}{codes/fast_exp.cpp}
\end{frame}

\begin{frame}[fragile]{Exponenciação em C/C++}

    \begin{itemize}
        \item A biblioteca \code{cpp}{math.h} de C ou a biblioteca \code{cpp}{cmath} de C++ implementam funções relacionadas a exponenciação

        \item A função \code{cpp}{pow(a, n)} computa o valor de $a^n$

        \item A função \code{cpp}{exp(x)} computa o valor de $e^x$

        \item A função \code{cpp}{sqrt(x)} computa a raiz quadrada de $x$

        \item A função \code{cpp}{cbrt(x)} computa a raiz cúbica de $x$

        \item Todas essas funções recebem e retornam variáveis do tipo \code{cpp}{double}
    \end{itemize}

\end{frame}
