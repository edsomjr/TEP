\section*{Simulações}

\begin{frame}[fragile]{Simulações em Programação Competitiva}

    \begin{itemize}
        \item Muitos problemas matemáticos em programação competitiva consistem em simular exatamente os passos ou critérios descritos no texto

        \item Neste sentido, a técnica fundamental para a solução de tais problemas é a busca completa (força bruta), sendo que a poda é crucial em problemas mais difíceis

        \item Vários destes problemas tem solução fechada, na forma de uma expressão geral, a qual pode não ser óbvia

        \item Nestes casos, se o tamanho da entrada permitir que a solução de busca completa seja aceita dentro do limite de tempo estabelecido, é melhor usar tal solução do que tentar encontrar a solução fechada
    \end{itemize}

\end{frame}

\begin{frame}[fragile]{Podas em simulações}

    \begin{itemize}
        \item A poda deve ser aplicada quando possível

        \item Nos problemas de simulação, a poda é baseada nas propriedades das expressões que modelam o problema (identidades trigonométricas, propriedades das operações fundamentais, zeros de polinômios, dentre outros)

        \item Contudo, a poda é um recurso para diminuir o tempo de execução (ou mesmo a complexidade assintótica, em alguns casos), de modo que deve ser considerada apenas quando a solução simples, sem poda, não atingir o limite de tempo estabelecido
    \end{itemize}

\end{frame}

\begin{frame}[fragile]{Exemplo}

    \metroset{block=fill}
    \begin{block}{Problema}
    \textit{Dado um inteiro $N$ positivo, qual é o maior inteiro $m$ tal que $N \geq m(m + 1)/2$?}
    \end{block}

    \begin{itemize}
        \item Há três abordagens possíveis para este problema, com diferentes complexidades e características.

        \item Uma possível solução para este problema consiste em testar, um a um, todos os inteiros menores ou iguais a $N$ em busca da resposta, com complexidade $O(N)$.
    \end{itemize}

    \inputcode{cpp}{codes/bf.cpp}
\end{frame}

\begin{frame}[fragile]{Busca completa}

    \begin{itemize}
        \item A implementação apresentada, embora correta para valores pequenos de $N$, falha no caso geral

        \item Se $1 \leq N \leq 10^9$, por exemplo, acontecerá um \textit{overflow} na condição do \code{cpp}{if}, comprometendo a corretude do resultado

        \item Além disso, o algoritmo testa vários valores desnecessariamente: se $i$ for maior do que a raiz quadrada de $2N$, a condição do laço sempre será falsa

        \item Fazendo este ajuste e corrigindo o tipo base para \code{cpp}{long long}, a nova solução terá complexidade $O(\sqrt{N})$
    \end{itemize}

\end{frame}

\begin{frame}[fragile]{Solução baseada em busca completa}

    \inputcode{cpp}{codes/bf2.cpp}

    \vspace{0.2in}

    Embora seja nítido o ganho de performance e de complexidade, ainda é possível melhorar esta complexidade, por meio de uma busca binária.
\end{frame}

\begin{frame}[fragile]{Busca binária}

    \begin{itemize}
        \item O valor de $m$ pode ser determinado por meio de uma busca binária

        \item Uma vez que a solução se encontra no intervalo $[1, N]$, é possível, a cada etapa, testar o elemento $c$ que ocupa a posição central do intervalo como possível solução

        \item Caso $c(c + 1)/2 > N$, a solução estará no intervalo $[1, c - 1]$

        \item Caso contrário, a resposta deve ser atualizada para $c$ e a busca deve prosseguir no intervalo $[c + 1, N]$ 

        \item Assim, a solução terá complexidade $O(\log N)$
    \end{itemize}

\end{frame}

\begin{frame}[fragile]{Solução baseada em busca binária}

    \inputcode{cpp}{codes/bs.cpp}

\end{frame}

\begin{frame}[fragile]{Busca binária}

    \begin{itemize}
        \item Esta abordagem, embora não seja a mais eficiente em termos de complexidade, tem uma grande vantagem

        \item Como a solução utiliza apenas aritmética inteira (a divisão $c(c + 1)/2$ resulta sempre em um inteiro), não há possíveis erros devido a precisão

        \item A solução fechada, em $O(1)$, depende de aritmética de ponto flutuante, o que pode levar ao resultado errado em determinados casos


    \end{itemize}

\end{frame}

\begin{frame}[fragile]{Solução fechada}

    \begin{itemize}
        \item Efetivamente o problema a ser resolvido consiste em determinar o zero positivo do polinômio $p(m) = m^2 + m - 2N$

        \item Pela Fórmula de Báskara segue que
$$m = \frac{-1 + \sqrt{1 + 8m}}{2}$$

        \item A solução desejada seria $\lfloor{m}\rfloor$, o maior inteiro menor ou igual a $m$

        \item Como pode ocorrer um erro de precisão numérica, a solução pode ficar errada para menos: isto pode ser testado e corrigido, se necessário
    \end{itemize}

\end{frame}

\begin{frame}[fragile]{Implementação da solução fechada em C++}

    \inputcode{cpp}{codes/sol.cpp}

\end{frame}
