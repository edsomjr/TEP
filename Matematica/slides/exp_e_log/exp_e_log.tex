\section*{Função exponencial e função logaritmo}

\begin{frame}[fragile]{Número de Euler}

    \metroset{block=fill}
    \begin{block}{Definição}
        O número de Euler é a constante $e$, dada por
        $$
            e = \lim_{n\to \infty} \left(1 + \frac{1}{n}\right)^n = 2,71828\ldots
        $$

    \end{block}

    \vspace{0.2in}

    Este limite corresponde a uma taxa de juros com capitalização instantânea.

\end{frame}

\begin{frame}[fragile]{Função exponencial}

    \metroset{block=fill}
    \begin{block}{Definição}
        A função exponencial $\exp(x)$ é definida, para qualquer $x$ real, por
        $$
            \exp(x) = \lim_{n\to \infty} \left(1 + \frac{x}{n}\right)^n = e^x
        $$
    \end{block}

    \vspace{0.2in}

    Observe que imagem de $\exp(x)$ é o conjunto dos números reais positivos.
\end{frame}

\begin{frame}[fragile]{Gráfico da função exponencial}

    \begin{figure}
        \centering

        \begin{tikzpicture}

        \draw[->] (-3.5, 0) -- (4,0) node[right] {$x$};
        \draw[->] (0, -0.5) -- (0,5) node[left] {$y$};

        \draw[smooth, domain=-3.5:2.3, thick] 
            plot (\x,{2^(\x)}) node [right] {\footnotesize $f(x)=e^x$};

        \fill (0, 1) circle (1.5pt) node at (0,1.2) [left] {\scriptsize $(0,1)$};

        \end{tikzpicture}
    \end{figure}

\end{frame}

\begin{frame}[fragile]{Função logaritmo}

    \metroset{block=fill}
    \begin{block}{Definição}
        A função logaritmo $\ln(x)$ é definida, para qualquer $x$ real positivo, por
        $$
            \ln(x) = \log_e(x) = \int_1^x \frac{1}{t}dt
        $$
    \end{block}

    \vspace{0.2in}

    Observe que, se $0 < x < 1$, então $\ln x < 0$, por conta da inversão dos limites de integração.

\end{frame}

\begin{frame}[fragile]{Gráfico da função logaritmo}

    \begin{figure}
        \centering

        \begin{tikzpicture}

            \draw[->] (-0.5, 0) -- (5,0) node[right] {$x$};
            \draw[->] (0, -3.5) -- (0,3) node[left] {$y$};

            \draw[smooth, domain = 0.1:5, thick] plot (\x,{log2(\x)});

            \fill (1, 0) circle (1.5pt) node at (1.4, 0)[below] {\scriptsize $(1,0)$};

            \node at (4.6,2.4)[above] {\footnotesize$y=\ln(x)$};

        \end{tikzpicture}


    \end{figure}

\end{frame}

\begin{frame}[fragile]{Relação entre as funções exponencial e logaritmo}

    \begin{itemize}
        \item Embora sejam definidas em contextos distintos (limite no caso da exponencial, integral no caso do logaritmo), ambas funções estão profundamente relacionadas

        \item De fato, ambas são mutuamente inversas, isto é, $\ln e^x = x$ e $e^{\ln x} = x$

        \item Esta relação permite manipular expressões envolvendo expoentes, por meio das propriedades das exponenciais e dos logaritmos
    \end{itemize}

\end{frame}

\begin{frame}[fragile]{Aplicação: Derivada da função exponencial}

    \begin{itemize}
        \item Considere a seguinte equação diferencial: $y'(x) = y(x)$, com $y(0) = 1$

        \item Uma solução desta equação é uma função que coincide com sua derivada

        \item Esta equação pode ser reescrita como
            $$
                \frac{y'(x)}{y(x)} = 1
            $$

        \item Integrando em ambos lados segue que
            $$
                \ln y(x) = x + C
            $$
    \end{itemize}

\end{frame}

\begin{frame}[fragile]{Aplicação: Derivada da função exponencial}

    \begin{itemize}
        \item Aplicando a exponencial em ambos lados obtém-se
            $$
                y(x) = e^{\ln y(x)} = e^{x + C} = e^Ce^x
            $$

        \item Do fato que $y(0) = 1$ segue que $e^C = 1$ e, portanto, que $y(x) = e^x$

        \item Ou seja, a derivada da função exponencial é a própria exponencial

        \item Uma consequência imediata deste fato é que
            $$
            \int e^u du = e^u + C
            $$
    \end{itemize}

\end{frame}

\begin{frame}[fragile]{Aplicação: Primeiros dígitos de uma exponenciação}

    \begin{itemize}
        \item É possível determinar os primeiros dígitos do resultado de uma exponencial da forma $a^k$ em uma base $b$ dada, com $a > 0$ e $b > 1$

        \item Observe que 
            $$
                a^k = b^{\log_b a^k} = b^{k\log_b a}
            $$

        \item Seja $r = \lfloor k\log_b a\rfloor$ e $s = k(\log_b a) - r$. Daí
            $$
            a^k = b^{k\log_b a} = b^{r + s}
            $$
    \end{itemize}

\end{frame}

\begin{frame}[fragile]{Aplicação: Primeiros dígitos de uma exponenciação}

    \begin{itemize}
        \item Como $r$ é inteiro positivo, $b^r$ adiciona $r$ zeros ao final da representação de $a^k$ em base $b$

        \item Assim, os dígitos não-nulos de $a^k$ provém de $a^s$

        \item Por exemplo, $2^{80} = 1208925819614629174706176$ e $80\log_{10} 2 = 24.082399653118497$

        \item Daí, $s = 0.082399653118497$ e 
            $$
                10^{0.082399653118497} = 1.2089258196146322
            $$
    \end{itemize}

\end{frame}

\begin{frame}[fragile]{Aplicação: Meia-vida}

    \begin{itemize}
        \item A meia-vida é o tempo necessário para desintegrar metade da massa de um radioisótopo

        \item Se a massa inicial é $M_0$ e o decaimento é exponencial, a massa no instante $t$ é dada por
            $$
                M(t) = M_0e^{kt},
            $$
        onde $k$ é uma constante que depende do material

        \item Assim, a meia-vida seria o instante $t_{1/2}$ tal que
            $$
                M(t_{1/2}) = \frac{M_0}{2} = M_0e^{kt_{1/2}}
            $$
    \end{itemize}

\end{frame}

\begin{frame}[fragile]{Aplicação: Meia-vida}

    \begin{itemize}
        \item Aplicando o logaritmo em ambas expressões obtém-se
            $$
                \ln M_0 - \ln 2 = \ln M_0 + kt_{1/2}
            $$

        \item Assim,
            $$
                t_{1/2} = - \frac{\ln 2}{k}
            $$

        \item Veja que esta expressão permite computar a constante $k$ se a meia-vida for conhecida
    \end{itemize}

\end{frame}

\begin{frame}[fragile]{Série da função exponencial}

    \metroset{block=fill}
    \begin{block}{Proposição}
        A função exponencial pode ser expandida na série de potências
        $$
            e^x = \sum_{i = 0}^\infty \frac{x^i}{i!} = 1 + x + \frac{x^2}{2!} + \frac{x^3}{3!} + \ldots
        $$

        Esta série converge para qualquer $x$ real.
    \end{block}

\end{frame}


\begin{frame}[fragile]{Série da função logaritmo}

    \metroset{block=fill}
    \begin{block}{Proposição}
        A função logaritmo deslocada pode ser expandida na série de potências
        $$
            \ln (x + 1) = \sum_{i = 1}^\infty (-1)^{i - 1}\frac{x^i}{i} = x - \frac{x^2}{2} + \frac{x^3}{3} - \frac{x^4}{4} + \ldots
        $$

        Esta série converge apenas no intervalo $-1 < x < 1$.
    \end{block}

\end{frame}

\begin{frame}[fragile]{Exponenciais complexas}

    \begin{itemize}
        \item Por meio da manipulação das séries de potência de $e^x, \cos x$ e $\sin x$ é possível mostrar que, para um número complexo $a + bi$, que
            $$
            e^{a + bi} = e^ae^{bi} = e^a(\cos b + i\sin b)
            $$

        \item Desta igualdade surge a identidade de Euler, considerada a mais bela de toda matemática:
            $$
                e^{i\pi} + 1 = 0
            $$
    \end{itemize}

\end{frame}

