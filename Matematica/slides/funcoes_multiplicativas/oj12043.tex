\begin{frame}[fragile]{OJ 12043 -- Divisors}

    \textbf{Versão resumida do problema}: compute as somas
    \begin{align*}
        g(a, b, k) &= \sum_i \tau(i) \\
        h(a, b, k) &= \sum_i \sigma(i)
    \end{align*}
    onde $a\leq i\leq b$ e $i$ é divisível por $k$.

    \vspace{0.1in}

    \textbf{Restrições}:
    \begin{itemize}
        \item $0 < a \leq b\leq 10^5$
        \item $0 < k < 2000$
    \end{itemize}
\end{frame}

\begin{frame}[fragile]{Solução em $O(B\log B)$}

    \begin{itemize}
        \item Para resolver este problema dentro do limite de tempo estabelecido, é preciso
            computar, de forma eficiente, as funções $\tau(m)$ e $\sigma(m)$ para todos 
            os inteiros $m$ de 1 a $N$

        \item Isto pode ser feito por meio de uma variante do crivo de Erastótenes

        \item Para os demais inteiros positivos tem, no mínimo, dois divisores: 1 e o próprio
            número

        \item A ideia portanto é iniciar os valores $\tau(m) = 2$ e $\sigma(m) = m + 1$

        \item Após esta inicialização, para cada inteiro positivo $d$ no intervalo de 2 a $N$,
            devemos identificar quais inteiros $m$ são divisíveis por $d$
    \end{itemize}

\end{frame}

\begin{frame}[fragile]{Solução em $O(B\log B)$}

    \begin{itemize}
        \item Para cada um destes inteiros os valores de $\tau(m)$ e $\phi(m)$ devem ser 
            atualizados, de acordo com o valor de $k = m/d$

        \item Se $d\neq k$, então $\tau(m)$ deve ser acrescido em duas unidades, pois são
            dois novos divisores de $m$ encontrados, e $\sigma(m)$ deve aumentar em $d + k$
            unidades

        \item Nos casos em que $d = k$, $\tau(m)$ deve ser incrementado em uma única unidade,
            e o valor de $\sigma(m)$ deve ser acrescido em $d$ unidades

        \item É preciso tomar cuidado para que nenhum divisor seja contabilizado mais de uma 
            veze

        \item Assim, os múltiplos de $d$ começarão a ser considerados a partir de $d^2$
    \end{itemize}

\end{frame}

\begin{frame}[fragile]{Solução em $O(B\log B)$}

    \begin{itemize}
        \item Conforme comentado no vídeo que apresentou o crivo de Erastótenes, ao proceder
            desta maneira os múltiplos de $d$ menores que $d^2$ já foram processados anteriormente

        \item De posse dos valores pré-computados de $\tau(n)$ e de $\sigma(n)$, a soma pode
            ser feita de forma linear 

        \item Para evitar iterar sobre valores que não são múltiplos de $k$, o laço deve iniciar
            no primeiro múltiplo $m$ de $k$ que é maior ou igual a $a$, e o incremento deve ser feito
            em passos de tamanho $k$

        \item Este múltiplo $m$ pode ser obtido por meio da expressão $m = kt$, onde
        $$
            t = \left\lceil \frac{a}{k} \right\rceil
        $$
    \end{itemize}

\end{frame}
\begin{frame}[fragile]{Solução $O(B\log B)$}
    \inputsnippet{cpp}{8}{17}{codes/12043.cpp}
\end{frame}

\begin{frame}[fragile]{Solução $O(B\log B)$}
    \inputsnippet{cpp}{18}{30}{codes/12043.cpp}
\end{frame}

\begin{frame}[fragile]{Solução $O(B\log B)$}
    \inputsnippet{cpp}{32}{45}{codes/12043.cpp}
\end{frame}
