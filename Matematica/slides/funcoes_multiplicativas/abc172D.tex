\begin{frame}[fragile]{AtCoder Beginner Contest 172D -- Sum of Divisors}

    \textbf{Versão resumida do problema}: calcule o valor $S$ dado pela soma
    $$
        S = \sum_{K = 1}^N  K\times \tau(K)
    $$

    \vspace{0.1in}
    \textbf{Restrição}: $1 \leq N \leq 10^7$

\end{frame}

\begin{frame}[fragile]{Solução com complexidade $O(N\log N)$}

    \begin{itemize}
        \item Assim como fizemos no caso da função $\varphi$ de Euler, é preciso computar o valor
            de $\tau$ para todos inteiros no intervalo $[1, N]$ de forma eficiente

        \item Uma vez que $\tau$ é uma função multiplicativa, isto pode ser feito por meio de
            uma variante do crivo de Erastótenes

        \item O crivo permite a identificação de um dos fatores primos de $n$, para $n\in [2, N]$

        \item No código da solução será mantido, para cada $n$, o maior primo $p$ que o divide

        \item Uma vez identificados estes fatores primos, o valor de $\tau(n)$ é computado em
            ordem crescente
    \end{itemize}

\end{frame}

\begin{frame}[fragile]{Solução com complexidade $O(N\log N)$}

    \begin{itemize}
        \item Lembre que $\tau(1) = 1$, de modo que esta computação inicia em $n = 2$

        \item Para cada $n$, utilizamos o fator $p$ para escrever $n = p^k \times m$, com
            $(p^k, m) = 1$

        \item Daí $$\tau(n) = \tau(p^k)\tau(m) = (k + 1)\tau(m)$$

        \item Como $m < n$, pois $p$ é primo, quando $\tau(n)$ estiver sendo computado o valor de
            $\tau(m)$ já estará disponível

        \item Como a fatoração parcial de $n$ tem complexidade $O(\log n)$, a solução terá 
            complexidade $O(N\log N)$, por conta do crivo modificado
    \end{itemize}

\end{frame}

\begin{frame}[fragile]{Solução com complexidade $O(N\log N)$}
    \inputsnippet{cpp}{8}{24}{codes/abc172D.cpp}
\end{frame}

\begin{frame}[fragile]{Solução com complexidade $O(N\log N)$}
    \inputsnippet{cpp}{26}{40}{codes/abc172D.cpp}
\end{frame}

\begin{frame}[fragile]{Solução com complexidade $O(N\log N)$}
    \inputsnippet{cpp}{42}{52}{codes/abc172D.cpp}
\end{frame}
