\begin{frame}[fragile]{AtCoder Beginner Contest 096D -- Five, Five Everywhere}

    \textbf{Versão resumida do problema}: dado um inteiro $n$, determine uma sequência de inteiros
        $a_1, a_2, \ldots, a_N$ tais que

    \begin{itemize}
        \item $a_i < 55555$ é primo
        \item Todos os elementos da sequência são distintos
        \item A soma de quaisquer 5 elementos da sequência resulta em um número composto
    \end{itemize}

    \vspace{0.1in}
    \textbf{Restrição}: $5 \leq n \leq 55$

\end{frame}

\begin{frame}[fragile]{Solução com complexidade $O(N\log \log N)$}

    \begin{itemize}
        \item A solução consiste em identificar um subconjuntos de primos que atendam a primeira
            e a terceira condições

        \item Os $\pi(55555) = 5637$ primos podem se gerados pelo crivo de Erastótenes

        \item Devem ser selecionados dentre os primos aqueles cujo resto da divisão por 5 seja
            $k$, com $k > 0$

        \item Qualquer $k$ no intervalo $[1, 4]$ gera uma lista de mais de 1.400 primos

        \item Assim, após o filtro todos os elementos selecionados serão da forma $5m + k$, e 
            daí
        $$
            (5m_1 + k) + (5m_2 + k) + \ldots + (5m_5 + k) = 5(m_1 + m_2 + \ldots + m_5 + k)
        $$

        \item Ou seja, a soma de quaisquer elementos dentre os $N$ escolhidos da lista é 
            divisível por 5, e portanto é um número composto
    \end{itemize}

\end{frame}

\begin{frame}[fragile]{Solução com complexidade $O(N\log \log N)$}
    \inputsnippet{cpp}{1}{21}{codes/abc096D.cpp}
\end{frame}
