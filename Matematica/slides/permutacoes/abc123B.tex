\begin{frame}[fragile]{AtCoder Beginner Contest 123B -- Five Dishes}

    \textbf{Versão resumida do problema}: determinar a sequência em que os pratos devem ser
        pedidos para que o tempo total para servir todos eles seja o menor possível.

    \vspace{0.1in}
    \textbf{Restrições}:
    \begin{itemize}
        \item somente um prato pode ser servido por vez,
        \item um prato só pode ser pedido em quando o tempo for um múltiplo de 10, e
        \item um novo pedido só pode ser feito quando o prato anterior for servido.
    \end{itemize}
\end{frame}

\begin{frame}[fragile]{Solução com complexidade $O(1)$}

    \begin{itemize}
        \item A solução consiste em determinar uma permutação dos pratos A, B, C, D e E que 
            minimize o tempo para serví-los

        \item Se não houvesse a restrição de que os pratos só podem ser pedidos em instantes de
            tempo que são múltiplos de 10, qualquer permutação levaria ao mesmo resultado

        \item Há $5! = 120$ permutações possíveis, as quais podem ser geradas por meio da 
            função \code{cpp}{next_permutation()}

        \item O $i$-ésimo prato requer $t_i$ minutos para ser servido após ser pedido e, exceto
            pelo último, é preciso esperar o próximo múltiplo de 10 para ser pedido

        \item Uma forma de tratar esta espera é substituir $t_i$ pelo menor múltiplo de 10 maior
            ou igual a $t_i$ para todos os pratos, exceto o último
    \end{itemize}

\end{frame}

\begin{frame}[fragile]{Solução com complexidade $O(1)$}
    \inputsnippet{cpp}{7}{22}{codes/B123B.cpp}
\end{frame}
