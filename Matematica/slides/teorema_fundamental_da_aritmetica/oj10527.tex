\begin{frame}[fragile]{OJ 10527 -- Persistent Numbers}

    \textbf{Versão resumida do problema}: compute o menor número $x$ tal que $n$ é igual ao 
        produto de seus dígitos
    \vspace{0.1in}

    \textbf{Restrições}: o número $n$ tem até 1000 dígitos decimais.
\end{frame}

\begin{frame}[fragile]{Solução $O(\log n)$}

    \begin{itemize}
        \item Se $n$ é o produto dos dígitos de $x$, então a fatoração prima de $n$ só pode
            conter primos cuja representação decimal só tem um dígito, a saber: 2, 3, 5 e 7

        \item Assim, se a fatoração de $n$ tem qualquer outro primo o problema não
            terá solução

        \item Nos demais casos, a fatoração prima de $n$ seria uma solução, embora nem sempre
            seja a mínima

        \item Para minimizar a solução, é preciso agrupar os fatores primos em dígitos compostos

        \item Antes de fazer esta redução, tratemos primeiro de um caso especial
    \end{itemize}

\end{frame}


\begin{frame}[fragile]{Solução $O(\log n)$}

    \begin{itemize}
        \item Mesmo não estando explícito no texto do problema, espera-se que $x$ tenha, no 
            mínimo, dois dígitos, conforme se observa nos exemplos de entrada e saída

        \item Assim, se $n$ tiver um único digo, a solução mínima seria o número $10 + n$

        \item Nos demais casos, para minimizar $x$ agruparemos os fatores primos nos compostos
            9, 8, 6 e 4, nesta ordem, de forma gulosa

        \item Feito este agrupamento, $x$ é formado por estes agrupamentos, ordenados do
            menor para o maior
    \end{itemize}

\end{frame}

\begin{frame}[fragile]{Solução $O(\log n)$}
    \inputsnippet{py3}{5}{17}{codes/10527.py}
\end{frame}

\begin{frame}[fragile]{Solução $O(\log n)$}
    \inputsnippet{py3}{20}{35}{codes/10527.py}
\end{frame}

\begin{frame}[fragile]{Solução $O(\log n)$}
    \inputsnippet{py3}{36}{51}{codes/10527.py}
\end{frame}

\begin{frame}[fragile]{Solução $O(\log n)$}
    \inputsnippet{py3}{54}{60}{codes/10527.py}
\end{frame}
