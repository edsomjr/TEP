\section*{Divisibilidade}

\begin{frame}[fragile]{Número de Euler}

    \metroset{block=fill}
    \begin{block}{Definição}
        Sejam $a$ e $b$ dois números inteiros. Dizemos que $a$ \textbf{divide} $b$ (ou que $b$ é divisível por $a$) se existe um $k$ inteiro tal que $b$ = $ak$. Caso não exista tal inteiro, dizemos que $a$ não divide $b$. Dizemos também que $b$ é um \textbf{múltiplo} de $a$.  

        \vspace{0.1in}
        \textbf{Notação:} $a|b$ (lê-se ``$a$ divide $b$'')
    \end{block}

\end{frame}

\begin{frame}[fragile]{Relações triviais}

    \begin{itemize}
        \item Qualquer número $a$ divide a si mesmo, pois $a = a\times 1$

        \item Observe que $1$ divide qualquer inteiro $m$, pois $m =  1 \times m$

        \item Segundo a definição de divisibilidade, zero divide zero, pois $0 = k \times 0$ para qualquer $k$ inteiro  

        \item De fato, qualquer inteiro $a$ divide zero, pois $0 = a\times 0$
    \end{itemize}

\end{frame}

\begin{frame}[fragile]{Unicidade de $k$}

    \begin{itemize}
        \item Se $a$ é diferente de zero e $a$ divide $b$, então o inteiro $k$ tal que $b = ak$ é único

        \item Suponha que exista um $t$ tal que $b = ak = at$. Como $a$ é diferente de zero, vale o cancelamento da multiplicação, de modo que $k = t$

        \item Observe que se $s \neq r$, ainda vale que $0 = 0 \times r = 0 \times s$

        \item Assim, $k$ não fica determinado (por isso que $0/0$ é uma indeterminação)

        \item Para todos os demais valores $a \neq 0$, o \textbf{quociente} $k$ de $b$ por $a$ é o inteiro $k$ tal que $b = ak$
    \end{itemize}

\end{frame}

\begin{frame}[fragile]{Propriedades da divisibilidade}

    Para quaisquer inteiros $a, b$ e $c$ vale que:

    \begin{enumerate}
        \item se $a|b$ e $b|c$ então $a|c$ (propriedade transitiva)
        \item $a|a$ (propriedade reflexiva)
        \item se $a|b$ e $b|a$, então $a = b$ ou $a = -b$
        \item se $a|b$ então $|a|\leq |b|$
        \item se $a|b$ e $a|c$ então $a|(bx + cy)$, para quaisquer $x,y$ inteiros
    \end{enumerate}

\end{frame}

\begin{frame}[fragile]{Divisão de Euclides}

    \metroset{block=fill}
    \begin{block}{Definição}
        Sejam $a,b$ inteiros, com $b \neq 0$. Segundo a \textbf{divisão de Euclides} existem dois inteiros $q,r$, únicos, com $0 \leq r < |b|$, tais que $a = bq + r$. O número $q$ é o \textbf{quociente} da divisão e $r$ é o \textbf{resto}.

    \end{block}

    \vspace{0.2in}

    Observe que, se $r = 0$, então $b$ divide $a$.
\end{frame}

\begin{frame}[fragile]{Resto da divisão em C++}

    O operador \mintinline{cpp} retorna um resto negativo, o que viola a condição $0 \leq r < |b|$ da divisão de Euclides

        \item Para determinar o resto euclidiano nestes casos, basta somar o valor absoluto de $b$ ao resto negativo
    \end{itemize}

\end{frame}

\begin{frame}[fragile]{Maior Divisor Comum}

    \metroset{block=fill}
    \begin{block}{Definição}
        Dados dois inteiros $a$ e $b$, o \textbf{maior divisor comum} (MDC) de $a$ e $b$ é o inteiro não-negativo $d$ tal que

        \begin{enumerate}
            \item $d$ divide $a$ e $d$ divide $b$;
            \item se $c$ divide $a$ e $c$ divide $b$, então $c$ divide $d$.
        \end{enumerate}

        \vspace{0.1in}

        \textbf{Notação:} $d = (a, b)$
    \end{block}

\end{frame}

\begin{frame}[fragile]{Observações sobre a definição do MDC}

    \begin{itemize}
        \item A primeira condição apresentada garante que $d$ é divisor comum de $a$ e $b$

        \item A segunda garante que ele é o maior dentre os divisores comuns de $a$ e $b$

        \item Pode-se observar que

        \begin{enumerate}
            \item $d = 0$ se, e somente se, $a = b = 0$;
            \item $(a, 0) = |a|$, para todo inteiro $a$.
        \end{enumerate}

        \item Como $(a, b) = (-a, b) = (a, -b) = (-a,-b)$, o problema de se determinar o MDC pode ser restrito aos números não-negativos
    \end{itemize}

\end{frame}

\begin{frame}[fragile]{Cálculo do MDC}

    \begin{itemize}
        \item Se $a$ e $b$ são dois inteiros não-negativos, com $a \geq b > 0$, por Euclides existem únicos $q$ e $r$ tais que $a = bq + r$, com $0 \leq r < b$

        \item Escrevendo $r = a - bq$, é possível mostrar que $(a, b) = (b, r)$

        \item Lembrando que $(a, 0) = a$, o MDC pode ser computado com complexidade $O(\log a)$

            \vspace{0.1in}
            \inputsnippet{cpp}{5}{8}{codes/gcd.cpp}

    \end{itemize}

\end{frame}

\begin{frame}[fragile]{Algoritmo de Euclides Estendido}

    \begin{itemize}
        \item É possível mostrar também que o MDC entre $a$ é $b$ é o menor número não-negativo que pode ser escrito como uma combinação linear $ax + by$ 

        \item Esta interpretação é fundamental para a demonstração de várias propriedades associadas ao MDC

        \item Para se determinar tais inteiros $x$ e $y$ (os quais não são únicos) pode-se usar uma versão estendida do algoritmo do MDC, denominada Algoritmo de Euclides Estendido
    \end{itemize}

\end{frame}

\begin{frame}[fragile]{Implementação do Algoritmo de Euclides Estendido em C++}
    \inputsnippet{cpp}{10}{26}{codes/gcd.cpp}
\end{frame}

\begin{frame}[fragile]{Equações Diofantinas Lineares}

    \begin{itemize}
        \item Uma importante aplicação do MDC e do algoritmo de Euclides estendido é a solução de equações diofantinas lineares

        \item Para $a, b, c, x, y$ inteiros, as equações diofantinas lineares são da forma
            $$
            ax + by = c
            $$

        \item Tais equações tem solução se, e somente se, $(a, b)$ divide $c$
    \end{itemize}

\end{frame}

\begin{frame}[fragile]{Solução particular}

    Uma solução \textbf{particular} $(x_0, y_0)$ de uma equação diofantina linear pode ser determinada da seguinte maneira:

    \begin{enumerate}
        \item Determine $x'$ e $y'$ tais que $ax' + by' = d$ (Algoritmo de Euclides estendido)
        \item Faça $k = c / d$
        \item Compute $x_0 = k \times x'$ e  $y_0 = k \times y'$
    \end{enumerate}

    \vspace{0.2in}

    Observe que $$ax_0 + by_0 = a(kx') + b(ky') = k(ax' + by') = kd = c$$

\end{frame}

\begin{frame}[fragile]{Solução geral das Equações Diofantinas Lineares}

    \begin{itemize}
        \item A solução particular não é única

        \item A solução geral de uma equação diofantina linear é dada por, para qualquer inteiro $t$, por
            $$
            \begin{matrix}
                x = x_0 + (a/d)t\\
                y = y_0 - (b/d)t
            \end{matrix}
            $$

        \item Estas expressões nos permitem determinar, por exemplo, soluções específicas, como a de menor $x$ (ou $y$), menor diferença entre $x$ e $y$, menor solução com $x$ e $y$ positivos, e assim por diante (se existirem)
    \end{itemize}

\end{frame}

\begin{frame}[fragile]{Números coprimos}

    \metroset{block=fill}
    \begin{block}{Definição}
        Dois números $a$ e $b$ são dito \textbf{coprimos}, ou primos entre si, se $(a, b)$ = 1.
    \end{block}

    \vspace{0.2in}

    Observe que, para dois inteiros $a$ e $b$ quaisquer, se $d$ = $(a, b)$, então 
    $$
    \left(\frac{a}{d}, \frac{b}{d}\right) = 1
    $$

\end{frame}

\begin{frame}[fragile]{Menor Múltiplo Comum}

    \metroset{block=fill}
    \begin{block}{Definição}
        Sejam $a$ e $b$ dois inteiros. O \textbf{menor múltiplo comum} (MMC) de $a$ e $b$ é o inteiro $m$ tal que

        \begin{enumerate}
            \item $a$ divide $m$ e $b$ divide $m$;
            \item se $a$ divide $n$ e $b$ divide $n$, então $m$ divide $n$.
        \end{enumerate}

        \vspace{0.2in}
        \textbf{Notação}: $m = [a, b]$
    \end{block}

\end{frame}

\begin{frame}[fragile]{Cálculo do MMC}

    \begin{itemize}
        \item De forma similar ao MDC, a primeira propriedade torna $m$ um múltiplo comum de $a$ e $b$; a segunda o torna o menor dentre os múltiplos comuns

        \item Uma importante relação entre o MDC e o MMC é que $ab = (a,b)[a,b]$

            \inputsyntax{cpp}{codes/mmc.cpp}

        \item Veja que, na implementação acima, a divisão é feita antes do produto: esta ordem pode evitar \textit{overflow} em alguns casos
    \end{itemize}

\end{frame}
