\section{Arranjos}

\begin{frame}[fragile]{Definição}

    \metroset{block=fill}
    \begin{block}{Definição de arranjo}
        Seja $A$ um conjunto com $n$ elementos distintos e $p$ um inteiro não negativo tal que 
        $p \leq n$.  Um \textbf{arranjo} destes $n$ elementos, tomados $p$ a $p$, consiste em uma 
        escolha de $p$ elementos distintos dentre os $n$ possíveis, onde cada arranjo difere dos 
        demais tanto pela qualidade quanto pela posição dos elementos.


        \textbf{Notação}: $A(n, p)$
    \end{block}

    \vspace{0.1in}

    Por exemplo, se $A = \{1, 2, 3, 4\}$ e $p = 2$, há $12$ arranjos distintos, a saber:
    $$
    12, 13, 14, 21, 23, 24, 31, 32, 34, 41, 42, 43
    $$
\end{frame}

\begin{frame}[fragile]{Cálculo de $A(n, p)$}

    \begin{itemize}
        \item Utilizando a mesma abordagem usada para computar $P(n)$, segue que
$$
        A(n, p) = n \times (n - 1) \times (n - 2) \times \ldots \times (n - (p - 1))
$$

        \item A lista acima contém $p$ fatores multiplicativos e se assemelha a um fatorial

        \item Se o termos remanescentes do fatorial forem multiplicados e a expressão dividida por
            estes mesmos elementos, obtém-se
$$
        A(n, p) = \frac{n \times (n - 1) \times \ldots \times (n - (p - 1)) \times (n - p) \times \ldots \times 2 \times 1}{(n - p) \times \ldots \times 2 \times 1} = \frac{n!}{(n - p)!}
$$
    \end{itemize}

\end{frame}

\begin{frame}[fragile]{Implementação de $A(n, p)$ em C++}
    \inputsnippet{cpp}{1}{21}{codes/A.cpp}
\end{frame}

\begin{frame}[fragile]{Caracterização dos arranjos}

    \begin{itemize}
        \item Assim como no caso das permutações, os arranjos podem ser interpretados como a
            retirada de bolas distintas de uma caixa, sem reposição, onde a ordem da retira é
            importante

        \item A diferença em relação às permutações é que número $p$ de bolas a serem removidas não
            é, necessariamente, igual a $n$

        \item Observe que $A(n, n) = P(n)$
    \end{itemize}

\end{frame}

\begin{frame}[fragile]{Arranjos com repetições}

    \begin{itemize}
        \item Nos arranjos com repetições, as bolas são repostas na caixa após cada retirada 

        \item Deste modo, a cada retirada há $n$ possíveis escolhas

        \item Assim,
    $$
        AR(n, p) = n \times n \times \ldots \times n = n^p
    $$
    \end{itemize}

\end{frame}

\begin{frame}[fragile]{Programação Dinâmica em problemas de contagem}

    \begin{itemize}
        \item Uma variante mais complicada do arranjo com repetições seria: \textit{Quanto são os
            arranjos de $n$ elementos, sendo que estes elementos não são, necessariamente, distintos?}

        \item Considere que existam apenas $k$ elementos distintos, que o $i$-ésimo elemento 
            distinto ocorre $n_i$ vezes e que $n_1 + n_2 + \ldots + n_k = n$

        \item Este problema, como muitos outros em combinatória, pode ser resolvido por uma relação
           de recorrência

        \item Estas relações podem ser implementadas, de forma eficiente, por meio de algoritmos de
            programação dinâmica
    \end{itemize}

\end{frame}

\begin{frame}[fragile]{Exemplo: Número de resultados distintos}

    \begin{itemize}
        \item Considere o seguinte cenário: $a$ equipes da Escola A e $b$ equipes da escola $B$
            participaram de uma gincana escolar. Quantos são os possíveis resultados da gincana,
            sendo que serão divulgadas as $k$ melhores equipes?

        \item Considere que dois resultados
            são diferentes se em ao menos uma das posições a escola vencedora seja diferente.

        \item Assuma que $k \leq a + b$

        \item Por exemplo, se $a = 2, b = 3$ e $k = 3$, então os resultados possíveis seriam
\begin{verbatim}
    AAB     ABA     ABB     BAA     BAB     BBA     BBB
\end{verbatim}

    \end{itemize}

\end{frame}

\begin{frame}[fragile]{Solução por recorrência}

    \begin{itemize}
        \item Seja $\sigma(k, a, b)$ o número de arranjos distintos para as $k$ primeiras posições
            levando-se em consideração $a$ equipes da escola A e $b$ equipes da escola B

        \item Observe que, a cada etapa da geração de um determinado arranjo, há duas opções:
            escolher uma equipe da escola A ou uma equipe de escola B

        \item  Assim,
$$
    \sigma(k, a, b) = \sigma(k - 1, a - 1, b) + \sigma(k - 1, a, b - 1)
$$
    \end{itemize}

\end{frame}

\begin{frame}[fragile]{Solução por recorrência}

    \begin{itemize}
        \item São três os casos-base:
        \begin{enumerate}
            \item $\sigma(0, a, b) = 1$
            \item $\sigma(k, a, 0) = 1$, se $a \geq k$, ou $\sigma(k, a, 0) = 0$, caso contrário
            \item $\sigma(k, 0, b) = 1$, se $b \geq k$, ou $\sigma(k, 0, b) = 0$, caso contrário
        \end{enumerate}

        \item Considerando que há $O(KAB)$ estados distintos, onde $K, A$ e $B$ são os valores
            máximos para $k, a$ e $b$, respectivamente, e que a transição é feita em $O(1)$, esta
            solução tem complexidade $O(KAB)$
    \end{itemize}

\end{frame}

\begin{frame}[fragile]{Solução com complexidade $O(KAB)$}
    \inputsnippet{cpp}{9}{24}{codes/dp.cpp}
\end{frame}
