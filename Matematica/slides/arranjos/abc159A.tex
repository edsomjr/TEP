\begin{frame}[fragile]{AtCoder Beginner Contest 159A -- The Number of Even Pairs}

    \textbf{Versão resumida do problema}: dadas $N + M$ bolas, determine o número de maneiras de
        se escolhar duas destas bolas de forma que a soma dos números escritos em ambas bolas seja
        par. A ordem da retirada das bolas deve ser desconsiderada, em $N$ bolas os números
        escritos são pares e nas outras $M$ estão escritos números ímpares.

    \vspace{0.1in}
    \textbf{Restrições}:
    \begin{itemize}
        \item $0\leq N, M\leq 100$
        \item $2\leq N + M$
    \end{itemize}
\end{frame}

\begin{frame}[fragile]{Solução com complexidade $O(1)$}

    \begin{itemize}
        \item Para que a soma dos números escritos nas bolas seja par há apenas duas
            possibilidades:
        \begin{itemize}
            \item escolher duas bolas com números pares, ou
            \item escolher duas bolas com números ímpares
        \end{itemize}

        \item No primeiro caso, há $A(N, 2)$ formas de se escolhar duas bolas com números
            pares, mas como a ordem de retirada não importa no problema e o arranjo contabiliza
            ordens distintas, é preciso dividir este resultado por 2

        \item O mesmo vale para as $M$ bolas ímpares

        \item Assim, a solução $S$ será dada por
        $$
            S = \frac{A(N, 2) + A(M, 2)}{2} = \frac{N(N - 1) + M(M - 1)}{2}
        $$
    \end{itemize}

\end{frame}

\begin{frame}[fragile]{Solução com complexidade $O(1)$}
    \inputsnippet{cpp}{1}{22}{codes/B159A.cpp}
\end{frame}
