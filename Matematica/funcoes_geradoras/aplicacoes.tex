\section*{Soluções das Equações Lineares com Coeficientes Constantes: Uma Nova Perspectiva}

\begin{frame}[fragile]{Função geradora da solução}

Voltemos ao problema de se contar o número de soluções distintas da equação
\[
        x_1 + x_2 + \ldots + x_r = n
\]
com $x_i \geq 0$. Associaremos o polinômio
\[
        1 + x + x^2 + \ldots
\]
a cada variável $x_i$, o qual representa todos os valores possíveis para cada variável. A função
geradora será dada pelo produto destes polinômio, isto é,
\[
        f(x) = (1 + x + x^2 + \ldots)(1 + x + x^2 + \ldots) \ldots (1 + x + x^2 + \ldots)
             = (1 + x + x^2 + \ldots)^r
\]

Usando a soma da PG infinita temos que
\[
        f(x) = \frac{1}{(1 - x)^r} = (1 - x)^{-r}
\]

\end{frame}

\begin{frame}[fragile]{Coeficiente binomial generalizado}

Para se determinar o coeficiente do termo $xn$ de $f(x)$, podemos usar a série de Taylor com 
de $(1 - x)^u$, com $ a=0$, para um $u$ real. Isto nos leva ao coeficiente binomial generalizado:
\[
        \binom{u}{k} = \frac{u(u - 1)(u - 2) \ldots (u - k + 1)}{k!}
\]
onde $k$ é um inteiro não-negativo e $u$ um número real. Por exemplo
\[
        \binom{\frac{1}{2}}{4} = \dfrac{\left(\dfrac{1}{2}\right)\left(\dfrac{-1}{2}\right)\left(\dfrac{-3}{2}\right)\left(\dfrac{-5}{2}\right)}{4!} = -\frac{15}{384}
\]
\end{frame}

\begin{frame}[fragile]{Equilavências entre coeficientes binomiais}

Assim, o termo $x_n$ de $f(x)$ terá coeficiente igual a
\[
\begin{split}
        (-1)^n \binom{-r}{n} &= (-1)^n \frac{(-r)(-r - 1)(-r - 2) \ldots (-r - n + 1)}{n!} \\
                             &= (-1)^n \left[\frac{(-1)^n r(r + 1)(r + 2) \ldots (r + n - 1)}{n!}\right] \\
                             &= \left[1 \times 2 \times \ldots \times (r - 1)\right]\times \frac{r(r + 1)(r + 2) \ldots (r + n - 1)}{(n!(r - 1)!} \\
                             &= \frac{(n + r - 1)!}{[n!(r - 1)!} \\
                             &= \binom{n + r - 1}{r - 1}
\end{split}
\]
a qual é a solução encontrada anteriormente. A igualdade acima é verdadeira sempre que $r$ for
um inteiro positivo.

\end{frame}


% TODO Adicionar texto sobre funções geradoras exponenciais, e as propriedades das funções geradoras
