\section{Coeficientes binomiais}

\begin{frame}[fragile]{Coeficientes binomiais}

    \metroset{block=fill}
    \begin{block}{Definição de coeficiente binomial}
        Sejam  $n$ e $p$ dois inteiros não-negativos tais que $n\geq p$. O coeficiente binomial é
        dado por
        $$
            \binom{n}{p} = \frac{n!}{(n - p)!p!}
        $$
    \end{block}

\end{frame}

\begin{frame}[fragile]{Implementação dos coeficientes binomiais}

    \begin{itemize}
        \item Na prática, pode ser que o valor de $\binom{n}{p}$ possa ser armazenado em uma
        variável inteira, mas o cálculo dos fatoriais envolvidos no numerador e no denominador pode
        resultar em um \textit{overflow}

        \item Há duas maneiras de contornar este problema: por cancelamento ou por recorrência

        \item A ideia do cancelamento é que, embora seja representado na forma de fração,
            $\binom{n}{p}$ é sempre um número inteiro

        \item Assim, é possível realizar os cancelamentos devidos antes de multiplicar os fatores
            restantes
    \end{itemize}

\end{frame}

\begin{frame}[fragile]{Implementação dos binomiais por cancelamento}
    \inputsnippet{cpp}{41}{57}{codes/cancelamento.cpp}
\end{frame}

\begin{frame}[fragile]{Triângulo de Pascal}

    \begin{itemize}
        \item Os números binomiais surgem nas expansões do monômio $(a + b)^n$, para $n$ não-negativo

        \item Estas expansões formam o Triângulo de Pascal:

\begin{verbatim}
        1
        1  1
        1  2  1
        1  3  3  1
        1  4  6  4  1
        1  5 10 10  5  1
        ...
\end{verbatim}
        \item A observação cuidadosa deste triângulo permite definir os coeficientes binomiais recursivamente
    \end{itemize}

\end{frame}

\begin{frame}[fragile]{Definição recursiva dos coeficientes binomiais}

    \begin{itemize}
        \item Sejam $n$ e $m$ inteiros não-negativos. Então os casos-base da recursão são
$$
        \binom{n}{0} = \binom{n}{n} = 1
$$
        \item A transição é dada por
$$
        \binom{n}{m} = \binom{n - 1}{m} + \binom{n - 1}{m - 1}
$$
    \end{itemize}

\end{frame}

\begin{frame}[fragile]{Implementação dos coeficientes binomiais por DP}
    \inputsnippet{cpp}{1}{21}{codes/dp.cpp}
\end{frame}

\begin{frame}[fragile]{Redução da complexidade de memória}

    \begin{itemize}
        \item A implementação acima tem complexidade $O(n^2)$ para execução e para memória

        \item É possível reduzir o uso de memória para $O(m)$ através de uma implementação
            cuidadosa, que se vale das propriedades da recorrência

        \item A ideia central é computar os coeficientes de cada linha da direita para a esquerda,
            uma vez que o coeficiente da próxima linha é computado a partir do coeficiente que
            ocupa a mesma posição na linha anterior e o coeficiente da linha anterior na posição
            anterior
    \end{itemize}

\end{frame}

\begin{frame}[fragile]{Implementação dos coeficientes binomiais em $O(nm)$ memória $O(m)$}
    \inputsnippet{cpp}{1}{21}{codes/dp2.cpp}
\end{frame}

\begin{frame}[fragile]{Propriedades dos coeficientes binomiais}

    \begin{itemize}
        \item Combinações complementares (permite a redução do espaço de memória em 50\%):
$$
    \binom{n}{p} = \binom{n}{n - p}
$$

        \item  Soma de uma linha (consequência da expansão do binômio $(1 + 1)^n$):
$$
    \binom{n}{0} + \binom{n}{1} + \ldots + \binom{n}{n} = 2^n
$$
    \end{itemize}

\end{frame}

\begin{frame}[fragile]{Propriedades dos coeficientes binomiais}

    \begin{itemize}
        \item Soma alternada de uma linha (consequência da expansão do binômio $(1 - 1)^n$):
$$
    \binom{n}{0} - \binom{n}{1} + \ldots + (-1)^n\binom{n}{n} = 0
$$

        \item Soma de uma coluna, com $n\geq p$ (\textit{Hockey-Stick Identity}):
$$
    \binom{p}{p} + \binom{p + 1}{p} + \binom{p + 2}{p} + \ldots + \binom{n}{p} = \binom{n + 1}{p + 1}
$$
    \end{itemize}

\end{frame}

\begin{frame}[fragile]{Identidades úteis}

    \begin{itemize}
        \item Soma de uma linha com coeficientes lineares:
$$
    \sum_{k = 0}^n k\binom{n}{k} = n2^{n - 1}
$$

        \item Soma de uma linha com coeficientes quadráticos:
$$
    \sum_{k = 0}^n k^2\binom{n}{k} = (n^2 + n)2^{n - 2}
$$
    \end{itemize}

\end{frame}

\begin{frame}[fragile]{Identidades úteis}

    \begin{itemize}
        \item Soma dos quadrados dos coeficientes de uma linha:
$$
    \sum_{k = 0}^n \binom{n}{k}^2 = \binom{2n}{n}
$$

        \item Se $F(n)$ é o $n$-ésimo número de Fibonacci, vale que
$$
    \sum_{k = 0}^{\lfloor \frac{n}{2}\rfloor} \binom{n - k}{k} = F(n + 1)
$$
    \end{itemize}

\end{frame}
