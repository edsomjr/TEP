\section*{Programação Competitiva e Matemática}

\begin{frame}[fragile]{Matemática e as grandes áreas}

A matemática é uma ciência ampla, presente, em maior ou menor grau, nas demais ciências, principalmente nas ciências exatas. 

Uma maneira de se tentar organizar o conhecimento matemático é usar a Lógica e a Teoria dos Conjuntos como base e separar o restante em 4 grandes áreas:

\begin{enumerate}
    \item Análise (estudo dos números reais)
    \item Álgebra (estudo dos números inteiros e das estruturas matemáticas)
    \item Geometria (estudo do espaço)
    \item Matemática Aplicada (por exemplo, a Teoria da Probabilidade)
\end{enumerate}

\end{frame}

\begin{frame}[fragile]{Matemática em Programação Competitiva}

    \begin{itemize}
        \item Embora esta divisão seja questionável, o importante é entender como estas áreas são tratadas na programação competitiva

        \item A mais rara é a Análise, aparecendo em problemas envolvendo, em geral, integrações ou a Transformada de Fourier

        \item A Geometria é bem representada, sendo considerada um tópico a parte (Geometria Computacional)

        \item A Probabilidade aparece também com frequência, em problemas cuja solução pode envolver programação dinâmica em algum nível
    \end{itemize}

\end{frame}

\begin{frame}[fragile]{Matemática em Programação Competitiva}

    \begin{itemize}
        \item Por fim, a mais frequente de todas é a Álgebra, principalmente no que diz respeito à questões envolvendo números primos e aritmética modular

        \item Este material tem como foco principal a Álgebra, embora não seja restrito especificamente a ela

        \item Também serão tratados conceitos e tópicos relacionados às demais áreas
    \end{itemize}

\end{frame}
