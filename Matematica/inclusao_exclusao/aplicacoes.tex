\section*{Aplicações do Princípio da Inclusão/Exclusão}

\begin{frame}[fragile]{Permutações caóticas}

    \metroset{block=fill}
    \begin{block}{Definição}
        Uma permutação de $N$ elementos é dita caótica se nenhum dos $N$ elementos ocupa, na 
        permutação, a mesma posição que ocupava na posicionamento original, isto é, o elemento $i$ não
        ocupa a $i$-ésima posição.
    \end{block}

\end{frame}

\begin{frame}[fragile]{Contando permutações caóticas}

    Podemos contar o total de permutações caóticas $D(N)$ (D de \textit{derangement}, em inglês) da
seguinte forma: seja $A_k$ o conjunto das permutações nas quais o elemento $k$ ocupa a $k$-ésima
posição. Assim
\[
        D(N) = P(N) - \sum_{i} | A_i | + \sum_{i<j}| A_i \cap  A_j | - \sum_{i<j<k}| A_i \cap  A_j \cap  A_k | 
             + ... + (-1)^N | A_1 \cap  A_2 \cap  ... \cap  A_N |
\]
onde $P(N)$ é o total de permutações de $N$ elementos. 

Observe que $|A_i| = (N - 1)!$ (pois o elemento $i$ está fixo), $|A_i \cap A_j| = (N - 2)!$ (dois elementos fixos), e assim por diante.
Logo
\[
        D(N) = N! - \binom{N}{1}(N - 1)! + \binom{N}{2}(N - 2)! - ... + (-1)^N \binom{N}{N}
\]
o que nos dá
\[
        D(N) = N!\left(1 - \frac{1}{1!} + \frac{1}{2!} - \frac{1}{3!} + ... + (-1)^N\frac{1}{N!}\right)
\]

\end{frame}

\begin{frame}[fragile]{Permutações caóticas e programação dinâmica}

Na prática, para computar $D(N)$ é melhor utilizar a seguinte recorrência. Para $N = 1$, não 
há permutações caóticas entre as $1!$ possíveis. Para o caso $N = 2$, há uma permutação caótica
entre as 2 possíveis (a saber, a permutação $(2\ 1)$). Assim os casos base da recorrência são:
\[
\begin{array}{l}
        D(1) = 0 \\
        D(2) = 1
\end{array}
\]

Para computar $D(N)$ devemos decidir o que fazer com o elemento que ocupa a posição $1$. Temos duas situações
possíveis
\begin{enumerate}
    \item trocar os elementos das posições $1$ e $j$;
    \item colocar o elemento da posição $j$ na posição 1, mas não colocar o elemento da posição $1$ na posição $j$.
\end{enumerate}

\end{frame}

\begin{frame}[fragile]{Permutações caóticas e programação dinâmica}

Tanto no primeiro quanto no segundo cenário temos $(N - 1)$ valores possíveis para $j$. No primeiro
caso, após a troca de ambos, restam $N - 2$ elementos a serem posicionados, o que pode ser 
feito de $D(N - 2)$ maneiras; no segundo caso, como $1$ não pode ocupar a posição $j$, ficamos
na situação de posicionar $N - 1$ elementos, pois o $1$ fará o papel do $j$, o que pode ser
feito de $D(N - 1)$ maneiras. Assim
\[
        D(N) = (N - 1)(D(N - 1) + D(N - 2))
\]

Os primeiros valores para esta sequência são
\[
1, 0, 1, 2, 9, 44, 265, 1854, 14833, 133496, 1334961, 14684570, 176214841, 2290792932, \ldots
\]

Uma notação alternativa para $D(N)$ é $!n$, que faz referência as permutações (que seriam $n!$).

\end{frame}

\begin{frame}[fragile]{Número de funções}

Sejam $A$ e $B$ dois conjuntos com $n$ e $m$ elementos, respectivamente. {\it Quantas são as funções $f: A \to B$?}

Esta é uma pergunta relativamente fácil de responder. Seja 
\[
\mathcal{F} = \{\ f\ |\  f: A \to B\ \}
\]

Cada um dos elementos de $A$ devem estar
associados a um único elemento de $B$. Assim temos, para cada um dos $n$ elementos, $m$ escolhas
possíveis. Pelo Princípio Multiplicativo temos que
\[
        |\mathcal{F}| = m^n
\]

\end{frame}


\begin{frame}[fragile]{Número de funções bijetoras}

A pergunta se torna mais interessante, porém, se adicionarmos algumas restrições. A primeira
variante seria: {\it quantas são as funções bijetoras de $A$ em $B$?}

Para que existam funções bijetoras de $A$ em $B$, é necessário que $n = m$. Neste caso, seja 
\[
\mathcal{B} = \{\ g\ |\ g: A \to B \ \mbox{e}\ g\ \mbox{é bijetora}\ \}
\]
Agora, cada um elemento de $A$ tem que estar associado a um
único elemento de $B$, cada elemento de $B$ deve estar associado a um único elemento de $A$,
e todos os elementos de $B$ devem estar associados a algum elemento de $A$.
Assim, o primeiro elemento de $A$ tem $n$ escolhas, o segundo tem $(n - 1)$, o terceiro $(n - 2)$,
e assim por diante. Portanto,
\[
        |\mathcal{B}| = n \times (n - 1) \times (n - 2) \times ... \times 2 \times 1 = P(n) = n!
\]

\end{frame}

\begin{frame}[fragile]{Número de funções injetoras}

A segunda variante da pergunta é: {\it quantas são as funções injetoras de $A$ em $B$}?

Para que existam funções injetoras de $A$ em $B$, é necessário que $n \leq m$. Neste caso, seja 
\[
\mathcal{I} = \{\ h\ |\ h: A \to B\ \mbox{e}\ h\ \mbox{é injetora}\ \}
\]

 Agora, cada um elemento de $A$ tem que estar associado a um
único elemento de $B$ e cada elemento de $B$ deve estar associado a um único elemento de $A$.
Assim, o primeiro elemento de $A$ tem $m$ escolhas, o segundo tem $(m - 1)$, o terceiro $(m - 2)$,
e assim por diante. Portanto,
\[
        |\mathcal{I}| = m \times (m - 1) \times (m - 2) \times ... \times (m - n + 1) = A(m, n) = \frac{m!}{(m - n)!}
\]

\end{frame}

\begin{frame}[fragile]{Número de funções sobrejetoras}

Por fim, a mais interessante variação da pergunta: {\it quantas são as funções sobrejetoras de $A$ em $B$?}

Precisamos, neste cenário, que $n \geq m$. Seja
\[
    \mathcal{S} = \{\ f: A \to B\ |\ f\ \mbox{é sobrejetora}\ \}
\]
e defina 
\[
\mathcal{C}_k = \{\ f: A \to B\ |\ f(a_i) \neq b_k, \forall a_i\in A\ \}
\]

Veja que uma função deixa de ser sobrejetora se pertencer a pelo menos um
dos conjuntos $C_k$. Pelo Princípio da Inclusão/Exclusão temos que
\[
        |\mathcal{S}| = |\mathcal{F} | - \sum_{i} | \mathcal{C}_i | + \sum_{i<j}| \mathcal{C}_i \cap \mathcal{C}_j | - \sum_{i<j<k}| \mathcal{C}_i \cap
            \mathcal{C}_j \cap \mathcal{C}_k | + \ldots + (-1)^N | \mathcal{C}_1 \cap \mathcal{C}_2 \cap \ldots \cap \mathcal{C}_N |
\]

\end{frame}

\begin{frame}[fragile]{Número de funções sobrejetoras}

Observe que $|\mathcal{C}_i| = (m - 1)^n$, $|\mathcal{C}_i \cap \mathcal{C}_j| = (m - 2)^n$, e assim por diante, de modo que
\[
        | \mathcal{S} | = m^n - \binom{m}{1} (m - 1)^n + \binom{m}{2} (m - 2)^n 
              + ... + (-1)^m \binom{m}{m} (m - m)^n
\]
o que nos dá
\[
        | \mathcal{S} | = \sum_{i} (-1)^i \binom{m}{i} (m - i)^n,
\]
para $i$ variando de 0 a $m$.

Este valor coincide com o número de maneiras de se distribuir $n$ bolas distintas em $m$ 
caixas distintas, sem que nenhuma caixa fique vazia.

Deste mesmo resultado pode-se deduzir este outro importante resultado: há $T = |\mathcal{S} | / m!$ maneiras
de se distribuir $n$ bolas distintas em $m$ caixas iguais, sem que nenhuma caixa fique vazia.


\end{frame}

