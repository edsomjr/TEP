\section{Motivação}

\begin{frame}[fragile]{\textit{Range sum query}}

    \begin{itemize}
        \item O problema conhecido como \textit{range sum query} ($RSQ(i, j)$) consiste em 
            determinar a 
            soma de todos os elementos de uma sequência $a_k$ cujos índices pertencem ao
            intervalo $[i, j]$

        \item Por exemplo, se $a_k = \lbrace 1, 2, 3, 4, 5\rbrace$, então $RSQ(2, 4) = 9,
            RSQ(3, 3) = 3$ e $RSQ(1, 5) = 15$

        \item Cada \textit{query} pode ser respondida em $O(N)$, onde $N$ é o número de elementos
            da sequência $a_k$, iterando em todos os elementos $a_i, a_{i + 1}, \ldots, a_j$

        \item Se, para um mesmo vetor $a_k$, forem realizadas $Q$ \textit{queries}, a complexidade
            do algoritmo seria $O(NQ)$

        \item Contudo, esta complexidade pode ser reduzida para $O(N + Q)$, se o sequência 
            for preprocessada, de modo que cada \textit{query} pode ser respondida em $O(1)$

    \end{itemize}

\end{frame}

\begin{frame}[fragile]{Implementação do $RSQ$ com complexidade $O(N)$ por \textit{query}}
    \inputcode{cpp}{rsq.cpp}
\end{frame}

\begin{frame}[fragile]{Soma dos prefixos de uma sequência}

    \begin{itemize}
        \item Seja $p_i$ a soma dos elementos do prefixo $a[1..i]$ da sequência $a_k$, isto é,
        \[
            p_i = \sum_{k = 1}^i a_k
        \]

        \item Por exemplo, para $a_k = \lbrace 1, 2, 3, 4, 5\rbrace$, segue que
            $p_k = \lbrace 1, 3, 6, 10, 15\rbrace$

        \item A sequência $p_k$, com $p_0 = 0$, pode ser utilizada para determinar $RSQ(i, j)$
            em $O(1)$

        \item De fato,
        \[
            RSQ(i, j) = p_j - p_{i - 1}
        \]

        \item Como a sequência $p_k$ é gerada com complexidade $O(N)$, a resposta para $Q$ tem
            complexidade $O(N + Q)$
    \end{itemize}

\end{frame}

\begin{frame}[fragile]{Implementação da soma dos prefixos}
    \inputcode{cpp}{prefix_sum.cpp}
\end{frame}

\begin{frame}[fragile]{\textit{Range sum queries} com sequência dinâmica}

    \begin{itemize}
        \item Embora a soma dos prefixos resolva o problema quando a sequência $a_k$ é 
            estática, ela não se aplica em sequências dinâmicas

        \item Se a sequência modificar um ou mais de seus elementos entre duas \textit{queries},
            a sequência $p_k$ também fica modifica, e esta atualização tem complexidade $O(N)$,
            voltando à complexidade da solução original

        \item Para responder \textit{range sum queries} em sequências dinâmicas com melhor
            complexidade
            é preciso o uso de estruturas mais sofisticadas, que permitam a atualização das
            somas pré-computadas de forma eficiente

        \item Uma destas estruturas é a \textit{Binary Indexed Tree} (\textit{BITree} ou 
            \textit{Fenwick Tree}), proposta pelo professor Peter M. Fenwick em 1994
    \end{itemize}

\end{frame}
