\section{{\it Hash} universal}

\begin{frame}[fragile]{{\it Hash} universal}

    \begin{itemize}
        \item Qualquer que seja a função de  \textit{hash} $h$, é possível construir uma
            sequência de chaves $K_1, K_2, \ldots, K_N$ tais que $h(K_i) = h(K_j)$

        \item Esta sequência levaria ao pior caso da inserção e da busca, com complexidade
            $O(T)$

        \item A ideia do \textit{hash} universal é a mesma do \textit{quicksort}: escolher,
            no início da execução do algoritmo, uma função de \textit{hash} dentre uma 
            família de \textit{hashes} possíveis

        \item Deste modo, diferentes execuções do algoritmo levariam a resultados diferentes,
            mesmo para uma entrada fixa

        \item Assim, uma única sequência não seria capaz de levar ao pior caso em todas as
            execuções, melhorando a performance no caso médio
             
    \end{itemize}

\end{frame}

\begin{frame}[fragile]{Conjunto universal}

    \begin{itemize}
        \item Seja $\mathcal{H}$ um conjunto de funções de \textit{hash} que mapeiam as chaves
            no intervalo $[0, T -1]$

        \item O conjunto $\mathcal{H}$ é dito universal de \textit{hashes} se, para todos os pares 
            de chaves distintas $K$ e $L$, o subconjunto $S_{KL}\subset \mathcal{H}$ tal que
            \[
                S_{KL} = \lbrace f, g\in\mathcal{H}\ |\ f\neq g \ \ \mbox{e}\ \ f(K) = g(L)\rbrace
            \]
            tem tamanho $|S_{KL}| \leq |\mathcal{H}|/T$

        \item Deste modo, escolhida aleatoriamente uma função $h\in\mathcal{H}$, a probabilidade 
            existam uma colisão entre duas chaves $K_1\neq K_2$ é igual a
            \[
                P(h(K_1) = h(K_2)) = \frac{|S_{K_1K_2}|}{|\mathcal{H}|} \leq \frac{|\mathcal{H}|/T}{|\mathcal{H}|} = \frac{1}{T}
            \]

    \end{itemize}

\end{frame}

\begin{frame}[fragile]{Exemplo de conjunto universal}

    \begin{itemize}
        \item Seja $p$ um número primo tal que o valor de qualquer chave $K$ seja menor do que $p$

        \item Seja $\mathbb{Z}_p = \lbrace 0, 1, 2, \ldots, p - 1\rbrace$ e $\mathbb{Z}_p^* = \lbrace 1, 2, \ldots, p - 1\rbrace$ 

        \item Defina
        \[
            h_{ab}(K) = \Mod{(\Mod{aK + b}{p})}{T},
        \]

        \item É possível demonstrar que
        \[
            \mathcal{H}_{pT} = \lbrace h_{ab}\ | \ a\in \mathbb{Z}_p^*\ \ \mbox{e}\ \  b \in \mathbb{Z}_p \rbrace
        \]
        é um conjunto universal de \textit{hashes} com $|\mathcal{H}_{pT}| = p(p - 1)$ elementos

        \item Observe que não há restrições impostas ao tamanho da tabela $T$

    \end{itemize}

\end{frame}

\begin{frame}[fragile]{Exemplo de uso do conjunto universal}

    Chaves a serem inseridas: $33, 17, 95, 27, 88, 15, 54, 62, 40$

    Tamanho da tabela: $T = 20$, $p = 101$

    \begin{table}[H]
        \centering

        \begin{tabular}{ccccccc}
            \toprule
            $K$ & $h_{1,0}(K)$ & $h_{2,1}(K)$ & $h_{44,37}(K)$ & $h_{51,97}(K)$ & $h_{5,11}(K)$ \\
            \midrule
            $33$ & $13$ & $7$ & $15$ & $\textcolor{red}{\mathbf{3}}$ & $15$ \\
            $17$ & $17$ & $\textcolor{red}{\mathbf{15}}$ & $\textcolor{red}{\mathbf{18}}$ & $15$& $16$  \\
            $95$ & $\textcolor{red}{\mathbf{15}}$ & $10$ & $16$ & $\textcolor{blue}{\mathbf{14}}$ & $2$ \\
            $27$ & $7$ & $\textcolor{red}{\mathbf{15}}$ & $13$ & $\textcolor{green!60!black}{\mathbf{0}}$& $5$  \\
            $88$ & $8$ & $16$ & $\textcolor{blue}{\mathbf{11}}$ & $\textcolor{green!60!black}{\mathbf{0}}$ & $7$ \\
            $15$ & $\textcolor{red}{\mathbf{15}}$ & $11$ & $\textcolor{blue}{\mathbf{11}}$ & $\textcolor{blue}{\mathbf{14}}$ & $6$ \\
            $54$ & $14$ & $8$ & $10$ & $\textcolor{red}{\mathbf{3}}$ & $19$ \\
            $62$ & $2$ & $4$ & $\textcolor{red}{\mathbf{18}}$ & $7$ & $18$ \\
            $40$ & $0$ & $1$ & $0$ & $16$ & $9$ \\
            \bottomrule
        \end{tabular}
    \end{table}

\end{frame}


\begin{frame}[fragile]{Implementação de conjunto universal de \textit{hashes}}
    \inputsnippet{cpp}{1}{18}{codes/universal.cpp}
\end{frame}

\begin{frame}[fragile]{Implementação de conjunto universal de \textit{hashes}}
    \inputsnippet{cpp}{20}{39}{codes/universal.cpp}
\end{frame}
