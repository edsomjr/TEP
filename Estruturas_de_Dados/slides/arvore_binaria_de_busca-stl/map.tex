\section{\texttt{map}}

\begin{frame}[fragile]{\texttt{map}}

    \begin{itemize}
        \item \texttt{map} é um tipo abstrato de dados da STL do C++ que abstrai o conceito de
            dicionário

        \item Cada elemento do \texttt{map} é composto de uma chave (\textit{key}) e um valor
            associado (\textit{value})

        \item Tanto o tipo da chave quanto do valor são paramétricos, e podem ser distintos

        \item Os elementos são ordenados por meio de suas chaves

        \item As chaves são únicas

        \item A inserção de um par (\textit{key}, \textit{value}) para uma chave já inserida 
            modifica o valor da chave existente

        \item As operações de inserção, remoção e busca são eficientes, com complexidade
            $O(\log N)$, onde $N$ é o número de elementos inseridos no \texttt{map}
    \end{itemize}

\end{frame}

\begin{frame}[fragile]{Operações um \texttt{map}}

    \begin{itemize}
        \item Embora sejam ADTs distintos, as interfaces do \texttt{map} e do \texttt{set}
            contém inúmeras interseções

        \item De fato, todos os métodos apresentados anteriormente para o \texttt{set} estão
            também disponíveis para o \texttt{map}

        \item A principal diferença reside no fato de que os iteradores do \texttt{map} são
            pares 

        \item O primeiro elemento de um iterador é a chave e o segundo elemento é o valor

        \item Além do \texttt{map}, a STL também oferece o \texttt{multimap}, o qual suporta
            chaves repetidas
            
    \end{itemize}

\end{frame}

\begin{frame}[fragile]{Exemplo de uso de \texttt{map} e \texttt{multimap}}
    \inputsnippet{cpp}{1}{20}{codes/map.cpp}
\end{frame}

\begin{frame}[fragile]{Exemplo de uso de \texttt{map} e \texttt{multimap}}
    \inputsnippet{cpp}{21}{41}{codes/map.cpp}
\end{frame}

\begin{frame}[fragile]{Exemplo de uso de \texttt{map} e \texttt{multimap}}
    \inputsnippet{cpp}{42}{62}{codes/map.cpp}
\end{frame}
