\section{\texttt{Set}}

\begin{frame}[fragile]{\texttt{Set}}

    \begin{itemize}
        \item O conjunto (\code{c}{set}) é um tipo de dado abstrato que representa um conjunto
            de elementos únicos

        \item Estes elementos são mantidos em ordem crescente, de acordo com a implementação
            do operator \code{c}{<} do tipo de elemento a ser armazenado

        \item O tipo de dado a ser armazenado é paramétrico e deve ser definido na instanciação
            do conjunto

        \item A principal característica dos conjuntos é a eficiência nas operações de inserção,
            remoção e busca

        \item Todas as três tem complexidade $O(\log N)$, onde $N$ é o número de elementos no conjunto
    \end{itemize}

\end{frame}

\begin{frame}[fragile]{Construção de um \texttt{set}}

    \begin{itemize}
        \item O padrão C++11 oferece cinco construtores distintos para um \code{c}{set}

        \item O primeiro deles, denominado \textit{default constructor},
                 não tem parâmetros e constrói um conjunto vazio

        \item O segundo, \textit{range constructor}, permite a construção de um conjunto a partir 
            de dois iteradores,
            \code{c}{first} e \code{c}{last}, que determinam um intervalo de valores a 
            serem inseridos, do primeiro ao penúltimo

        \item Este construtor também permite a definição de um alocador de memória customizado

        \item O terceiro, \textit{copy constructor}, cria uma cópia exata do \code{c}{set}
            passado como parâmetro

        \item O quarto, \textit{move constructor}, move o conteúdo do \code{c}{set} passado
            como parâmetro para o novo conjunto

        \item O quinto, \textit{initializer-list constructor}, cria um novo \code{c}{set} com
            os elementos passados na lista de inicialização
    \end{itemize}

\end{frame}

\begin{frame}[fragile]{Exemplo de uso dos construtores do \texttt{set}}

    \inputcode{c++}{codes/set_constructor.cpp}

\end{frame}

\begin{frame}[fragile]{Principais operações}

    \begin{itemize}
        \item As principais operações em um conjunto são a inserção, remoção e busca,
        todas com complexidade $O(\log N)$, onde $N$ é o número de elementos armazenados
        no conjunto

        \item A inserção é feita através do método \code{c}{insert()}, que pode receber ou o
            valor a ser inserido ou uma lista de inicialização com os elementos a serem
            inseridos

        \item Outro método de inserção é o \code{c}{emplace()}, que recebe como parâmetros os mesmos
            parâmetros do construtor do elemento a ser inserido e constrói o elemento durante a inserção

        \item A inserção de um valor que já existe no conjunto não tem efeito

        \item O método \code{c}{erase()} remove o nó que contém o valor passado como parâmetro,
            se existir tal valor no conjunto
    \end{itemize}

\end{frame}

\begin{frame}[fragile]{Principais operações}

    \begin{itemize}
        \item O retorno do método pode ser utilizado para se determinar quantos elementos foram
            removidos

        \item Para se determinar se um elemento está ou não no conjunto há duas alternativas

        \item A primeira é utilizar o método \code{c}{count()}, cujo retorno significa o número
            de ocorrências do valor passado como parâmetro

        \item A segunda é utilizar o método \code{c}{find()}: ele retorna o iterador para o
            elemento que contém o valor, ou o iterador \code{cpp}{end()}, caso contrário
    \end{itemize}

\end{frame}

\begin{frame}[fragile]{Exemplo de uso das principais operações do \texttt{set}}
    \inputcode{cpp}{codes/set_operations.cpp}
\end{frame}

\begin{frame}[fragile]{Operações relevantes}

    \begin{itemize}
        \item O método \code{c}{empty()} verifica se o conjunto está ou não vazio

        \item O método \code{c}{size()} determina o número de elementos armazenado no conjunto

        \item Tanto \code{c}{empty()} quando \code{c}{size()} tem complexidade constante

        \item O método \code{c}{lower_bound()} retorna um iterator para o primeiro elemento do
            conjunto cuja informação é maior ou igual ao valor passado como parâmetro

        \item O método \code{c}{upper_bound()} tem comportamento semelhante, retornando um iterador
            para o elemento cuja informação é estritamente maior do que valor passado como parâmetro

        \item Ambos métodos tem complexidade $O(\log N)$
    \end{itemize}

\end{frame}

\begin{frame}[fragile]{Exemplo de uso de operações relevantes no \code{c}{set}}
    \inputcode{cpp}{codes/set_ops.cpp}
\end{frame}

\begin{frame}[fragile]{\texttt{multiset}}

    \begin{itemize}
        \item A STL também oferece a implementação de um conjunto que permite a inserção de 
            elementos repetidos, denominado \code{c}{multiset}

        \item O retorno do método \code{c}{count()} corresponde ao número de ocorrências de um 
            mesmo valor

        \item O método \code{c}{erase()} deve ser usado com cuidado: ele apaga todas as ocorrências
            do valor passado como parâmetro

        \item Para remover somente uma ocorrência, esta ocorrência deve ser localizada com o método
            \code{c}{find()} e o iterador de retorno deve ser passado como parâmetro para o
            método \code{c}{erase()}

        \item Uma travessia usando \textit{range for} passa uma vez em cada ocorrência de cada
            elemento

        \item O método \code{c}{equal_range()} retorna um par de iteradores que delimitam o
            intervalo de valores idênticos ao valor passado como parâmetro
    \end{itemize}

\end{frame}

\begin{frame}[fragile]{Exemplo de uso de \texttt{multiset}}
    \inputcode{cpp}{codes/multiset.cpp}
\end{frame}
