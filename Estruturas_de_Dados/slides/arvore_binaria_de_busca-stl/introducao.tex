\section{Introdução}

\begin{frame}[fragile]{Árvores Binárias de Busca na STL}

    \begin{itemize}
        \item A STL (\textit{Standard Template Library}) da linguagem C++ não oferece uma
            implementação básica de árvores binárias de busca que permita o acesso direto
            aos nós e seus ponteiros

        \item Entretanto, ela oferece tipos de dados abstratos cujas implementações utilizam
            árvores binárias de busca auto-balanceáveis

        \item O padrão da linguagem não especifica qual árvore deve ser utilizada na implementação,
            e sim as complexidades assintóticas esperadas para cada operação

        \item Segundo o site CppReference\footnote{https://en.cppreference.com/w/}, em geral 
            são utilizadas árvores \textit{red-black} 

        \item Os principais tipos de dados abstratos implementados são os conjuntos 
            (\textit{sets}) e os dicionários (\textit{maps})
    \end{itemize}

\end{frame}
