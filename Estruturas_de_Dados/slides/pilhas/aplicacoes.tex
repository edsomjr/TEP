\section{Aplicações de Pilhas}

\begin{frame}[fragile]{Identificação de delimitadores}

    \begin{itemize}
        \item Delimitadores são caracteres de marcação que delimitam um conjunto de informações,
            e devem ter um símbolo (ou conjunto de símbolos) que determine o início e o fim do
            conjunto

        \item Em C++, os caracteres \code{c}{(), [], {}} e os pares de caracteres \code{c}{/*, */} são delimitadores

        \item As pilhas podem ser utilizadas para verificar se, em uma determinada expressão,
            os delimitadores foram abertos e fechados corretamente

        \item Por exemplo, as expressões \code{c}{(()), [()()], ()[]} são válidas

        \item Já as expressões \code{c}{[), )(, ()], [[][]} são inválidas

        \item O algoritmo é simples: cada símbolo que abre um conjunto é colocado no topo da
            pilha

        \item A cada símbolo que fecha o topo da pilha é observado: se contiver o símbolo que
            abre correspondente, ele é removido e o algoritmo continua; caso contrário, a 
            expressão é inválida

        \item Ao final do algoritmo a expressão será válida se a pilha estiver vazia

    \end{itemize}

\end{frame}

\begin{frame}[fragile]{Exemplo de identificação de delimitadores}
    \inputsnippet{c++}{1}{19}{codes/delimitadores.cpp}
\end{frame}

\begin{frame}[fragile]{Exemplo de identificação de delimitadores}
    \inputsnippet{c++}{21}{41}{codes/delimitadores.cpp}
\end{frame}

\begin{frame}[fragile]{Soma de grandes números}

    \begin{itemize}
        \item As pilhas também podem ser utilizadas para somar números com um grande número de
            dígitos

        \item Os tipos primitivos integrais do C/C++ tem restrições de tamanho (em \textit{bytes})
            e podem levar a erros de \textit{overflow} caso o resultado seja demasiadamente 
            grande

        \item Com pilhas é possível somar números de qualquer magnitude

        \item A ideia é armazenar os números como pilhas de dígitos, de modo que as unidades 
            fiquem nos topos das pilhas

        \item Daí é só colocar os resultados das somas dos topos em uma terceira pilha, tomando
            cuidado com o vai um (\textit{carry}), quando for o caso
    \end{itemize}

\end{frame}

\begin{frame}[fragile]{Exemplo de adição de grandes números}
    \inputsnippet{c++}{1}{20}{codes/bigint.cpp}
\end{frame}

\begin{frame}[fragile]{Exemplo de adição de grandes números}
    \inputsnippet{c++}{22}{41}{codes/bigint.cpp}
\end{frame}

\begin{frame}[fragile]{Exemplo de adição de grandes números}
    \inputsnippet{c++}{43}{64}{codes/bigint.cpp}
\end{frame}
