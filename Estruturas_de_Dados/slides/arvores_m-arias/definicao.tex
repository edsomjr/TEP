\section{Árvores-B}

\begin{frame}[fragile]{Árvores múltiplas}

	\begin{itemize}
		\item Segundo a definição formal de árvores, não há restrição quanto ao número de filhos 
            que um nó pode ter 

		\item Uma árvore múltipla de ordem $m$ é um árvore cujos nós possuem, no máximo, $m$ filhos

		\item As árvores binárias de busca que são árvores múltiplas de ordem 2 que impõem 
        condições sobre as chaves dos nós com o intuito de agilizar o processo de busca.

		\item  As árvores binárias de busca podem ser generalizadas como árvores de busca de ordem 
            $m$ 

	\end{itemize}

\end{frame}

\begin{frame}[fragile]{Árvores de busca de ordem $m$}

    \metroset{block=fill}
	\begin{block}{Definição}
		Uma árvore de busca de ordem $m$ é uma árvore que satisfaz as seguintes condições:

		\begin{enumerate}
			\item Cada nó tem, no máximo, $m$ filhos e $m - 1$ chaves.
			\item As chaves de cada nó são armazenadas em ordem crescente.
			\item As chaves dos $i$ primeiros filhos são menores do que a chave $i$.
			\item As chaves dos $m - i$ últimos filhos são maiores do que a chave $i$.
		\end{enumerate}
	\end{block}

\end{frame}

\begin{frame}[fragile]{Exemplo de árvore de busca de ordem 4}

	\begin{figure}
        \centering

        \begin{tikzpicture}
            \draw (4, 7) grid (7, 8);
            \draw (0, 5) grid (2, 6);
            \draw (3, 5) grid (5, 6);
            \draw (6, 5) grid (9, 6);
            \draw (10, 5) grid (11, 6);
            \draw (2, 3) grid (4, 4);
            \draw (5, 3) grid (7, 4);
            \draw (4, 1) grid (5, 2);

            \draw (4, 7) -- (0.5, 6);
            \draw (5, 7) -- (3.5, 6);
            \draw (6, 7) -- (6.5, 6);
            \draw (7, 7) -- (10.5, 6);
            \draw (3, 5) -- (2.5, 4);
            \draw (6, 5) -- (5.5, 4);
            \draw (4, 3) -- (4.5, 2);

            \node at (4.5, 7.5) { 50 };
            \node at (5.5, 7.5) { 60 };
            \node at (6.5, 7.5) { 80 };

            \node at (0.5, 5.5) { 30 };
            \node at (1.5, 5.5) { 35 };

            \node at (3.5, 5.5) { 58 };
            \node at (4.5, 5.5) { 59 };

            \node at (6.5, 5.5) { 63 };
            \node at (7.5, 5.5) { 70 };
            \node at (8.5, 5.5) { 73 };

            \node at (10.5, 5.5) { 99 };

            \node at (2.5, 3.5) { 52 };
            \node at (3.5, 3.5) { 54 };

            \node at (5.5, 3.5) { 61 };
            \node at (6.5, 3.5) { 62 };

            \node at (4.5, 1.5) { 57 };
        \end{tikzpicture}
	\end{figure}

\end{frame}
 
\begin{frame}[fragile]{Notas sobre árvores de busca de ordem $m$}

	\begin{itemize}
		\item As árvores de busca de ordem $m$ tem o mesmo objetivo das árvores de busca binárias: 
            aumentar a eficiência da rotina de busca

        \item Observe que, em cada nó, é preciso localizar, a partir da informação a ser encontrada
            e das chaves armazenadas, identificar o filho que dará sequência a busca

        \item A ordenação das chaves permite esta identificação em ordem $O(\log m)$, desde que
            o contêiner que armazena as chaves permita a busca binária

		\item Assim como as árvores binárias de busca, as árvores de busca de ordem $m$ 
            também podem ter problemas relativos ao balanceamento

        \item Para evitar tal problemas, existem especilizações destas árvores, como as árvores-B
	\end{itemize}

\end{frame}

\begin{frame}[fragile]{Definição de Árvores-B}

    \metroset{block=fill}
	\begin{block}{Definição}
		Uma árvore-B de ordem $m$ é uma árvore de busca de ordem $m$ com as seguintes propriedades:

		\begin{enumerate}
			\item A raiz tem, no mínimo, dois filhos, caso não seja uma folha.
			\item Cada nó que não é nem folha nem raiz tem $k-1$ chaves e $k$ ponteiros para 
                subárvores, onde $\lceil m/2\rceil \leq k \leq m$.
			\item Cada folha tem $k-1$ chaves, onde $\lceil m/2\rceil \leq k \leq m$.
			\item Todas as folhas estão no mesmo nível.
		\end{enumerate}
	\end{block}

\end{frame} 

\begin{frame}[fragile]{Exemplo de árvore-B de ordem 5}

	\begin{figure}
        \centering

        \begin{tikzpicture}[scale=1.12]
            \draw[step=0.3] (0, 0) grid (1.2, 0.3);
            \draw[step=0.3] (1.5, 0) grid (2.7, 0.3);
            \draw[step=0.3] (3.0, 0) grid (4.2, 0.3);
            \draw[step=0.3] (4.5, 0) grid (5.7, 0.3);
            \draw[step=0.3] (6.0, 0) grid (7.2, 0.3);
            \draw[step=0.3] (7.5, 0) grid (8.7, 0.3);
            \draw[step=0.3] (9.0, 0) grid (10.2, 0.3);
            \draw[step=0.3] (1.5, 3) grid (2.7, 3.3);
            \draw[step=0.3] (4.5, 6) grid (5.7, 6.3);
            \draw[step=0.3] (7.5, 3) grid (8.7, 3.3);
 
            \node at (0.15, 0.15) { \tiny 6 };
            \node at (0.45, 0.15) { \tiny 8 };
            \node at (1.65, 0.15) { \tiny 11 };
            \node at (1.95, 0.15) { \tiny 12 };
            \node at (3.15, 0.15) { \tiny 16 };
            \node at (3.45, 0.15) { \tiny 18 };
            \node at (4.65, 0.15) { \tiny 21 };
            \node at (4.95, 0.15) { \tiny 25 };
            \node at (5.25, 0.15) { \tiny 27 };
            \node at (5.55, 0.15) { \tiny 29 };
            \node at (6.15, 0.15) { \tiny 54 };
            \node at (6.45, 0.15) { \tiny 56 };
            \node at (7.65, 0.15) { \tiny 74 };
            \node at (7.95, 0.15) { \tiny 76 };
            \node at (9.15, 0.15) { \tiny 81 };
            \node at (9.45, 0.15) { \tiny 89 };
            \node at (1.65, 3.15) { \tiny 10 };
            \node at (1.95, 3.15) { \tiny 15 };
            \node at (2.25, 3.15) { \tiny 20 };
            \node at (7.65, 3.15) { \tiny 70 };
            \node at (7.95, 3.15) { \tiny 80 };
            \node at (4.65, 6.15) { \tiny 50 };

            \draw (4.5, 6) -- (1.65, 3.3);
            \draw (4.8, 6) -- (7.65, 3.3);
            \draw (1.5, 3) -- (0.15, 0.3);
            \draw (1.8, 3) -- (1.65, 0.3);
            \draw (2.1, 3) -- (3.15, 0.3);
            \draw (2.4, 3) -- (4.65, 0.3);
            \draw (7.5, 3) -- (6.15, 0.3);
            \draw (7.8, 3) -- (7.65, 0.3);
            \draw (8.1, 3) -- (9.15, 0.3);

        \end{tikzpicture}
	\end{figure}

\end{frame}
 
\begin{frame}[fragile]{Árvores-B}

	\begin{itemize}
		\item As árvores-B foram propostas por Bayer e McCreigh em 1972.

		\item O número de chaves armazenadas em uma árvore-B é proporcional a metade de 
            sua capacidade máxima

		\item Devido às suas propriedades, uma árvore-B tem poucos níveis

		\item Uma árvore-B está sempre perfeitamente balanceada

		\item Um nó de uma árvore-B possui dois contêiners: um para armazenar as $m-1$ chaves e 
            outro para os  $m$ ponteiros para os filhos

		\item Na implementação dos nós de árvores-B costuma-se adicionar informações extras que 
            facilitem a manutenção da árvore, como o número de chaves do nó e uma indicação se o 
            nó é folha ou não
	\end{itemize}

\end{frame}

\begin{frame}[fragile]{Exemplo de implementação de uma árvore-B}
    \inputsnippet{cpp}{1}{21}{btree.cpp}
\end{frame}

\begin{frame}[fragile]{Exemplo de implementação de uma árvore-B}
    \inputsnippet{cpp}{22}{42}{btree.cpp}
\end{frame}

\begin{frame}[fragile]{Exemplo de implementação de uma árvore-B}
    \inputsnippet{cpp}{44}{64}{btree.cpp}
\end{frame}

\begin{frame}[fragile]{Exemplo de implementação de uma árvore-B}
    \inputsnippet{cpp}{65}{72}{btree.cpp}
\end{frame}
