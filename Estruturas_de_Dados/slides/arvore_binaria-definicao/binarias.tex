\section{Árvores Binárias}

\begin{frame}

    \frametitle{Árvores binárias}

    \begin{itemize}
        \item A definição de árvores {não impõem} qualquer restrição no 
        {número de filhos} que um nó pode ter
        

        \item Uma árvore é dita {binária} se cada nó tem, no máximo, dois filhos: o 
            {esquerdo} e o {direito}
        
        \item O estabelecimento de uma ordem entre as informações armazenadas em um nó e seus
            filhos leva a especilizações úteis de uma árvore binária

        \item Por exemplo, uma \textit{max heap} é uma árvore binária cuja informação contida no
            pai é maior ou igual que as informações contidas em seus filhos

        \item Já em uma árvore binária de busca, as informações contidas em qualquer nó da
            subárvore à esquerda de um nó $N$ devem ser menores do que a informação contida em $N$

        \item Do mesmo modo, as informações contidas nos nós da subárvore à direita de $N$ devem ser
            maiores do que a informação armazenada em $N$
    \end{itemize}

\end{frame}

\begin{frame}[fragile]{Exemplo de \textit{max heap}} 

    \begin{tikzpicture}
        \node[opacity=0] at (0, 7) { x };
        \node[opacity=0] at (14, 0) { x };

        \node[circle,draw] (D) at (7, 6) { $80$ };
        \node[circle,draw] (C) at (5, 4) { $63$ };
        \node[circle,draw] (E) at (9, 4) { $47$ };
        \node[circle,draw] (F) at (8, 2) { $25$ };
        \node[circle,draw] (A) at (4, 2) { $1$ };
        \node[circle,draw] (B) at (6, 2) { $23$ };
        \node[circle,draw] (G) at (10, 2) { $9$ };

        \draw (A) -- (C);
        \draw (B) -- (C);
        \draw (C) -- (D);
        \draw (D) -- (E);
        \draw (E) -- (F);
        \draw (E) -- (G);
    \end{tikzpicture}


\end{frame}

\begin{frame}[fragile]{Exemplo de árvore binária de busca}

    \begin{tikzpicture}

        \node[opacity=0] at (0, 7) { x };
        \node[opacity=0] at (14, 0) { x };

        \node[circle,draw] (D) at (7, 6) { $25$ };
        \node[circle,draw] (C) at (5, 4) { $9$ };
        \node[circle,draw] (E) at (9, 4) { $63$ };
        \node[circle,draw] (F) at (8, 2) { $47$ };
        \node[circle,draw] (A) at (4, 2) { $1$ };
        \node[circle,draw] (B) at (6, 2) { $23$ };
        \node[circle,draw] (G) at (10, 2) { $80$ };

        \draw (A) -- (C);
        \draw (B) -- (C);
        \draw (C) -- (D);
        \draw (D) -- (E);
        \draw (E) -- (F);
        \draw (E) -- (G);

    \end{tikzpicture}


\end{frame}
