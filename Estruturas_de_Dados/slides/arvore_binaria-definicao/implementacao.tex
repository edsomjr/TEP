\section{Implementação}

\begin{frame}
    \frametitle{Implementação de árvores binárias}

    \begin{itemize}
        \item Uma árvore binária pode ser {implementada}, no mínimo, de 
        {duas} maneiras: utilizando vetores ou utilizando ponteiros
        
        \item No caso dos vetores, a informação da raiz fica armazenada no índice 1 (o índice
            zero não é utilizado)
            
        \item Dado um nó pai armazenado no índice $p$, o nó à esquerda ocupa o índice $2p$ e o
            nó à direita ocupa o índice $2p + 1$

        \item Se um nó ocupa o índice $i\neq 1$, seu pai está armazenado no índice $i/2$

        \item No caso da implementação via ponteiros, a estrutura que representa um {nó} contém 
            dois ponteiros: um para cada filho 
    \end{itemize}

\end{frame}

\begin{frame}[fragile]{Implementação de uma árvore usando ponteiros}
    \inputcode{cpp}{codes/pointers.cpp}
\end{frame}

\begin{frame}[fragile]{Implementação de uma árvore usando vetores}
    \inputcode{cpp}{codes/vector.cpp}
\end{frame}
