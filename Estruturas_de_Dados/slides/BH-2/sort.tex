\section{Ordenação e heaps na STL}

\begin{frame}[fragile]{Heaps e ordenação}

    \begin{itemize}
        \item A biblioteca \code{c}{algorithm} da linguagem C++ contém três rotinas de 
            ordenação, a saber: \code{c}{sort(), stable_sort()} e \code{c}{partial_sort()}

        \item As \textit{heaps} estão presentes em dois deles

        \item A função \code{c}{sort()} é implementada através de uma estratégia mista: ela
            começa com o \textit{Introsort} (que combina o \textit{Quicksort} com o \textit{Heapsort}) e 
            finaliza com o \textit{Insert sort}

        \item Já o \code{c}{partial_sort()} é implementada por meio do \textit{Heapsort}: é mantida
            uma \textit{max heap} com exatamante $k$ elementos, removendo o $(k + 1)$-ésimo
            elemento sempre que o tamanho da \textit{heap} for maior do que $k$

        \item Em seguida, os elementos são ordenados, retirando os elementos máximo, um por vez,
            e colocando-os nas posições apropriadas
    \end{itemize}

\end{frame}

\begin{frame}[fragile]{Exemplo de uso das funções de ordenação em C++}
    \inputcode{cpp}{sort.cpp}
\end{frame}
