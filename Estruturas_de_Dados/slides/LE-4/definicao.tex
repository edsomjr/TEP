\section{Definição}

\begin{frame}[fragile]{Listas auto-organizáveis}

    \begin{itemize}
        \item Listas encadeadas e duplamente encadeadas não são eficientes ($O(N)$) em relação 
            à busca de elementos

        \item Diferentes organizações dos elementos da lista podem melhorar o tempo de execução
            dos algoritmos de busca

        \item Estas organizações geram as listas auto-organizáveis, que são listas que se 
            alteram dinamicamente a cada busca

        \item Em geral, estas listas não contém repetições de elementos (isto é, não há repetição
            de valores nos campos \texttt{info} dos nós

        \item Para efeitos de análise e implementação, se uma busca não localizar uma dada
            informação, a mesma será inserida na lista ao final da busca

        \item O comportamento dinâmico destas listas faz com que as buscas ganhem desempenho
            através do posicionamento estratégico dos elementos já localizados
    \end{itemize}

\end{frame}

\begin{frame}[fragile]{Organizações possíveis}

    \begin{enumerate}
        \item Move o elemento localizado para o início da lista

        \item Trocar o elemento localizado com o seu antecessor, caso não seja o primeiro 
            elemento da lista

        \item Ordenar a lista de acordo com a frequência de acesso de cada elemento

        \item Ordenar a lista de acordo com um critério pré-estabelecido (ordem alfabética,
            lexicográfica, etc)
    \end{enumerate}

\end{frame}

\begin{frame}[fragile]{Observações sobre as formas de organização}

    \begin{itemize}
        \item As três primeiras formas incluem um novo nó ao final da lista se a informação
            procurada não for localizada

        \item Já a quarta forma adiciona o novo nó de acordo com o critério estabelecido

        \item As três primeiras formas visam fazer com que o elemento seja localizado o mais
            cedo possível

        \item A última forma é útil em buscas de elementos que não estão na lista, uma vez que
            sua organização subjacente pode suspender a busca quando for o caso

        \item Se uma informação não consta na lista, nas três primeiras formas a lista deve ser
            percorrida na íntegra, o que não acontece na quarta forma
    \end{itemize}

\end{frame}

\begin{frame}[fragile]{Exemplo das formas de organização}

    \begin{center}
    \begin{scriptsize}

    \begin{tabular}{p{0.4in}lp{0.5in}lll}
        \toprule
        \textbf{Elemento acessado} & \textbf{Nenhuma forma} & 
            \textbf{Mover para o início} & \textbf{Transposição} & 
            \textbf{Frequência} & \textbf{Ordenação} \\
        \toprule
        A & A & A & A & A & A  \\
        C & AC & AC & AC & AC & AC \\
        B & ACB & ACB & ACB & ACB  & ABC \\
        C  & ACB  & CAB  & CAB  & CAB  & ABC  \\
        D  & ACBD & CABD & CABD & CABD & ABCD \\
        A  & ACBD  & ACBD  & ACBD  & CABD  & ABCD  \\
        D  & ACBD  & DACB  & ACDB  & CADB  & ABCD  \\
        A  & ACBD  & ADCB  & ACDB  & ACDB  & ABCD  \\
        C  & ACBD  & CADB  & CADB  & ACDB  & ABCD  \\
        A  & ACBD  & ACDB  & ACDB  & ACDB  & ABCD  \\
        C  & ACBD  & CADB  & CADB  & ACDB  & ABCD  \\
        C  & ACBD  & CADB  & CADB  & CADB  & ABCD  \\
        E  & ACBDE  & CADBE  & CADBE  & CADBE  & ABCDE  \\
        E  & ACBDE  & ECADB  & CADEB  & CADEB  & ABCDE  \\
        \bottomrule
    \end{tabular}

    \end{scriptsize}
    \end{center}

\end{frame}
