\section{Otimizações e Aplicações}

\begin{frame}[fragile]{Compressão de caminho}

    \begin{itemize}
        \item O método \code{cpp}{find_set(x)} pode ser modificado para que, ao percorrer o
            caminho de $x$ até a raiz, os nós intermediários passem a ter a raiz da árvore 
            como pai

        \item Esta técnica é conhecida como compressão de caminho

        \item A medida que o método for invocado para diferentes valores de $x$, as árvores tendem
            a ter apenas dois níveis

        \item Usada em conjunto com a união por tamanho (ou por ranqueamento), a complexidade 
            amortizada de
            ambas operações (união e identificação de representante) passa a ser $O(\alpha(n))$,
            onde $\alpha(n)$ é a função inversa de Ackermann

        \item Para qualquer $n$ representável no universo físico, $\alpha(n) < 5$, de modo que
            as operações tem, na essência, complexidade constante
    \end{itemize}

\end{frame}

\begin{frame}[fragile]{Visualização da compressão de caminho}

    \begin{figure}
        \centering

        \begin{tikzpicture}
            \node[anchor=west] at (0, 6) { \tt find\_set(6) };

            \node[draw,circle] (A) at (4, 5) { 1 };
            \node[draw,circle] (B) at (2, 3) { 2 };
            \node[draw,circle] (C) at (6, 3) { 3 };
            \node[draw,circle] (D) at (5, 1) { 4 };
            \node[draw,circle] (E) at (8, 1) { 5 };
            \node[draw,circle] (F) at (7, -1) { 6 };

            \draw[->] (B) edge (A);
            \draw[->] (C) edge (A);
            \draw[->] (D) edge (C);
            \draw[->] (E) edge (C);
            \draw[->] (F) edge (E);

        \end{tikzpicture}

    \end{figure}

\end{frame}

\begin{frame}[fragile]{Visualização da compressão de caminho}

    \begin{figure}
        \centering

        \begin{tikzpicture}
            \node[anchor=west] at (0, 6) { \tt find\_set(6) };

            \node[draw,circle] (A) at (4, 5) { 1 };
            \node[draw,circle] (B) at (2, 3) { 2 };
            \node[draw,circle] (C) at (6, 3) { 3 };
            \node[draw,circle] (D) at (5, 1) { 4 };
            \node[draw,circle,fill=red!30] (E) at (8, 1) { 5 };
            \node[draw,circle,fill=green!80!black] (F) at (7, -1) { 6 };

            \draw[->] (B) edge (A);
            \draw[->] (C) edge (A);
            \draw[->] (D) edge (C);
            \draw[->] (E) edge (C);
            \draw[->] (F) edge (E);

        \end{tikzpicture}

    \end{figure}

\end{frame}

\begin{frame}[fragile]{Visualização da compressão de caminho}

    \begin{figure}
        \centering

        \begin{tikzpicture}
            \node[anchor=west] at (0, 6) { \tt find\_set(6) };

            \node[draw,circle] (A) at (4, 5) { 1 };
            \node[draw,circle] (B) at (2, 3) { 2 };
            \node[draw,circle,fill=red!30] (C) at (6, 3) { 3 };
            \node[draw,circle] (D) at (5, 1) { 4 };
            \node[draw,circle,fill=green!80!black] (E) at (8, 1) { 5 };
            \node[draw,circle] (F) at (4, 3) { 6 };
            \node[draw,circle,opacity=0] (G) at (6, -1) { 6 };

            \draw[->] (B) edge (A);
            \draw[->] (C) edge (A);
            \draw[->] (D) edge (C);
            \draw[->] (E) edge (C);
            \draw[->] (F) edge (A);

        \end{tikzpicture}

    \end{figure}

\end{frame}

\begin{frame}[fragile]{Visualização da compressão de caminho}

    \begin{figure}
        \centering

        \begin{tikzpicture}
            \node[anchor=west] at (0, 6) { \tt find\_set(6) };

            \node[draw,circle,fill=blue!30] (A) at (4, 5) { 1 };
            \node[draw,circle] (B) at (2, 3) { 2 };
            \node[draw,circle,fill=green!80!black] (C) at (6, 3) { 3 };
            \node[draw,circle] (D) at (5, 1) { 4 };
            \node[draw,circle] (E) at (8, 3) { 5 };
            \node[draw,circle] (F) at (4, 3) { 6 };
            \node[draw,circle,opacity=0] (G) at (6, -1) { 6 };

            \draw[->] (B) edge (A);
            \draw[->] (C) edge (A);
            \draw[->] (D) edge (C);
            \draw[->] (E) edge (A);
            \draw[->] (F) edge (A);

        \end{tikzpicture}

    \end{figure}

\end{frame}

\begin{frame}[fragile]{Visualização da compressão de caminho}

    \begin{figure}
        \centering

        \begin{tikzpicture}
            \node[anchor=west] at (0, 6) { \tt find\_set(6) = \textcolor{blue}{1}};

            \node[draw,circle] (A) at (4, 5) { 1 };
            \node[draw,circle] (B) at (2, 3) { 2 };
            \node[draw,circle] (C) at (6, 3) { 3 };
            \node[draw,circle] (D) at (5, 1) { 4 };
            \node[draw,circle] (E) at (8, 3) { 5 };
            \node[draw,circle] (F) at (4, 3) { 6 };
            \node[draw,circle,opacity=0] (G) at (6, -1) { 6 };

            \draw[->] (B) edge (A);
            \draw[->] (C) edge (A);
            \draw[->] (D) edge (C);
            \draw[->] (E) edge (A);
            \draw[->] (F) edge (A);

        \end{tikzpicture}

    \end{figure}

\end{frame}


\begin{frame}[fragile]{Implementação da compressão de caminho}
    \inputsnippet{cpp}{1}{20}{codes/compression.h}
\end{frame}

\begin{frame}[fragile]{Aplicações}

    \begin{itemize}
        \item A UFDS pode ser aplicada em vários contextos

        \item Uma aplicação comum é a implementação do algoritmo de Kruskall para determinar a
            árvore mínima geradora (MST) de um grafo

        \item De modo geral, ela pode ser usada para identificar os componentes conectados de
            um grafo não-direcionado

        \item Por fim, seja $R$ uma relação de equivalência

        \item Se $(a, b)\in R$ (isto é, $a$ está relacionado a $b$), então 
            \code{cpp}{same_set(a, b)} retorna verdadeiro
    \end{itemize}

\end{frame}
