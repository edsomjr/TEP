\section{Live Archive -- Problem 6139: Interval Product}

\begin{frame}[fragile]{Problema}

It’s normal to feel worried and tense the day before a programming contest. To relax, you went out 
for a drink with some friends in a nearby pub. To keep your mind sharp for the next day, you 
decided to play the following game. To start, your friends will give you a sequence of $N$ 
integers $X_1, X_2, \ldots, X_N$.  Then, there will be $K$ rounds; at each round, your friends 
will issue a command, which can be:

\begin{itemize}
    \item a change command, when your friends want to change one of the values in the sequence; or
    \item a product command, when your friends give you two values $I$, $J$ and ask you if the 
        product $X_I \times X_{I+1} \times \ldots \times X_{J-1} \times X_J$ is positive, negative 
        or zero.
\end{itemize}

\end{frame}


\begin{frame}[fragile]{Problema}

Since you are at a pub, it was decided that the penalty for a wrong answer is to drink a pint 
of beer. You are worried this could affect you negatively at the next day’s contest, and you don’t
want to check if Ballmer’s peak theory is correct. Fortunately, your friends gave you the right to
use your notebook. Since you trust more your coding skills than your math, you decided to write a 
program to help you in the game.

\end{frame}

\begin{frame}[fragile]{Entrada e saída}

\textbf{Input}

Each test case is described using several lines. The first line contains two integers $N$ and 
$K$, indicating respectively the number of elements in the sequence and the number of rounds of 
the game ($1\leq N, K\leq 10^5$). The second line contains $N$ integers $X_i$ that represent the 
initial values of the sequence ($-100\leq X_i\leq 100\ \mbox{for}\ i = 1, 2, \ldots, N$). Each of the next 
$K$ lines describes a command and starts with an uppercase letter that is either `\texttt{C}' or 
`\texttt{P}'. If the letter is `\texttt{C}', the line describes a change command, and
the letter is followed by two integers $I$ and $V$ indicating that $X_I$ must receive the value 
$V (1\leq I\leq N and -100\leq V\leq 100)$. If the letter is `\texttt{P}', the line describes a 
product command, and the letter is followed by two integers $I$ and $J$ indicating that the 
product from $X_I$ to $X_J$, inclusive must be calculated $(1\leq I\leq J\leq N)$. Within each 
test case there is at least one product command.

\end{frame}

\begin{frame}[fragile]{Entrada e saída}

\textbf{Output}

For each test case output a line with a string representing the result of all the product commands 
in the test case. The $i$-th character of the string represents the result of the $i$-th product 
command. If the result of the command is positive the character must be `\texttt{+}' (plus); if 
the result is negative the character must be `\texttt{-}' (minus); if the result is zero the 
character must be `\texttt{0}' (zero).

\end{frame}


\begin{frame}[fragile]{Exemplo de entradas e saídas}

\begin{footnotesize}
\begin{minipage}[t]{0.6\textwidth}
\textbf{Sample Input}
\begin{verbatim}
4 6
-2 6 0 -1
C 1 10
P 1 4
C 3 7
P 2 2
C 4 -5
P 1 4
5 9
1 5 -2 4 3
P 1 2
P 1 5
C 4 -5
P 1 5
P 4 5
C 3 0
P 1 5
C 4 -5
C 4 -5
\end{verbatim}
\end{minipage}
\begin{minipage}[t]{0.35\textwidth}
\textbf{Sample Output}
\begin{verbatim}
0+-
+-+-0
\end{verbatim}
\end{minipage}
\end{footnotesize}

\end{frame}

\begin{frame}[fragile]{Solução com complexidade $O(K\log N + K)$}

    \begin{itemize}
        \item Embora fique claro a necessidade do uso de uma árvore de segmentos (ou uma árvore
            de Fenwick), alguns cuidados são necessários

       \item Em primeiro lugar, se os números forem armazenados de acordo com os valores dados
            na entrada, ocorrerá o erro de \textit{overflow}

        \item Além disso, o zero não é inversível na operação de multiplicação

        \item Assim, os zeros devem ser armazenados e tratados à parte

        \item Uma saída é manter duas árvores, que mantenham o registro dos números negativos e 
            dos zeros

        \item Assim, em uma consulta, se o número de zeros no intervalo for maior que zero, a 
            resposta será zero

        \item Caso contrário, o resultado depende da paridade do número de negativos no intervalo
   \end{itemize}

\end{frame}

\begin{frame}[fragile]{Solução com complexidade $O(K\log N + K)$}
    \inputsnippet{cpp}{1}{21}{codes/6139.cpp}
\end{frame}

\begin{frame}[fragile]{Solução com complexidade $O(K\log N + K)$}
    \inputsnippet{cpp}{22}{42}{codes/6139.cpp}
\end{frame}

\begin{frame}[fragile]{Solução com complexidade $O(K\log N + K)$}
    \inputsnippet{cpp}{43}{63}{codes/6139.cpp}
\end{frame}

\begin{frame}[fragile]{Solução com complexidade $O(K\log N + K)$}
    \inputsnippet{cpp}{64}{84}{codes/6139.cpp}
\end{frame}

\begin{frame}[fragile]{Solução com complexidade $O(K\log N + K)$}
    \inputsnippet{cpp}{85}{105}{codes/6139.cpp}
\end{frame}
