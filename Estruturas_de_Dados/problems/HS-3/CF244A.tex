\section{Codeforces Round \#150 (Div. 2) -- Problem A: Dividing Orange}

\begin{frame}[fragile]{Problema}

One day Ms Swan bought an orange in a shop. The orange consisted of $n\cdot k$ segments, numbered 
with integers from $1$ to $n\cdot k$.

There were $k$ children waiting for Ms Swan at home. The children have recently learned about the 
orange and they decided to divide it between them. For that each child took a piece of paper and 
wrote the number of the segment that he would like to get: the $i$-th ($1\leq i \leq k$) child 
wrote the number $a_i$ $(1\leq a_i \leq n\cdot k)$. All numbers $a_i$ accidentally turned out to be 
different.

\end{frame}

\begin{frame}[fragile]{Problema}

Now the children wonder, how to divide the orange so as to meet these conditions:

\begin{itemize}
    \item each child gets exactly $n$ orange segments;
    \item the $i$-th child gets the segment with number $a_i$ for sure;
    \item no segment goes to two children simultaneously.
\end{itemize}

Help the children, divide the orange and fulfill the requirements, described above.

\end{frame}

\begin{frame}[fragile]{Entrada e saída}

\textbf{Input}

The first line contains two integers $n, k$ $(1\leq n, k\leq 30)$. The second line contains 
$k$ space-separated integers $a_1, a_2, \ldots, a_k$ $(1\leq a_i\leq n\cdot k)$, where $a_i$ is 
the number of the orange segment that the $i$-th child would like to get.

It is guaranteed that all numbers $a_i$ are distinct.

\textbf{Output}

Print exactly $n\cdot k$ distinct integers. The first $n$ integers represent the indexes of the 
segments the first child will get, the second $n$ integers represent the indexes of the segments 
the second child will get, and so on. Separate the printed numbers with whitespaces.

You can print a child's segment indexes in any order. It is guaranteed that the answer always 
exists. If there are multiple correct answers, print any of them.

\end{frame}

\begin{frame}[fragile]{Exemplo de entradas e saídas}

\begin{minipage}[t]{0.5\textwidth}
\textbf{Sample Input}
\begin{verbatim}
2 2
4 1

3 1
2
\end{verbatim}
\end{minipage}
\begin{minipage}[t]{0.45\textwidth}
\textbf{Sample Output}
\begin{verbatim}
2 4 
1 3 

3 2 1
\end{verbatim}
\end{minipage}
\end{frame}

\begin{frame}[fragile]{Solução com complexidade $O(N)$}

    \begin{itemize}
        \item Este problema pode ser resolvido com complexidade linear por meio do uso de um
            \code{c}{unordered_set} $S$

        \item Este conjunto representa os segmentos da laranja que já foram atribuídos a uma
            das crianças

        \item Inicialmente deve ser inseridos em $S$ todos os valores $a_i$

        \item Como o \code{cpp}{unordered_set} utiliza \textit{hashes} em sua implementação,
            cada consulta ou inserçaõ em $S$ tem complexidade média $O(1)$

        \item Em seguida, inicializa-se uma variável $i$ apontando para o próximo segmento
            a ser considerado

        \item Assim, para cada uma das crianças, avalia-se se $i$ está ou não disponível

        \item Se estiver disponível, $i$ é atribuído à criança e acrescido em $S$

        \item Quando uma criança tiver $N$ segmentos o processamento para a criança seguinte

        \item Como cada segmento será processado uma única vez, a complexidade da solução
            é $O(N)$
   \end{itemize}

\end{frame}

\begin{frame}[fragile]{Solução AC com complexidade $O(N)$}
    \inputsnippet{cpp}{1}{21}{244A.cpp}
\end{frame}

\begin{frame}[fragile]{Solução AC com complexidade $O(N)$}
    \inputsnippet{cpp}{22}{42}{244A.cpp}
\end{frame}

\begin{frame}[fragile]{Solução AC com complexidade $O(N)$}
    \inputsnippet{cpp}{43}{63}{244A.cpp}
\end{frame}
