\section{UVA 11111 -- Generalized Matrioshkas}

\begin{frame}[fragile]{Problema}

Vladimir worked for years making matrioshkas, those nesting dolls that certainly represent truly 
Russian craft. A matrioshka is a doll that may be opened in two halves, so that one finds another 
doll inside.  Then this doll may be opened to find another one inside it. This can be repeated 
several times, till a final doll -- that cannot be opened -- is reached.

Recently, Vladimir realized that the idea of nesting dolls might be generalized to nesting toys.
Indeed, he has designed toys that contain toys but in a more general sense. One of these toys may
be opened in two halves and it may have more than one toy inside it. That is the new feature that
Vladimir wants to introduce in his new line of toys.
\end{frame}

\begin{frame}[fragile]{Problema}

Vladimir has developed a notation to describe how nesting toys should be constructed. A toy
is represented with a positive integer, according to its size. More precisely: if when opening the 
toy represented by $m$ we find the toys represented by $n_1, n_2, \ldots, n_r$, it must be true 
that $n_1+n_2+. . .+n_r < m$. And if this is the case, we say that toy $m$ contains directly the 
toys $n_1, n_2, \ldots, n_r$. It should be clear that toys that may be contained in any of the toys 
$n_1, n_2, \ldots, n_r$ are not considered as directly contained in the toy $m$.

A \textit{generalized matrioshka} is denoted with a non-empty sequence of non zero integers of the 
form:
\[
a_1\ \ a_2\ \ \ldots\ \ a_N
\]
such that toy $k$ is represented in the sequence with two integers $-k$ and $k$, with the negative 
one occurring in the sequence first that the positive one.

\end{frame}

\begin{frame}[fragile]{Problema}
For example, the sequence
\[
-9\ \ -7\ \ -2\ \ 2\ \ -3\ \ -2\ \ -1\ \ 1\ \ 2\ \ 3\ \ 7\ \ 9
\]
represents a generalized matrioshka conformed by six toys, namely, 1, 2 (twice), 3, 7 and 9. 
Note that toy 7 contains directly toys 2 and 3. Note that the first copy of toy 2 occurs left 
from the second one and that the second copy contains directly a toy 1. It would be wrong to 
understand that the first -2 and the last 2 should be paired.

On the other hand, the following sequences do not describe generalized matrioshkas:
\begin{itemize}
\item
\[
-9\ \ -7\ \ -2\ \ 2\ \ -3\ \ -1\ \ -2\ \ 2\ \ 1\ \ 3\ \ 7\ \ 9
\]
because toy 2 is bigger than toy 1 and cannot be allocated inside it.
\end{itemize}
\end{frame}

\begin{frame}[fragile]{Problema}
\begin{itemize}
    \item
\[
-9\ \ -7\ \ -2\ \ 2\ \ -3\ \ -2\ \ -1\ \ 1\ \ 2\ \ 3\ \ 7\ \ -2\ \ 2\ \ 9
\]
because 7 and 2 may not be allocated together inside 9.
    \item
\[
-9\ \ -7\ \ -2\ \ 2\ \ -3\ \ -1\ \ -2\ \ 3\ \ 2\ \ 1\ \ 7\ \ 9
\]
because there is a nesting problem within toy 3.
\end{itemize}

Your problem is to write a program to help Vladimir telling good designs from bad ones.
\end{frame}

\begin{frame}[fragile]{Entrada e saída}

\textbf{Input}

The input file contains several test cases, each one of them in a separate line. Each test case is a
sequence of non zero integers, each one with an absolute value less than $10^7$.

\textbf{Output}

Output texts for each input case are presented in the same order that input is read.

For each test case the answer must be a line of the form

\begin{flushleft}
\texttt{:-) Matrioshka!}
\end{flushleft}

if the design describes a generalized matrioshka. In other case, the answer should be of the form

\begin{flushleft}
\texttt{:-( Try again.}
\end{flushleft}

\end{frame}


\begin{frame}[fragile]{Exemplo de entradas e saídas}

\begin{minipage}[t]{0.6\textwidth}
\textbf{Sample Input}
\begin{verbatim}
-9 -7 -2 2 -3 -2 -1 1 2 3 7 9
-9 -7 -2 2 -3 -1 -2 2 1 3 7 9
-9 -7 -2 2 -3 -1 -2 3 2 1 7 9
-100 -50 -6 6 50 100
-100 -50 -6 6 45 100
-10 -5 -2 2 5 -4 -3 3 4 10
-9 -5 -2 2 5 -4 -3 3 4 9
\end{verbatim}
\end{minipage}
\begin{minipage}[t]{0.35\textwidth}
\textbf{Sample Output}
\begin{verbatim}
:-) Matrioshka!
:-( Try again.
:-( Try again.
:-) Matrioshka!
:-( Try again.
:-) Matrioshka!
:-( Try again.
\end{verbatim}
\end{minipage}
\end{frame}

\begin{frame}[fragile]{Solução com complexidade $O(N)$}

    \begin{itemize}
        \item Primeiramente, é preciso avaliar se cada brinquedo aberto ($-x$) é fechado
            corretamente (tem um $-x$ associado)

        \item Entre os valores $-x$ e $x$ devem haver brinquedos que abrem e fecham corretamente,
            cuja soma dos valores seja menor do que $x$

        \item Este problema é uma variante do problema do pareamento de parêntesis

        \item O uso de uma pilha permite a verificação de ambas condições simultaneamente

        \item Para cada valor $x$ da entrada, duas providências devem ser tomadas

        \item Se $x$ por negativo, ele deve ser inserido no topo da pilha
   \end{itemize}

\end{frame}

\begin{frame}[fragile]{Solução com complexidade $O(N)$}

    \begin{itemize}
        \item Se $x$ for positivo, deve se contabilizar os valores de todos os brinquedos que
            estão dentro de $x$

        \item Para tal, basta acumular os valores positivos que forem removidos da pilha até
            se alcançar um valor negativo

        \item Caso o valor negativo seja diferente de $-x$, ou se a pilha se tornar vazia antes
            de atingir um valor negativo, ou se o total dos valores dos brinquedos dentro de $x$
            for maior ou igual a $x$, o design não é válido

        \item Caso o valor atingido seja $-x$ e o total acumulado seja menor que $x$, o valor
            $-x$ deve ser retirado da pilha e o valor $x$ inserido no topo

        \item Ao final do processo, o design será válido se a pilha contiver apenas valores
            positivos
    \end{itemize}

\end{frame}

\begin{frame}[fragile]{Solução com complexidade $O(N)$}
    \inputsnippet{cpp}{1}{21}{11111.cpp}
\end{frame}

\begin{frame}[fragile]{Solução com complexidade $O(N)$}
    \inputsnippet{cpp}{22}{42}{11111.cpp}
\end{frame}

\begin{frame}[fragile]{Solução com complexidade $O(N)$}
    \inputsnippet{cpp}{43}{63}{11111.cpp}
\end{frame}
