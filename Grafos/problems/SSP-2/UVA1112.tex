\section{UVA 1112 -- Mice and Maze}

\begin{frame}[fragile]{Problema}

A set of laboratory mice is being trained to escape a maze. The maze is made up of cells, and each 
cell is connected to some other cells. However, there are obstacles in the passage between cells 
and therefore there is a time penalty to overcome the passage Also, some passages allow mice to go 
one-way, but not the other way round.

Suppose that all mice are now trained and, when placed in an arbitrary cell in the maze, take a
path that leads them to the exit cell in minimum time.

\end{frame}

\begin{frame}[fragile]{Problema}

We are going to conduct the following experiment: a mouse is placed in each cell of the maze and
a count-down timer is started. When the timer stops we count the number of mice out of the maze.

Write a program that, given a description of the maze and the time limit, predicts the number of
mice that will exit the maze. Assume that there are no bottlenecks is the maze, i.e. that all 
cells have room for an arbitrary number of mice.

\end{frame}

\begin{frame}[fragile]{Entrada e saída}

\textbf{Input}

The input begins with a single positive integer on a line by itself indicating the number of the 
cases following, each of them as described below. This line is followed by a blank line, and there 
is also a blank line between two consecutive inputs.

The maze cells are numbered $1, 2, \ldots, N$, where $N$ is the total number of cells. You can 
assume that $N\leq 100$.

The first three input lines contain $N$, the number of cells in the maze, $E$, the number of the 
exit cell, and the starting value $T$ for the count-down timer (in some arbitrary time unit).

\end{frame}

\begin{frame}[fragile]{Entrada e saída}

The fourth line contains the number $M$ of connections in the maze, and is followed by $M$ lines, 
each specifying a connection with three integer numbers: two cell numbers $a$ and $b$ (in the range 
$1, \ldots, N$) and the number of time units it takes to travel from $a$ to $b$.

Notice that each connection is one-way, i.e., the mice can’t travel from b to a unless there is 
another line specifying that passage. Notice also that the time required to travel in each 
direction might be different.

\textbf{Output}

For each test case, the output must follow the description below. The outputs of two consecutive 
cases will be separated by a blank line.

The output consists of a single line with the number of mice that reached the exit cell $E$ in at 
most $T$ time units.

\end{frame}

\begin{frame}[fragile]{Exemplo de entradas e saídas}
\begin{minipage}[t]{0.6\textwidth}
\textbf{Sample Input}
\begin{verbatim}
1

4
2
1
8
1 2 1
1 3 1
2 1 1
2 4 1
3 1 1
3 4 1
4 2 1
4 3 1
\end{verbatim}
\end{minipage}
\begin{minipage}[t]{0.35\textwidth}
\textbf{Sample Output}
\begin{verbatim}
3
\end{verbatim}
\end{minipage}
\end{frame}

\begin{frame}[fragile]{Solução com complexidade $O(T(V + E\log E))$}

    \begin{itemize}
        \item O problema consiste em determinar o caminho mínimo de todos os vértices do grafo
            até a saída $E$

        \item O custo de cada caminho mínimmo pode ser obtido através do algoritmo de Dijkstra

        \item Contudo, ao invés de executar o algoritmo $N - 1$ vezes, é mais eficiente inverter
            todas as arestas do grafo, e computar todos os caminhos a partir de $E$
        
        \item Deste modo, é necessário executar o algoritmo de Dijkstra uma única vez, de modo
            que a solução tem complexidade $O(T(V + E\log E))$, onde $T$ é o número de casos
            de teste

        \item Observe que um rato é posto também na saída, de modo que a resposta sempre será
            maior ou igual a um
   \end{itemize}

\end{frame}

\begin{frame}[fragile]{Solução com complexidade $O(T(V + E\log E))$}
    \inputsnippet{cpp}{1}{21}{1112.cpp}
\end{frame}

\begin{frame}[fragile]{Solução com complexidade $O(T(V + E\log E))$}
    \inputsnippet{cpp}{23}{43}{1112.cpp}
\end{frame}

\begin{frame}[fragile]{Solução com complexidade $O(T(V + E\log E))$}
    \inputsnippet{cpp}{44}{63}{1112.cpp}
\end{frame}

\begin{frame}[fragile]{Solução com complexidade $O(T(V + E\log E))$}
    \inputsnippet{cpp}{64}{84}{1112.cpp}
\end{frame}

\begin{frame}[fragile]{Solução com complexidade $O(T(V + E\log E))$}
    \inputsnippet{cpp}{85}{105}{1112.cpp}
\end{frame}
