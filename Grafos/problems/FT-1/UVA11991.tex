\section{UVA 11991 -- Easy Problem from Rujia Liu?}

\begin{frame}[fragile]{Problema}

``\textit{Though Rujia Liu usually sets hard problems for contests (for example, regional contests 
like Xi’an 2006, Beijing 2007 and Wuhan 2009, or UVa OJ contests like Rujia Liu’s Presents 1
and 2), he occasionally sets easy problem (for example, ‘the Coco-Cola Store’ in UVa OJ),
to encourage more people to solve his problems :D}''

Given an array, your task is to find the $k$-th occurrence (from left to right) of an integer 
$v$. To make the problem more difficult (and interesting!), you’ll have to answer $m$ such queries.

\end{frame}

\begin{frame}[fragile]{Entrada e saída}

\textbf{Input}

There are several test cases. The first line of each test case contains two integers 
$n, m$ $(1\leq n, m\leq 100,000)$, the number of elements in the array, and the number of 
queries. The next line contains $n$ positive integers not larger than 1,000,000. Each of the 
following $m$ lines contains two integer $k$ and $v$ ($1\leq k\leq n, 1\leq v\leq 1,000,000$). 
The input is terminated by end-of-file (EOF).

\textbf{Output}

For each query, print the $1$-based location of the occurrence. If there is no such element, 
output ‘\texttt{0}’ instead.

\end{frame}


\begin{frame}[fragile]{Exemplo de entradas e saídas}

\begin{minipage}[t]{0.6\textwidth}
\textbf{Sample Input}
\begin{verbatim}
8 4
1 3 2 2 4 3 2 1
1 3
2 4
3 2
4 2
\end{verbatim}
\end{minipage}
\begin{minipage}[t]{0.35\textwidth}
\textbf{Sample Output}
\begin{verbatim}
2
0
7
0
\end{verbatim}
\end{minipage}
\end{frame}

\begin{frame}[fragile]{Solução com complexidade $O(M)$}

    \begin{itemize}
        \item Cada \textit{query} pode ser respondida em $O(1)$, se o problema for interpretado
            como uma lista de adjacências

        \item Para tal, associe a um vértica cada número inteiro positivo de $1$ a $N$ e a cada 
            valor distinto do vetor $a$

        \item Se o valor $v$ ocorre na $i$-ésima posição do vetor $a$, adicione a aresta
            direcionada $(a, i)$ ao grafo $G$

        \item A \textit{query} $(v, k)$ pode ser respondida em $O(1)$ se o grafo $G$ for 
            representado por uma lista de adjacências, usando um \code{c}{vector} para cada
            lista

        \item Basta verificar o tamanho do \code{c}{vector} associado ao vértice $v$: se ele tem 
            $k$ ou mais elementos, basta retornar o valor que ocupa a posição $k - 1$

        \item Caso contrário, retorne zero
   \end{itemize}

\end{frame}

\begin{frame}[fragile]{Solução com complexidade $O(M)$}
    \inputsnippet{cpp}{1}{21}{11991.cpp}
\end{frame}

\begin{frame}[fragile]{Solução com complexidade $O(M)$}
    \inputsnippet{cpp}{23}{43}{11991.cpp}
\end{frame}

\begin{frame}[fragile]{Solução com complexidade $O(M)$}
    \inputsnippet{cpp}{44}{64}{11991.cpp}
\end{frame}
