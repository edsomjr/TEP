\section{Educational Codeforces Round 33 (Rated for Div. 2) -- Problem C: Rumor}

\begin{frame}[fragile]{Problema}

In Summer Informatics School, if a student doesn't behave well, teachers make a hole in his badge. 
And today one of the teachers caught a group of $n$ students doing yet another trick.

Let's assume that all these students are numbered from 1 to $n$. The teacher came to student $a$ 
and put a hole in his badge. The student, however, claimed that the main culprit is some other 
student $p_a$.

After that, the teacher came to student $p_a$ and made a hole in his badge as well. The student in 
reply said that the main culprit was student $p_{pa}$.

This process went on for a while, but, since the number of students was finite, eventually the teacher came to the student, who already had a hole in his badge.
\end{frame}

\begin{frame}[fragile]{Problema}
After that, the teacher put a second hole in the student's badge and decided that he is done with this process, and went to the sauna.

You don't know the first student who was caught by the teacher. However, you know all the numbers pi. Your task is to find out for every student $a$, who would be the student with two holes in the badge if the first caught student was $a$.
\end{frame}

\begin{frame}[fragile]{Entrada e saída}

\textbf{Input}

The first line of the input contains the only integer $n$ $(1\leq n\leq 1000)$ -- the number of the naughty students.

The second line contains $n$ integers $p_1, \ldots, p_n$ $(1\leq p_i\leq n)$, where $p_i$ indicates 
the student who was reported to the teacher by student $i$.

\textbf{Output}

For every student $a$ from 1 to $n$ print which student would receive two holes in the badge, if $a$ was the first student caught by the teacher.

\end{frame}

\begin{frame}[fragile]{Exemplo de entradas e saídas}

\begin{minipage}[t]{0.5\textwidth}
\textbf{Sample Input}
\begin{verbatim}
3
2 3 2

3
1 2 3
\end{verbatim}
\end{minipage}
\begin{minipage}[t]{0.45\textwidth}
\textbf{Sample Output}
\begin{verbatim}
2 2 3 


1 2 3 
\end{verbatim}
\end{minipage}
\end{frame}

\begin{frame}[fragile]{Solução com complexidade $O(N^2)$}

    \begin{itemize}
        \item Como a entrada é pequena $N\leq 10^3$, é possível resolver este problema
            com um algoritmo quadrático

        \item Considere que cada estudante seja um vértice do grafo $G$, e que uma aresta
            $(u, v)$ indique que o estudante $u$ indicou o estudante $v$ como principal
            responsável

        \item Logo $|V| = |E| = N$, de modo que cada travessia tem complexidade $O(N)$

        \item Assim, basta realizar $N$ travessias por profundidade, interrompendo a 
            mesma caso ela indifique um vértice que já foi encontrado anteriormente na travessia
   \end{itemize}

\end{frame}

\begin{frame}[fragile]{Solução AC com complexidade $O(N^2)$}
    \inputsnippet{cpp}{1}{21}{893C.cpp}
\end{frame}
