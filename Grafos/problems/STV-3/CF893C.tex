\section{Educational Codeforces Round 33 (Rated for Div. 2) -- Problem C: Rumor}

\begin{frame}[fragile]{Problema}

Vova promised himself that he would never play computer games... But recently Firestorm -- a 
well-known game developing company -- published their newest game, World of Farcraft, and it became 
really popular. Of course, Vova started playing it.

Now he tries to solve a quest. The task is to come to a settlement named Overcity and spread a 
rumor in it.

Vova knows that there are n characters in Overcity. Some characters are friends to each other, and they share information they got. Also Vova knows that he can bribe each character so he or she starts spreading the rumor; $i$-th character wants $c_i$ gold in exchange for spreading the rumor. When a character hears the rumor, he tells it to all his friends, and they start spreading the rumor to their friends (for free), and so on.
\end{frame}


\begin{frame}[fragile]{Problema}
The quest is finished when all n characters know the rumor. What is the minimum amount of gold Vova needs to spend in order to finish the quest?

Take a look at the notes if you think you haven't understood the problem completely.

\end{frame}

\begin{frame}[fragile]{Entrada e saída}

\textbf{Input}

The first line contains two integer numbers $n$ and $m$ $(1\leq n\leq 10^5, 0\leq m\leq 10^5)$ -- 
the number of characters in Overcity and the number of pairs of friends.

The second line contains $n$ integer numbers $c_i$ $(0\leq c_i\leq 10^9)$ -- the amount of gold 
$i$-th character asks to start spreading the rumor.

Then $m$ lines follow, each containing a pair of numbers $(x_i, y_i)$ which represent that 
characters $x_i$ and $y_i$ are friends $(1\leq x_i, y_i\leq n, x_i\neq y_i)$. It is guaranteed 
that each pair is listed at most once.

\textbf{Output}

Print one number -- the minimum amount of gold Vova has to spend in order to finish the quest.

\end{frame}

\begin{frame}[fragile]{Exemplo de entradas e saídas}

\begin{minipage}[t]{0.5\textwidth}
\textbf{Sample Input}
\begin{verbatim}
5 2
2 5 3 4 8
1 4
4 5

10 0
1 2 3 4 5 6 7 8 9 10

10 5
1 6 2 7 3 8 4 9 5 10
1 2
3 4
5 6
7 8
9 10
\end{verbatim}

\end{minipage}
\begin{minipage}[t]{0.45\textwidth}
\textbf{Sample Output}
\begin{verbatim}
10




55


15
\end{verbatim}
\end{minipage}
\end{frame}

\begin{frame}[fragile]{Solução com complexidade $O(N + M)$}

    \begin{itemize}
        \item Observe que este problema pode ser modelado como um grafo não-direcionado,
            onde cada vértice (e não aresta) tem um custo associado

        \item Como a pessoa é capaz de espalhar o rumo para todas as pessoas que estejam
            em seu componente conectado, basta escolher o menor custo em cada um dos
            componentes conectados

        \item Estes componentes podem ser identificados através de uma DFS ou uma BFS

        \item Atente ao fato de que, no pior caso, o custo mínimo pode extrapolar a capacidade
            de um número inteiro, sendo necessário utilizar o tipo \code{c}{long long}
   \end{itemize}

\end{frame}

\begin{frame}[fragile]{Solução AC com complexidade $O(N + M)$}
    \inputsnippet{cpp}{1}{21}{893C.cpp}
\end{frame}

\begin{frame}[fragile]{Solução AC com complexidade $O(N + M)$}
    \inputsnippet{cpp}{22}{41}{893C.cpp}
\end{frame}

\begin{frame}[fragile]{Solução AC com complexidade $O(N + M)$}
    \inputsnippet{cpp}{42}{62}{893C.cpp}
\end{frame}

\begin{frame}[fragile]{Solução AC com complexidade $O(N + M)$}
    \inputsnippet{cpp}{63}{83}{893C.cpp}
\end{frame}
