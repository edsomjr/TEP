\section{UVA 10505 -- Montesco vs Capuleto}

\begin{frame}[fragile]{Problema}

Romeo and Juliet have finally decided to get married. But preparing the wedding party will not be
so easy, as it is well-known that their respective families --the Montesco and the Capuleto-- are 
bloody enemies. In this problem you will have to decide which person to invite and which person 
not to invite, in order to prevent a slaugther.

We have a list of $N$ people who can be invited to the party or not. For every person $i$, we have 
a list of his enemies: $E_1, E_2, \ldots, E_p$. The “enemy” relationship has the following 
properties:

\begin{description}
    \item [Anti-transitive.] If $a$ is an enemy of $b$, and $b$ is an enemy of $c$, then $a$ is a 
        friend of $c$. Also, the enemies of the friends of $a$ are his enemies, and the friends of 
        the friends of $a$ are his friends.

    \item [Symmetrical.] If $a$ is an enemy of $b$, then $b$ is an enemy of $a$ (although it may 
        not be indicated in his list of enemies).
\end{description}

\end{frame}

\begin{frame}[fragile]{Problema}
One person will accept an invitation to the party if, and only if, he is invited, all his friends 
are invited and none of his enemies is invited. You have to find the maximum number of people that 
can be invited, so that all of them accept their invitation.

For instance, if $N = 5$, and we know that: 1 is enemy of 3, 2 is enemy of 1, and 4 is enemy of 5,
then we could invite a maximum of 3 people. These people could be 2, 3 and 4, but for this problem
we only want the number of people invited.

\end{frame}

\begin{frame}[fragile]{Entrada e saída}

\textbf{Input}

The first line of the input file contains the number $M$ of test cases in this file. A blank line 
follows this number, and a blank line is also used to separate test cases. The first line of each 
test case contains an integer $N$, indicating the number of people who have to be considered. You 
can assume that $N\leq 200$.  For each of these $N$ people, there is a line with his list of 
enemies. The first line contains the list of enemies of person 1, the second line contains the 
list of enemies of person 2, and so on. Each list of enemies starts with an integer $p$ (the number 
of known enemies of that person) and then, there are $p$ integers (the $p$ enemies of that person). 
So, for example, if a person’s enemies are 5 and 7, his list of enemies would be: ‘\texttt{2 5 7}’.

\end{frame}

\begin{frame}[fragile]{Entrada e saída}
\textbf{Output}

For each test case, the output should consist of a single line containing an integer, the maximum 
number of people who can be invited, so that all of them accept their invitation.

\end{frame}

\begin{frame}[fragile]{Exemplo de entradas e saídas}
\begin{scriptsize}
\begin{minipage}[t]{0.6\textwidth}
\textbf{Sample Input}
\begin{verbatim}
3

5
1 3
1 1
0
1 5
0

8
2 4 5
2 1 3
0
0
0
1 3
0
1 5

3
2 2 3
1 3
1 1
\end{verbatim}
\end{minipage}
\begin{minipage}[t]{0.35\textwidth}
\textbf{Sample Output}
\begin{verbatim}
3
5
0
\end{verbatim}
\end{minipage}
\end{scriptsize}
\end{frame}

\begin{frame}[fragile]{Solução com complexidade $O(M(V + E))$}

    \begin{itemize}
        \item Observe que o grafo que corresponde à entrada do problema é direcionado e não 
            necessariamente conectado

        \item Logo, para cada componente conectado, se as duas propriedades forem verificadas
            o componente será bipartido

        \item Neste caso, deve ser adicionado à resposta maior entre os números $B$ e $R$, os 
            quais correspondem ao número de nós azuis e vermelhos no componente, respectivamente
        
        \item Se o componente for bipartido, ele não contribuirá para a resposta final

        \item É preciso tomar, porém, alguns cuidados na implementação

        \item Como são múltiplos casos de teste, é preciso reiniciar as variáveis globais que
            foram utilizadas

        \item Além disso, na lista de inimigos da entrada podem aparecer pessoas cujo índice 
            está fora do intervalo $[1, N]$, as quais devem ser desprezadas
   \end{itemize}

\end{frame}

\begin{frame}[fragile]{Solução com complexidade $O(M(V + E))$}
    \inputsnippet{cpp}{1}{21}{10505.cpp}
\end{frame}

\begin{frame}[fragile]{Solução com complexidade $O(M(V + E))$}
    \inputsnippet{cpp}{23}{43}{10505.cpp}
\end{frame}

\begin{frame}[fragile]{Solução com complexidade $O(M(V + E))$}
    \inputsnippet{cpp}{44}{63}{10505.cpp}
\end{frame}

\begin{frame}[fragile]{Solução com complexidade $O(M(V + E))$}
    \inputsnippet{cpp}{64}{84}{10505.cpp}
\end{frame}
