\section{UVA 10687 -- Monitoring the Amazon}

\begin{frame}[fragile]{Problema}

A network of autonomous, battery-powered, data acquisition stations has been installed to monitor 
the climate in the region of Amazon. An order-dispatch station can initiate transmission of 
instructions to the control stations so that they change their current parameters. To avoid 
overloading the battery, each station (including the order-dispatch station) can only transmit to 
two other stations. The destinataries of a station are the two closest stations. In case of draw, 
the first criterion is to chose the westernmost (leftmost on the map), and the second criterion is 
to chose the southernmost (lowest on the map).

You are commissioned by Amazon State Government to write a program that decides if, given the
localization of each station, messages can reach all stations.

\end{frame}

\begin{frame}[fragile]{Entrada e saída}

\textbf{Input}

The input consists of an integer $N$, followed by $N$ pairs of integers $X_i, Y_i$, indicating the 
localization coordinates of each station. The first pair of coordinates determines the position of 
the order-dispatch station, while the remaining $N - 1$ pairs are the coordinates of the other 
stations. The following constraints are imposed: $-20\leq X_i, Y_i\leq 20$, and $1\leq N\leq 1000$.
The input is terminated with $N = 0$.

\textbf{Output}

For each given expression, the output will echo a line with the indicating if all stations can be 
reached or not (see sample output for the exact format).

\end{frame}


\begin{frame}[fragile]{Exemplo de entradas e saídas}
\begin{footnotesize}
\begin{minipage}[t]{0.55\textwidth}
\textbf{Sample Input}
\begin{verbatim}
4
1 0 0 1 -1 0 0 -1
8
1 0 1 1 0 1 -1 1 -1 0 -1 -1 0 -1 1 -1
6
0 3 0 4 1 3 -1 3 -1 -4 -2 -5
0
\end{verbatim}
\end{minipage}
\begin{minipage}[t]{0.4\textwidth}
\textbf{Sample Output}
\begin{verbatim}
All stations are reachable.
All stations are reachable.
There are stations that are unreachable.
\end{verbatim}
\end{minipage}
\end{footnotesize}
\end{frame}

\begin{frame}[fragile]{Solução com complexidade $O(TN^2)$}

    \begin{itemize}
        \item Observe que a transmissão se inicia na estação 1, e vai se propagando para as
            demais estações

        \item Também é importante notar a restrição de retransmissão para, no máximo, duas outras
            estações, sendo estas as mais próximas possíveis

        \item Estas condições do problema fazem com que a travessia mais apropriada seja a BFS

        \item Como o grafo é complexo, cada travessia teria complexidade $O(N^2)$ no pior caso

        \item Porém, a restrição para apenas duas retransmissões reduz a complexidade da
            travessia para $O(N)$

        \item A complexidade $O(N^2)$ se dá pelo processo de identificação das estações mais
            próximas
   \end{itemize}

\end{frame}

\begin{frame}[fragile]{Solução com complexidade $O(TN^2)$}
    \inputsnippet{cpp}{1}{21}{10687.cpp}
\end{frame}

\begin{frame}[fragile]{Solução com complexidade $O(TN^2)$}
    \inputsnippet{cpp}{22}{42}{10687.cpp}
\end{frame}

\begin{frame}[fragile]{Solução com complexidade $O(TN^2)$}
    \inputsnippet{cpp}{43}{61}{10687.cpp}
\end{frame}

\begin{frame}[fragile]{Solução com complexidade $O(TN^2)$}
    \inputsnippet{cpp}{62}{80}{10687.cpp}
\end{frame}

\begin{frame}[fragile]{Solução com complexidade $O(TN^2)$}
    \inputsnippet{cpp}{81}{101}{10687.cpp}
\end{frame}
