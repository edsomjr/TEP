\section{URI 1205 -- Cerco a Leningrado}

\begin{frame}[fragile]{Problema}

A cidade de São Petersburgo mudou de nome depois da revolção russa em 1914 para Petrogrado. Após a morte de Lênin, em homenagem ao grande líder o nome da cidade mudou novamente para Leningrado em 1924, e assim permaneceu até o fim da União Soviética. Em 1991, a cidade voltou a ter o nome antigo. Durante a segunda guerra mundial a cidade de Leningrado sofreu um cerco das tropas alemãs que durou cerca de 900 dias. Foi uma época terrível, de muita fome e perdas humanas, que terminou em 27 de janeiro de 1944 com a vitória dos soviéticos. É considerada uma das vitórias mais custosas da história em termos de vidas humanas perdidas.

No auge da ofensiva alemã, no ano de 1942, vários atiradores de elite foram espalhados pela cidade, inclusive, em alguns pontos estratégicos da cidade mais de um atirador aguardavam soldados inimigos. A espionagem russa tinha informações detalhadas das habilidades desses atiradores, mas seus esconderijos eram excelentes, tornando a tarefa de um soldado soviético que desejasse cruzar a cidade extremamente difícil. 

\end{frame}

\begin{frame}[fragile]{Problema}

Os soldados soviéticos eram bem treinados, mas com o passar do tempo e a continuação do cerco à cidade, os melhores soldados foram sendo dizimados, uma vez que se errassem o alvo na primeira tentativa certamente eram mortos pelos soldados alemães na tocaia.

Sabendo a probabilidade de um soldado em matar um atirador alemão e sabendo também o número de balas que ele tinha à sua disposição, desejamos saber a probabilidade desse soldado conseguir chegar a um ponto estratégico de destino, partindo de um ponto estratégico de origem. O soldado, sendo muito experiente, sempre usava um caminho que maximizava a probabilidade de sucesso. Note que o soldado deve matar todos os atiradores presentes no caminho usado, inclusive os que estiverem nos pontos estratégicos de origem e destino.

\end{frame}

\begin{frame}[fragile]{Entrada e saída}

\textbf{Entrada}

A entrada é composta por diversas instâncias e termina com final de arquivo (EOF). A primeira linha 
de cada instância contém 3 inteiros, $N$ $(2\leq N\leq 1000)$, $M$, e $K$ $(0\leq K\leq 1000)$ e a 
probabilidade $P$ $(0\leq P\leq 1)$ do soldado matar um atirador. Os inteiros $N, M$, e $K$ 
representam respectivamente os números de pontos estratégicos, estradas ligando pontos estratégicos 
e balas carregadas pelo soldado soviético. Os pontos estratégicos são numerados de 1 a $N$.

Cada uma das próximas $M$ linhas contém um par de inteiros $i$ e $j$ indicando que existe uma 
estrada ligando o ponto $i$ ao $j$. Em seguida tem uma linha contendo um inteiro $A$ 
$(0\leq A\leq 2000)$, correspondendo ao número de atiradores na cidade, seguido por $A$ inteiros 
indicando a posição de cada atirador. A última linha de cada instância contém dois inteiros 
indicando o ponto de partida e o destino do soldado.

\end{frame}

\begin{frame}[fragile]{Entrada e saída}
\textbf{Saída}

Para cada instância imprima uma linha contendo a probabilidade de sucesso do soldado soviético. A probabilidade deve ser impressa com 3 casas decimais.

\end{frame}

\begin{frame}[fragile]{Exemplo de entradas e saídas}

\begin{minipage}[t]{0.5\textwidth}
\textbf{Exemplo de Entrada}
\begin{verbatim}
3 2 10 0.1
1 2
2 3
10 1 1 3 3 1 3 1 1 3 3
1 3
5 5 10 0.3
1 2
2 4
2 5
4 5
5 3
6 3 3 3 3 3 3
1 3
\end{verbatim}
\end{minipage}
\begin{minipage}[t]{0.45\textwidth}
\textbf{Exemplo de Saída}
\begin{verbatim}
0.000
0.001
\end{verbatim}
\end{minipage}
\end{frame}

\begin{frame}[fragile]{Solução com complexidade $O(TVE)$}

    \begin{itemize}
        \item Observe que, para maximizar a probabilidade de sucesso, o soldado deve encontrar
            o mínimo de atiradores possível

        \item Desta forma, o problema se reduz a se determinar o caminho mínimo entre o ponto de
            partida $s$ e o ponto de chegada $e$

        \item Como o número de vértices é pequeno, é possível usar o algoritmo de Bellman-Ford

        \item Porém é necessário tomar dois cuidados

        \item O primeiro deles é que a distância inicial de $s$ a $s$ não é zero, e sim o número
            de soldados em $s$

        \item Em segundo lugar é preciso acelerar o algoritmo interompendo o laço caso não 
            exista nenhuma atualização ao final de uma iteração

        \item Por fim, se o número de soldados no caminho mínimo for maior do que $K$, não
            há chance de sucesso
   \end{itemize}

\end{frame}

\begin{frame}[fragile]{Solução com complexidade $O(TVE)$}
    \inputsnippet{cpp}{1}{21}{1205.cpp}
\end{frame}

\begin{frame}[fragile]{Solução com complexidade $O(TVE)$}
    \inputsnippet{cpp}{22}{42}{1205.cpp}
\end{frame}

\begin{frame}[fragile]{Solução com complexidade $O(TVE)$}
    \inputsnippet{cpp}{43}{63}{1205.cpp}
\end{frame}

\begin{frame}[fragile]{Solução com complexidade $O(TVE)$}
    \inputsnippet{cpp}{64}{82}{1205.cpp}
\end{frame}

\begin{frame}[fragile]{Solução alternativa SPFA complexidade $O(TVE)$}
    \inputsnippet{cpp}{1}{21}{ac.cpp}
\end{frame}

\begin{frame}[fragile]{Solução alternativa SPFA complexidade $O(TVE)$}
    \inputsnippet{cpp}{23}{43}{ac.cpp}
\end{frame}

\begin{frame}[fragile]{Solução alternativa SPFA complexidade $O(TVE)$}
    \inputsnippet{cpp}{45}{64}{ac.cpp}
\end{frame}

\begin{frame}[fragile]{Solução alternativa SPFA complexidade $O(TVE)$}
    \inputsnippet{cpp}{65}{84}{ac.cpp}
\end{frame}

\begin{frame}[fragile]{Solução alternativa SPFA complexidade $O(TVE)$}
    \inputsnippet{cpp}{85}{105}{ac.cpp}
\end{frame}


