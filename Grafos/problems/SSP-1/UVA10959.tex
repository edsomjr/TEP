\section{UVA 10959 -- The Party, Part I}

\begin{frame}[fragile]{Problema}

Don Giovanni likes to dance–especially with girls! And everyone else in the party enjoyed the 
dance, too. Getting a chance to dance with the host (that is Don Giovanni) is the greatest honour; 
failing that, dancing with someone who has danced with the host or will dance with the host is the 
second greatest honour. This can go further. Define the Giovanni number of a person as follows, at 
the time after the party is over and therefore who has danced with whom is completely known and 
fixed:

\begin{enumerate}
    \item No one has a negative Giovanni number.
    \item The Giovanni number of Don Giovanni is 0.
    \item If a person $p$ is not Don Giovanni himself, and has danced with someone with Giovanni 
        number $n$, and has not danced with anyone with a Giovanni number smaller than $n$, then 
        $p$ has Giovanni number $n + 1$
\end{enumerate}
\end{frame}

\begin{frame}[fragile]{Problema}
\begin{enumerate}
    \setcounter{enumi}{3}
    \item If a person’s Giovanni number cannot be determined from the above rules (he/she has not 
        danced with anyone with a finite Giovanni number), his/her Giovanni number is 
        $\infty$. Fortunately, you will not need this rule in this problem.
\end{enumerate}

Your job is to write a program to compute Giovanni numbers.

\end{frame}

\begin{frame}[fragile]{Entrada e saída}

\textbf{Input}

The input begins with a single positive integer on a line by itself indicating the number of the 
cases following, each of them as described below. This line is followed by a blank line, and there 
is also a blank line between two consecutive inputs.

The first line has two numbers $P$ and $D$; this means there are $P$ persons in the party 
(including Don Giovanni) and $D$ dancing couples ($P\leq 1000$ and $D\leq P(P - 1)/2$). Then 
$D$ lines follow, each containing two distinct persons, meaning the two persons has danced. 
Persons are represented by numbers between 0 and $P - 1$; Don Giovanni is represented by 0.

As noted, we design the input so that you will not need the $\infty$ rule in computing Giovanni 
numbers.

\end{frame}

\begin{frame}[fragile]{Entrada e saída}

We have made our best effort to eliminate duplications in listing the dancing couples, e.g., if 
there is a line “\texttt{4 7}” among the $D$ lines, then this is the only occurrence of 
“\texttt{4 7}”, and there is no occurrence of “\texttt{7 4}”. But just in case you see a 
duplication, you can ignore it (the duplication, not the first occurrence).

\textbf{Output}

For each test case, the output must follow the description below. The outputs of two consecutive 
cases will be separated by a blank line.

Output $P - 1$ lines. Line $i$ is the Giovanni number of person $i$, for $1\leq i\leq P - 1$. Note 
that it is $P - 1$ because we skip Don Giovanni in the output.

\end{frame}

\begin{frame}[fragile]{Exemplo de entradas e saídas}
\begin{minipage}[t]{0.6\textwidth}
\textbf{Sample Input}
\begin{verbatim}
1

5 6
0 1
0 2
3 2
2 4
4 3
1 2
\end{verbatim}
\end{minipage}
\begin{minipage}[t]{0.35\textwidth}
\textbf{Sample Output}
\begin{verbatim}
1
1
2
2
\end{verbatim}
\end{minipage}
\end{frame}

\begin{frame}[fragile]{Solução com complexidade $O(T(V + E))$}

    \begin{itemize}
        \item Observe que o número de Giovanni é, de fato, a distância do nó em questão até o
         nó zero, em número de arestas

        \item Esta distância pode ser computada por meio de uma BFS

        \item Como há garantia de conectividade do grafo de entrada (condição 4 do problema),
            uma única travessia é suficiente
        
        \item Como a complexidade da BFS é $O(V + E)$, a complexidade da solução é 
            $O(T(V + E))$, onde $T$ é o número de casos de teste
   \end{itemize}

\end{frame}

\begin{frame}[fragile]{Solução com complexidade $O(T(V + E))$}
    \inputsnippet{cpp}{1}{21}{10959.cpp}
\end{frame}

\begin{frame}[fragile]{Solução com complexidade $O(T(V + E))$}
    \inputsnippet{cpp}{22}{42}{10959.cpp}
\end{frame}

\begin{frame}[fragile]{Solução com complexidade $O(T(V + E))$}
    \inputsnippet{cpp}{43}{63}{10959.cpp}
\end{frame}

\begin{frame}[fragile]{Solução com complexidade $O(T(V + E))$}
    \inputsnippet{cpp}{64}{84}{10959.cpp}
\end{frame}

