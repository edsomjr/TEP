\section{UVA 10113 -- Exchange Rates}

\begin{frame}[fragile]{Problema}

Using money to pay for goods and services usually makes life easier, but sometimes people prefer 
to trade items directly without any money changing hands. In order to ensure a consistent 
``price'', traders set an exchange rate between items. The exchange rate between two items $A$ and 
$B$ is expressed as two positive integers $m$ and $n$, and says that $m$ of item $A$ is worth 
$n$ of item $B$. For example, 2 stoves might be worth 3 refrigerators. (Mathematically, 1 stove 
is worth 1.5 refrigerators, but since it’s hard to find half a refrigerator, exchange rates are 
always expressed using integers.)

Your job is to write a program that, given a list of exchange rates, calculates the exchange rate
between any two items.

\end{frame}

\begin{frame}[fragile]{Entrada e saída}

\textbf{Input}

The input file contains one or more commands, followed by a line beginning with a period that 
signals the end of the file. Each command is on a line by itself and is either an assertion or a 
query. An assertion begins with an exclamation point and has the format
\[
!\ m\ itema = n\ itemb
\]

where $itema$ and $itemb$ are distinct item names and $m$ and $n$ are both positive integers less 
than 100.  This command says that $m$ of $itema$ are worth $n$ of $itemb$. A query begins with a 
question mark, is of the form
\[
?\ itema = itemb
\]
and asks for the exchange rate between $itema$ and $itemb$, where $itema$ and $itemb$ are distinct 
items that have both appeared in previous assertions (although not necessarily the same assertion).
\end{frame}

\begin{frame}[fragile]{Entrada e saída}
\textbf{Output}

For each query, output the exchange rate between itema and itemb based on all the assertions made 
up to that point. Exchange rates must be in integers and must be reduced to lowest terms. If no 
exchange rate can be determined at that point, use question marks instead of integers. Format all 
output exactly as shown in the example.
\end{frame}

\begin{frame}[fragile]{Entrada e saída}
\textbf{Note}:

\begin{itemize}
    \item Item names will have length at most 20 and will contain only lowercase letters.
    \item Only the singular form of an item name will be used (no plurals).
    \item There will be at most 60 distinct items.
    \item There will be at most one assertion for any pair of distinct items.
    \item There will be no contradictory assertions. For example, “2 pig = 1 cow”, “2 cow = 1 horse”, and “2 horse = 3 pig” are contradictory.
    \item Assertions are not necessarily in lowest terms, but output must be.
    \item Although assertions use numbers less than 100, queries may result in larger numbers that will not exceed 10000 when reduced to lowest terms
\end{itemize}

\end{frame}


\begin{frame}[fragile]{Exemplo de entradas e saídas}

\begin{minipage}[t]{0.6\textwidth}
\textbf{Sample Input}
\begin{verbatim}
! 6 shirt = 15 sock
! 47 underwear = 9 pant
? sock = shirt
? shirt = pant
! 2 sock = 1 underwear
? pant = shirt
.
\end{verbatim}
\end{minipage}
\begin{minipage}[t]{0.35\textwidth}
\textbf{Sample Output}
\begin{verbatim}
5 sock = 2 shirt
? shirt = ? pant
45 pant = 188 shirt
\end{verbatim}
\end{minipage}
\end{frame}

\begin{frame}[fragile]{Solução com complexidade $O(QN^2)$}

    \begin{itemize}
        \item Para simplificar a solução, pode-se associar um número inteiro distinto para cada 
            item

        \item A cada \textit{query} do tipo `\code{c}{!}' deve se associar o identificador 
            inteiro a cada item ainda não encontrado e registrar, em uma matriz de adjacências,
            o par $(m, n)$ que corresponde à taxa de conversão entre ambos itens

        \item Esta taxa deve ser convertida a um par de elementos co-primos através da divisão de
            ambos pelo maior divisor comum entre eles

        \item Para cada \textit{query} do tipo `\code{c}{?}' deve se determinar os identificadores
            $u$ e $w$ de cada item e reiniciar a tabela de vetores encontrados DFS

        \item Para cada item $v$ ainda não encontrado pela DFS, computa-se a taxa de conversão
            entre $u$ e $v$

        \item Esta taxa pode já estar registrada na tabela de conversões ou deve ser estabelecida
            se a busca já visitou anteriormente outro nó

        \item Se $v = w$ a busca retorna esta taxa; caso contrário, a busca continua com $u = v$
   \end{itemize}

\end{frame}

\begin{frame}[fragile]{Solução com complexidade $O(QN^2)$}
    \inputsnippet{cpp}{1}{21}{10113.cpp}
\end{frame}

\begin{frame}[fragile]{Solução com complexidade $O(QN^2)$}
    \inputsnippet{cpp}{22}{42}{10113.cpp}
\end{frame}

\begin{frame}[fragile]{Solução com complexidade $O(QN^2)$}
    \inputsnippet{cpp}{43}{63}{10113.cpp}
\end{frame}

\begin{frame}[fragile]{Solução com complexidade $O(QN^2)$}
    \inputsnippet{cpp}{64}{84}{10113.cpp}
\end{frame}

\begin{frame}[fragile]{Solução com complexidade $O(QN^2)$}
    \inputsnippet{cpp}{85}{105}{10113.cpp}
\end{frame}
