\section{UVA 10171 -- Meeting Prof. Miguel...}

\begin{frame}[fragile]{Problema}

I have always thought that someday I will meet Professor Miguel,
who has allowed me to arrange so many contests. But I have
managed to miss all the opportunities in reality. At last with the
help of a magician I have managed to meet him in the magical
\textbf{City of Hope}. The city of hope has many roads. Some of
them are bi-directional and others are unidirectional. Another
important property of these streets are that some of the streets
are for people whose age is less than thirty and rest are for the
others. This is to give the minors freedom in their activities.
Each street has a certain length. Given the description of such
a city and our initial positions, you will have to find the most
suitable place where we can meet. The most suitable place is the
place where our combined effort of reaching is minimum. You can
assume that I am 25 years old and Prof. Miguel is 40+.

\end{frame}

\begin{frame}[fragile]{Problema}

\begin{figure}
    \includegraphics[scale=1.2]{figure.png}
    \caption{Shahriar Manzoor and Miguel A.  Revilla (Shanghai, 2005).  First meeting after five years of collaboration}
\end{figure}

\end{frame}

\begin{frame}[fragile]{Entrada e saída}

\textbf{Input}

The input contains several descriptions of cities. Each description of city is started by a 
integer $N$, which indicates how many streets are there. The next $N$ lines contain the 
description of $N$ streets.  

The description of each street consists of four uppercase alphabets and an integer. The first 
alphabet is either ‘\texttt{Y}’ (indicates that the street is for young) or ‘\texttt{M}’ 
(indicates that the road is for people aged 30 or more) and the second character is either 
‘\texttt{U}’ (indicates that the street is unidirectional) or ‘\texttt{B}’ (indicates that the 
street is bi-directional). The third and fourth characters, $X$ and $Y$ can be any uppercase 
alphabet and they indicate that place named $X$ and $Y$ of the city are connected (in case of
unidirectional it means that there is a one-way street from $X$ to $Y$) and the last non-negative 
integer $C$ indicates the energy required to walk through the street. If we are in the same place 
we can meet each other in zero cost anyhow. Every energy value is less than 500.

\end{frame}

\begin{frame}[fragile]{Entrada e saída}

After the description of the city the last line of each input contains two place names, which are 
the initial position of me and Prof. Miguel respectively.

A value zero for $N$ indicates end of input.

\textbf{Output}

For each set of input, print the minimum energy cost and the place, which is most suitable for us to
meet. If there is more than one place to meet print all of them in lexicographical order in the same
line, separated by a single space. If there is no such places where we can meet then print the line ‘\texttt{You will never meet.}’

\end{frame}

\begin{frame}[fragile]{Exemplo de entradas e saídas}
\begin{minipage}[t]{0.6\textwidth}
\textbf{Sample Input}
\begin{verbatim}
4
Y U A B 4
Y U C A 1
M U D B 6
M B C D 2
A D
2
Y U A B 10
M U C D 20
A D
0
\end{verbatim}
\end{minipage}
\begin{minipage}[t]{0.35\textwidth}
\textbf{Sample Output}
\begin{verbatim}
10 B
You will never meet.
\end{verbatim}
\end{minipage}
\end{frame}

\begin{frame}[fragile]{Solução com complexidade $O(TV^3)$}

    \begin{itemize}
        \item Observe que o valor de $N$ da entrada corresponde ao número de arestas, e não
            de vértices

        \item Como cada vértice corresponde a uma letra maiúscula do alfabeto, vale que
            $V \leq 26$

        \item Para a solução do problema é necessário computar as distâncias mínimas entre todos
            os vértices dos dois grafos: o grafo associado ao professor Manzoor, cujas
            arestas correspondem às pessoas com menos de 30 anos, e o grafo associado ao professor
            Revilla, para pessoas com mais de 30 anos
        
        \item Em seguida, basta avaliar todos os 26 possíveis pontos de 
            encontro, e escolher os que levam ao custo mínimo, que é a soma das distâncias
            mínimas do ponto de partida de ambos até o destino

        \item Deve-se ter cuidado em avaliar se ambos podem chegar ao destino, e também com
            a possibilidade de \textit{autoloops}, que podem levar a respostas erradas, uma
            vez que o texto atribui custo zero quando ambos estão na mesma posição
   \end{itemize}

\end{frame}

\begin{frame}[fragile]{Solução com complexidade $O(TV^3)$}
    \inputsnippet{cpp}{1}{21}{10171.cpp}
\end{frame}

\begin{frame}[fragile]{Solução com complexidade $O(TV^3)$}
    \inputsnippet{cpp}{22}{42}{10171.cpp}
\end{frame}

\begin{frame}[fragile]{Solução com complexidade $O(TV^3)$}
    \inputsnippet{cpp}{43}{63}{10171.cpp}
\end{frame}

\begin{frame}[fragile]{Solução com complexidade $O(TV^3)$}
    \inputsnippet{cpp}{65}{85}{10171.cpp}
\end{frame}

\begin{frame}[fragile]{Solução com complexidade $O(TV^3)$}
    \inputsnippet{cpp}{87}{107}{10171.cpp}
\end{frame}

\begin{frame}[fragile]{Solução com complexidade $O(TV^3)$}
    \inputsnippet{cpp}{108}{128}{10171.cpp}
\end{frame}
