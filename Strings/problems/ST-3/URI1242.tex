\section{URI 1242 -- Ácido Ribonucleico Alienígena}

\begin{frame}[fragile]{Problema}

Foi descoberta uma espécie alienígena de ácido ribonucleico (popularmente conhecido como RNA). Os cientistas, por falta de criatividade, batizaram a descoberta de ácido ribonucleico alienígena (RNAA). Similar ao RNA que conhecemos, o RNAA é uma fita composta de várias bases. As bases são B C F S e podem ligar-se em pares. Os únicos pares possíveis são entre as bases B e S e as bases C e F.
Enquanto está ativo, o RNAA dobra vários intervalos da fita sobre si mesma, realizando ligações entre suas bases. Os cientistas perceberam que:

\begin{itemize}
\item Quando um intervalo da fita de RNAA se dobra, todas as bases neste intervalo se ligam com suas bases correspondentes;
\item Cada base pode se ligar a apenas uma outra base;
\item As dobras ocorrem de forma a maximizar o número de ligações feitas sobre fitas;
\end{itemize}

\end{frame}

\begin{frame}[fragile]{Problema}

As figuras abaixo ilustram dobras e ligacões feitas sobre fitas.

\begin{figure}
    \centering
    \includegraphics[scale=0.7]{UOJ_1242.jpg}
\end{figure}

Sua tarefa será, dada a descrição de uma tira de RNAA, determinar quantas ligações serão realizadas entre suas bases se a tira ficar ativa.

\end{frame}

\begin{frame}[fragile]{Entrada e saída}

\textbf{Entrada}

A entrada é composta por diversos casos de teste e termina com EOF. Cada caso de teste possui uma linha descrevendo a sequência de bases da fita de RNAA. Uma fita de RNAA na entrada contém pelo menos 1 e no máximo 300 bases. Não existem espaços entre bases de uma fita da entrada. As bases são 'B', 'C', 'F' e 'S'.

\vspace{0.2in}

\textbf{Saída}

Para cada instância imprima uma linha contendo o número total de ligações que ocorre quando a fita descrita é ativada.

\end{frame}


\begin{frame}[fragile]{Exemplo de entradas e saídas}

\begin{minipage}[t]{0.5\textwidth}
\textbf{Exemplo de Entrada}
\begin{verbatim}
SBC
FCC
SFBC
SFBCFSCB
CFCBSFFSBCCB
\end{verbatim}
\end{minipage}
\begin{minipage}[t]{0.45\textwidth}
\textbf{Exemplo de Saída}
\begin{verbatim}
1
1
0
4
5
\end{verbatim}
\end{minipage}
\end{frame}

\begin{frame}[fragile]{Solução com complexidade $O(N^2)$}

    \begin{itemize}
        \item A estratégia para a solução do problema é utilizar duas pilhas

        \item A primeira deve ser preenchida com todos os elementos da string dada, na ordem da
            entrada

        \item A segunda guardará os elementos a serem pareadas, e inicialmente estará vazia

        \item Em seguida, todos os elementos da primeira fila devem ser processados, do topo à
            base

        \item Se a segunda pilha estiver vazia, o elemento processado vai para o seu topo

   \end{itemize}

\end{frame}

\begin{frame}[fragile]{Solução com complexidade $O(N^2)$}

    \begin{itemize}
        \item Caso contrário, tenta-se o pareamento do elemento processado com o topo da 
            pilha

        \item Se é um pareamento válido, a resposta é incrementada e ambos elementos descartados

        \item Caso contrário, o elemento processado vai para segunda pilha

        \item Ao final do processamento, se houve ao menos um pareamento válido, o processamento
            deve recomeçar, tendo as filas trocadas de posição

        \item Esta solução é quadrática no número de elementos da string dada
   \end{itemize}

\end{frame}

\begin{frame}[fragile]{Solução com complexidade $O(N^2)$}
    \inputsnippet{cpp}{1}{21}{1242.cpp}
\end{frame}

\begin{frame}[fragile]{Solução com complexidade $O(N^2)$}
    \inputsnippet{cpp}{22}{42}{1242.cpp}
\end{frame}

\begin{frame}[fragile]{Solução com complexidade $O(N^2)$}
    \inputsnippet{cpp}{43}{63}{1242.cpp}
\end{frame}
