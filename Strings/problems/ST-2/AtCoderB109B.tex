\section{AtCoder Beginner Contest 109 -- Problem B: Shiritori}

\begin{frame}[fragile]{Problema}

Takahashi is practicing \textit{shiritori} alone again today.

Shiritori is a game as follows:

\begin{itemize}
    \item In the first turn, a player announces any one word.
    \item In the subsequent turns, a player announces a word that satisfies the following conditions:
        \begin{itemize}
            \item That word is not announced before.
            \item The first character of that word is the same as the last character of the last word announced.
        \end{itemize}
\end{itemize}

In this game, he is practicing to announce as many words as possible in ten seconds.

You are given the number of words Takahashi announced, $N$, and the $i$-th word he announced, 
$W_i$, for each $i$. Determine if the rules of shiritori was observed, that is, every word announced by him satisfied the conditions.

\end{frame}

\begin{frame}[fragile]{Entrada e saída}

\textbf{Constraints}

\begin{itemize}
    \item $N$ is an integer satisfying $2\leq N\leq 100$.
    \item $W_i$is a string of length between 1 and 10 (inclusive) consisting of lowercase English letters.
\end{itemize}

\textbf{Input}

Input is given from Standard Input in the following format:
\begin{align*}
&N\\
&W_1\\
&W_2\\
&\vdots \\
&W_N\\
\end{align*}
\end{frame}

\begin{frame}[fragile]{Entrada e saída}
\textbf{Output}

If every word announced by Takahashi satisfied the conditions, print `\textcolor{red}{\texttt{Yes}}'; otherwise, print `\textcolor{red}{\texttt{No}}'.

\end{frame}

\begin{frame}[fragile]{Exemplo de entradas e saídas}
\begin{footnotesize}
\begin{minipage}[t]{0.5\textwidth}
\textbf{Exemplo de Entrada}
\begin{verbatim}
4
hoge
english
hoge
enigma

9
basic
c
cpp
php
python
nadesico
ocaml
lua
assembly
\end{verbatim}
\end{minipage}
\begin{minipage}[t]{0.45\textwidth}
\textbf{Exemplo de Saída}
\begin{verbatim}
No





Yes
\end{verbatim}
\end{minipage}
\end{footnotesize}
\end{frame}

\begin{frame}[fragile]{Solução}

    \begin{itemize}
        \item A solução do problema consiste em observar as duas regras básicas do jogo

        \item Para tal, seja $s_1$ a primeira palavra dita e $c$ o último caractere de $s$

        \item Para manter a primeira regra, as palavras já ditas devem ser mantidas em um
            conjunto

        \item Logo $s_1$ deve ser inserida neste conjunto

        \item Para as demais palavras $s_i$, com $2\leq i\leq N$, deve-se verificar primeiramente
            a segunda regra, isto é, checar se o primeiro caractere de  $s_i$ é igual a $c$

        \item Caso não seja, a resposta do problema é `\textcolor{red}{\texttt{No}}'

        \item Em caso afirmativo, $s_i$ deve ser inserida no conjunto: caso já esteja lá,
            a resposta do problema também é `\textcolor{red}{\texttt{No}}'

        \item Se não estiver no conjunto, o valor de $c$ deve ser atualizado para o último
            caractere de $s_i$ e continuar a rotina

        \item Esta solução tem complexidade $O(N\log N)$, devido as inserções no conjunto
    \end{itemize}

\end{frame}


\begin{frame}[fragile]{Solução AC com complexidade $O(N\log N)$}
    \inputsnippet{cpp}{1}{21}{B109B.cpp}
\end{frame}

\begin{frame}[fragile]{Solução AC com complexidade $O(N\log N)$}
    \inputsnippet{cpp}{22}{42}{B109B.cpp}
\end{frame}
