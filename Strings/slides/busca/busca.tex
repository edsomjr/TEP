\section{Busca em Strings}


\begin{frame}[fragile]{Definição de busca em strings}

    \begin{itemize}
        \item A busca é o algoritmo fundamental dentre os algoritmos de strings 
        \pause

        \item Ela equivalente, em importância, aos algoritmos de ordenação no estudo de algoritmos
        \pause

        \item A busca em strings consiste em determinar se uma string $P$, de tamanho $m$, ocorre 
            ou não em uma string $S$, de tamanho $n$
        \pause

        \item Uma variante comum é determinar o número de ocorrências de $P$ em $S$
    \end{itemize}

\end{frame}

\begin{frame}[fragile]{Algoritmos de busca em strings}

    \begin{itemize}
        \item Os principais algoritmos de busca em strings são
        \begin{enumerate}
            \item a busca completa
            \item o algoritmo de Rabin-Karp
            \item o algoritmo de Knuth-Morris-Pratt
            \item a função $z$
            \item o algoritmo de Boyer-Moore
        \end{enumerate}
        \pause


        \item O primeiro deles é de fácil entendimento e codificação
        \pause


        \item Os demais são conceitualmente mais sofisticados e podem compreender duas etapas:
            pré-processamento e busca
        \pause


        \item Esta sofistificação dificulta a implementação, mas traz ganhos na complexidade
            assintótica em relação à busca completa
    \end{itemize}

\end{frame}
