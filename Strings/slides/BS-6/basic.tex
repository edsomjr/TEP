\section{Algoritmos elementares}

\begin{frame}[fragile]{Métodos {\tt find()} e {\tt rfind()} da classe string}

    \begin{itemize}
        \item A classe string da linguagem C++ oferece dois métodos de busca em strings

        \item O método \code{cpp}{find()} procura pela substring \code{cpp}{str} na 
            substring $S[\mathtt{pos}..(n - 1)]$
        \vspace{.1in}
        \inputsyntax{cpp}{find.st}
        \vspace{.1in}

        \item O retorno é o índice da primeira ocorrência de \code{cpp}{str} na substring em
            questão, ou \code{cpp}{string::npos}, caso \code{cpp}{str} não ocorra em
            em nenhuma posição do intervalo especificado

        \item A complexidade assintótica é $O(nm)$, onde $n = |S|$ e $m = |\mathtt{str}|$

        \item O método \code{cpp}{rfind()} tem o mesmo comportamento e retorno, porém busca
            a última ocorrência de \code{cpp}{str} em $S[0..\mathtt{pos}]$

        \vspace{.1in}
        \inputsyntax{cpp}{rfind.st}
        \vspace{.1in}
    \end{itemize}

\end{frame}

\begin{frame}[fragile]{Exemplo de uso dos métodos {\tt find()} e {\tt rfind()}}
    \inputsnippet{cpp}{1}{21}{find.cpp}
\end{frame}

\begin{frame}[fragile]{Método {\tt find\_first\_of()}}

    \begin{itemize}
        \item Outro método relacionado à busca de strings é o \code{cpp}{find_first_of()}, cuja
            assinatura é

        \vspace{.1in}
        \inputsyntax{cpp}{find_first_of.st}
        \vspace{.1in}
        
        \item Ele retorna a primeira posição $i$ em $S$ tal que $S[i]$ é igual a um dos 
            caracteres de \code{cpp}{str}, ou \code{cpp}{string::npos}, caso não encontre
            nenhum correspondente de \code{cpp}{str} em $S$

        \item A complexidade é a mesma do método \code{cpp}{find()}: $O(nm)$ 

        \item O método \code{cpp}{find_first_not_of()} é semelhante, porém retorna o primeiro
            caractere de $S$ que é diferente de todos os caracteres de \code{cpp}{str}

        \item Os métodos \code{cpp}{find_last_of()} e \code{cpp}{find_last_not_of()} são
            equivalentes a ambos, porém iniciando sua busca no sentido oposto
    \end{itemize}

\end{frame}

\begin{frame}[fragile]{Exemplo de uso dos métodos {\tt find\_first\_of()} e {\tt find\_last\_of()}}
    \inputsnippet{cpp}{1}{21}{find_first_of.cpp}
\end{frame}


