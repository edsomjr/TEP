\section{Motivação}

\begin{frame}[fragile]{Motivações para o estudo de strings}

    \begin{itemize}
        \item Sequências de caracteres, ou strings, constituem uma maneira natural de representar informações
        \pause

        \item As strings aparecem em diversas áreas além da Computação, como a Biologia (estudo das moléculas e DNA), Letras (ortografia, sintaxe e morfologia), Criptografia (codificação e decodificação de mensagens), dentre outras
        \pause

        \item O algoritmo fundamental para o estudo e entendimento de strings é o \textit{pattern 
        matching}, que consiste na localização informações (padrões) em um texto (string)
        \pause

        \item A importância do \textit{pattern matching} para o estudo das strings equivale à 
            importância dos algoritmos de ordenação no estudo de algoritmos

    \end{itemize}

\end{frame}

\begin{frame}[fragile]{Motivações para o estudo de strings}

    \begin{itemize}
        \item Os padrões a serem localizados podem ser exatos, ou escritos em uma representação 
            que utiliza caracteres especiais para marcar sequências ou repetições, denominada 
            \textit{regex} (\textit{regular expressions})
        \pause

        \item A linguagem awk (Aho, Weinberger, Kernighan) é inteiramente baseada em expressões 
            regulares e é focada na manipulação de strings
        \pause

        \item O ambiente UNIX dispõe de várias ferramentas para textos (\texttt{grep, cat, more, less sed, diff}, etc), que permitem \textit{pattern matching}, exibição, busca, identificação, filtragem, e manipulação de strings, dentre outros
        \pause

        \item Estas ferramentas podem ser utilizadas isoladamente ou em conjunto, oferecendo uma grande gama de opções aos seus usuários
    \end{itemize}

\end{frame}
