\section{Variantes da LCS}

\begin{frame}[fragile]{Identificação da LCS}

    \begin{itemize}
        \item Assim como o problema de \textit{edit distance}, uma variante comum do LCS é
            determinar a sequência de operações que leva à maior subsequência comum

        \item A implementação é idêntica à proposta para $edit(S, T)$, uma vez aplicada a 
            modificação dos pesos e a alteração da operação \code{cpp}{min()} por 
            \code{cpp}{max()}, de modo que a complexidade permanece sendo $O(nm)$

        \item A maior subsequência comum corresponde aos caracteres onde os caracteres foram
            mantidos

        \item Assim esta rotina pode ser modificada para exibir a sequência, e não as operações
            que levaram a ela
    \end{itemize}

\end{frame}

\begin{frame}[fragile]{Identificação da LCS em C++}
    \inputsnippet{cpp}{74}{94}{lcs2.cpp}
\end{frame}

\begin{frame}[fragile]{Identificação da LCS em C++}
    \inputsnippet{cpp}{95}{115}{lcs2.cpp}
\end{frame}

\begin{frame}[fragile]{Identificação da LCS em C++}
    \inputsnippet{cpp}{116}{132}{lcs2.cpp}
\end{frame}
