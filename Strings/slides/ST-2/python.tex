\section{Strings em Python}

\begin{frame}[fragile]{Representação de strings em Python}

    \begin{itemize}
        \item Embora as strings em Python também sejam implementadas através de classes, elas 
            podem ser vistas informalmente como listas de caracteres

        \item Em Python, constantes do tipo string podem ser representados usando-se aspas
            simples ou duplas, ou mesmo triplas (para strings com múltiplas linhas)

        \item Para maratonas de programação, o módulo \code{py}{string} da linguagem Python traz 
            constantes bastantes úteis, como listagens de caracteres comuns:

            \inputsyntax{py}{constants.py}
    \end{itemize}

\end{frame}

\begin{frame}[fragile]{Representação de strings em Python}

    \begin{itemize}
        \item Mesmo que a solução proposta pela equipe seja escrita em outra linguagem, estas 
            constantes podem ser facilmente acessadas via terminal, importando o módulo e usando o 
            comando (ou função, no Python 3) \code{py}{print}

        \item Outra particularidade do Python é que, ao contrário das linguagens C e C++, ele 
            suporta índices negativos para strings

        \item Por exemplo, \code{py}{s[-1]} se refere ao último caractere, \code{py}{s[-2]} ao 
            penúltimo, e assim por diante

        \item Outra notação útil é \code{py}{s[::-1]}, que indica o reverso da string 
            \code{py}{s} (isto é, \code{py}{s} lida do fim para o começo)

        \item A API para strings em Python contempla ainda muitas outras funções úteis, como 
            \code{py}{strip(), join()} e \code{py}{split()}
    \end{itemize}

\end{frame}

\begin{frame}[fragile]{Exemplo de uso de strings em Python}
    \inputsnippet{py}{1}{21}{strings.py}
\end{frame}

\begin{frame}[fragile]{Exemplo de uso de strings em Python}
    \inputsnippet{py}{22}{42}{strings.py}
\end{frame}

\begin{frame}[fragile]{Exemplo de uso de strings em Python}
    \inputsnippet{py}{43}{60}{strings.py}
\end{frame}

\begin{frame}[fragile]{Exemplo de uso de strings em Python}
    \inputsnippet{py}{61}{81}{strings.py}
\end{frame}
