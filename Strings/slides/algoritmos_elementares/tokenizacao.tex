\section{Tokenização}

\begin{frame}[fragile]{Definição}

    \begin{itemize}
        \item Tokenização é o processo de quebra de um texto em palavras, frases, símbolos, etc, 
            denominados \textit{tokens}
        \pause

        \item Para realizar a tokenização, é necessário primeiramente definir precisamente um 
            \textit{token}, e esta definição depende do contexto onde os \textit{tokens} serão 
            utilizados
        \pause

        \item Uma abordagem simples é definir uma lista de caracteres delimitadores (como espaços 
            em branco, pontuações, etc) e dividir a string a cada ocorrência de um delimitador
        \pause

        \item Por exemplo, a frase \code{cpp}{"O rato roeu a roupa do rei de Roma"} seria dividida 
            em 9 tokens: \code{cpp}{"O"}, \code{cpp}{"rato"}, \code{cpp}{"roeu"}, \code{cpp}{"a"}, 
            \code{cpp}{"roupa"}, \code{cpp}{"do"}, \code{cpp}{"rei"}, \code{cpp}{"de"}, \code{cpp}{"Roma"}
    \end{itemize}

\end{frame}

\begin{frame}[fragile]{\code{c}{strtok()}}

    \begin{itemize}
        \item A linguagem C oferece uma função para tokenização na biblioteca \code{c}{string.h}, 
            denominada \code{c}{strtok()}
        \pause

        \item Ela recebe como primeiro parâmetro um ponteiro para a string a ser tokenizada e como 
            segundo parâmetro uma lista de delimitadores
        \pause

        \item Ela retorna um ponteiro para o início do próximo token, ou \code{c}{NULL}, caso não 
            exista mais tokens
        \pause

        \item Esta função tem um comportamento incomum, devido três aspectos importantes:
            \begin{enumerate}
        \pause
                \item ela mantém, internamente, um ponteiro para o início do próximo \textit{token}
        \pause
                \item esta função altera o parâmetro de entrada, escrevendo o caractere \code{cpp}{'\0'} nas posições onde são encontrados delimitadores
        \pause
                \item a função strtok() não retorna \textit{tokens} vazios (isto é, de tamanho zero)
            \end{enumerate}
    \end{itemize}

\end{frame}

\begin{frame}[fragile]{\code{c}{strtok()}}

    \begin{itemize}
        \item Devido ao aspecto 1, para obter vários tokens de uma mesma string, as chamadas 
            subsequentes à primeira devem receber \code{c}{NULL} como primeiro parâmetro
        \pause

        \item Este comportamento faz com que esta função não seja segura num contexto \textit{multithread}
        \pause

        \item Já em relação ao aspecto 2, se for necessário preservar o conteúdo original da string, deve ser passada uma cópia da mesma como primeiro parâmetro
        \pause

        \item Por fim, por conta do aspecto 3, a chamada de \code{c}{strtok()} na string
            \code{cpp}{"aaaaa"} com delimitador \code{cpp}{"a"} retorna \code{c}{NULL} já na 
            primeira chamada
        \pause

        \item A biblioteca \code{c}{string.h} oferece uma versão \textit{thread safe} de \code{c}{strtok()}, denominada \code{c}{strtok_r()}
        \pause

        \item Nesta função há um terceiro parâmetro, que é utilizado para memorizar a posição do próximo \textit{token}
    \end{itemize}

\end{frame}

\begin{frame}[fragile]{Exemplo de uso da função \code{c}{strtok()}}
    \inputsnippet{c}{1}{21}{codes/cpf.c}
\end{frame}

\begin{frame}[fragile]{Tokenização em C++}

    \begin{itemize}
        \item Em C++, a função \code{c}{getline()} da biblioteca \code{c}{string} pode ser 
            utilizada para tokenização
        \pause

        \item Ela recebe como primeiro parâmetro um fluxo de entrada (\code{c}{cin}, um arquivo, 
            um fluxo de caracteres, etc)
        \pause

        \item O segundo parâmetro é a string que conterá o \textit{token} identificado 
        \pause

        \item O terceiro parâmetro é o caractere delimitador
        \pause

        \item Se omitido, o caractere \code{cpp}{'\n'} será considerado o delimitador
        \pause

        \item Esta função é útil quando há apenas um delimitador
        \pause

        \item Para processos de tokenização mais elaborados pode ser necessário escrever um \textit{parser}

    \end{itemize}

\end{frame}

\begin{frame}[fragile]{Exemplo de uso da função \code{c}{getline()}}
    \inputsnippet{cpp}{1}{21}{codes/cpf.cpp}
\end{frame}
