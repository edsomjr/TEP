\section{Anagramas}

\begin{frame}[fragile]{Definição e identificação de anagramas}

    \begin{itemize}
        \item Anagramas são palavras formadas pelo rearranjo dos caracteres de um conjunto fixo

        \item Por exemplo, ``\texttt{iracema}"\  e ``\texttt{america}"\  são anagramas, enquanto que ``\texttt{amora}"\  e ``\texttt{roma}"\  não são anagramas

        \item ``\texttt{roma}"\  tem os mesmo caracteres, mas não a mesma quantidade: tem um ``\texttt{a}"\ a menos que ``\texttt{amora}"

        \item Para se determinar se determinar se duas strings $s$ e $t$ são anagramas há duas abordagens possíveis:
            \begin{enumerate}
                \item obter os histogramas de ambas strings e compará-los: caso sejam iguais, as strings serão anagramas
                \item ordenar ambas strings segundo a ordem lexicográfica: se após a ordenação as strings são iguais, ambas são anagramas
            \end{enumerate}
    \end{itemize}

\end{frame}

\begin{frame}[fragile]{Implementação da verificação de anagramas}
    \inputcode{cpp}{anagram.cpp}
\end{frame}

\begin{frame}[fragile]{Listagem de todos os anagramas}

    \begin{itemize}
        \item Um problema comum é determinar o número de anagramas distintos que uma determinada 
            palavra tem

        \item Segundo a Análise Combinatória, este número é dado por um arranjo com repetição

        \item Se $s$ tem $n$ caracteres ($r$ deles distintos) e $n_1, n_2, \ldots, n_r$ é o número 
            de ocorrências de cada um dos $r$ caracteres em $s$, então o número de anagramas 
            distintos $A(s)$ de $s$ é dado por
            \[
                A(s) = \frac{n!}{n_1!n_2!...n_r!}
            \]

        \item Para listar todos os possíveis anagramas de uma string $s$, pode-se utilizar a 
            função \code{cpp}{next_permutation()} da biblioteca \code{cpp}{algorithm} do C++

        \item Ela retorna verdadeiro, e modifica a string passada, enquanto houver uma próxima 
            permutação distinta de seus caracteres

        \item Deve-se tomar o cuidado, porém, de ordenar a string $s$ antes das sucessivas 
            chamadas da função \code{cpp}{next_permutation()}
    \end{itemize}

\end{frame}

\begin{frame}[fragile]{Exemplo de listagem de todos os anagramas de uma palavra}
    \inputcode{cpp}{next_permutation.cpp}
\end{frame}
