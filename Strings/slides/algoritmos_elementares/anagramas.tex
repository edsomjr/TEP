\section{Anagramas}

\begin{frame}[fragile]{Definição e identificação de anagramas}

    \begin{itemize}
        \item Anagramas são palavras formadas pelo rearranjo dos caracteres de um conjunto fixo
        \pause

        \item Por exemplo, \code{cpp}{"iracema"}\  e \code{cpp}{"america"}\  são anagramas, enquanto que \code{cpp}{"amora"}\  e \code{cpp}{"roma"}\  não são anagramas
        \pause

        \item \code{cpp}{"roma"}\  tem os mesmo caracteres, mas não a mesma quantidade: tem um \code{cpp}{"a"}\ a menos que \code{cpp}{"amora"}
        \pause

        \item Para se determinar se determinar se duas strings $s$ e $t$ são anagramas há duas abordagens possíveis:
        \pause
            \begin{enumerate}
                \item obter os histogramas de ambas strings e compará-los: caso sejam iguais, as strings serão anagramas
        \pause
                \item ordenar ambas strings segundo a ordem lexicográfica: se após a ordenação as strings são iguais, ambas são anagramas
            \end{enumerate}
    \end{itemize}

\end{frame}

\begin{frame}[fragile]{Implementação da verificação de anagramas}
    \inputcode{cpp}{codes/anagram.cpp}
\end{frame}

\begin{frame}[fragile]{Listagem de todos os anagramas}

    \begin{itemize}
        \item Um problema comum é determinar o número de anagramas distintos que uma determinada 
            palavra tem
        \pause

        \item Segundo a Análise Combinatória, este número é dado por um arranjo com repetição
        \pause

        \item Se $s$ tem $n$ caracteres ($r$ deles distintos) e $n_1, n_2, \ldots, n_r$ é o número 
            de ocorrências de cada um dos $r$ caracteres em $s$, então o número de anagramas 
            distintos $A(s)$ de $s$ é dado por
            \[
                A(s) = \frac{n!}{n_1!n_2!...n_r!}
            \]
        \pause

        \item Para listar todos os possíveis anagramas de uma string $s$, pode-se utilizar a 
            função \code{cpp}{next_permutation()} da biblioteca \code{cpp}{algorithm} da linguagem C++
        \pause

        \item Ela retorna verdadeiro, e modifica a string passada, enquanto houver uma próxima 
            permutação distinta de seus caracteres
        \pause

        \item Deve-se tomar o cuidado, porém, de ordenar a string $s$ antes das sucessivas 
            chamadas da função \code{cpp}{next_permutation()}
    \end{itemize}

\end{frame}

\begin{frame}[fragile]{Exemplo de listagem de todos os anagramas de uma palavra}
    \inputcode{cpp}{codes/next_permutation.cpp}
\end{frame}
