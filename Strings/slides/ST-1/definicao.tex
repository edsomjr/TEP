\section{Definições}

\begin{frame}[fragile]{Definição de string}

    \begin{itemize}
        \item Um alfabeto $A$ é um conjunto finito de símbolos

        \item Os elementos de um alfabeto $A$ são denominados letras, caracteres ou símbolos

        \item Exemplos de alfabeto comumente utilizados são as letras maiúsculas e minúsculas, 
            os dígitos decimais e a tabela ASCII

        \item Uma string $s$ (ou texto ou palavra) é uma sequência ordenada $s = \lbrace a_1, a_2, 
        \ldots, a_N\rbrace$ de caracteres $a_i$ de um alfabeto $A$

        \item O $i$-ésimo termo de $s$ também é denotado por $s_i$ ou $s[i]$

        \item O número $N$ de elementos da sequência $s$ pode ser notado como $|s|$

    \end{itemize}

\end{frame}


\begin{frame}[fragile]{Substrings}

    \begin{itemize}
        \item Um intervalo é uma subsequência contígua 
        \[
            s[i..j] = s[i]s[i+1]\ldots s[j]
        \] de elementos de $s$

        \item Observe que $|s[i..j]| = j - i + 1$ 

        \item Uma substring $b$ de $s$, com $|b| = M$, é uma string $b$ tal que $b = s[(i+1)..(i+M)]$ 
            para algum inteiro $i$

        \item Uma subsequência $a = s[i_1]s[i_2]\ldots s[i_M]$ de uma string $s$ pode ser obtida a 
            partir da remoção de zero ou mais elementos de $s$, não necessariamente consecutivos

        \item Os inteiros $i_1, i_2, ..., i_M$ formam uma sequência crescente de índices de $s$ 
            (isto é, $i_u < i_v$ para $u < v$)
    \end{itemize}

\end{frame}


\begin{frame}[fragile]{Prefixos e sufixos}

    \begin{itemize}
        \item Um prefixo de uma string $s$ é uma substring $p$, de tamanho $|p| = M$, tal que 
            $p = s[1..M]$

        \item Um sufixo $x$ de $s$, de tamanho $|x| = T$, é uma substring de $s$ tal que 
            $x = s[(N - T + 1)..N]$, onde $|s| = N$

        \item Uma borda $B$ de uma string $s$ é uma substring que é, simultaneamente, prefixo e 
            sufixo de $s$

        \item Uma vez que a string vazia (isto é, $|s| = 0$) e a própria string $s$ são sempre 
            bordas (triviais) de $s$, define-se $border(s)$ como a mais longa 
            (de maior tamanho) dentre as bordas de $s$ que são distintas da própria string $s$

        \item Por exemplo, as strings \lq\lq\texttt{ame}", \lq\lq\texttt{rica}"\ e 
            \lq\lq\texttt{a}"\ são exemplos 
            de prefixo, sufixo e borda da string \lq\lq\texttt{america}", respectivamente
    \end{itemize}

\end{frame}


\begin{frame}[fragile]{Período de uma string}

    \begin{itemize}

        \item Um período de uma string $s$ é um inteiro $p$, $0 < p \leq |s|$ tal que 
            $s[i] = s[i + p]$, para todo $i = {1, \ldots, |s| - p}$

        \item Para qualquer string, $|s|$ é um período, de modo que define-se 
            $period(s)$ como o menor período de $s$

        \item A string $s$ é dita periódica se $period(s) \leq |s|/2$

        \item Por exemplo, para as strings $s_1$ = \lq\lq\texttt{marítima}", 
            $s_2$ = \lq\lq\texttt{ticotico}"\ e $s_3$ = \lq\lq\texttt{Brasilia}", temos 
            $period(s_1) = 6, period(s_2) = 4$ e $period(s_3) = 8$

        \item Dentre as três, apenas $s_2$ é períodica
    \end{itemize}

\end{frame}

\begin{frame}[fragile]{Lemas de Periodicidade}

    Os diferentes períodos de uma mesma string $s$ se relacionam de uma maneira não trivial, que 
        pode ser expressa pelos dois lemas a seguir.

    \metroset{block=fill}                                                                           
    \begin{block}{Lema da Periodicidade}
        Sejam $p$ e $q$ dois períodos de uma string $s$. Se $p + q < |s|$, então 
            $\mbox{mdc}(p, q)$ também é período de $s$.
    \end{block} 

    \vspace{0.1in}

    \metroset{block=fill}                                                                           
    \begin{block}{Lema Forte da Periodicidade}
        Sejam $p$ e $q$ dois períodos de uma string $s$. Se $p + q - \mbox{mdc}(p, q) \leq |s|$, 
            então $\mbox{mdc}(p, q)$ também é período de $s$.
    \end{block} 

\end{frame}

\begin{frame}[fragile]{Relação entre períodos e bordas}

    \begin{itemize}
        \item Há uma interessante relação entre bordas e períodos

        \item A sequência
        \[
            |s| - |border(s)|, |s| - |border^2(s)|, ..., |s| - |border^k(s)|
        \]
        é a sequência crescente de todos os possíveis períodos de $s$, onde $k$ é o menor inteiro 
        positivo tal que $border^k(s)$ é uma string vazia

        \item Por exemplo, para a string $s$ = \lq\lq\texttt{teteatete}", tem-se $|s| = 9$ e
        \begin{align*}
            & border(s) = ``\mbox{\texttt{tete}}" \\
            & border^2(s) = border(``\mbox{\texttt{tete}}") = ``\mbox{\texttt{te"}} \\
            & border^3(s) = border(``\mbox{\texttt{te}}") = ``\,"
        \end{align*}
        os quais formam a sequência de períodos $9 - 4 = 5, 9 - 2 = 7$ e $9 - 0 = 9$
    \end{itemize}

\end{frame}

\begin{frame}[fragile]{\textit{Pattern matching}}

    \begin{itemize}
        \item O problema fundamental de strings é o \textit{pattern matching}

        \item Dada uma string $P$, que representa um padrão, o \textit{pattern matching} consiste
            em determinar se $P$ ocorre ou não em $s$

        \item O \textit{pattern matching} é um problema de decisão, isto é, a resposta é booleana: 
            o padrão ocorre ou não, embora uma variante comum é determina o índice da primeira
            posição onde $P$ ocorre ou um valor sentinela, caso não ocorra

        \item Em geral, $|P| \leq |s|$

        \item Uma notação possível é $match(P, s)$

    \end{itemize}

\end{frame}
