\section{Maior subsequência palíndroma}

\begin{frame}[fragile]{Definição}

    \begin{itemize}
        \item Uma variante da maior sequência comum é o problema de se encontrar a maior 
            subsequência de uma string $S$ que forma um palíndromo (\textit{Longest Palindrome 
            Subsequence} -- $LPS$)

        \item Uma maneira de se enunciar este problema é a seguinte: qual é o maior palíndromo que 
            pode ser formado removendo $m$ ($0 \leq m \leq n$) caracteres, de quaisquer posições, 
            de uma string $S$ de tamanho $n$?

        \item Este problema sempre tem solução, pois uma string com apenas um caractere é um 
            palíndromo (o mesmo vale para strings vazias)

        \item O tamanho do maior palíndromo, ou o palíndromo em si, pode ser determinado por
            meio de programação dinâmica
    \end{itemize}
\end{frame}

\begin{frame}[fragile]{Formulação em programação dinâmica da LPS}

    \begin{itemize}
        \item O fato de uma string vazia ou com um único caractere serem palíndromos fornecem os
            casos bases

        \item Se $LPS[i,j]$ é o tamanho da maior subsequência palíndroma da substring 
            $S[i..j]$, então
            \[
                LPS[i,i] = 1, \ \ \ \ \ \ LPS[i,j] = 0, \ \ \ \ \mbox{se}\ i > j
            \]

        \item São três transições possíveis:
        a primeira é remover o caractere mais à esquerda de $S[i..j]$, isto é,
        \[
             LPS[i,j] = LPS[i+1,j]
        \]

        \item A segunda transição remove o caractere mais à direita $S[i..j]$:
        \[
            LPS[i, j] = LPS[i, j - 1]
        \] 

        \item No último caso, casos os caracteres que estão nos extremos da strings sejam iguais, 
            eles podem ser parte do palíndromo: 
        \[
            LPS[i,j] = LPS[i+1,j-1] + 2,\ \ \ \ \mbox{se}\ S[i] = S[j]
        \]

    \end{itemize}

\end{frame}

\begin{frame}[fragile]{Implementação {\it top-down} da LPS}
    \inputsnippet{cpp}{8}{25}{codes/lps.cpp}
\end{frame}

\begin{frame}[fragile]{Implementação {\it top-down} da LPS}
    \inputsnippet{cpp}{27}{47}{codes/lps.cpp}
\end{frame}

\begin{frame}[fragile]{Identificação da maior subsequência palíndroma}

    \begin{itemize}
        \item O algoritmo proposto para a LPS tem complexidade $O(n^2)$ tanto para a execução 
            quanto para a memória

        \item Para recuperar a string que corresponde à LPS, é preciso manter o registro das
            operações utilizadas em cada transição:
            \begin{itemize}
                \item \code{cpp}{'B'}: os caracteres dos extremos são mantidos
                \item \code{cpp}{'K'}: o único caractere da string é mantido
                \item \code{cpp}{'L'}: remover o primeiro caractere 
                \item \code{cpp}{'R'}: remover o último caractere 
            \end{itemize}

        \item Usando um valor sentinela (zero) para estados que não forem atingidos, é possível
            remontar o palíndromo em $O(n^2)$
    \end{itemize}

\end{frame}

\begin{frame}[fragile]{Implementação da identificação da LPS}
    \inputsnippet{cpp}{9}{28}{codes/lps2.cpp}
\end{frame}

\begin{frame}[fragile]{Implementação da identificação da LPS}
    \inputsnippet{cpp}{30}{47}{codes/lps2.cpp}
\end{frame}

\begin{frame}[fragile]{Implementação da identificação da LPS}
    \inputsnippet{cpp}{49}{68}{codes/lps2.cpp}
\end{frame}

\begin{frame}[fragile]{Implementação da identificação da LPS}
    \inputsnippet{cpp}{70}{85}{codes/lps2.cpp}
\end{frame}
