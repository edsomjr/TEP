\section{Strings em C++}

\begin{frame}[fragile]{Representação de strings em C++}

    \begin{itemize}
        \item Embora seja possível utilizar a abordagem e as bibliotecas da linguagem C em C++, 
            existe uma representação em C++ de strings baseada em classes
        \pause

        \item O uso de classes para representar strings traz a vantagem de poder manter outras 
            informações sobre a string sempre atualizadas e com acesso em $O(1)$
        \pause

        \item Por outro lado, esta representação demanda mais memória do que a representação em C
        \pause

        \item Existem técnicas para tentar reduzir a memória utilizada, como a \textit{small string optimization}
        \pause

        \item A classe fundamental dentre as várias classes que representam strings em C++ é a
            \code{cpp}{std::string}.
        \pause

        \item O método \code{cpp}{c_str()} permite obter, a partir de uma string C++, uma 
            representação compatível com a utilizada em C
        \pause

        \item Deste modo, é possível utilizar funções escritas para strings em C a partir de 
            instâncias da classe da linguagem C++
    \end{itemize}

\end{frame}

\begin{frame}[fragile]{Exemplo de uso da classe string em C++}
    \inputsnippet{cpp}{1}{19}{codes/string.cpp}
\end{frame}

\begin{frame}[fragile]{Exemplo de uso da classe string em C++}
    \inputsnippet{cpp}{21}{38}{codes/string.cpp}
\end{frame}

\begin{frame}[fragile]{Exemplo de uso da classe string em C++}
    \inputsnippet{cpp}{40}{57}{codes/string.cpp}
\end{frame}

\begin{frame}[fragile]{Exemplo de uso da classe string em C++}
    \inputsnippet{cpp}{59}{65}{codes/string.cpp}
\end{frame}
