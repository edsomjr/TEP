\section{Busca em strings na STL}

\begin{frame}[fragile]{Função {\tt search()} da STL}

    \begin{itemize}
        \item A função \code{cpp}{search()} da STL busca a primeira ocorrência da sequência de
            elementos $[\mathtt{a}, \mathtt{b})$ no intervalo $[\mathtt{first},\mathtt{last})$:

            \vspace{.1in}
            \inputsyntax{cpp}{codes/search1.st}
            \vspace{.1in}
        \pause

        \item Sendo uma função paramétrica, ela pode ser aplicada no contexto de busca em
            strings
        \pause

        \item Por exemplo, para procurar a primeira ocorrência do padrão $P$ em $S$ a chamada
            seria

            \vspace{.1in}
            \inputcode{cpp}{codes/search1.cpp}
            \vspace{.1in}
        \pause

        \item A string $S$ tem tamanho $n$ e o padrão $P$ tem tamanho $m$, a complexidade será
            $O(nm)$

    \end{itemize}

\end{frame}

\begin{frame}[fragile]{Função {\tt search()} da STL}

    \begin{itemize}
        \item A versão C++17 da STL trouxe uma assinatura adicional para a função 
            \code{cpp}{search()}:

            \vspace{.1in}
            \inputsyntax{cpp}{codes/search2.st}
            \vspace{.1in}
        \pause

        \item Deste modo, é possível especificar o algoritmo de busca a ser utilizado para
            o localizar o padrão indicado no construtor de \code{cpp}{search} na string
            delimitada pelo intervalo $[\mathtt{begin}, \mathtt{last})$
        \pause

        \item A biblioteca padrão fornece três algoritmos:
        \begin{enumerate}
            \item \code{cpp}{default_searcher}
            \item \code{cpp}{boyer_moore_searcher}
            \item \code{cpp}{boyer_moore_horspool_searcher}
        \end{enumerate}
        \pause

        \item É possível implementar um \code{cpp}{Searcher} customizado
    \end{itemize}

\end{frame}

\begin{frame}[fragile]{Função {\tt search()} da STL}

    \begin{itemize}

        \item O primeiro é o algoritmo utilizado nas demais versões da função \code{cpp}{search()}
        \pause

        \item O segundo implementa o algoritmo de Boyer-Moore, cuja complexidade assintótica é
            $O(n + m)$ no pior caso
        \pause

        \item O terceiro algoritmo é uma versão simplificado do algoritmo de Boyer-Moore, que 
            exige menos memória
        \pause

        \item Esta redução de memória, porém, implica em uma complexidade $O(nm)$ no pior caso
        \pause

        \item Embora o algoritmo de Boyer-Moore tenha sido proposto inicialmente como um algoritmo
            de busca em strings, no caso da STL ele pode ser utilizado em um contêiner que
            armazena um tipo \code{cpp}{T} arbitrário
    \end{itemize}

\end{frame}

\begin{frame}[fragile]{Teste de performance dos algoritmos de busca da STL}
    \inputsnippet{cpp}{1}{20}{codes/search2.cpp}
\end{frame}

\begin{frame}[fragile]{Teste de performance dos algoritmos de busca da STL}
    \inputsnippet{cpp}{22}{41}{codes/search2.cpp}
\end{frame}

\begin{frame}[fragile]{Teste de performance dos algoritmos de busca da STL}
    \inputsnippet{cpp}{43}{64}{codes/search2.cpp}
\end{frame}
