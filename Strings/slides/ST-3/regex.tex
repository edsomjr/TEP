\section{Expressões Regulares}

\begin{frame}[fragile]{Regex}

    \begin{itemize}
        \item Expressão regular (\textit{regular expression} ou \textit{regex}) é uma 
            representação que utiliza símbolos especiais para marcar sequências de caracteres 
            ou repetições

        \item As \textit{regexes} são uma forma compacta e poderosa de representar textos e padrões

        \item Porém, sem o devido cuidado, pode levar a \textit{bugs} sutis relacionados a falsos 
            positivos (o texto identificado não é o desejado) e a falsos negativos 
            (textos desejados não são identificados)

        \item Algumas linguagens de programação, como Java e Python, tem suporte nativo para 
            expressões regulares

        \item A linguagem C++ incorporou suporte às \textit{regexes} a partir da versão C++11, 
            com padrão de sintaxe distinta das outras duas linguagens já citadas
    \end{itemize}

\end{frame}

\begin{frame}[fragile]{Características de uma regex}

    \begin{enumerate}
        \item cada caractere da \textit{regex} corresponde ao mesmo caractere no texto
        \item existem combinações especiais de caracteres para corresponder à sequências de caracteres: por exemplo, \verb|\d| corresponde a qualquer um dos dez dígitos decimais, \verb|\D| a qualquer caractere, exceto os dez dígitos decimais
        \item o caractere \lq .\rq\ é o coringa: ele representa qualquer caractere. O ponto final é representado pela sequência \verb|\.|
        \item um conjunto de caracteres válidos para o padrão pode ser representado por meio de colchetes. Por exemplo, a notação \verb|[abc]| significa ou \texttt{a}, ou \texttt{b} ou \texttt{c}. Se os caracteres são consecutivos, esta notação pode ser abreviada com o uso do símbolo \texttt{-}. Por exemplo, \verb|[0-9]| tem o mesmo significado que \verb|\d|
    \end{enumerate}

\end{frame}

\begin{frame}[fragile]{Características de uma regex}

    \begin{enumerate}
        \setcounter{enumi}{4}

        \item a notação de colchetes pode ser usada para excluir caracteres, se usada com conjunto com o símbolo \verb|^|. Por exemplo, a notação \verb|[^abc]| significa ``todos os caracteres, exceto \texttt{a}, \texttt{b} e \texttt{c}"
        \item a sequência \verb|\w| corresponde aos caracteres alfanuméricos \verb|[a-zA-Z0-9_]|;
        \item Sequências especiais de caracteres podem ser usadas para representar repetições de caracteres ou padrões:
        \begin{enumerate}
            \item um número entre chaves após o caractere/padrão indica o número de repetições ou 
                as quantidades válidas de repetições. Por exemplo, \verb|a{5}| significa o mesmo 
                que ``\texttt{aaaaa}"; \verb|a{1-3}| indica ``de um a três caracteres a"; 
                \verb|[abc]{2}| indica dois caracteres seguidos dentres os indicados no colchete
            \item o caractere \lq \verb|*|\rq\ significa ``zero ou mais repetições"
            \item o caractere \lq \verb|+|\rq\ significa ``uma ou mais repetições"

        \end{enumerate}
    \end{enumerate}

\end{frame}

\begin{frame}[fragile]{Características de uma regex}

    \begin{enumerate}
        \setcounter{enumi}{7}
        \item[] \begin{enumerate}
            \setcounter{enumii}{3}
            \item o símbolo \verb|?| significa que o caractere ou padrão é opcional, isto é, que pode ou não ocorrer no texto
            \item O caractere \lq \verb|?|\rq\ pode ser representado pela sequência de escape 
                \verb|\?|
            \item a sequência \verb|\s| indica espaços em branco (\lq\ \rq, \lq\verb|\t|\rq, 
                \lq\verb|\r|\rq, \lq\verb|\n|\rq)
            \item os símbolos \verb|^| e \verb|$| representam, respectivamente, o início e o fim 
                do texto
            \item parêntesis podem ser utilizados para armazenar trechos do texto que correspondem à expressão entre parêntesis para posterior uso. Eles podem ser aninhados
            \item o símbolo \texttt{|} pode ser utilizado como ou lógico para separar grupos de padrões possíveis: a expressão \texttt{(abc|123)} significa ``ou \texttt{a, b, c} ou \texttt{1, 2, 3}"
        \end{enumerate}
    \end{enumerate}

\end{frame}

\begin{frame}[fragile]{Exemplo de uso de regex em Java}
    \inputcode{java}{Main.java}
\end{frame}
