\section{Algoritmo de Rabin-Karp}

\begin{frame}[fragile]{Definição}

    \begin{itemize}
        \item O algoritmo de Rabin-Karp é um algoritmo que contabiliza o número de ocorrências
            da string $P$, de tamanho $m$, na string $S$, de tamanho $n$
        \pause

        \item Ele foi proposto por Michael O. Rabin e Richard M. Karp em 1987
        \pause

        \item A ideia principal do algoritmo é computar o \textit{hash} $h_P = h(P)$ e compará-lo
            com todas as substrings $h_{ij} = S[i..j]$ de $S$ de tamanho $m$
        \pause

        \item Caso $h_P \neq h_{ij}$, segue que $P\neq S[i..j]$ e o algoritmo pode prosseguir
        \pause

        \item Se $h_P = h_{ij}$, as strings ou são iguais ou houve uma colisão
        \pause

        \item Esta dúvida pode ser sanada através da comparação direta, enquanto strings, entre
            $S[i..j]$ e $P$
        \pause

        \item O algoritmo tem complexidade $O(mn)$ no pior caso, por conta do custo do cálculo
            dos {\it hashes} e das possíveis comparações diretas entre as strings

    \end{itemize}

\end{frame}

\begin{frame}[fragile]{Pseudocódigo do algoritmo de Rabin-Karp}

    \begin{algorithm}[H]
        \caption{Algoritmo de Rabin-Karp -- Naive}
        \begin{algorithmic}[1]
            \Require Duas strings $P$ e $S$ e uma função de \textit{hash} $h$
            \Ensure O número de ocorrências $occ$ de $P$ em $S$

            \Function{RabinKarp}{$P$,$S$}
                \State $m \gets |P|$
                \State $n \gets |S|$
                \State $occ \gets 0$
                \State $h_P \gets h(P)$

                \For {$i \gets 1 \forto n - m + 1$}
                    \State $h_S \gets h(S[i..(i + m - 1)])$
                    \If {$h_S = h_P$}
                        \If {$S[i..(i + m - 1)] = P$}
                            \State $occ \gets occ + 1$
                        \EndIf
                    \EndIf
                \EndFor

                \State \Return $occ$
            \EndFunction
        \end{algorithmic}
    \end{algorithm}

\end{frame}

\begin{frame}[fragile]{Implementação do algoritmo de Rabin-Karp em Haskell}
    \inputsnippet{haskell}{1}{19}{codes/rk.hs}
\end{frame}

\begin{frame}[fragile]{Implementação do algoritmo de Rabin-Karp em C++}
    \inputsnippet{cpp}{1}{19}{codes/rk.cpp}
\end{frame}

\begin{frame}[fragile]{Implementação do algoritmo de Rabin-Karp em C++}
    \inputsnippet{cpp}{21}{32}{codes/rk.cpp}
\end{frame}
