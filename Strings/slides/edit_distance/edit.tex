\section{\it Edit Distance}

\begin{frame}[fragile]{Falha no algoritmo de busca em string}

    \begin{itemize}
        \item  Sejam $S$ e $T$ duas strings, com $S \neq T$. Quando comparadas no que diz respeito
            a igualdade, há três motivos comuns que levam ao veredito falso 
        \pause

        \item O primeiro é que ambas strings tem mesmo tamanho, mas diferem em um ou mais símbolos
        \pause

        \item Por exemplo, se $S = $ \code{cpp}{"banana"} e $T = $ \code{cpp}{"bacana"}, 
            $|S| = |T|$, mas $S[3] \neq T[3]$
        \pause

        \item O segundo ocorre quando a primeira string é mais longa que a segunda, e poderíam se 
            tornar iguais se removidos os caracteres excedentes
        \pause

        \item Por exemplo, se $S = $ \code{cpp}{"aspectos"} e $T = $ \code{cpp}{"seco"}, segue que
            $|S| > |T|$, mas teríamos $S = T$ se removidos os caracteres das posições 
            1, 3, 6 e 8 de $S$
    \end{itemize}

\end{frame}

\begin{frame}[fragile]{Falha no algoritmo de busca em string}

    \begin{itemize}
        \item O terceiro motivo é o fato da primeira string é mais curta do que a segunda, e 
            poderiam se igualar se adicionados os caracateres ausentes
        \pause

        \item Por exemplo, se $S = $ \code{cpp}{"fga"} e $T = $ \code{cpp}{"formigas"}, temos
            $|S| < |T|$, e ambas se tornariam iguais com a adição dos caracteres 
                \code{cpp}{"ormis"} em $S$, nas devidas posições
        \pause

        \item Na prática, é possível obter qualquer string $T$ a partir de uma string $S$ dada, 
            usando uma sequência de operações baseado nos três motivos listados
        \pause

        \item Estas operações são: alterar um caractere, adicionar um caractere ou remover um caractere
    \end{itemize}

\end{frame}

\begin{frame}[fragile]{\it Edit Distance}

    \begin{itemize}
        \item O problema denominado \textit{edit distance} consiste em determinar o número mínimo 
            de operações a serem feitas para transformar a string $S$ na string $T$
        \pause

        \item Em termos mais gerais, o problema é computar o custo mínimo desta transformação, se a cada operação for 
            associado um determinado custo
        \pause

        \item Este custo mínimo é denotado $edit(S, T)$ e tem as seguintes propriedades:
        \begin{enumerate}
            \item $edit(S, T)\geq 0$
            \item $edit(S, T) = 0$ se, e somente se, $S = T$
            \item $edit(S, T) = edit(T, S)$ (simetria)
            \item $edit(S, T) \leq edit(S, U) + edit(U, T)$ (desigualdade triangular)
        \end{enumerate}
    \end{itemize}

\end{frame}

\begin{frame}[fragile]{$edit(S, T)$ -- Caso base}

    \begin{itemize}
        \item Considere que $|S| = m, |T| = n$. Para determinar $edit(S, T)$ deve-se construir uma 
        tabela auxiliar de estados $st$, onde $st(i, j) = edit(S[1..i], T[1..j])$, com 
        $0 < i \leq m, 0 < j \leq n$
        \pause

        \item O casos bases acontecem quando uma das duas strings é vazia: nestes casos, o mínimo 
            de operações a serem feitas é igual  ao tamanho 
            da string não vazia
        \pause

        \item Em notação simbólica,
        \begin{align*}
            st(0, j) &= j \\
            st(i, 0) &= i
        \end{align*}
        \pause

        \item Se os custos de inserção, de remoção e de alteração forem $c_i, c_r, c_s$, 
            respectivamente, então os casos bases devem ser
        \begin{align*}
            st(0, j) &= j\times c_i \ \ \ \ \ \ \ \ \textcolor{gray}{\mbox{\tt \#\ } j\ \mbox{\tt inserções}}\\
            st(i, 0) &= i\times c_r \ \ \ \ \ \ \ \ \textcolor{gray}{\mbox{\tt \#\ } i\ \mbox{\tt remoções}}
        \end{align*}
    \end{itemize}

\end{frame}

\begin{frame}[fragile]{$edit(S, T)$ -- Transições}

    \begin{itemize}
        \item A transição entre os estados será, dentre as três operações, a de menor custo
        \pause

        \item Uma transição por inserção seria dada por
        \[
            st(i, j) = st(i, j - 1) + c_i,\ \ \ \ \ \ \ \ \textcolor{gray}{\mbox{\tt \# adicionar}\ T[j]}
        \]
        \pause

        \item Uma transição por remoção seria igual a
        \[
            st(i, j) = c_r + st(i - 1, j),\ \ \ \ \ \ \ \ \textcolor{gray}{\mbox{\tt \# remover}\ S[i]}
        \]
        \pause

        \item Uma transição por alteração corresponde a
        \begin{align*}
            &\textcolor{gray}{\mbox{\tt \# Mantém}\ S[i]\ \mbox{\tt ou substitui}\ S[i]\ \mbox{\tt por}\ T[j]} \\
            &st(i, j) = st(i - 1, j - 1) + \left\lbrace\begin{array}{ll}0, & \mbox{se}\ S[i] = T[j] \\ c_s,& \mbox{caso contrário}\end{array}\right.   
        \end{align*}
    \end{itemize}

\end{frame}

\begin{frame}[fragile]{$edit(S, T)$}

    \begin{itemize}
        \item Assim,
        \begin{align*}
            st(i, 0) &= i \times c_r \\
            st(0, j) &= j \times c_i \\
            st(i, j) &= \min\lbrace st(i, j - 1) + c_i, st(i - 1, j) + c_r, st(i - 1, j - 1) + k \rbrace
        \end{align*}
        onde $k = 0$ se $S[i] = T[j]$, ou $k = c_s$, caso contrário
        \pause

        \item Portanto, $edit(S, T) = st(m, n)$
        \pause

        \item Como a tabela tem $(m + 1)(n + 1)$ estados, e cada transição é feita em $O(1)$, o algoritmo tem 
            complexidade $O(mn)$
    \end{itemize}

\end{frame}

\input{edit_view}

\begin{frame}[fragile]{Implementação {\it bottom-up} de $edit(S, T)$}
    \inputsnippet{cpp}{9}{28}{codes/edit.cpp}
\end{frame}

