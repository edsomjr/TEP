\section{Variações de $edit(S, T)$}

\begin{frame}[fragile]{Implementação de $edit(S, T)$ com memória $O(n)$}

    \begin{itemize}
        \item A implementação anterior tem complexidade $O(nm)$ tanto para o tempo de execução
            quanto para a memória
        \pause

        \item Isto se deve à tabela de estados \code{cpp}{st}, que tem dimensões $m \times n$
        \pause

        \item  É possível implementar o mesmo algoritmo usando apenas $O(n)$ de memória, uma vez 
            que é necesário apenas a linha anterior para computar os valores da próxima linha
        \pause

        \item Esta segunda implementação pode ser necessária em competições ou ambientes com 
            restrição de memória
        \pause

        \item A complexidade do tempo de execução, porém, se mantém igual

    \end{itemize}

\end{frame}

\begin{frame}[fragile]{Implementação de $edit(S, T)$ com memória $O(n)$}
    \inputsnippet{cpp}{1}{19}{codes/edit2.cpp}
\end{frame}

\begin{frame}[fragile]{Implementação de $edit(S, T)$ com memória $O(n)$}
    \inputsnippet{cpp}{21}{42}{codes/edit2.cpp}
\end{frame}

\begin{frame}[fragile]{Sequência de operações ótima}

    \begin{itemize}
        \item Uma variante do problema {\it edit distance} é retornar um conjunto de operações
            $O = \lbrace o_1, o_2, \ldots, o_s\rbrace$ tal que $s = edit(S, T)$ e $o_j$ é uma
            das três operações: inserção, remoção ou alteração
        \pause

        \item Para obter tal sequência, na implementação com memória $O(nm)$, basta manter um 
            registro da operação responsável pela atualização de cada elemento da tabela
            \code{cpp}{st}
        \pause

        \item Ao final do algoritmo, uma sequência de operações que leva ao custo mínimo pode
            ser reconstruída por meio da recursão
        \pause

        \item A construção da tabela tem complexidade $O(nm)$ e a identificação da sequência
            de operações tem complexidade $O(n + m)$
    \end{itemize}

\end{frame}

\begin{frame}[fragile]{Implementação da sequência de operações ótima}
    \inputsnippet{cpp}{9}{28}{codes/edit3.cpp}
\end{frame}

\begin{frame}[fragile]{Implementação da sequência de operações ótima}
    \inputsnippet{cpp}{30}{49}{codes/edit3.cpp}
\end{frame}

\begin{frame}[fragile]{Implementação da sequência de operações ótima}
    \inputsnippet{cpp}{51}{70}{codes/edit3.cpp}
\end{frame}
