\section{Aplicações da $z$-{\it Function}}

\begin{frame}[fragile]{Aplicação \#1 -- Número de ocorrências de $P$ em $S$}

    \begin{itemize}
        \item A função $z$ também pode ser utilizada para determinar o número de ocorrências de 
            uma string $P$, de tamanho $m$, em uma string $S$, de tamanho $n$

        \item Defina a string $T$ como
        \[
            T = P + \mbox{\textcolor{red!80!black}{\texttt{`\#'}}} + S
        \]

        \item Assim, $T$ é a concatenação da string $P$, o caractere separador
        \textcolor{red!80!black}{\texttt{`\#'}} e a string $S$

        \item O separador pode ser qualquer caractere que não apareça nem em $S$ e nem em $T$ 

        \item A string $P$ ocorre na posição $i$ de $S$ se, e somente se, 
        \[
            z_T(i + m + 1) = m
        \]

        \item Esta abordagem tem complexidade $O(n + m)$

    \end{itemize}

\end{frame}

\begin{frame}[fragile]{Exemplo de uso da função $z$ para contagem de ocorrências de $P$ em $S$}

    \begin{center}
    \begin{large}
    \begin{tabular}{c|cccccccccc}
        $i$ & 1 & 2 & 3 & 4 & 5 & 6 & 7 & 8 & 9 & 10 \\
        \hline
        $T$
& \texttt{\textcolor{red!80!black}{a}} 
& \texttt{\textcolor{red!80!black}{n}} 
& \texttt{\textcolor{red!80!black}{a}} 
& \texttt{\textcolor{red!80!black}{\#}} 
& \texttt{\textcolor{red!80!black}{b}} 
& \texttt{\textcolor{red!80!black}{a}} 
& \texttt{\textcolor{red!80!black}{n}} 
& \texttt{\textcolor{red!80!black}{a}} 
& \texttt{\textcolor{red!80!black}{n}} 
& \texttt{\textcolor{red!80!black}{a}} 
\\
        $z_T(i)$
& \textcolor{blue}{0} 
& \textcolor{blue}{0} 
& \textcolor{blue}{1} 
& \textcolor{blue}{0} 
& \textcolor{blue}{0} 
& \underline{\textcolor{blue}{3}}
& \textcolor{blue}{0} 
& \underline{\textcolor{blue}{3}}
& \textcolor{blue}{0} 
& \textcolor{blue}{1} 
\\
    \end{tabular}
    \end{large}
    \end{center}

    \begin{align*}
        m &= 3 \\
        occ &= 2 \\
        pos &= 2\ (6 - 3 - 1)\ \mbox{e}\ 4\ (8 - 3 - 1)
    \end{align*}
\end{frame}

\begin{frame}[fragile]{Aplicação \#2 -- Busca inexata}

    \begin{itemize}
        \item A função $z$ também pode ser utilizada para localizar o número de ocorrências
            de uma substring $P$, de tamanho $m$, em $S$, de tamanho $n$, permitindo que $P$
            e a substring de $S$ sejam distintas em, no máximo, um caractere

        \item Sejam $A$ e $B$ duas strings de mesmo tamanho $n$. A distância de Hamming $dist(A, B)$
            de $A$ a $B$ é dada por
        \[
            dist(A, B) = | \lbrace i\ | \ i\in [1,n]\ \mbox{e}\ A[i] \neq B[i]\rbrace |
        \]
       
        \item Em termos mais precisos, é possível determinar o tamanho do conjunto
        \[
            M = \lbrace S[i..j]\ | \ j - i + 1 = m\ \mbox{e}\ dist(P, S[i..j])\leq 1\rbrace
        \]
            
    \end{itemize}

\end{frame}

\begin{frame}[fragile]{Aplicação \#2 -- Busca inexata}

    \begin{itemize}
        \item Para computar o tamanho de $M$ é preciso montar duas strings

        \item A primeira delas é a string $T$, definida na primeira aplicação, onde
         \[
            T = P + \mbox{\textcolor{red!80!black}{\texttt{`\#'}}} + S
        \]

        \item Deste modo, o caractere $S[i]$ corresponde ao caractere $T[i + m + 1]$

        \item Seja $S'$ a string reversa de $S$, isto é, $S'[i] = S[n - i + 1]$, para todo
            $i\in [1,n]$

        \item A segunda string $R$ é definida por
         \[
            R = P' + \mbox{\textcolor{red!80!black}{\texttt{`\#'}}} + S'
        \]

        \item O caractere $S[i]$ corresponde ao caractere $S'[i']$ da string 
            reversa, com $i' = n - i + 1$

        \item Como $j = i + m - 1$, segue que $j' = i' - m + 1$ 

    \end{itemize}

\end{frame}

\begin{frame}[fragile]{Aplicação \#2 -- Busca inexata}

    \begin{itemize}
        \item Assim, o último caractere da substring $S[i..j]$ corresponde
            ao caractere $R[k]$, onde
        \begin{align*}
            k &= j' + m + 1 \\
              &= (i' - m + 1) + m + 1 \\
              &= (n - i + 1) + 2 \\
              &= n - i + 3
        \end{align*}

        \item Portanto, $S[i..j]\in M$ se, e somente se
        \[
            z_T[i + m + 1] + z_T[n - i + 3] \geq m - 1
        \]

        \item Outras palavras, se o tamanhos do prefixo comum entre $S[i..j]$ e $P$, somado ao
            tamanho do sufixo comum entre $S[i..j]$ e $P$, resultar em $m - 1$, há apenas um
            caractere de diferença entre ambos

        \item Se a soma for maior ou igual a $m$ (de fato, igual a $2m$),  então $S[i..j] = P$
    \end{itemize}

\end{frame}

\begin{frame}[fragile]{Exemplo de uso da função $z$ para busca inexata}

    \begin{center}
    \begin{tabular}{c|ccccccccccccc}
        $i$ & 1 & 2 & 3 & 4 & 5 & 6 & 7 & 8 & 9 & 10 & 11 & 12 \\
        \hline
        $T$
& \texttt{\textcolor{red!80!black}{a}} 
& \texttt{\textcolor{red!80!black}{n}} 
& \texttt{\textcolor{red!80!black}{a}} 
& \texttt{\textcolor{red!80!black}{\#}} 
& \texttt{\textcolor{red!80!black}{r}} 
& \texttt{\textcolor{red!80!black}{a}} 
& \texttt{\textcolor{red!80!black}{b}} 
& \texttt{\textcolor{red!80!black}{a}} 
& \texttt{\textcolor{red!80!black}{n}} 
& \texttt{\textcolor{red!80!black}{e}} 
& \texttt{\textcolor{red!80!black}{t}} 
& \texttt{\textcolor{red!80!black}{e}} 
\\
        $z_T(i)$
& \textcolor{blue}{0} 
& \textcolor{blue}{0} 
& \textcolor{blue}{1} 
& \textcolor{blue}{0} 
& \textcolor{blue}{0} 
& \underline{\textcolor{blue}{3}}
& \textcolor{blue}{0} 
& \underline{\textcolor{blue}{3}}
& \textcolor{blue}{0} 
& \textcolor{blue}{1} 
\\
    \end{tabular}
    \end{center}

    \begin{align*}
        m &= 3 \\
        occ &= 2 \\
        pos &= 2\ (6 - 3 - 1)\ \mbox{e}\ 4\ (8 - 3 - 1)
    \end{align*}
\end{frame}
