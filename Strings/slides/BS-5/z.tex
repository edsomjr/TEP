\section{$z$-{\it Function}}

\begin{frame}[fragile]{Definição}

    \begin{itemize}
        \item Seja $S$ uma string de tamanho $n$

        \item A função $z$ ($z$-\textit{function}) é definida por
        \begin{align*}
            z_S: \mathbb{N} & \to \mathbb{N}\cup \lbrace 0\rbrace \\
               i & \to z_S(i) = \max\lbrace k\ |\ S[1..k]\ \mbox{é prefixo de}\ S[i..n] \rbrace
        \end{align*}

        \item O caso especial $z_S(1)$ depende se o conjunto usado na definição acima inclui
            apenas sufixos próprios ou não

        \item Em geral, considera-se apenas sufixos próprios, de modo que $z_S(1) = 0$

        \item A tabela abaixo ilustra a função $z$ para a string $S = $ \code{cpp}{"abaaba"}:
    \end{itemize}

    \begin{center}
        \begin{tabular}{c|cccccc}
        $i$ & 1 & 2 & 3 & 4 & 5 & 6 \\
        \hline
        $S$ & \texttt{\textcolor{red!80!black}{a}} & \texttt{\textcolor{red!80!black}{b}} & \texttt{\textcolor{red!80!black}{a}} & \texttt{\textcolor{red!80!black}{a}} & \texttt{\textcolor{red!80!black}{b}} & \texttt{\textcolor{red!80!black}{a}} \\
        $z_S(i)$ & \textcolor{blue}{0} & \textcolor{blue}{0} & \textcolor{blue}{1} & \textcolor{blue}{3} & \textcolor{blue}{0} & \textcolor{blue}{1} \\
        \end{tabular}
    \end{center}
\end{frame}

\begin{frame}[fragile]{Pseudocódigo da função $z$ -- \textit{Naive}}

    \begin{algorithm}[H]
        \caption{Função $z$}
        \begin{algorithmic}[1]
            \Require Uma string $S$
            \Ensure Um vetor $zs$ tal que $zs[i] = z_S(i)$

            \Function{z}{$S$}
                \State $n \gets |S|$
                \State $zs[1] \gets 0$
                \State 
                \For {$i \gets 2 \forto n$}
                    \State $j \gets 0$
                    \While {$i + j \leq n\algand S[1 + j] = S[i + j]$}
                        \State $j \gets j + 1$
                    \EndWhile

                    \State $zs[i] \gets j$
                \EndFor
                \State 
                \State \Return $zs$
            \EndFunction
        \end{algorithmic}
    \end{algorithm}

\end{frame}

\begin{frame}[fragile]{Cálculo da função $z$ em $O(n)$}

    \begin{itemize}
        \item O pseudocódigo para a função $z$ apresentado anteriormente tem complexidade
            $O(n^2)$

        \item É possível modificar este algoritmo de modo que seja possível computar todos os
            valores $z_S(i)$, $i = 1, 2, \ldots, n$ em $O(n)$

        \item A ideia central é utilizar os valores já computados da função $z$ evitar 
            comparações já feitas

        \item De fato, a implementação difere da versão \textit{naive} em apenas dois pontos,
            referentes a duas condicionais

        \item A seguir será apresentada a implementação em C++ desta versão modificada, e
            as mudanças promovidas serão explicadas adiante
    \end{itemize}

\end{frame}

\begin{frame}[fragile]{Implementação da função $z$ em C++}
    \inputsnippet{cpp}{1}{21}{z.cpp}
\end{frame}

\begin{frame}[fragile]{Detalhamento da implementação da função $z$}

    \begin{itemize}
        \item A ideia principal da implementação é o uso dos dois ponteiros $L$ e $R$

        \item Para qualquer posição $i$ (na implementação a string é indexada de $0$ a $n - 1$), 
            $L$ e $R$ representam o início e o fim de prefixo comum 
            entre $S$ e algum sufixo $S[k..(n-1)]$, para $k < i$

        \item Este prefixo deve ser não nulo, e caso exista mais de um prefixo comum já 
            identificado, deve ser escolhido aquele termina mais à direita possível

        \item Por exemplo, $S$ = \code{cpp}{"abacababac"}, a função $z$ assumiria os seguintes
            valores:


    \end{itemize}

    \begin{center}
        \begin{tabular}{c|cccccccccc}
        $i$ & 0 & 1 & 2 & 3 & 4 & 5 & 6 & 7 & 8 & 9 \\
        \hline
        $S$ & \texttt{\textcolor{red!80!black}{a}} & \texttt{\textcolor{red!80!black}{b}} & \texttt{\textcolor{red!80!black}{a}} & \texttt{\textcolor{red!80!black}{c}} & \texttt{\textcolor{red!80!black}{a}} & \texttt{\textcolor{red!80!black}{b}}
& \texttt{\textcolor{red!80!black}{a}} 
& \texttt{\textcolor{red!80!black}{b}}
& \texttt{\textcolor{red!80!black}{a}}
& \texttt{\textcolor{red!80!black}{c}} \\
        $z_S(i)$ & \textcolor{blue}{0} & \textcolor{blue}{0} & \textcolor{blue}{1} & \textcolor{blue}{0} & \textcolor{blue}{3} & \textcolor{blue}{0}
& \textcolor{blue}{4}
& \textcolor{blue}{0}
& \textcolor{blue}{1}
& \textcolor{blue}{0} \\
        \end{tabular}
    \end{center}


\end{frame}

\begin{frame}[fragile]{Detalhamento da implementação da função $z$}

    \begin{itemize}
        \item Suponha que o laço \code{cpp}{for} está na nona iteração (isto é, $i = 8$)

        \item Neste ponto, os prefixos comuns não nulos já encontrados são:

        \vspace{0.1in}
        \begin{center}
            \begin{tabularx}{0.7\textwidth}{Xccc}
            \toprule
            \textbf{Prefixo} & \textbf{Posição} & \textbf{Tamanho} \\
            \midrule
            \textcolor{red!80!black}{\verb|"a"|} & $S[2..2]$ & 1 \\
            \textcolor{red!80!black}{\verb|"aba"|} & $S[4..6]$ & 3 \\
            \textcolor{red!80!black}{\verb|"abac"|} & $S[6..9]$ & 8 \\
            \bottomrule
            \end{tabularx}
        \end{center}
        \vspace{0.1in}

        \item As posições destas substrings são os candidatos a valores de $L$ e $R$

        \item Como a substring que termina mais à direita é $S[6..9]$, nesta iteração vale que
            $L = 6$ e $R = 9$
    \end{itemize}

\end{frame}

\begin{frame}[fragile]{Detalhamento da implementação da função $z$}

    \begin{itemize}
        \item Nesta iteração o primeiro \code{cpp}{if} tem sua condição verdadeira: 

        \vspace{0.1in}
        \inputsnippet{cpp}{8}{9}{z.cpp}
        \vspace{0.1in}

        \item O fato de que $i$ pertence ao intervalo $[L, R]$ significa que a substring 
            $S[i..R] = S[(i-L)..(R-L)]$, pois $S[0..(R-L)] = S[L..R]$

        \item O índice $i-L$ corresponde à posição do caractere de $S[0..(R-L)]$ equivalente 
            $i$ em $S[L..R]$

        \item Como $zs[i-L]$ já foi computado, é conhecido o tamanho do maior prefixo comum entre 
            $S$ e $S[i..R]$

        \item Assim, $zs[i]$ será  mínimo entre $zs[i-L]$ e $|S[i..R]| = R-i+1$, pois caso 
            $zs[i-L]$ seja maior que o tamanho de $S[i..R]$, não há garantias de que os caracteres
            que sucedem $S[R]$ coincidam com os caracteres que sucedem $S[R-L]$

    \end{itemize}

\end{frame}

\begin{frame}[fragile]{Detalhamento da implementação da função $z$}

    \begin{itemize}
        \item O segundo \code{cpp}{if} atualiza os ponteiros $L$ e $R$ caso o prefixo
            comum da substring $S[i..(i+z[i]-1)]$ termine mais à direita do que o prefixo
            comum armazenado em $L$ e $R$

        \item Assim, se $i+z[i]-1 > R$, então o $L$ e $R$ passam a apontar para a substring 
            $S[i..i+z[i]-1]$

        \vspace{0.1in}
        \inputsnippet{cpp}{14}{15}{z.cpp}
        \vspace{0.1in}

        \item Observe que, caso $z[i] = 0$, $S[i..i-1]$ é uma string vazia, e os valores de
            $L$ e $R$ correspondem a um intervalo degenerado

        \item Pode ser mostrado que a condição do laço \code{cpp}{while} pode ser verdadeira, no
            máximo, $n - 1$ vezes, de modo que o algoritmo tem complexidade $O(n)$

    \end{itemize}

\end{frame}

