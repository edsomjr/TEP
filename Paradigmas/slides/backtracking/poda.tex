\section{Poda}

\begin{frame}[fragile]{Poda}

    \begin{itemize}
        \item A poda (\textit{prunning}) é uma estratégia de otimização dos algoritmos que 
            utilizam o \textit{backtracking}

        \item A poda avalia a solução parcial $xs$ e os candidatos disponíveis $cs$

        \item Se os candidatos ainda disponíveis não são mais capazes de tornar $xs$ uma solução
            válida, então o \textit{backtracking} deve retornar imediatamente, sem realizar
            as chamadas recursivas

        \item A poda não necessariamente reduz a complexidade assintótica do algoritmo, mas 
            impacta no valor da constante 

        \item O ganho obtido em não processar ramos da árvore de decisão que não levam a
            soluções pode resultar em AC onde um \textit{backtracking} sem poda resulta em
            TLE
    \end{itemize}

\end{frame}

\begin{frame}[fragile]{Listagem de combinações com poda}
    \inputsnippet{cpp}{1}{19}{codes/combinacoes_poda.cpp}
\end{frame}

\begin{frame}[fragile]{Listagem de combinações com poda}
    \inputsnippet{cpp}{21}{40}{codes/combinacoes_poda.cpp}
\end{frame}

\begin{frame}[fragile]{Listagem de combinações com poda}
    \inputsnippet{cpp}{42}{61}{codes/combinacoes_poda.cpp}
\end{frame}

\begin{frame}[fragile]{Listagem de combinações com poda}
    \inputsnippet{cpp}{62}{83}{codes/combinacoes_poda.cpp}
\end{frame}
