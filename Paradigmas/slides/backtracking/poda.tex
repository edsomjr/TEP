\section{Poda}

\begin{frame}[fragile]{Poda}

    \begin{itemize}
        \item A poda (\textit{prunning}) é uma estratégia de otimização dos algoritmo que 
            utilizam o \textit{backtracking}

        \item A poda avalia a solução parcial $xs$ e os candidatos disponíveis $cs$

        \item Se os candidatos ainda disponíveis não são mais capazes de tornar $xs$ uma solução
            válida, então o \textit{backtracking} deve retornar imediatamente, sem realizar
            as chamadas recursivas

        \item A poda pode não reduzir a complexidade assintótica do algoritmo, mas pode reduzir
            a constante

        \item O ganho com o não-processamentos de ramos da árvore de decisão que não levam à
            soluções pode resultar em um AC onde um \textit{backtracking} sem poda resulta em
            TLE
    \end{itemize}

\end{frame}

\begin{frame}[fragile]{Listagem de combinações com poda}
    \inputsnippet{cpp}{1}{20}{codes/combinacoes_poda.cpp}
\end{frame}

\begin{frame}[fragile]{Listagem de combinações com poda}
    \inputsnippet{cpp}{21}{41}{codes/combinacoes_poda.cpp}
\end{frame}

\begin{frame}[fragile]{Listagem de combinações com poda}
    \inputsnippet{cpp}{42}{62}{codes/combinacoes_poda.cpp}
\end{frame}

\begin{frame}[fragile]{Listagem de combinações com poda}
    \inputsnippet{cpp}{63}{83}{codes/combinacoes_poda.cpp}
\end{frame}
