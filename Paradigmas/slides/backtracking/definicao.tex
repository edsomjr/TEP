\section{\it Backtracking}

\begin{frame}[fragile]{Definição}

    \begin{itemize}
        \item \textit{Backtracking} é uma técnica de busca completa que, por meio de recursão,
            investiga todo o espaço de soluções

        \item Ela parte de uma solução, inicialmente vazia, e tenta construir uma nova solução
            estendendo a solução parcial um elemento por vez

        \item A cada passo são avaliados todos os possíveis elementos que podem integrar
            uma nova solução, a depender dos elementos já inseridos na solução parcial

        \item Quando uma solução é encontrada, ela é processada

        \item Em seguida, ou o algoritmo para ou ele segue para identificar todas as demais soluções, se existirem

    \end{itemize}

\end{frame}

\begin{frame}[fragile]{Pseudo-código do {\it backtracking}}

    \begin{algorithm}[H]
        %\algsetup{linenosize=\tiny}
        \small

        \caption{\it Backtracking}

        \begin{algorithmic}[1]
            \Require um vetor solução $xs$ e os parâmetros $P$ do problema
            \Ensure O processamento de todas as soluções possíveis

            \State $xs \gets \emptyset$
            \State 
 
            \Function{backtracking}{$xs, P$}
                \If { \Call{is\_solution}{$xs$, $P$} }
                    \State \Call{process\_solution}{$xs$, $P$}
                \Else
                    \State $cs \gets $ \Call{candidades}{$xs$, $P$}

                    \ForAll { $c \in cs$ }
                        \State \Call{push\_back}{$c, xs$}
                        \State \Call{update}{$P, c$}
                        \State \Call{backtracking}{$xs, P$}
                        \State \Call{restore}{$P, c$}
                        \State \Call{pop\_back}{$xs$}
                    \EndFor
                \EndIf
            \EndFunction
        \end{algorithmic}
    \end{algorithm}

\end{frame}

\begin{frame}[fragile]{Observações sobre o {\it backtracking}}

    \begin{itemize}
        \item No pseudocódigo apresentado, para cada algoritmo em particular é necessário
            implementar as funções \code{cpp}{is_solution()}, \code{cpp}{process_solution()} e
             \code{cpp}{candidates()}

        \item O vetor $xs$ pode ser tanto um \code{cpp}{vector} do C++ ou um \textit{array} 
            estático de C

        \item A construção dos possíveis candidatos depende do estado do vetor $xs$ e dos
            parâmetros $P$ do problema

        \item Cada candidato deve ser inserido ao final de $xs$, e após o retorno da chamada
            recursiva, ele deve ser removido

        \item Assim, não é necessário fazer novas cópias de $xs$ a cada chamada: basta passar este
            vetor por referência

        \item A inserção de um candidato em $xs$ pode modificar os parâmetros $P$: esta
            atualização também deve ser desfeita após a chamada recursiva
    \end{itemize}

\end{frame}
