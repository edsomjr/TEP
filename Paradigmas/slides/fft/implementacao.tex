\section{Implementação}

\begin{frame}[fragile]{Implementação recursiva {\it top-down}}

    \begin{itemize}
        \item Uma forma de implementar a FFT é por meio de recursão

        \item A cada etapa, são criados dois subvetores $e_k$ e $o_k$, de tamanho $N/2$, e a 
            transformada é chamada em cada um destes subvetores

        \item O caso base acontece quando $N = 1$, onde a transformada é igual à própria amostra

        \item Na etapa de síntese ou fusão, os coeficientes da transformada $X_k$ podem ser
            computado atráves das transformadas $E_k$ e $O_k$ dos subvetores:
        \[
            X_j = \left\{ \begin{array}{ll}
                E_j + S_jO_j,& \mbox{se}\ 0 \leq j < N/2,\\
                E_{j - N/2} - S_{j - N/2}O_{j - N/2},& \mbox{caso contrário}
            \end{array}\right.
        \]
        onde
        \[
            S_j = e^{-2\pi ji/N}
        \]
    \end{itemize}

\end{frame}

\begin{frame}[fragile]{Implementação recursiva {\it top-down}}

    \begin{itemize}
        \item É possível implementar tanto a transformada quanto a inversa em uma única função

        \item Primeiramente, é preciso uniformizar o tipo dos vetores

        \item Uma opção é que todos sejam vetores de complexos

        \item Em segundo lugar, são duas as diferenças entre a transformada e sua inversa:
        \begin{enumerate}
            \item os sinais dos ângulos são opostos
            \item na inversa os coeficientes são divididos por $N$
        \end{enumerate}

        \item Uma \textit{flag} booleana pode ser passada como parâmetro para decidir o sentido
            da transformada

        \item Como $N$ é uma potência de 2, uma forma de realizar esta divisão por $N$ é dividir,
            a cada etapa, os coeficientes por 2
    \end{itemize}

\end{frame}

\begin{frame}[fragile]{Implementação {\it top-down} da FFT}
    \inputsnippet{cpp}{1}{21}{codes/fft.cpp}
\end{frame}

\begin{frame}[fragile]{Implementação {\it top-down} da FFT}
    \inputsnippet{cpp}{22}{39}{codes/fft.cpp}
\end{frame}

\begin{frame}[fragile]{Implementação {\it bottom-up}, {\it in-place}}

    \begin{itemize}
        \item Utilizando a ordenação por \textit{bits} invertidos, é possível implementar a FFT
            \textit{bottom-up}

        \item Além de dispensar a recursão, o custo de memória é reduzido, pois só é preciso
            copiar dois coeficientes a cada atualização, sem precisar copiar os subvetores a cada
            iteração

        \item A ordenação baseada em \textit{bits} pode ser feita em $O(N)$: basta trocar de 
            posição dos elementos de índices $i$ e $j$ tais que $i < j$ e que $i$ e $j$ são
            mutuamente reversos 
    \end{itemize}

\end{frame}

\begin{frame}[fragile]{Implementação {\it bottom-up, in-place} da FFT}
    \inputsnippet{cpp}{1}{20}{codes/fft_bu.cpp}
\end{frame}

\begin{frame}[fragile]{Implementação {\it bottom-up, in-place} da FFT}
    \inputsnippet{cpp}{21}{40}{codes/fft_bu.cpp}
\end{frame}

\begin{frame}[fragile]{Implementação {\it bottom-up, in-place} da FFT}
    \inputsnippet{cpp}{41}{61}{codes/fft_bu.cpp}
\end{frame}
