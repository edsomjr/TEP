\section{Transformada de Fourier}

\begin{frame}[fragile]{Série de Fourier}

    \begin{itemize}
        \item Uma série de Fourier consiste na expansão de uma função períodica $f(x)$ em termos
            de senos e cosenos

        \item Isto possível porque as funções $\sin(mx)$ e $\sin(ny)$ são ortogonais para 
            $m\neq n$ no intervalo $[-\pi, \pi]$:

        \begin{align*}
            \int_{-\pi}^\pi \sin(mx)\sin(nx) dx &= 
            \int_{-\pi}^\pi \sin(mx)\cos(nx) dx \\
            &= \int_{-\pi}^\pi \cos(mx)\cos(nx) dx = 0
        \end{align*}

        \item Para $m = n$, segue que
        \[
            \int_{-\pi}^\pi \sin^2(mx) dx = 
            \int_{-\pi}^\pi \cos^2(mx) dx = \pi
        \]

    \end{itemize}

\end{frame}

\begin{frame}[fragile]{Série de Fourier}

    \begin{itemize}
        \item Deste modo,
        \[
            f(x) = \frac{1}{2}a_0 + \sum_{n=1}^\infty a_n\cos(n x) + \sum_{n=1}^\infty b_n\sin(nx),
        \]
        onde
        \begin{align*}
            a_0 &= \frac{1}{\pi}\int_{-\pi}^{\pi} f(x)dx \\    
            a_n &= \frac{1}{\pi}\int_{-\pi}^{\pi} f(x)\cos(nx)dx \\    
            b_n &= \frac{1}{\pi}\int_{-\pi}^{\pi} f(x)\sin(nx)dx    
        \end{align*}
    
    \end{itemize}

\end{frame}

\begin{frame}[fragile]{Exemplo}

    \begin{itemize}
        \item
    \end{itemize}

\end{frame}
