\section{Definição}

\begin{frame}[fragile]{{\it Definição}}

    \metroset{block=fill}
    \begin{block}{Problema do Troco}
        Seja $C = \{c_1, c_2, \ldots, c_N\}$ uma sequência ordenada de $N$ inteiros positivos
        distintos
        e $M$ um inteiro positivo. O problema do troco consiste em determinar um vetor de inteiros
        não-negativos $x = \{ x_1, x_2, \ldots, x_N \}$ tal que
        \[
            M = \sum_{i = 1}^N x_ic_i
        \]
        e que a soma
        \[
            \sum_{i = 1}^N x_i
        \] seja mínima.
    \end{block}

\end{frame}

\begin{frame}[fragile]{Características do problema do troco}

    \begin{itemize}
        \item Os elementos do conjunto $C$ são denominados \textit{moedas}

        \item $M$ é o \textit{troco}

        \item O problema pode ser definido informalmente como: \textit{Qual é o menor número de
            moedas necessárias para dar o troco $M$?}

        \item Se $c_1 = 1$, há solução para qualquer $M$

        \item Se $C$ é o conjunto de moedas utilizadas no sistema financeiro da maioria dos
            países, o problema do troco pode ser resolvido por meio de um algoritmo guloso
    \end{itemize}

\end{frame}

\begin{frame}[fragile]{Algoritmo guloso para o problema do troco}

    \begin{itemize}
        \item O algoritmo guloso para o problema do troco escolhe, dentre as moedas, a maior delas
            ($c_k$) que é menor ou igual a $M$

        \item Em seguida, ele atribui a $x_k$ o valor $M/c_k$ e subtrai de $M$ o valor $x_kc_k$

        \item O algoritmo então prossegue até que $M$ se torne igual a zero
            
        \item Para todos os valores $x_i$ não atribuídos durante o algoritmo, vale que $x_i = 0$ 

    \end{itemize}

\end{frame}

\begin{frame}[fragile]{Implementação do algoritmo guloso para o problema do troco}
    \inputsnippet{cpp}{1}{18}{codes/greedy.cpp}
\end{frame}

\begin{frame}[fragile]{Implementação do algoritmo guloso para o problema do troco}
    \inputsnippet{cpp}{19}{39}{codes/greedy.cpp}
\end{frame}
