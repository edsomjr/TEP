\section{Definição}

\begin{frame}[fragile]{{\it Definição}}

    \metroset{block=fill}
    \begin{block}{Problema do Caixeiro Viajante}
        Dados $N$ vértices $v_1, v_2, \ldots, v_N$ e os custos $w(v_i, v_j)$ das arestas, para 
        todos $i\neq j$, o problema do caixeiro viajante (\textit{travelling salesman problem} --
        TSP) é determinar uma travessia
        \[
            s = v_{i_1}, v_{i_2}, \ldots, v_{i_N}, s
        \]
        que tenha início no vértice $s$ qualquer, passe por todos os demais uma única vez e retorne
        a $s$ com custo 
        \[
            C = \sum_{k = 1}^N w(v_{i_k}, v_{i_{k + 1}})
        \]
        mínimo.
    \end{block}

\end{frame}

\begin{frame}[fragile]{Características do problema do caixeiro viajante}

    \begin{itemize}
        \item Embora a definição apresentada exija que o grafo $G$ seja completo, basta que
            seja conectado

        \item Caso não exista uma aresta entre $v_i$ e $v_j$, basta fazer $w(v_i, v_j) = \infty$

        \item A travessia corresponde a uma permutação dos $N$ vértices de $G$, sendo a única
            diferença a duplicação do primeiro vértice ao final da travessia

        \item Na teoria dos grafos, tal travessia é denonimada um ciclo hamiltoniano, ou seja,
            um caminho cíclico que passa por todos os vértices uma única vez

    \end{itemize}

\end{frame}
