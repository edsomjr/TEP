\section{Busca Ternária}

\begin{frame}[fragile]{Motivação}

    \begin{itemize}
        \item A busca ternária também utiliza a divisão e conquista para reduzir significativamente
            o espaço de busca a cada iteração do algoritmo

        \item Ela pode ser utilizada para localizar o valor máximo ou
            mínimo de uma função unimodal em um intervalo $[a, b]$

        \item Uma função $f(x)$ é unimodal no intervalo $I = [a, b]$ se existe um ponto 
            $c\in I$ tal que
            \begin{enumerate}
                \item $f'(x) > 0$ se $x \in[a, c)$, $f'(c) = 0$ e $f'(x) < 0$ se $x\in (c, b]$; ou
                \item $f'(x) < 0$ se $x \in[a, c)$, $f'(c) = 0$ e $f'(x) > 0$ se $x\in (c, b]$
            \end{enumerate}

        \item Observe que a busca binária não é capaz de localizar tal máximo diretamente
            neste cenário
    \end{itemize}

\end{frame}

\input{unimodal_view}

\begin{frame}[fragile]{Algoritmo}

    \begin{itemize}
        \item Seja $f(x)$ uma função unimodal no intervalo $I = [a, b]$ e $m_1, m_2\in I$ tais
            que $a < m_1 < m_2 < b$, com um valor máximo no ponto $c\in I$

        \item Os valores $f(m_1)$ e $f(m_2)$ se relacionam de uma das três maneiras seguintes:
        \begin{enumerate}
            \item $f(m_1) < f(m_2)$
            \item $f(m_1) > f(m_2)$
            \item $f(m_1) = f(m_2)$
        \end{enumerate}

        \item No primeiro caso, o máximo não pode estar no intervalo $[a, m_1]$, pois 
            área de crescimento da função está à direita de $m_1$

        \item Assim $c > m_1$ e a busca deve prosseguir no intervalo $[m_1, b]$

        \item O segundo caso é simétrico ao primeiro: a região de decrescimento está à 
            direita de $m_2$, logo $c$ está no intervalo $[a, m_2]$
    \end{itemize}

\end{frame}

\begin{frame}[fragile]{Algoritmo}

    \begin{itemize}
        \item No terceiro caso ocorre ou quando $m_1 = m_2$ ou se $m_1$ está na área de crescimento
            e $m_2$ na área de decrescimento, ou vice-versa

        \item Assim, $c\in [m_1, m_2]$

        \item Para simplificar o algoritmo, o terceiro caso pode ser reduzido a um dos dois
            primeiros

        \item Se $m_1$ e $m_2$ dividirem $[a, b]$ em três regiões iguais, a cada etapa
            o intervalo de busca é reduzido em um terço de seu tamanho

        \item Para esta divisão os valores a serem escolhidos são
        \begin{align*}
            m_1 &= a + \left(\dfrac{b - a}{3}\right) \\
            m_2 &= b - \left(\dfrac{b - a}{3}\right)
        \end{align*}
    \end{itemize}

\end{frame}

\input{ternary_view}

\begin{frame}[fragile]{Implementação iterativa da busca ternária}
    \inputsnippet{cpp}{1}{19}{codes/ternary.cpp}
\end{frame}

\begin{frame}[fragile]{Implementação recursiva da busca ternária}
    \inputsnippet{cpp}{1}{20}{codes/recursive_ternary.cpp}
\end{frame}
