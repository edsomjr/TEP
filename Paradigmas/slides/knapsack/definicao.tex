\section{Definição}

\begin{frame}[fragile]{{\it Definição}}

    \metroset{block=fill}
    \begin{block}{Problema da Mochila Binária}
        Considere uma conjunto $C = \{ c_1, c_2, \ldots, c_N \}$, onde $c_i = (w_i, v_i)$,
        e seja $M$ um inteiro positivo.

        O problema da mochila binária consiste em determinar um subjconjunto $S\subset 
        \{ 1, 2, \ldots, N \}$ de índices de $C$ tal que
        \[
            W = \sum_{j\in S} w_j \leq M
        \]
        e que a soma
        \[
            V = \sum_{j \in S} v_j
        \] seja máxima.
    \end{block}

\end{frame}

\begin{frame}[fragile]{Características do problema da mochila binária}

    \begin{itemize}
        \item Os elementos do conjunto $C$ são denominados objetos ou items

        \item $M$ é o capacidade da mochila

        \item $w_i$ é o peso/tamanho do objeto, o qual determina se o objeto pode ou não ser
            transportado na mochila, de acordo com a capacidade ainda disponível na mesma

        \item $v_i$ é o valor/ganho do objeto, e a soma dos valores dos objetos selecionados
            deve ser maximizada

        \item O termo ``mochila binária'' remete ao fato de que, para cada um dos objetos,
            há duas opções: escolhê-lo ou não
    \end{itemize}

\end{frame}

