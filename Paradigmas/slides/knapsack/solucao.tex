\section{Solução do problema da mochila binária}

\begin{frame}[fragile]{Solução do problema da mochila binária}

    \begin{itemize}
        \item Como $|C| = N$, há $2^N$ subconjuntos de índices a serem avaliados

        \item Como cada subconjunto pode ser avaliado em $O(N)$, uma solução de busca
            completa tem complexidade $O(N2^N)$

        \item É possível, entretando, reduzir esta complexidade por meio de um algoritmo
            de programação dinâmica

        \item Seja $v(i, m)$ a soma máxima dos valores que pode ser obtida a partir dos
            primeiros $i$ elementos de $C$ e uma mochila com capacidade $m$

        \item São dois casos-base: o primeiro deles acontece quando não resta mais nenhum
            elemento a ser considerado

        \item Neste casos, temos $v(0, m) = 0$
    \end{itemize}

\end{frame}

\begin{frame}[fragile]{Solução do problema da mochila binária}

    \begin{itemize}
        \item O segundo caso-base acontece quando não há mais espaço disponível na mochila:
            $v(i, 0) = 0$

        \item Também são duas as transições possíveis:
        \begin{enumerate}
            \item ignorar o $i$-ésimo elemento e considerar apenas os $i - 1$ primeiros; ou
            \item caso possar ser transportado, pegar o $i$-ésimo elemento e colocá-lo na
                mochila
        \end{enumerate}

        \item A primeira transição não modifica o estado da mochila e nem o total dos valores
            transportados

        \item Caso a mochila não consiga transportar o $i$-ésimo elemento, esta será a única
        transição possível

        \item Assim,
        \[
            v(i, m) = v(i - 1, m), \ \ \mbox{se}\ w_i > m
        \]

    \end{itemize}

\end{frame}

\begin{frame}[fragile]{Solução do problema da mochila binária}

    \begin{itemize}
        \item A segunda transição só é possível se $w_i \leq m$

        \item Caso esta condição seja atendida, a capacidade da mochila é reduzida em $w_i$
            unidades, e o total dos valores transportados é acrescido em $v_i$

        \item Assim, deve-se optar pela transição que produz o maior valor:
        \[
            v(i, m) = \max\{\ v(i -1, m), v(i - 1, m - w_i) + v_i\ \}, \ \ \mbox{se}\ w_i \leq m
        \]

        \item A solução do problema será dada por $v(N, M)$

        \item O número de estados distintos é $O(NM)$ e cada transição é feita em $O(1)$

        \item Portanto a solução de programação dinâmica para o problema do troco tem
            complexidade $O(NM)$
    \end{itemize}

\end{frame}

\begin{frame}[fragile]{Implementação {\it top-down} da mochila binária}
    \inputsnippet{cpp}{1}{21}{codes/knapsack.cpp}
\end{frame}

\begin{frame}[fragile]{Implementação {\it top-down} da mochila binária}
    \inputsnippet{cpp}{22}{35}{codes/knapsack.cpp}
\end{frame}

\begin{frame}[fragile]{Recuperação dos elementos selecionados}

    \begin{itemize}
        \item Uma variante comum do problema é exibir a lista dos elementos selecionados
            que maximizaram a soma dos valores, respeitando a capacidade da mochila

        \item Para recuperar os elementos escolhidos é precisa uma tabela adicional $p$, com
            as mesmas dimensões da tabela de memorização

        \item Se $p(i, m) = 1$, a transição escolhida por $v(i, m)$ foi a segunda, isto é,
            o elemento $c_i$ foi escolhido

        \item Caso contrário, $p(i, m) = 0$

        \item Com esta informação basta recuperar os itens, usando os valores de $p(i, m)$
            para rastrear os estados que levaram à solução ótima
    \end{itemize}

\end{frame}

\begin{frame}[fragile]{Implementação {\it bottom-up} da mochila binária}
    \inputsnippet{cpp}{1}{21}{codes/knapsack_bu.cpp}
\end{frame}

\begin{frame}[fragile]{Implementação {\it bottom-up} da mochila binária}
    \inputsnippet{cpp}{22}{40}{codes/knapsack_bu.cpp}
\end{frame}

\begin{frame}[fragile]{Implementação {\it bottom-up} da mochila binária}
    \inputsnippet{cpp}{41}{57}{codes/knapsack_bu.cpp}
\end{frame}
