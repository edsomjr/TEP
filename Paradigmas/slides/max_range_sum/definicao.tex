\section{Definição}

\begin{frame}[fragile]{{\it Definição}}

    \metroset{block=fill}
    \begin{block}{Max Range Sum}
        Seja $a = \{a_1, a_2, \ldots, a_N\}$ uma sequência de $N$ elementos. O \textit{max range
            sum} de $a$ é o intervalo $[i, j]$, com $i \leq j$, tal que a soma
        \[
            \sum_{k = i}^j a_k = a_i + a_{i + 1} + \ldots + a_j
        \]
        é máxima.
    \end{block}

\end{frame}

\begin{frame}[fragile]{Características do {\it max range sum}}

    \begin{itemize}
        \item Observe que o \textit{max range sum} de uma sequência não é, necessariamente,
            único

        \item Por exemplo, para a sequência
        \[
            a = \{ 1, 1, 1, -5, 3, -2, 0 \}
        \]
        os intervalos $[1, 3]$ e $[5, 5]$ tem, ambos, soma máxima (igual a 3)

        \item Variante comuns deste prolema é  escolher um dentre estes intervalos, ou retornar
            somente o valor da soma máxima, ignorando o intervalo que a gerou

        \item Há $N(N - 1)/2$ intervalos definidos pelos índices $[i, j]$, com $i\leq j$, da sequência $a$

        \item Uma solução de busca completa que computa a soma de cada intervalo em $O(N)$
            solucionaria o problema \textit{max range sum} com complexidade $O(N^3)$

    \end{itemize}

\end{frame}

\begin{frame}[fragile]{Implementação em $O(N^3)$ do MRS}
    \inputsnippet{cpp}{7}{26}{codes/max-range-sum_complete-search.cpp}
\end{frame}

\begin{frame}[fragile]{Soma dos prefixos}

    \begin{itemize}
        \item Uma forma de reduzir a complexidade da solução é computar as somas de cada
            intervalo em $O(1)$

        \item Assim a complexidade seria reduzida de $O(N^3)$ para $O(N^2)$

        \item Esta soma em $O(1)$ pode ser feita precomputando a soma de todos os prefixos
            de $a$

        \item Um prefixo $p_j$ de $a$ é uma subsequência que tem início no primeiro elemento $a_1$
            de $a$ e termina no elemento $a_j$

        \item Seja $ps(i)$ a soma dos elementos do prefixo $p_i$ de $a$, isto é,
        \[
            ps(i) = \sum_{k = 1}^i a_1 + a_2 + \ldots + a_i
        \]

    \end{itemize}

\end{frame}

\begin{frame}[fragile]{Soma dos prefixos}

    \begin{itemize}
        \item Como $p_0$ é o prefixo vazio, então $ps(0) = 0$

        \item Para $1\leq i\leq N$, vale que
        \[
            ps(i) = a_1 + a_2 + \ldots + a_i = (a_1 + a_2 + \ldots + a_{i - 1}) + a_i =
                ps(i - 1) + a_i
        \]

        \item Assim, é possível computar $ps(i)$ para todos os $i$ possíveis em $O(N)$

        \item A soma dos elementos cujos índices pertencem ao intervalo $[i, j]$ então é dada
            por
        \begin{align*}
            \sum_{k = i}^j a_k &= a_i + a_{i + 1} + \ldots + a_j \\
                &= (a_1 + \ldots + a_{i - 1}) + (a_i + \ldots + a_j) - (a_1 + \ldots + a_{i - 1}) \\
                &= (a_1 + a_2 + \ldots + a_j) - (a_1 + \ldots + a_{i - 1}) \\
                &= ps(j) - ps(i - 1)
        \end{align*}

    \end{itemize}

\end{frame}

\begin{frame}[fragile]{Implementação do MRS em $O(N^2)$}
    \inputsnippet{cpp}{7}{23}{codes/max-range-sum_prefix-sum.cpp}
\end{frame}

\begin{frame}[fragile]{Algoritmo de Kadane}

    \begin{itemize}
        \item De forma surpreendente, é possível resolver solucionar o \textit{max range sum} em
            $O(N)$

        \item Em geral, o algoritmo de Kadane é apresentado de tal forma que leva a crer que o 
            mesmo seja um algoritmo guloso

        \item Porém ele é, de fato, um algoritmo de programação dinâmica

        \item Seja $s(k)$ a maior soma dentre todos os intervalos da forma $[i, k]$, com 
            $1\leq i\leq k$

        \item O caso base ocorre quando $k = 1$: neste caso, $s(1) = a_1$

        \item Para $k > 1$ há duas transições possíveis:
        \begin{enumerate}
            \item estender o intervalo anterior, adicionando $a_k$
            \item desprezar o intervalo anterior e retornar $a_k$
        \end{enumerate}
    \end{itemize}

\end{frame}

\begin{frame}[fragile]{Algoritmo de Kadane}

    \begin{itemize}
        \item Destas duas possibilidades, o algoritmo seleciona a melhor das duas

        \item A segunda transição só é a melhor quando $s(k - 1) < 0$, e esta característica é
            que gera a percepção de estratégia gulosa do algoritmo

        \item Em notação matemática,
        \[
            s(k) = \max\{\ s(k - 1) + a_k, a_k\ \}
        \]

        \item Uma vez computado os valores de $s(k)$ para $k\in[1, N]$, a solução do problema
            é dada por
        \[
            MRS(a) = \max\{\ s(1), s(2), \ldots, s(N)\ \}
        \]

        \item Como há $O(N)$ estados possíveis e as transições são feitas em $O(1)$, a complexidade
            da solução é $O(N)$
    \end{itemize}

\end{frame}

\begin{frame}[fragile]{Implementação do algoritmo de Kadane}
    \inputsnippet{cpp}{5}{14}{codes/kadane.cpp}
\end{frame}

\begin{frame}[fragile]{Implementação ``gulosa'' do algoritmo de Kadane}
    \inputsnippet{cpp}{5}{19}{codes/kadane2.cpp}
\end{frame}
