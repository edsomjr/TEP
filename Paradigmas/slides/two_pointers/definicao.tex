\section{\it Two pointers}

\begin{frame}[fragile]{Definição}

    \begin{itemize}
        \item \textit{Two pointers} é uma técnica de gulosa aplicada em problemas que usam
            vetores

        \item Nela são utilizados dois ponteiros unidirecionais $L$ e $R$

        \item Assim, a cada iteração do algoritmo, estes ponteiros só podem avançar na direção
            pré-definida, sem recuar

        \item Isto faz com que cada ponteiro observe cada elemento do vetor uma única vez

        \item Para tal, é preciso considerar quais valores ainda podem fazer parte da solução,
            e quais podem, e devem, ser descartados

        \item Desde modo, esta é uma técnica gulosa, uma vez que, fixado um dos ponteiros, o
            segundo se move o máximo possível e, em seguida, os ponteiros são reposicionados para
            as melhores posições possíveis

    \end{itemize}

\end{frame}

\begin{frame}[fragile]{Implementação}

    \begin{itemize}
        \item Em geral, o ponteiro $L$ (\textit{left}) aponta para o início do vetor e é
            incrementado a cada passo do algoritmo

        \item O ponteiro $R$ (\textit{right}) geralmente parte de $L$ (ou $L + 1$) e avança
            enquanto o intervalo $[L, R)$ constituir uma subsolução válida do problema

        \item Em alguns problemas, o ponteiro $L$ pode saltar diretamente para $R$, caso $R$ não
            possa mais avançar

        \item Também há problemas onde $R$ inicia no último elemento do vetor e caminha em direção
            ao início do mesmo

        \item Na maioria dos casos, o uso desta técnica leva a algoritmos $O(N)$, onde $N$ é o tamanho
            do vetor
    \end{itemize}

\end{frame}
