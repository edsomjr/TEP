\section{Exemplos de aplicação de {\it two pointers}}

\begin{frame}[fragile]{Maior substring em ordem lexicográfica}

    \begin{itemize}
        \item Seja $S$ uma string com $N$ caracteres

        \item O problema consiste em determinar o tamanho $M$ da maior substring $b$ de $S$ tal
            que os caracteres de $b$ estão em ordem lexicográfica

        \item Em outras palavras, o problema é determinar a maior substring $b=b[1..M]$ de
            $S$ tal que $b_{i - 1} \leq b_i, \forall i\in[2, M]$

        \item A string $S$ tem $O(N^2)$ substrings, e a verificação se uma substring está em
            ordem lexicográfica ou não é feita em $O(N)$

        \item Assim um algoritmo de busca completa tem complexidade $O(N^3)$

    \end{itemize}

\end{frame}

\begin{frame}[fragile]{Maior substring em ordem lexicográfica}

    \begin{itemize}
        \item Porém, é possível utilizar a técnica \textit{two pointers} neste problema

        \item Inicie $L$ no primeiro caractere de $S$

        \item Para cada valor de $L$, faça $R = L + 1$

        \item A substring $S[L..(R - 1)]$ tem, inicialmente, um único caractere, de modo que
            está em ordem lexicográfica

        \item Enquanto $R\leq N$ e $S[R - 1] \leq S[R]$, incremente $R$

        \item Ao final do processo, a substring $S[L..(R - 1)]$ terá $R - L$ caracteres e estará
            em ordem lexicográfica

        \item Atualize $L$ para $R$ e prossiga enquanto $L\leq N$

        \item Como tanto $L$ quanto $R$ observam cada caractere de $S$ uma única vez, esta 
            solução tem complexidade $O(N)$ e memória $O(1)$
    \end{itemize}

\end{frame}

\begin{frame}[fragile]{Tamanho da maior substring ordenada}
    \inputsnippet{cpp}{1}{21}{codes/substring.cpp}
\end{frame}

\begin{frame}[fragile]{Tamanho da maior substring ordenada}
    \inputsnippet{cpp}{22}{42}{codes/substring.cpp}
\end{frame}

\begin{frame}[fragile]{Maior subvetor com, no máximo, $K$ números ímpares}

    \begin{itemize}
        \item Seja $\vec{v}$ um vetor com $N$ números inteiros

        \item O problema consiste em terminar o maior subvetor $\vec{x} = \vec{v}[i..j]$ de 
            $\vec{v}$ que contenha, no máximo, $K$ números ímpares

        \item Novamente, $\vec{v}$ em $O(N^2)$ subvetores, de modo que uma solução de busca
            completa que verifique cada um deles terá complexidade $O(N^3)$

        \item A ideia central da solução usando \textit{two pointers} é identificar intervalos
            $[L, R)$ tais que os subvetores $\vec{v}[L..(R - 1)]$ tenham, no máximo, $K$ números
            ímpares

        \item Observe que $|\vec{v}[L..(R-1)| = R - L$
 
        \item O ponteiro $L$ observará, um a um, os elementos de $\vec{v}$, do primeiro para o 
            último

        \item O ponteiro $R$ iniciará apontando para o primeiro elemento
    \end{itemize}

\end{frame}

\begin{frame}[fragile]{Maior subvetor com, no máximo, $K$ números ímpares}

    \begin{itemize}
        \item Um contador $c$, inicialmente igual a zero, manterá o registro do número de elementos
            ímpares dentre os elementos de $\vec{v}$ cujos índices estão no intervalo $[L, R)$

        \item Para cada valor de $L$, o ponteiro $R$ apontará para $L$ ou manterá o seu valor, o 
            que estiver mais distante do início do vetor

        \item Enquanto $R$ apontar para um elemento par, ou apontar para um ímpar com o contador
            $c < K$, o contador é atualizado com o valor de $\vec{v}[R]$ e $R$ é incrementado

        \item Ao final deste processo, a resposta é atualizada em relação ao tamanho do subvetor
            válido ($R - L$)

        \item Por fim, o contador é ajustado, caso o valor de $\vec{v}[L]$ tenha sido
            considerado, e $L$ é incrementado

        \item A complexidade da solução é $O(N)$
    \end{itemize}

\end{frame}

\begin{frame}[fragile]{Visualização do algoritmo do maior subvetor com, no máximo, $K=1$ números
ímpares}

        \begin{tikzpicture}
            \node[anchor=west] at (0, 7) { $ans = 0$ };
            \node[anchor=west] at (0, 6.5) { $c = 0$ };
            \node[opacity=0,anchor=west] at (0, 2) { $c = 0$ };

            \draw (3, 4) grid (8, 5);

            \node at (3.5, 4.5) { \tt 1 };
            \node at (4.5, 4.5) { \tt 2 };
            \node at (5.5, 4.5) { \tt 3 };
            \node at (6.5, 4.5) { \tt 4 };
            \node at (7.5, 4.5) { \tt 5 };
 
%            \draw[->] (3.5, 5.5) node[anchor=south] { $L$ } -- (3.5, 5.2);
%            \draw[->] (3.5, 3.5) node[anchor=north] { $R$ } -- (3.5, 3.8);

        \end{tikzpicture}

\end{frame}

\begin{frame}[fragile]{Visualização do algoritmo do maior subvetor com, no máximo, $K=1$ números
ímpares}

        \begin{tikzpicture}
            \node[anchor=west] at (0, 7) { $ans = 0$ };
            \node[anchor=west] at (0, 6.5) { $c = \textcolor{blue}{1}$ };
            \node[opacity=0,anchor=west] at (0, 2) { $c = 0$ };

            \draw (3, 4) grid (8, 5);

            \node at (3.5, 4.5) { \tt 1 };
            \node at (4.5, 4.5) { \tt 2 };
            \node at (5.5, 4.5) { \tt 3 };
            \node at (6.5, 4.5) { \tt 4 };
            \node at (7.5, 4.5) { \tt 5 };
 
            \draw[->] (3.5, 5.5) node[anchor=south] { $L$ } -- (3.5, 5.2);
%            \draw[->] (3.5, 3.5) node[anchor=north] { $R$ } -- (3.5, 3.8);

        \end{tikzpicture}

\end{frame}

\begin{frame}[fragile]{Visualização do algoritmo do maior subvetor com, no máximo, $K=1$ números
ímpares}

        \begin{tikzpicture}
            \node[anchor=west] at (0, 7) { $ans = 0$ };
            \node[anchor=west] at (0, 6.5) { $c = \textcolor{black}{1}$ };
            \node[opacity=0,anchor=west] at (0, 2) { $c = 0$ };

            \draw (3, 4) grid (8, 5);

            \node at (3.5, 4.5) { \tt 1 };
            \node at (4.5, 4.5) { \tt 2 };
            \node at (5.5, 4.5) { \tt 3 };
            \node at (6.5, 4.5) { \tt 4 };
            \node at (7.5, 4.5) { \tt 5 };
 
            \draw[->] (3.5, 5.5) node[anchor=south] { $L$ } -- (3.5, 5.2);
            \draw[->] (4.5, 3.5) node[anchor=north] { $R$ } -- (4.5, 3.8);

        \end{tikzpicture}

\end{frame}

\begin{frame}[fragile]{Visualização do algoritmo do maior subvetor com, no máximo, $K=1$ números
ímpares}

        \begin{tikzpicture}
            \node[anchor=west] at (0, 7) { $ans = 0$ };
            \node[anchor=west] at (0, 6.5) { $c = \textcolor{black}{1}$ };
            \node[opacity=0,anchor=west] at (0, 2) { $c = 0$ };

            \draw (3, 4) grid (8, 5);

            \node at (3.5, 4.5) { \tt 1 };
            \node at (4.5, 4.5) { \tt 2 };
            \node at (5.5, 4.5) { \tt 3 };
            \node at (6.5, 4.5) { \tt 4 };
            \node at (7.5, 4.5) { \tt 5 };
 
            \draw[->] (3.5, 5.5) node[anchor=south] { $L$ } -- (3.5, 5.2);
            \draw[->] (5.5, 3.5) node[anchor=north] { $R$ } -- (5.5, 3.8);

        \end{tikzpicture}

\end{frame}

\begin{frame}[fragile]{Visualização do algoritmo do maior subvetor com, no máximo, $K=1$ números
ímpares}

        \begin{tikzpicture}
            \node[anchor=west] at (0, 7) { $ans = \textcolor{blue}{2}$ };
            \node[anchor=west] at (0, 6.5) { $c = \textcolor{black}{1}$ };
            \node[opacity=0,anchor=west] at (0, 2) { $c = 0$ };

            \draw (3, 4) grid (8, 5);

            \node at (3.5, 4.5) { \tt \textcolor{red}{1} };
            \node at (4.5, 4.5) { \tt \textcolor{blue}{2} };
            \node at (5.5, 4.5) { \tt 3 };
            \node at (6.5, 4.5) { \tt 4 };
            \node at (7.5, 4.5) { \tt 5 };
 
            \draw[->] (3.5, 5.5) node[anchor=south] { $L$ } -- (3.5, 5.2);
            \draw[->] (5.5, 3.5) node[anchor=north] { $R$ } -- (5.5, 3.8);

        \end{tikzpicture}

\end{frame}

\begin{frame}[fragile]{Visualização do algoritmo do maior subvetor com, no máximo, $K=1$ números
ímpares}

        \begin{tikzpicture}
            \node[anchor=west] at (0, 7) { $ans = \textcolor{black}{2}$ };
            \node[anchor=west] at (0, 6.5) { $c = \textcolor{blue}{0}$ };
            \node[opacity=0,anchor=west] at (0, 2) { $c = 0$ };

            \draw (3, 4) grid (8, 5);

            \node at (3.5, 4.5) { \tt \textcolor{black}{1} };
            \node at (4.5, 4.5) { \tt \textcolor{black}{2} };
            \node at (5.5, 4.5) { \tt 3 };
            \node at (6.5, 4.5) { \tt 4 };
            \node at (7.5, 4.5) { \tt 5 };
 
            \draw[->] (4.5, 5.5) node[anchor=south] { $L$ } -- (4.5, 5.2);
            \draw[->] (5.5, 3.5) node[anchor=north] { $R$ } -- (5.5, 3.8);

        \end{tikzpicture}

\end{frame}

\begin{frame}[fragile]{Visualização do algoritmo do maior subvetor com, no máximo, $K=1$ números
ímpares}

        \begin{tikzpicture}
            \node[anchor=west] at (0, 7) { $ans = \textcolor{black}{2}$ };
            \node[anchor=west] at (0, 6.5) { $c = \textcolor{blue}{1}$ };
            \node[opacity=0,anchor=west] at (0, 2) { $c = 0$ };

            \draw (3, 4) grid (8, 5);

            \node at (3.5, 4.5) { \tt \textcolor{black}{1} };
            \node at (4.5, 4.5) { \tt \textcolor{black}{2} };
            \node at (5.5, 4.5) { \tt 3 };
            \node at (6.5, 4.5) { \tt 4 };
            \node at (7.5, 4.5) { \tt 5 };
 
            \draw[->] (4.5, 5.5) node[anchor=south] { $L$ } -- (4.5, 5.2);
            \draw[->] (6.5, 3.5) node[anchor=north] { $R$ } -- (6.5, 3.8);

        \end{tikzpicture}

\end{frame}

\begin{frame}[fragile]{Visualização do algoritmo do maior subvetor com, no máximo, $K=1$ números
ímpares}

        \begin{tikzpicture}
            \node[anchor=west] at (0, 7) { $ans = \textcolor{black}{2}$ };
            \node[anchor=west] at (0, 6.5) { $c = \textcolor{black}{1}$ };
            \node[opacity=0,anchor=west] at (0, 2) { $c = 0$ };

            \draw (3, 4) grid (8, 5);

            \node at (3.5, 4.5) { \tt \textcolor{black}{1} };
            \node at (4.5, 4.5) { \tt \textcolor{black}{2} };
            \node at (5.5, 4.5) { \tt 3 };
            \node at (6.5, 4.5) { \tt 4 };
            \node at (7.5, 4.5) { \tt 5 };
 
            \draw[->] (4.5, 5.5) node[anchor=south] { $L$ } -- (4.5, 5.2);
            \draw[->] (7.5, 3.5) node[anchor=north] { $R$ } -- (7.5, 3.8);

        \end{tikzpicture}

\end{frame}

\begin{frame}[fragile]{Visualização do algoritmo do maior subvetor com, no máximo, $K=1$ números
ímpares}

        \begin{tikzpicture}
            \node[anchor=west] at (0, 7) { $ans = \textcolor{blue}{3}$ };
            \node[anchor=west] at (0, 6.5) { $c = \textcolor{black}{1}$ };
            \node[opacity=0,anchor=west] at (0, 2) { $c = 0$ };

            \draw (3, 4) grid (8, 5);

            \node at (3.5, 4.5) { \tt \textcolor{black}{1} };
            \node at (4.5, 4.5) { \tt \textcolor{blue}{2} };
            \node at (5.5, 4.5) { \tt \textcolor{red}{3} };
            \node at (6.5, 4.5) { \tt \textcolor{blue}{4} };
            \node at (7.5, 4.5) { \tt 5 };
 
            \draw[->] (4.5, 5.5) node[anchor=south] { $L$ } -- (4.5, 5.2);
            \draw[->] (7.5, 3.5) node[anchor=north] { $R$ } -- (7.5, 3.8);

        \end{tikzpicture}

\end{frame}

\begin{frame}[fragile]{Visualização do algoritmo do maior subvetor com, no máximo, $K=1$ números
ímpares}

        \begin{tikzpicture}
            \node[anchor=west] at (0, 7) { $ans = \textcolor{black}{3}$ };
            \node[anchor=west] at (0, 6.5) { $c = \textcolor{black}{1}$ };
            \node[opacity=0,anchor=west] at (0, 2) { $c = 0$ };

            \draw (3, 4) grid (8, 5);

            \node at (3.5, 4.5) { \tt \textcolor{black}{1} };
            \node at (4.5, 4.5) { \tt \textcolor{black}{2} };
            \node at (5.5, 4.5) { \tt \textcolor{black}{3} };
            \node at (6.5, 4.5) { \tt \textcolor{black}{4} };
            \node at (7.5, 4.5) { \tt 5 };
 
            \draw[->] (5.5, 5.5) node[anchor=south] { $L$ } -- (5.5, 5.2);
            \draw[->] (7.5, 3.5) node[anchor=north] { $R$ } -- (7.5, 3.8);

        \end{tikzpicture}

\end{frame}

\begin{frame}[fragile]{Visualização do algoritmo do maior subvetor com, no máximo, $K=1$ números
ímpares}

        \begin{tikzpicture}
            \node[anchor=west] at (0, 7) { $ans = \textcolor{black}{3}$ };
            \node[anchor=west] at (0, 6.5) { $c = \textcolor{black}{1}$ };
            \node[opacity=0,anchor=west] at (0, 2) { $c = 0$ };

            \draw (3, 4) grid (8, 5);

            \node at (3.5, 4.5) { \tt \textcolor{black}{1} };
            \node at (4.5, 4.5) { \tt \textcolor{black}{2} };
            \node at (5.5, 4.5) { \tt \textcolor{red}{3} };
            \node at (6.5, 4.5) { \tt \textcolor{blue}{4} };
            \node at (7.5, 4.5) { \tt 5 };
 
            \draw[->] (5.5, 5.5) node[anchor=south] { $L$ } -- (5.5, 5.2);
            \draw[->] (7.5, 3.5) node[anchor=north] { $R$ } -- (7.5, 3.8);

        \end{tikzpicture}

\end{frame}

\begin{frame}[fragile]{Visualização do algoritmo do maior subvetor com, no máximo, $K=1$ números
ímpares}

        \begin{tikzpicture}
            \node[anchor=west] at (0, 7) { $ans = \textcolor{black}{3}$ };
            \node[anchor=west] at (0, 6.5) { $c = \textcolor{blue}{0}$ };
            \node[opacity=0,anchor=west] at (0, 2) { $c = 0$ };

            \draw (3, 4) grid (8, 5);

            \node at (3.5, 4.5) { \tt \textcolor{black}{1} };
            \node at (4.5, 4.5) { \tt \textcolor{black}{2} };
            \node at (5.5, 4.5) { \tt \textcolor{black}{3} };
            \node at (6.5, 4.5) { \tt \textcolor{black}{4} };
            \node at (7.5, 4.5) { \tt 5 };
 
            \draw[->] (6.5, 5.5) node[anchor=south] { $L$ } -- (6.5, 5.2);
            \draw[->] (7.5, 3.5) node[anchor=north] { $R$ } -- (7.5, 3.8);

        \end{tikzpicture}

\end{frame}

\begin{frame}[fragile]{Visualização do algoritmo do maior subvetor com, no máximo, $K=1$ números
ímpares}

        \begin{tikzpicture}
            \node[anchor=west] at (0, 7) { $ans = \textcolor{black}{3}$ };
            \node[anchor=west] at (0, 6.5) { $c = \textcolor{blue}{1}$ };
            \node[opacity=0,anchor=west] at (0, 2) { $c = 0$ };

            \draw (3, 4) grid (8, 5);

            \node at (3.5, 4.5) { \tt \textcolor{black}{1} };
            \node at (4.5, 4.5) { \tt \textcolor{black}{2} };
            \node at (5.5, 4.5) { \tt \textcolor{black}{3} };
            \node at (6.5, 4.5) { \tt \textcolor{black}{4} };
            \node at (7.5, 4.5) { \tt 5 };
 
            \draw[->] (6.5, 5.5) node[anchor=south] { $L$ } -- (6.5, 5.2);
            \draw[->] (8.5, 3.5) node[anchor=north] { $R$ } -- (8.5, 3.8);

        \end{tikzpicture}

\end{frame}

\begin{frame}[fragile]{Visualização do algoritmo do maior subvetor com, no máximo, $K=1$ números
ímpares}

        \begin{tikzpicture}
            \node[anchor=west] at (0, 7) { $ans = \textcolor{black}{3}$ };
            \node[anchor=west] at (0, 6.5) { $c = \textcolor{black}{1}$ };
            \node[opacity=0,anchor=west] at (0, 2) { $c = 0$ };

            \draw (3, 4) grid (8, 5);

            \node at (3.5, 4.5) { \tt \textcolor{black}{1} };
            \node at (4.5, 4.5) { \tt \textcolor{black}{2} };
            \node at (5.5, 4.5) { \tt \textcolor{black}{3} };
            \node at (6.5, 4.5) { \tt \textcolor{blue}{4} };
            \node at (7.5, 4.5) { \tt \textcolor{red}{5} };
 
            \draw[->] (6.5, 5.5) node[anchor=south] { $L$ } -- (6.5, 5.2);
            \draw[->] (8.5, 3.5) node[anchor=north] { $R$ } -- (8.5, 3.8);

        \end{tikzpicture}

\end{frame}

\begin{frame}[fragile]{Visualização do algoritmo do maior subvetor com, no máximo, $K=1$ números
ímpares}

        \begin{tikzpicture}
            \node[anchor=west] at (0, 7) { $ans = \textcolor{black}{3}$ };
            \node[anchor=west] at (0, 6.5) { $c = \textcolor{black}{1}$ };
            \node[opacity=0,anchor=west] at (0, 2) { $c = 0$ };

            \draw (3, 4) grid (8, 5);

            \node at (3.5, 4.5) { \tt \textcolor{black}{1} };
            \node at (4.5, 4.5) { \tt \textcolor{black}{2} };
            \node at (5.5, 4.5) { \tt \textcolor{black}{3} };
            \node at (6.5, 4.5) { \tt \textcolor{black}{4} };
            \node at (7.5, 4.5) { \tt \textcolor{red}{5} };
 
            \draw[->] (7.5, 5.5) node[anchor=south] { $L$ } -- (7.5, 5.2);
            \draw[->] (8.5, 3.5) node[anchor=north] { $R$ } -- (8.5, 3.8);

        \end{tikzpicture}

\end{frame}


\begin{frame}[fragile]{Implementação do maior subvetor com, no máximo, $K$ ímpares}
    \inputsnippet{cpp}{1}{21}{codes/subarray.cpp}
\end{frame}

\begin{frame}[fragile]{Implementação do maior subvetor com, no máximo, $K$ ímpares}
    \inputsnippet{cpp}{22}{42}{codes/subarray.cpp}
\end{frame}

\begin{frame}[fragile]{Menor elemento em um subvetor de tamanho $K$}

    \begin{itemize}
        \item Quando o intervalo delimitador por $[L, R)$ tem um tamanho $K$ fixo, a técnica dos
            dois ponteiros também é conhecida como \textit{sliding window}

        \item Seja $\vec{v}$ um veto com $N$ inteiros

        \item O problema de se determinar o menor elemento de cada um dos $N - K + 1$ subintervalos
            de tamanho $K$ pode ser resolvido com dois ponteiros e duas pilhas $P_{in}$ e 
            $P_{out}$

        \item As pilhas armazenarão pares de valores $(x, m)$, onde $x$ é o elemento do vetor a
            ser inserido, e $m$ é o menor dentre os valores contidos na pilha e o próprio $x$
    \end{itemize}

\end{frame}

\begin{frame}[fragile]{Menor elemento em um subvetor de tamanho $K$}

    \begin{itemize}
        \item Observe que $m$ será igual ao próprio $x$, caso a pilha esteja vazia

        \item Nos demais casos, $m = \min\{ x, M \}$, onde $M$ é o segundo elemento do par que
            está no topo da pilha

        \item Inicialmente, insera os primeiros $K$ elementos de $\vec{v}$ na pilha $P_{in}$

        \item O segundo elemento do par do topo de $P_{in}$ será a resposta para o primeiro 
            intervalo

        \item Faça $L = 0$ e $R = K$

        \item Caso a pilha $P_{out}$ esteja vazia mova, um a um, os elementos de $P_{in}$ para
            $P_{out}$, atualizando corretamente os valores de $m$

    \end{itemize}

\end{frame}

\begin{frame}[fragile]{Menor elemento em um subvetor de tamanho $K$}

    \begin{itemize}
        \item Exclua o elemento do topo de $P_{out}$: isto removerá o elemento
            $\vec{v}[L]$ das pilhas

        \item Em seguida, incremente $L$

        \item Agora, insira $\vec{v}[R]$ na pilha $P_{in}$  e incremente $R$

        \item Após estes ajustes, as pilhas conterão os elementos do intervalo de tamanho $K$
            adjacente ao anterior (isto é, a janela se moveu de $[L, R)$ para $[L + 1, R + 1)$)

        \item A resposta para $L$ será o menor entre os valores $m$ dos topos das pilhas

        \item Como cada ponteiros observa cada elemento de $\vec{v}$ no máximo uma vez, e cada
            elemento passa por, no máximo, duas pilhas, a complexidade do algoritmo é $O(N)$
    \end{itemize}

\end{frame}

\begin{frame}[fragile]{Visualização do menor elemento em cada um dos subvetores de tamanho $K = 3$}

    \begin{figure}
        \centering

        \begin{tikzpicture}
            \draw (1, 4) grid (11, 5);

            \node at (1.5, 4.5) { \tt 1 };
            \node at (2.5, 4.5) { \tt 3 };
            \node at (3.5, 4.5) { \tt 2 };
            \node at (4.5, 4.5) { \tt 3 };
            \node at (5.5, 4.5) { \tt 4 };
            \node at (6.5, 4.5) { \tt 5 };
            \node at (7.5, 4.5) { \tt 8 };
            \node at (8.5, 4.5) { \tt 2 };
            \node at (9.5, 4.5) { \tt 7 };
            \node at (10.5, 4.5) { \tt 5 };

            \draw (1, 0) -- (11, 0);

%            \draw[->] (1.5, 5.5) node[anchor=south] { $L$ } -- (1.5, 5.2);
            \draw[->,opacity=0] (1.5, 5.5) node[anchor=south] { $R$ } -- (1.5, 5.2);

            \draw[opacity=0] (3, 0) rectangle (5, 1);
            \draw[opacity=0] (3, 1) rectangle (5, 2);
            \draw[opacity=0] (3, 2) rectangle (5, 3);

            \draw[opacity=0] (7, 0) rectangle (9, 1);
            \draw[opacity=0] (7, 1) rectangle (9, 2);
            \draw[opacity=0] (7, 2) rectangle (9, 3);

            \node at (4, -0.5) { $P_{in}$ };
            \node at (8, -0.5) { $P_{out}$ };

        \end{tikzpicture}

    \end{figure}

\end{frame}

\begin{frame}[fragile]{Visualização do menor elemento em cada um dos subvetores de tamanho $K = 3$}

    \begin{figure}
        \centering

        \begin{tikzpicture}
            \draw (1, 4) grid (11, 5);

            \node at (1.5, 4.5) { \tt 1 };
            \node at (2.5, 4.5) { \tt 3 };
            \node at (3.5, 4.5) { \tt 2 };
            \node at (4.5, 4.5) { \tt 3 };
            \node at (5.5, 4.5) { \tt 4 };
            \node at (6.5, 4.5) { \tt 5 };
            \node at (7.5, 4.5) { \tt 8 };
            \node at (8.5, 4.5) { \tt 2 };
            \node at (9.5, 4.5) { \tt 7 };
            \node at (10.5, 4.5) { \tt 5 };

            \draw (1, 0) -- (11, 0);

%            \draw[->] (1.5, 5.5) node[anchor=south] { $L$ } -- (1.5, 5.2);
            \draw[->,opacity=1] (1.5, 5.5) node[anchor=south] { $R$ } -- (1.5, 5.2);

            \draw[opacity=1] (3, 0) rectangle (5, 1);
            \node at (4, 0.5) { \tt (1, 1) };

            \draw[opacity=0] (3, 1) rectangle (5, 2);
            \draw[opacity=0] (3, 2) rectangle (5, 3);

            \draw[opacity=0] (7, 0) rectangle (9, 1);
            \draw[opacity=0] (7, 1) rectangle (9, 2);
            \draw[opacity=0] (7, 2) rectangle (9, 3);

            \node at (4, -0.5) { $P_{in}$ };
            \node at (8, -0.5) { $P_{out}$ };

        \end{tikzpicture}

    \end{figure}

\end{frame}

\begin{frame}[fragile]{Visualização do menor elemento em cada um dos subvetores de tamanho $K = 3$}

    \begin{figure}
        \centering

        \begin{tikzpicture}
            \draw (1, 4) grid (11, 5);

            \node at (1.5, 4.5) { \tt 1 };
            \node at (2.5, 4.5) { \tt 3 };
            \node at (3.5, 4.5) { \tt 2 };
            \node at (4.5, 4.5) { \tt 3 };
            \node at (5.5, 4.5) { \tt 4 };
            \node at (6.5, 4.5) { \tt 5 };
            \node at (7.5, 4.5) { \tt 8 };
            \node at (8.5, 4.5) { \tt 2 };
            \node at (9.5, 4.5) { \tt 7 };
            \node at (10.5, 4.5) { \tt 5 };

            \draw (1, 0) -- (11, 0);

%            \draw[->] (1.5, 5.5) node[anchor=south] { $L$ } -- (1.5, 5.2);
            \draw[->,opacity=1] (2.5, 5.5) node[anchor=south] { $R$ } -- (2.5, 5.2);

            \draw[opacity=1] (3, 0) rectangle (5, 1);
            \node at (4, 0.5) { \tt (1, 1) };

            \draw[opacity=1] (3, 1) rectangle (5, 2);
            \node at (4, 1.5) { \tt (3, 1) };

            \draw[opacity=0] (3, 2) rectangle (5, 3);

            \draw[opacity=0] (7, 0) rectangle (9, 1);
            \draw[opacity=0] (7, 1) rectangle (9, 2);
            \draw[opacity=0] (7, 2) rectangle (9, 3);

            \node at (4, -0.5) { $P_{in}$ };
            \node at (8, -0.5) { $P_{out}$ };

        \end{tikzpicture}

    \end{figure}

\end{frame}

\begin{frame}[fragile]{Visualização do menor elemento em cada um dos subvetores de tamanho $K = 3$}

    \begin{figure}
        \centering

        \begin{tikzpicture}
            \draw (1, 4) grid (11, 5);

            \node at (1.5, 4.5) { \tt 1 };
            \node at (2.5, 4.5) { \tt 3 };
            \node at (3.5, 4.5) { \tt 2 };
            \node at (4.5, 4.5) { \tt 3 };
            \node at (5.5, 4.5) { \tt 4 };
            \node at (6.5, 4.5) { \tt 5 };
            \node at (7.5, 4.5) { \tt 8 };
            \node at (8.5, 4.5) { \tt 2 };
            \node at (9.5, 4.5) { \tt 7 };
            \node at (10.5, 4.5) { \tt 5 };

            \draw (1, 0) -- (11, 0);

%            \draw[->] (1.5, 5.5) node[anchor=south] { $L$ } -- (1.5, 5.2);
            \draw[->,opacity=1] (3.5, 5.5) node[anchor=south] { $R$ } -- (3.5, 5.2);

            % Pin
            \draw[opacity=1] (3, 0) rectangle (5, 1);
            \node at (4, 0.5) { \tt (1, 1) };

            \draw[opacity=1] (3, 1) rectangle (5, 2);
            \node at (4, 1.5) { \tt (3, 1) };

            \draw[opacity=1] (3, 2) rectangle (5, 3);
            \node at (4, 2.5) { \tt (2, 1) };

            % Pout
            \draw[opacity=0] (7, 0) rectangle (9, 1);
            \draw[opacity=0] (7, 1) rectangle (9, 2);
            \draw[opacity=0] (7, 2) rectangle (9, 3);

            \node at (4, -0.5) { $P_{in}$ };
            \node at (8, -0.5) { $P_{out}$ };

        \end{tikzpicture}

    \end{figure}

\end{frame}

\begin{frame}[fragile]{Visualização do menor elemento em cada um dos subvetores de tamanho $K = 3$}

    \begin{figure}
        \centering

        \begin{tikzpicture}
            \draw (1, 4) grid (11, 5);

            \node at (1.5, 4.5) { \texttt{\textbf{1}} };
            \node at (2.5, 4.5) { \tt \textcolor{blue}{3} };
            \node at (3.5, 4.5) { \tt \textcolor{blue}{2} };
            \node at (4.5, 4.5) { \tt 3 };
            \node at (5.5, 4.5) { \tt 4 };
            \node at (6.5, 4.5) { \tt 5 };
            \node at (7.5, 4.5) { \tt 8 };
            \node at (8.5, 4.5) { \tt 2 };
            \node at (9.5, 4.5) { \tt 7 };
            \node at (10.5, 4.5) { \tt 5 };

            \draw (1, 0) -- (11, 0);

%            \draw[->] (1.5, 5.5) node[anchor=south] { $L$ } -- (1.5, 5.2);
            \draw[->,opacity=1] (3.5, 5.5) node[anchor=south] { $R$ } -- (3.5, 5.2);

            % Pin
            \draw[opacity=1] (3, 0) rectangle (5, 1);
            \node at (4, 0.5) { \tt (1, 1) };

            \draw[opacity=1] (3, 1) rectangle (5, 2);
            \node at (4, 1.5) { \tt (3, 1) };

            \draw[opacity=1] (3, 2) rectangle (5, 3);
            \node at (4, 2.5) { \tt (2, \texttt{\textbf{1}}) };

            % Pout
            \draw[opacity=0] (7, 0) rectangle (9, 1);
            \draw[opacity=0] (7, 1) rectangle (9, 2);
            \draw[opacity=0] (7, 2) rectangle (9, 3);

            \node at (4, -0.5) { $P_{in}$ };
            \node at (8, -0.5) { $P_{out}$ };

        \end{tikzpicture}

    \end{figure}

\end{frame}

\begin{frame}[fragile]{Visualização do menor elemento em cada um dos subvetores de tamanho $K = 3$}

    \begin{figure}
        \centering

        \begin{tikzpicture}
            \draw (1, 4) grid (11, 5);

            \node at (1.5, 4.5) { {1} };
            \node at (2.5, 4.5) { \tt \textcolor{black}{3} };
            \node at (3.5, 4.5) { \tt \textcolor{black}{2} };
            \node at (4.5, 4.5) { \tt 3 };
            \node at (5.5, 4.5) { \tt 4 };
            \node at (6.5, 4.5) { \tt 5 };
            \node at (7.5, 4.5) { \tt 8 };
            \node at (8.5, 4.5) { \tt 2 };
            \node at (9.5, 4.5) { \tt 7 };
            \node at (10.5, 4.5) { \tt 5 };

            \draw (1, 0) -- (11, 0);

            \draw[->] (1.5, 5.5) node[anchor=south] { $L$ } -- (1.5, 5.2);
            \draw[->,opacity=1] (4.5, 5.5) node[anchor=south] { $R$ } -- (4.5, 5.2);

            % Pin
            \draw[opacity=1] (3, 0) rectangle (5, 1);
            \node at (4, 0.5) { \tt (1, 1) };

            \draw[opacity=1] (3, 1) rectangle (5, 2);
            \node at (4, 1.5) { \tt (3, 1) };

            \draw[opacity=1] (3, 2) rectangle (5, 3);
            \node at (4, 2.5) { \tt (2, {{1}}) };

            % Pout
            \draw[opacity=0] (7, 0) rectangle (9, 1);
            \draw[opacity=0] (7, 1) rectangle (9, 2);
            \draw[opacity=0] (7, 2) rectangle (9, 3);

            \node at (4, -0.5) { $P_{in}$ };
            \node at (8, -0.5) { $P_{out}$ };

        \end{tikzpicture}

    \end{figure}

\end{frame}

\begin{frame}[fragile]{Visualização do menor elemento em cada um dos subvetores de tamanho $K = 3$}

    \begin{figure}
        \centering

        \begin{tikzpicture}
            \draw (1, 4) grid (11, 5);

            \node at (1.5, 4.5) { {1} };
            \node at (2.5, 4.5) { \tt \textcolor{black}{3} };
            \node at (3.5, 4.5) { \tt \textcolor{black}{2} };
            \node at (4.5, 4.5) { \tt 3 };
            \node at (5.5, 4.5) { \tt 4 };
            \node at (6.5, 4.5) { \tt 5 };
            \node at (7.5, 4.5) { \tt 8 };
            \node at (8.5, 4.5) { \tt 2 };
            \node at (9.5, 4.5) { \tt 7 };
            \node at (10.5, 4.5) { \tt 5 };

            \draw (1, 0) -- (11, 0);

            \draw[->] (1.5, 5.5) node[anchor=south] { $L$ } -- (1.5, 5.2);
            \draw[->,opacity=1] (4.5, 5.5) node[anchor=south] { $R$ } -- (4.5, 5.2);

            % Pin
            \draw[opacity=1] (3, 0) rectangle (5, 1);
            \node at (4, 0.5) { \tt (1, 1) };

            \draw[opacity=1] (3, 1) rectangle (5, 2);
            \node at (4, 1.5) { \tt (3, 1) };

            %\draw[opacity=1] (3, 2) rectangle (5, 3);
            %\node at (4, 2.5) { \tt (2, {{1}}) };

            % Pout
            \draw[opacity=1] (7, 0) rectangle (9, 1);
            \node at (8, 0.5) { \tt (2, {{2}}) };

            \draw[opacity=0] (7, 1) rectangle (9, 2);
            \draw[opacity=0] (7, 2) rectangle (9, 3);

            \node at (4, -0.5) { $P_{in}$ };
            \node at (8, -0.5) { $P_{out}$ };

        \end{tikzpicture}

    \end{figure}

\end{frame}

\begin{frame}[fragile]{Visualização do menor elemento em cada um dos subvetores de tamanho $K = 3$}

    \begin{figure}
        \centering

        \begin{tikzpicture}
            \draw (1, 4) grid (11, 5);

            \node at (1.5, 4.5) { {1} };
            \node at (2.5, 4.5) { \tt \textcolor{black}{3} };
            \node at (3.5, 4.5) { \tt \textcolor{black}{2} };
            \node at (4.5, 4.5) { \tt 3 };
            \node at (5.5, 4.5) { \tt 4 };
            \node at (6.5, 4.5) { \tt 5 };
            \node at (7.5, 4.5) { \tt 8 };
            \node at (8.5, 4.5) { \tt 2 };
            \node at (9.5, 4.5) { \tt 7 };
            \node at (10.5, 4.5) { \tt 5 };

            \draw (1, 0) -- (11, 0);

            \draw[->] (1.5, 5.5) node[anchor=south] { $L$ } -- (1.5, 5.2);
            \draw[->,opacity=1] (4.5, 5.5) node[anchor=south] { $R$ } -- (4.5, 5.2);

            % Pin
            \draw[opacity=1] (3, 0) rectangle (5, 1);
            \node at (4, 0.5) { \tt (1, 1) };

            %\draw[opacity=1] (3, 1) rectangle (5, 2);
            %\node at (4, 1.5) { \tt (3, 1) };

            %\draw[opacity=1] (3, 2) rectangle (5, 3);
            %\node at (4, 2.5) { \tt (2, {{1}}) };

            % Pout
            \draw[opacity=1] (7, 0) rectangle (9, 1);
            \node at (8, 0.5) { \tt (2, {{2}}) };

            \draw[opacity=1] (7, 1) rectangle (9, 2);
            \node at (8, 1.5) { \tt (3, {{2}}) };

            \draw[opacity=0] (7, 2) rectangle (9, 3);

            \node at (4, -0.5) { $P_{in}$ };
            \node at (8, -0.5) { $P_{out}$ };

        \end{tikzpicture}

    \end{figure}

\end{frame}

\begin{frame}[fragile]{Visualização do menor elemento em cada um dos subvetores de tamanho $K = 3$}

    \begin{figure}
        \centering

        \begin{tikzpicture}
            \draw (1, 4) grid (11, 5);

            \node at (1.5, 4.5) { {1} };
            \node at (2.5, 4.5) { \tt \textcolor{black}{3} };
            \node at (3.5, 4.5) { \tt \textcolor{black}{2} };
            \node at (4.5, 4.5) { \tt 3 };
            \node at (5.5, 4.5) { \tt 4 };
            \node at (6.5, 4.5) { \tt 5 };
            \node at (7.5, 4.5) { \tt 8 };
            \node at (8.5, 4.5) { \tt 2 };
            \node at (9.5, 4.5) { \tt 7 };
            \node at (10.5, 4.5) { \tt 5 };

            \draw (1, 0) -- (11, 0);

            \draw[->] (1.5, 5.5) node[anchor=south] { $L$ } -- (1.5, 5.2);
            \draw[->,opacity=1] (4.5, 5.5) node[anchor=south] { $R$ } -- (4.5, 5.2);

            % Pin
            %\draw[opacity=1] (3, 0) rectangle (5, 1);
            %\node at (4, 0.5) { \tt (1, 1) };

            %\draw[opacity=1] (3, 1) rectangle (5, 2);
            %\node at (4, 1.5) { \tt (3, 1) };

            %\draw[opacity=1] (3, 2) rectangle (5, 3);
            %\node at (4, 2.5) { \tt (2, {{1}}) };

            % Pout
            \draw[opacity=1] (7, 0) rectangle (9, 1);
            \node at (8, 0.5) { \tt (2, {{2}}) };

            \draw[opacity=1] (7, 1) rectangle (9, 2);
            \node at (8, 1.5) { \tt (3, {{2}}) };

            \draw[opacity=1] (7, 2) rectangle (9, 3);
            \node at (8, 2.5) { \tt (1, {{1}}) };

            \node at (4, -0.5) { $P_{in}$ };
            \node at (8, -0.5) { $P_{out}$ };

        \end{tikzpicture}

    \end{figure}

\end{frame}

\begin{frame}[fragile]{Visualização do menor elemento em cada um dos subvetores de tamanho $K = 3$}

    \begin{figure}
        \centering

        \begin{tikzpicture}
            \draw (1, 4) grid (11, 5);

            \node at (1.5, 4.5) { {1} };
            \node at (2.5, 4.5) { \tt \textcolor{black}{3} };
            \node at (3.5, 4.5) { \tt \textcolor{black}{2} };
            \node at (4.5, 4.5) { \tt 3 };
            \node at (5.5, 4.5) { \tt 4 };
            \node at (6.5, 4.5) { \tt 5 };
            \node at (7.5, 4.5) { \tt 8 };
            \node at (8.5, 4.5) { \tt 2 };
            \node at (9.5, 4.5) { \tt 7 };
            \node at (10.5, 4.5) { \tt 5 };

            \draw (1, 0) -- (11, 0);

            \draw[->] (2.5, 5.5) node[anchor=south] { $L$ } -- (2.5, 5.2);
            \draw[->,opacity=1] (4.5, 5.5) node[anchor=south] { $R$ } -- (4.5, 5.2);

            % Pin
            %\draw[opacity=1] (3, 0) rectangle (5, 1);
            %\node at (4, 0.5) { \tt (1, 1) };

            %\draw[opacity=1] (3, 1) rectangle (5, 2);
            %\node at (4, 1.5) { \tt (3, 1) };

            %\draw[opacity=1] (3, 2) rectangle (5, 3);
            %\node at (4, 2.5) { \tt (2, {{1}}) };

            % Pout
            \draw[opacity=1] (7, 0) rectangle (9, 1);
            \node at (8, 0.5) { \tt (2, {{2}}) };

            \draw[opacity=1] (7, 1) rectangle (9, 2);
            \node at (8, 1.5) { \tt (3, {{2}}) };

            %\draw[opacity=1] (7, 2) rectangle (9, 3);
            %\node at (8, 2.5) { \tt (1, {{1}}) };

            \node at (4, -0.5) { $P_{in}$ };
            \node at (8, -0.5) { $P_{out}$ };

        \end{tikzpicture}

    \end{figure}

\end{frame}

\begin{frame}[fragile]{Visualização do menor elemento em cada um dos subvetores de tamanho $K = 3$}

    \begin{figure}
        \centering

        \begin{tikzpicture}
            \draw (1, 4) grid (11, 5);

            \node at (1.5, 4.5) { {1} };
            \node at (2.5, 4.5) { \tt \textcolor{black}{3} };
            \node at (3.5, 4.5) { \tt \textcolor{black}{2} };
            \node at (4.5, 4.5) { \tt 3 };
            \node at (5.5, 4.5) { \tt 4 };
            \node at (6.5, 4.5) { \tt 5 };
            \node at (7.5, 4.5) { \tt 8 };
            \node at (8.5, 4.5) { \tt 2 };
            \node at (9.5, 4.5) { \tt 7 };
            \node at (10.5, 4.5) { \tt 5 };

            \draw (1, 0) -- (11, 0);

            \draw[->] (2.5, 5.5) node[anchor=south] { $L$ } -- (2.5, 5.2);
            \draw[->,opacity=1] (5.5, 5.5) node[anchor=south] { $R$ } -- (5.5, 5.2);

            % Pin
            \draw[opacity=1] (3, 0) rectangle (5, 1);
            \node at (4, 0.5) { \tt (3, 3) };

            %\draw[opacity=1] (3, 1) rectangle (5, 2);
            %\node at (4, 1.5) { \tt (3, 1) };

            %\draw[opacity=1] (3, 2) rectangle (5, 3);
            %\node at (4, 2.5) { \tt (2, {{1}}) };

            % Pout
            \draw[opacity=1] (7, 0) rectangle (9, 1);
            \node at (8, 0.5) { \tt (2, {{2}}) };

            \draw[opacity=1] (7, 1) rectangle (9, 2);
            \node at (8, 1.5) { \tt (3, {{2}}) };

            %\draw[opacity=1] (7, 2) rectangle (9, 3);
            %\node at (8, 2.5) { \tt (1, {{1}}) };

            \node at (4, -0.5) { $P_{in}$ };
            \node at (8, -0.5) { $P_{out}$ };

        \end{tikzpicture}

    \end{figure}

\end{frame}

\begin{frame}[fragile]{Visualização do menor elemento em cada um dos subvetores de tamanho $K = 3$}

    \begin{figure}
        \centering

        \begin{tikzpicture}
            \draw (1, 4) grid (11, 5);

            \node at (1.5, 4.5) { {1} };
            \node at (2.5, 4.5) { \tt \textcolor{blue}{3} };
            \node at (3.5, 4.5) { \tt \texttt{\textbf{2}} };
            \node at (4.5, 4.5) { \tt \textcolor{blue}{3} };
            \node at (5.5, 4.5) { \tt 4 };
            \node at (6.5, 4.5) { \tt 5 };
            \node at (7.5, 4.5) { \tt 8 };
            \node at (8.5, 4.5) { \tt 2 };
            \node at (9.5, 4.5) { \tt 7 };
            \node at (10.5, 4.5) { \tt 5 };

            \draw (1, 0) -- (11, 0);

            \draw[->] (2.5, 5.5) node[anchor=south] { $L$ } -- (2.5, 5.2);
            \draw[->,opacity=1] (5.5, 5.5) node[anchor=south] { $R$ } -- (5.5, 5.2);

            % Pin
            \draw[opacity=1] (3, 0) rectangle (5, 1);
            \node at (4, 0.5) { \tt (3, 3) };

            %\draw[opacity=1] (3, 1) rectangle (5, 2);
            %\node at (4, 1.5) { \tt (3, 1) };

            %\draw[opacity=1] (3, 2) rectangle (5, 3);
            %\node at (4, 2.5) { \tt (2, {{1}}) };

            % Pout
            \draw[opacity=1] (7, 0) rectangle (9, 1);
            \node at (8, 0.5) { \tt (2, {{2}}) };

            \draw[opacity=1] (7, 1) rectangle (9, 2);
            \node at (8, 1.5) { \tt (3, \texttt{\textbf{2}}) };

            %\draw[opacity=1] (7, 2) rectangle (9, 3);
            %\node at (8, 2.5) { \tt (1, {{1}}) };

            \node at (4, -0.5) { $P_{in}$ };
            \node at (8, -0.5) { $P_{out}$ };

        \end{tikzpicture}

    \end{figure}

\end{frame}

\begin{frame}[fragile]{Visualização do menor elemento em cada um dos subvetores de tamanho $K = 3$}

    \begin{figure}
        \centering

        \begin{tikzpicture}
            \draw (1, 4) grid (11, 5);

            \node at (1.5, 4.5) { {1} };
            \node at (2.5, 4.5) { \tt \textcolor{black}{3} };
            \node at (3.5, 4.5) { \tt {{2}} };
            \node at (4.5, 4.5) { \tt \textcolor{black}{3} };
            \node at (5.5, 4.5) { \tt 4 };
            \node at (6.5, 4.5) { \tt 5 };
            \node at (7.5, 4.5) { \tt 8 };
            \node at (8.5, 4.5) { \tt 2 };
            \node at (9.5, 4.5) { \tt 7 };
            \node at (10.5, 4.5) { \tt 5 };

            \draw (1, 0) -- (11, 0);

            \draw[->] (2.5, 5.5) node[anchor=south] { $L$ } -- (2.5, 5.2);
            \draw[->,opacity=1] (5.5, 5.5) node[anchor=south] { $R$ } -- (5.5, 5.2);

            % Pin
            \draw[opacity=1] (3, 0) rectangle (5, 1);
            \node at (4, 0.5) { \tt (3, 3) };

            %\draw[opacity=1] (3, 1) rectangle (5, 2);
            %\node at (4, 1.5) { \tt (3, 1) };

            %\draw[opacity=1] (3, 2) rectangle (5, 3);
            %\node at (4, 2.5) { \tt (2, {{1}}) };

            % Pout
            \draw[opacity=1] (7, 0) rectangle (9, 1);
            \node at (8, 0.5) { \tt (2, {{2}}) };

            \draw[opacity=1] (7, 1) rectangle (9, 2);
            \node at (8, 1.5) { \tt (3, {{2}}) };

            %\draw[opacity=1] (7, 2) rectangle (9, 3);
            %\node at (8, 2.5) { \tt (1, {{1}}) };

            \node at (4, -0.5) { $P_{in}$ };
            \node at (8, -0.5) { $P_{out}$ };

        \end{tikzpicture}

    \end{figure}

\end{frame}

\begin{frame}[fragile]{Visualização do menor elemento em cada um dos subvetores de tamanho $K = 3$}

    \begin{figure}
        \centering

        \begin{tikzpicture}
            \draw (1, 4) grid (11, 5);

            \node at (1.5, 4.5) { {1} };
            \node at (2.5, 4.5) { \tt \textcolor{black}{3} };
            \node at (3.5, 4.5) { \tt {{2}} };
            \node at (4.5, 4.5) { \tt \textcolor{black}{3} };
            \node at (5.5, 4.5) { \tt 4 };
            \node at (6.5, 4.5) { \tt 5 };
            \node at (7.5, 4.5) { \tt 8 };
            \node at (8.5, 4.5) { \tt 2 };
            \node at (9.5, 4.5) { \tt 7 };
            \node at (10.5, 4.5) { \tt 5 };

            \draw (1, 0) -- (11, 0);

            \draw[->] (3.5, 5.5) node[anchor=south] { $L$ } -- (3.5, 5.2);
            \draw[->,opacity=1] (5.5, 5.5) node[anchor=south] { $R$ } -- (5.5, 5.2);

            % Pin
            \draw[opacity=1] (3, 0) rectangle (5, 1);
            \node at (4, 0.5) { \tt (3, 3) };

            %\draw[opacity=1] (3, 1) rectangle (5, 2);
            %\node at (4, 1.5) { \tt (3, 1) };

            %\draw[opacity=1] (3, 2) rectangle (5, 3);
            %\node at (4, 2.5) { \tt (2, {{1}}) };

            % Pout
            \draw[opacity=1] (7, 0) rectangle (9, 1);
            \node at (8, 0.5) { \tt (2, {{2}}) };

            %\draw[opacity=1] (7, 1) rectangle (9, 2);
            %\node at (8, 1.5) { \tt (3, {{2}}) };

            %\draw[opacity=1] (7, 2) rectangle (9, 3);
            %\node at (8, 2.5) { \tt (1, {{1}}) };

            \node at (4, -0.5) { $P_{in}$ };
            \node at (8, -0.5) { $P_{out}$ };

        \end{tikzpicture}

    \end{figure}

\end{frame}

\begin{frame}[fragile]{Visualização do menor elemento em cada um dos subvetores de tamanho $K = 3$}

    \begin{figure}
        \centering

        \begin{tikzpicture}
            \draw (1, 4) grid (11, 5);

            \node at (1.5, 4.5) { {1} };
            \node at (2.5, 4.5) { \tt \textcolor{black}{3} };
            \node at (3.5, 4.5) { \tt {{2}} };
            \node at (4.5, 4.5) { \tt \textcolor{black}{3} };
            \node at (5.5, 4.5) { \tt 4 };
            \node at (6.5, 4.5) { \tt 5 };
            \node at (7.5, 4.5) { \tt 8 };
            \node at (8.5, 4.5) { \tt 2 };
            \node at (9.5, 4.5) { \tt 7 };
            \node at (10.5, 4.5) { \tt 5 };

            \draw (1, 0) -- (11, 0);

            \draw[->] (3.5, 5.5) node[anchor=south] { $L$ } -- (3.5, 5.2);
            \draw[->,opacity=1] (6.5, 5.5) node[anchor=south] { $R$ } -- (6.5, 5.2);

            % Pin
            \draw[opacity=1] (3, 0) rectangle (5, 1);
            \node at (4, 0.5) { \tt (3, 3) };

            \draw[opacity=1] (3, 1) rectangle (5, 2);
            \node at (4, 1.5) { \tt (4, 3) };

            %\draw[opacity=1] (3, 2) rectangle (5, 3);
            %\node at (4, 2.5) { \tt (2, {{1}}) };

            % Pout
            \draw[opacity=1] (7, 0) rectangle (9, 1);
            \node at (8, 0.5) { \tt (2, {{2}}) };

            %\draw[opacity=1] (7, 1) rectangle (9, 2);
            %\node at (8, 1.5) { \tt (3, {{2}}) };

            %\draw[opacity=1] (7, 2) rectangle (9, 3);
            %\node at (8, 2.5) { \tt (1, {{1}}) };

            \node at (4, -0.5) { $P_{in}$ };
            \node at (8, -0.5) { $P_{out}$ };

        \end{tikzpicture}

    \end{figure}

\end{frame}

\begin{frame}[fragile]{Visualização do menor elemento em cada um dos subvetores de tamanho $K = 3$}

    \begin{figure}
        \centering

        \begin{tikzpicture}
            \draw (1, 4) grid (11, 5);

            \node at (1.5, 4.5) { {1} };
            \node at (2.5, 4.5) { \tt \textcolor{black}{3} };
            \node at (3.5, 4.5) { \tt \texttt{\textbf{2}} };
            \node at (4.5, 4.5) { \tt \textcolor{blue}{3} };
            \node at (5.5, 4.5) { \tt \textcolor{blue}{4} };
            \node at (6.5, 4.5) { \tt 5 };
            \node at (7.5, 4.5) { \tt 8 };
            \node at (8.5, 4.5) { \tt 2 };
            \node at (9.5, 4.5) { \tt 7 };
            \node at (10.5, 4.5) { \tt 5 };

            \draw (1, 0) -- (11, 0);

            \draw[->] (3.5, 5.5) node[anchor=south] { $L$ } -- (3.5, 5.2);
            \draw[->,opacity=1] (6.5, 5.5) node[anchor=south] { $R$ } -- (6.5, 5.2);

            % Pin
            \draw[opacity=1] (3, 0) rectangle (5, 1);
            \node at (4, 0.5) { \tt (3, 3) };

            \draw[opacity=1] (3, 1) rectangle (5, 2);
            \node at (4, 1.5) { \tt (4, 3) };

            %\draw[opacity=1] (3, 2) rectangle (5, 3);
            %\node at (4, 2.5) { \tt (2, {{1}}) };

            % Pout
            \draw[opacity=1] (7, 0) rectangle (9, 1);
            \node at (8, 0.5) { \tt (2, \texttt{\textbf{2}}) };

            %\draw[opacity=1] (7, 1) rectangle (9, 2);
            %\node at (8, 1.5) { \tt (3, {{2}}) };

            %\draw[opacity=1] (7, 2) rectangle (9, 3);
            %\node at (8, 2.5) { \tt (1, {{1}}) };

            \node at (4, -0.5) { $P_{in}$ };
            \node at (8, -0.5) { $P_{out}$ };

        \end{tikzpicture}

    \end{figure}

\end{frame}


\begin{frame}[fragile]{Implementação}
    \inputsnippet{cpp}{1}{21}{codes/minimum.cpp}
\end{frame}

\begin{frame}[fragile]{Implementação}
    \inputsnippet{cpp}{22}{42}{codes/minimum.cpp}
\end{frame}

\begin{frame}[fragile]{Implementação}
    \inputsnippet{cpp}{43}{63}{codes/minimum.cpp}
\end{frame}

\begin{frame}[fragile]{Implementação}
    \inputsnippet{cpp}{64}{84}{codes/minimum.cpp}
\end{frame}

\begin{frame}[fragile]{2SUM}

    \begin{itemize}
        \item O 2SUM é um problema bastante conhecido: dado um vetor $\vec{v}$ com $N$ inteiros,
            determine se, para um dado S, existem dois elementos $v_i$ e $v_j$ tais que 
            $v_i + v_j = S$

        \item Há $O(N^2)$ pares de elementos de $\vec{v}$, de modo que uma solução que verificasse
            cada um destes pares teria complexidade $O(N^2)$

        \item É possível usar a técnica dos dois ponteiros para reduzir complexidade

        \item Se $\vec{v}$ não estiver ordenado, ordene-o em ordem não-decrescente

        \item Inicie os ponteiros $L = 0$ e $R = N - 1$

        \item Enquanto $R > L$ (ou $R\geq L$, se um elemento puder se utilizado duas vezes) e
            $\vec{v}[L] + \vec{v}[R] > S$, decremente $R$

    \end{itemize}

\end{frame}

\begin{frame}[fragile]{2SUM}

    \begin{itemize}
        \item Caso $\vec{v}[L] + \vec{v}[R] = S$, a solução foi encontrada e o algoritmo
            termina

        \item Caso contrário, incremente $L$ e repita o processo

        \item Se, em algum momento, $R < L$, o algoritmo finalizará
            e o problema não tem solução

        \item Este problema é um exemplo onde os ponteiros avançam em direções opostas

        \item Como cada elemento é observado, no máximo, uma única vez por cada ponteiros, a
            complexidade do algoritmo é $O(N)$

        \item Caso o vetor não esteja inicialmente ordenado, a complexidade passa a ser
            $O(N\log N)$, devido ao algoritmo de ordenação
    \end{itemize}

\end{frame}

\begin{frame}[fragile]{Implementação do 2SUM}
    \inputsnippet{cpp}{1}{21}{codes/2sum.cpp}
\end{frame}

\begin{frame}[fragile]{Implementação do 2SUM}
    \inputsnippet{cpp}{22}{42}{codes/2sum.cpp}
\end{frame}
