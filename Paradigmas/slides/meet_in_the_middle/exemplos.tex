\section{Exemplos de aplicação do encontro no meio}

\begin{frame}[fragile]{Soma de subconjuntos}

    \begin{itemize}
        \item Considere o seguinte problema: dados $N$ inteiros $x_1, x_2, \ldots, x_N$, e um
            inteiro $x$, determine se existem $k$ ($1\leq k\leq N$) inteiros 
            $x_{i_1}, x_{i_2}, \ldots, x_{i_k}$ tais que
            \[
                x = x_{i_1} + x_{i_2} + \ldots + x_{i_k}
            \] 

        \item Uma solução por força bruta avaliaria todos os $2^N$ subconjuntos possíveis, de modo
            que a complexidade seria igual a $O(N2^N)$, sendo viável para valores de $N \approx 20$

        \item Contudo, o encontro no meio pode ser utilizado para obter uma complexidade
            igual a $O(N2^{N/2}\log N)$, de modo que o problema poderia ser resolvido para valores de
            $N\approx 40$
       
    \end{itemize}

\end{frame}

\begin{frame}[fragile]{Soma de subconjuntos}

    \begin{itemize}
        \item A solução iniciaria subdividindo os inteiros em dois grupos
        \[
            G_1 = \lbrace x_1, x_2, \ldots, x_{\lfloor N/2\rfloor}\rbrace
        \]
        e
        \[
            G_2 = \lbrace x_{\lfloor N/2\rfloor + 1}, x_{\lfloor N/2\rfloor + 2}, \ldots, x_{N}\rbrace
        \]

        \item Em seguida, geraria os subconjuntos $S_1$ e $S_2$ das somas de todos os subconjuntos de
            $G_1$ e $G_2$, respectivamente

        \item Assim, ambos subconjuntos teriam, aproximadamente, $2^{N/2}$ elementos

        \item Em seguida, para cada elemento $s\in S_1$, uma busca binária tentaria localizar o
            elemento $r = x - s$ no conjunto $S_2$

        \item Assim, caso $r\in S_2$, a solução seria $x + r = s$, e a complexidade seria
            $O(N2^{N/2}\log N)$
    \end{itemize}

\end{frame}

\input{sum_view}

\begin{frame}[fragile]{Implementação da soma de subconjuntos}
    \inputsnippet{cpp}{1}{20}{codes/sum.cpp}
\end{frame}

\begin{frame}[fragile]{Implementação da soma de subconjuntos}
    \inputsnippet{cpp}{22}{36}{codes/sum.cpp}
\end{frame}

\begin{frame}[fragile]{\it 4sum}

    \begin{itemize}
        \item \textit{4sum} é um problema comum em entrevistas de emprego

        \item Dado um vetor de inteiros $xs = \lbrace x_1, x_2, \ldots, x_N\rbrace$, determine 
            $a, b, c, d\in xs$ tais que $a + b + c + d = 0$, se existirem

        \item Observe que um mesmo elemento de $xs$ pode ser escolhido mais de uma vez

        \item Uma solução \textit{naive} computaria e checaria todas as $O(N^4)$ quádruplas

        \item Uma solução, também de busca completa, computaria $O(N^3)$ triplas $(a, b, c)$ e 
            procuraria pelo elemento $d = -(a + b + c)$ em $xs$ usando busca binária, melhorando
            a complexidade para $O(N^3\log N)$

        \item Observado que $a + b = -(c + d)$ e usando \textit{hashes} em conjunto com o encontro
            no meio, é possível resolver este problema com complexidade $O(N^2)$
    \end{itemize}

\end{frame}

\begin{frame}[fragile]{Implementação de {\it 4sum} em $O(N^2)$}
    \inputsnippet{cpp}{1}{19}{codes/4sum.cpp}
\end{frame}
