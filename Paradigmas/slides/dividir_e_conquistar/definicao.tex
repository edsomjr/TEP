\section{Dividir e Conquistar}

\begin{frame}[fragile]{Definição}

    \begin{itemize}
        \item Dividir e conquistar (ou divisão e conquista) é um paradigma de solução de problemas
            que divide o problema em subproblemas menores sucessivas vezes até que o problema
            possa ser (trivialmente) resolvido

        \item Em seguida, ele combina as soluções dos subproblemas na solução do problema original

        \item Sistematicamente, são três etapas:

        \begin{enumerate}
            \item dividir o problema em subproblemas menores (divisão);

            \item resolver os subproblemas recursivamente, se ainda forem grandes o suficiente,
                ou diretamente, se forem pequenos o bastante (conquista);

            \item combinar as soluções dos subproblemas para formar a solução do problema
                que foi dividido (fusão).
        \end{enumerate}

    \end{itemize}

\end{frame}

\begin{frame}[fragile]{Complexidade de soluções baseadas em divisão e conquista}

    \begin{itemize}
        \item Dada a natureza do paradigma, algoritmos baseados em divisão e conquista são
            implementados por meio de recursão

        \item Esta característica facilita a implementação e deixa evidente as etapas do
            paradigma

        \item Contudo, a complexidade assintótica do algoritmo pode não ser óbvia

        \item Para computar a complexidade o primeiro passo é escrever a relação de recorrência
            que associa o número de operações $T(n)$ necessárias para a resolução do problema
            com o número de operações necessárias para a solução dos subproblemas

        \item Uma vez estabelecida a recorrência, deve-se utilizar, se possível, o Teorema Mestre
    \end{itemize}

\end{frame}

\begin{frame}[fragile]{Teorema Mestre}

    \metroset{block=fill}
    \begin{block}{Teorema Mestre}
        Sejam $a\geq 1$ e $b > 1$ constantes, $f(n)$ uma função tais que
        \[
            T(n) = aT\left(\left\lfloor n/b\right\rfloor\right) + f(n)
        \]

        Se
        \begin{enumerate} 
            \item $f(n) = \Theta(n^{\log_b a - \varepsilon})$ para alguma constante $\varepsilon > 0$,
                então $T(n) = \Theta(n^{\log_b a})$

            \item $f(n) = \Theta(n^{\log_b a})$ então $T(n) = \Theta(n^{\log_b a}\log n)$

            \item $f(n) = \Theta(n^{\log_b a + \varepsilon})$ para alguma constante $\varepsilon > 0$,
                e se $af(n/b) \leq cf(n)$ para alguma constante $c > 1$ e $n$ grande o suficiente,
                então $T(n) = \Theta(f(n))$
        \end{enumerate} 
    \end{block}

\end{frame}

\begin{frame}[fragile]{Variáveis do Teorema Mestre}

    \begin{itemize}
        \item Considere a relação de recorrência citada no Teorema Mestre:
        \[
            T(n) = aT\left(\left\lfloor n/b\right\rfloor\right) + f(n),
        \]

        \item $T(n)$ é o número de operações para resolver um problema cuja entrada tem tamanho $n$

        \item $a$ é o número de subproblemas após a divisão

        \item $\lfloor n/b\rfloor$ é o tamanho de cada subproblema

        \item Observe que cada subproblema deve ter o mesmo tamanho, e este tamanho deve ser
            um número inteiro

        \item $f(n)$ é o número de operações necessárias para dividir o problema e para 
            combinar as soluções dos subproblemas para formar a solução do problema
    \end{itemize}

\end{frame}

\begin{frame}[fragile]{Interpretação do Teorema Mestre}

    \begin{itemize}
        \item O Teorema compara a função que representa o custo $f(n)$ das etapas de divisão e
            fusão com a função $g(n) = n^{\log_a b}$, onde o número de subproblemas é $O(g(n))$

        \item No primeiro caso, o número de subproblemas é maior do que o custo da divisão e da
            fusão, e domina a complexidade

        \item No terceiro caso, o custo da divisão e da fusão é maior do que solução dos
            subproblemas, e domina a complexidade

        \item No segundo, ambas são equivalentes, de modo que o custo de divisão e fusão é
            multiplicado por um fator logarítmico, pois 
            \[
                \Theta(n^{\log_b a}\log n) = \Theta(f(n)\log n)
            \]

        \item Note que o Teorema Mestre não cobre todos os cenários possíveis
    \end{itemize}

\end{frame}
