\section{Implementação {\it top-down}}

\begin{frame}[fragile]{Implementação {\it top-down}}

    \begin{itemize}
        \item Uma implementação \textit{top-down} de um algoritmo de programação dinâmica parte
            da relação de recorrência para produzir uma função recursiva

        \item Após a verificação dos casos-base, a tabela é consultada para se determinar se o 
            estado já foi computado ou não

        \item Em caso afirmativo, o valor armazenado na tabela é retornado

        \item Caso contrário, o subproblema associado ao estado é solucionado por meio de recursão

        \item Em seguida, a solução obtida é armazenada na tabela e então retornado
    \end{itemize}

\end{frame}

\begin{frame}[fragile]{Tabela de memorização}

    \begin{itemize}
        \item A tabela de memorização deve ter tamanho suficiente para armazenar todos os estados
            possíveis (dentro dos limites das variáveis que compõem o estado)

        \item Além disso, esta tabela deve ser iniciada com um valor que sinalize que o estado
            associado não foi computado ainda

        \item Este valor sentinela deve corresponder a um valor que não pode ser uma solução de
            nenhum estado (-1, $\infty$, etc)

        \item Assim, a complexidade em memória do algoritmo será, no mínimo, $O(S)$, onde $S$
            é o total de estados possíveis
    \end{itemize}

\end{frame}

\begin{frame}[fragile]{Características da implementação {\it top-down}}

    \begin{itemize}
        \item A principal vantagem da implementação \textit{top-down} é a simplicidade: em geral,
            é uma tradução literal da relação de recorrência que representa a solução do problema

        \item Outra vantagem é que apenas os estados necessários à solução são computados

        \item Por outro lado, a tabela deve ter dimensões que comportem todos os estados possíveis

        \item Também há um custo de execução associado às chamadas recursivas 

        \item A ordem em que os subproblemas associados aos estados são resolvidos corresponde
            a uma DFS no grafo cujos vértices são os estados e as arestas as transições
    \end{itemize}

\end{frame}

\begin{frame}[fragile]{Exemplo: Coeficientes Binomiais}

    \begin{itemize}

        \item O coeficiente binomial $\binom{n}{m}$ é dado
        \[
            \binom{n}{m} = \frac{n!}{(n - m)!m!},
        \]
        se $n \geq m$, e $\binom{n}{m} = 0$, caso contrário

        \item Os coeficientes $\binom{i}{j}$, com $0\leq j\leq i$, formam a $i$-ésima linha do
            Triângulo de Pascal:

            \begin{figure}
    \centering

    \begin{tikzpicture}
        \node at (5, 5) { \tt 1 };
        \node at (4.5, 4.5) { \tt 1 };
        \node at (5.5, 4.5) { \tt 1 };
        \node at (4, 4) { \tt 1 };
        \node at (5, 4) { \tt 2 };
        \node at (6, 4) { \tt 1 };
        \node at (3.5, 3.5) { \tt 1 };
        \node at (4.5, 3.5) { \tt 3 };
        \node at (5.5, 3.5) { \tt 3 };
        \node at (6.5, 3.5) { \tt 1 };
        \node at (3, 3) { \tt 1 };
        \node at (4, 3) { \tt 4 };
        \node at (5, 3) { \tt 6 };
        \node at (6, 3) { \tt 4 };
        \node at (7, 3) { \tt 1 };
        \node at (2.5, 2.5) { \tt 1 };
        \node at (3.5, 2.5) { \tt 5 };
        \node at (4.5, 2.5) { \tt 10 };
        \node at (5.5, 2.5) { \tt 10 };
        \node at (6.5, 2.5) { \tt 5 };
        \node at (7.5, 2.5) { \tt 1 };

    \end{tikzpicture}

\end{figure}


    \end{itemize}
 
\end{frame}

\begin{frame}[fragile]{Exemplo: Coeficientes Binomiais}

    \begin{itemize}
        \item O Triângulo de Pascal permite a visualização de uma importante relação dos
            coeficientes binomiais

        \item Se $m > 0$ e $m < n$, então o coeficiente $\binom{n}{m}$ é dado pela soma de
            dois coeficientes da linha anterior: o imediatamente acima e seu antecessor

        \item Em notação matemática,
        \[
            \binom{n}{m} = \binom{n - 1}{m} + \binom{n - 1}{m - 1}
        \], se $0 < m < n$

        \item Se $m = 0$ ou $m = n$, então $\binom{n}{m} = 1$

        \item Estas relação caracterizam a recorrência e os casos bases que permitem uma
            implementação de programação dinâmica para os coeficientes binomiais
    \end{itemize}

\end{frame}

\begin{frame}[fragile]{Implementação dos coeficientes binomiais com DP}
    \inputsnippet{cpp}{1}{21}{codes/binom_td.cpp}
\end{frame}

\begin{frame}[fragile]{Melhoria e correção da implementação}

    \begin{itemize}
        \item A implementação anterior pode ser melhorada em um ponto, e também corrigida,
            uma vez que não produz a saída correta para todas as entradas

        \item Cada linha do Triângulo de Pascal é simétrica: assim
        \[
            \binom{n}{m} = \binom{n}{n - m}
        \]

        \item Este fato permite computa apenas a metade das colunas, e este espaço extra pode
            ser repassado para as linhas

        \item Os valores dos coeficientes crescem muito rapidamente, de modo que estouram a
            capacidade de armazenamento de um \code{cpp}{long long}

        \item Uma forma de tratar isso é computar os restos destes coeficientes para um módulo
            $M$ dado (em competições, é comum o valor $M = 10^9 + 7$)
    \end{itemize}

\end{frame}

\begin{frame}[fragile]{Implementação dos coeficientes binomiais melhorada}
    \inputsnippet{cpp}{1}{20}{codes/binom_td2.cpp}
\end{frame}

\begin{frame}[fragile]{Implementação dos coeficientes binomiais melhorada}
    \inputsnippet{cpp}{21}{41}{codes/binom_td2.cpp}
\end{frame}
