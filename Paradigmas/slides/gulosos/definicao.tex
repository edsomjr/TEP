\section{Algoritmos Gulosos}

\begin{frame}[fragile]{Definição}

    \begin{itemize}
        \item Um algoritmo é dito guloso (\textit{greedy}, em inglês) se, a cada iteração,
            ele faz uma escolha local ótima que converge para a solução global ótima

        \item Para a estratégia gulosa levar a um algoritmo correto para todas as entradas
            é preciso que o problema tenha duas características:

        \begin{enumerate}
            \item ter subestruturas ótimas
            \item ter a propriedade gulosa
        \end{enumerate}

        \item Ter  subestruturas ótimas significa que a solução global ótima contém soluções
            ótimas para os subproblemas que o compõem o problema principal

        \item A propriedade gulosa garante que, uma vez tomada a decisão local ótima, ela não
            precisará ser reconsiderada 

        \item Em geral, é difícil provar que um problema tem ambas características
    \end{itemize}

\end{frame}

\begin{frame}[fragile]{Algoritmos gulosos em programação competitiva}

    \begin{itemize}
        \item É fácil pensar em algoritmos gulosos: difícil é provar a corretude destes algoritmos

        \item Em competições, soluções gulosas ou estão corretas ou levam ao WA

        \item Raramente obtém um TLE, por terem complexidades assintóticas relativamente baixas

        \item Mesmo que o problema possa ser resolvido por um algoritmo guloso, a estratégia 
            de escolha das soluções ótimas para os subproblemas pode não ser óbvia

        \item Em geral, se o tamanho da entrada não for proibitivo, é melhor tentar uma abordagem
            por busca completa ou por programação dinâmica antes de tentar uma abordagem gulosa
    \end{itemize}

\end{frame}
