\section{Exemplos de algoritmos gulosos}

\begin{frame}[fragile]{Problema do troco}

    \metroset{block=fill}
    \begin{block}{Problema do Troco}
        Seja $C = \lbrace c_1, c_2, \ldots, c_N\rbrace\subset \mathbb{N}$, com $c_i < c_j$ se 
        $i < j$, o conjunto dos tipos de moedas disponíveis.
        O problema do troco consiste em, dada uma quantia $Q\in\mathbb{N}$, determinar $N$ inteiros 
        não-negativos $k_i$ tais que
        \[
            k_1c_1 + k_2c_2 + \ldots + k_Nc_N = Q
        \] e que a soma
        \[
            \sum_{i = 1}^N k_i
        \]
        seja mínima.
    \end{block}

\end{frame}

\begin{frame}[fragile]{Algoritmo guloso para o problema do troco}

    \begin{itemize}
        \item A estratégia gulosa para o problema do troco é escolher o maior número possíveis
            de moedas do tipo $c_N$, em seguida o maior número de moedas do tipo $c_{N - 1}$,
            e assim por diante, até a moeda $c_1$

        \item Por exemplo, se $C = \lbrace 1, 2, 5, 10, 25, 50\rbrace$ e $Q = 198$, inicialmente
            se escolhe $k_6 = 3$ moedas de $50$

        \item Em seguida, toma-se uma moeda de $k_5 = 25$, totalizando $3\times 150 + 1\times 25 
            = 175$

        \item Agora, seguindo o mesmo raciocínio, escolhe-se $k_4 = 2$, $k_3 = 0$, $k_2 = 1$ e 
            $k_1 = 1$, de modo que
        \[
            3\times 50 + 1\times 25 + 2\times 10 + 0\times 5 + 1\times 2 + 1\times 1 = 198
        \]
        e
        \[
            \sum_{i = 1}^N k_i = 3 + 1 + 2 + 0 + 1 + 1 = 8
        \]

        \item A complexidade desta solução é $O(N)$, pois são feitas, no máximo, $N$ divisões,
            uma para cada tipo de moeda
            
    \end{itemize}

\end{frame}

\begin{frame}[fragile]{Implementação da estratégia gulosa para o problema do troco}
    \inputsnippet{cpp}{5}{20}{codes/coin_change.cpp}
\end{frame}

\begin{frame}[fragile]{Observações sobre a estratégia gulosa para o problema do troco}

    \begin{itemize}
        \item Observe que, se as moedas fossem processadas da menor para a maior, o algoritmo
            não produziria a resposta correta

        \item No exemplo anterior, a resposta seria $k_1 = 198$ e $k_i = 0$ para $i > 1$

        \item Além disso, considere $C = \lbrace 1, 4, 5\rbrace$ e $Q = 8$

        \item O algoritmo guloso para o problema do troco produziria $k_3 = 1$, $k_2 = 0$ e 
            $k_1 = 3$, num total de $4$ moedas

        \item Porém, é possível gerar $8$ com apenas duas moedas, fazendo $k_3 = k_1 = 0$ e 
            $k_2 = 2$

        \item Assim, o algoritmo guloso para o problema do troco não produz a saída correta para
            todas as entradas

        \item Um conjunto $C$ para o qual o algoritmo guloso produz sempre a saída correta 
        para o problema do troco é
            denominada base canônica

        \item Dentre as bases canônicas mais comuns estão o sistema monetário vigente (base do
            primeiro exemplo), as potências de uma base $b > 1$ e os números de Fibonacci
        
    \end{itemize}

\end{frame}

\begin{frame}[fragile]{Agendamento de eventos}

    \metroset{block=fill}
    \begin{block}{Agendamento de eventos}
        Seja $E = \lbrace e_1, e_2, \ldots, e_N\rbrace$ um conjunto de $N$ eventos $e_i$ cuja
            duração é determinada pelos intervalos $[a_i, b_i)$. O problema do agendamento de
            eventos consiste em determinar um subconjunto $S\subset E$ tal que
        \[
            \forall s_i, s_j\in S, s_i \cap s_j = \emptyset, \ \ \mbox{se}\ \ i\neq j
        \]
        e $|S|$ é o maior possível.
    \end{block}

\end{frame}

\begin{frame}[fragile]{Algoritmo guloso para o agendamento de eventos}

    \begin{itemize}
        \item O problema do agendamento de eventos pode ser resolvido com um algoritmo guloso
            para todas as entradas possíveis, embora não seja óbvio à primeira vista qual seria
            a escolha ótima para cada subproblema

        \item Por exemplo, escolher pelo evento de menor duração ainda não escolhido e que não
            conflita com os já escolhidos leva a resposta errada quando $E = \lbrace 
                [1, 5), [4, 7), [6, 10)\rbrace$, pois o algoritmo escolheria apenas o evento 
            $e_2$, sendo que a solução correta seria a escolha dos eventos $e_1$ e $e_3$

        \item Outra abordagem incorreta seria escolher os eventos que começam o mais cedo possível
            ainda não selecionados

        \item O algoritmo correto resulta da escolha dos eventos que terminam o mais cedo
            possível e que não conflitam com os já escolhidos

        \item Este algoritmo tem complexidade $O(N\log N)$, pois é preciso ordenar os eventos
            pelos valores $b_i$
    \end{itemize}

\end{frame}

\begin{frame}[fragile]{Implementação do algoritmo guloso para o agendamento de eventos}
    \inputsnippet{cpp}{1}{17}{codes/scheduling.cpp}
\end{frame}

\begin{frame}[fragile]{Implementação do algoritmo guloso para o agendamento de eventos}
    \inputsnippet{cpp}{19}{31}{codes/scheduling.cpp}
\end{frame}
