\section{Definição}

\begin{frame}[fragile]{{\it Definição}}

    \metroset{block=fill}
    \begin{block}{Problema da Maior Subsequência Crescente}
        Considere uma sequência $a = \{ a_1, a_2, \ldots, a_N \}$.
        Uma subsequência 
        \[
            b = \{ b_1, b_2, \ldots, b_k \} = \{ a_{i_1}, a_{i_2}, \ldots, a_{i_k} \}
        \]
        de $a$ é a maior subsequência crescente (\textit{longest increasing subsequence} --
        LIS) se valem as seguintes condições:
        \begin{enumerate}[i.]
            \item $i_1 < i_2 < \ldots < i_k$, 
            \item $b_i < b_j$ se $i < j$, e
            \item $k$ é máximo.
        \end{enumerate}
    \end{block}

\end{frame}

\begin{frame}[fragile]{Características da maior subsequência crescente}

    \begin{itemize}
        \item O problema da maior subsequência crescente tem solução para qualquer 
            sequência $a$, uma vez que qualquer subsequência composta por um único elemento
            é uma subsequência crescente (não necessariamente a maior)

        \item A maior subsequência não é única: por exemplo, a sequência $a = \{ 4, 1, 5, 2,
            6, 3 \}$ tem várias subsequências com três elementos ($b_1 = \{ 4, 5 ,6 \}$ e 
            $b_2 = \{ 1, 2, 3 \}$ são duas delas)

        \item Os elementos de $a$ podem ser de qualquer tipo, desde que o operador 
            \texttt{<} esteja definido
    \end{itemize}

\end{frame}
