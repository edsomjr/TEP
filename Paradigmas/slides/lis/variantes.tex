\section{Variantes}

\begin{frame}[fragile]{Solução linearítmica}

    \begin{itemize}
        \item A LIS pode ser determinada por meio de um algoritmo de programação dinâmica linearítmico

        \item Este algoritmo se baseia em um estado diferente do utilizado no algoritmo
            quadrático e em uma transição mais eficiente

        \item Seja $lis(k, i)$ o menor elemento que finaliza uma subsequência crescente de
            $\{ a_1, a_2, \ldots, a_i \}$ de tamanho $k$

        \item A segunda dimensão deste estado será implícita, sem necessidade de alocação de
            memória (caso contrário, a alocação da tabela de memória já tornaria este 
            algoritmo quadrático)

        \item Os casos bases acontecem com $i = 0$, isto é, antes de se considerar qualquer
            elemento da sequência

    \end{itemize}

\end{frame}

\begin{frame}[fragile]{Solução linearítmica}

    \begin{itemize}
        \item Para simplificar a implementação, pode-se fazer $lis(k, 0) = \infty$, para 
            todo $k\in [1, N]$ e $lis(0, 0) = 0$ (ou qualquer outro valor sentinela, desde
            que seja estritamente menor do que qualquer $a_i$)

        \item Para cada $i$, apenas um dos $lis(k, i)$ será atualizado

        \item Importante notar que os elementos da sequência $lis(1, i), lis(2, i), \ldots$
            estarão em ordem crescente

        \item Por meio de uma busca binária, deve-se identificar o primeiro índice $j$ tal que
            $lis(j, i - 1)$ seja estritamente maior do que $a_i$

        \item Daí, $lis(j, i) = a_i$

        \item O tamanho da LIS será igual ao maior índice $j$ tal que $lis(j, N) < \infty$
    \end{itemize}

\end{frame}

\input{lis_view}

\begin{frame}[fragile]{Implementação linearítmica da LIS}
    \inputsnippet{cpp}{5}{24}{codes/lis_linearitmico.cpp}
\end{frame}
