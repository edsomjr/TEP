\section{Conjuntos notáveis}

\begin{frame}[fragile]{Produto cartesiano}

    \begin{itemize}
        \item Dados dois conjuntos $A$ e $B$, o produto cartesiano $A\times B$ de $A$ por
            $B$ é o conjunto de todos os pares ordenados cujo primeiro elemento pertence
            a $A$ e o segundo pertence a $B$, isto é,
        \[
            A\times B = \lbrace (a, b)\ |\ a\in A, b\in B\rbrace
        \]

        \item Se $|A| = m$ e $|B| = n$ então $|A\times B| = mn$

        \item Por exemplo, se $A = \lbrace a, b, c\rbrace$ e $B = \lbrace 1, 2\rbrace$, então
        \[
            A\times B = \lbrace (a, 1), (a, 2), (b, 1), (b, 2), (c, 1), (c, 2)\rbrace
        \]

        \item Observe que $A\times B \neq B\times A$ e que $A\times A = A^2$

        \item Se ambos conjuntos são finitos, o produto cartesiano pode ser obtido diretamente
            por meio de um laço duplo
    \end{itemize}

\end{frame}

\begin{frame}[fragile]{Implementação da geração do produto cartesiano}
    \inputsnippet{cpp}{1}{15}{codes/produto.cpp}
\end{frame}

\begin{frame}[fragile]{Subconjuntos}

    \begin{itemize}
        \item Um conjunto $A$ composto por $N$ elementos tem $2^N$ subconjuntos

        \item Por exemplo, para $N = 2$, os subconjuntos de $A = \lbrace 1, 2\rbrace$ são:
            $\emptyset, \lbrace 1\rbrace, \lbrace 2\rbrace, \lbrace 1, 2\rbrace$

        \item É possível listar todos estes conjuntos, em aproximadamente 1 segundo, para
            $N\leq 23$, pois $2^{23} \approx 10^7$

        \item As formas de se gerar estes subconjuntos estão vinculadas à forma escolhida para
            representar estes subconjuntos

        \item Se cada subconjunto $S\subset A$ é um vetor contendo os índices dos elementos de
            $A$ que pertencem a $S$, a estratégia mais adequada é usar recursão

        \item Se $S$ corresponde a um vetor de $N$ \textit{bits}, onde o $i$-ésimo \textit{bit}
            indica a presença ou a ausência do elemento $a_i$ em $S$, os subconjuntos podem ser
            listados diretamente, por meio de um laço 
    \end{itemize}

\end{frame}

\begin{frame}[fragile]{Implementação iterativa dos subconjuntos de $A$}
    \inputsnippet{cpp}{1}{11}{codes/subsets.cpp}
\end{frame}

\begin{frame}[fragile]{Implementação iterativa dos subconjuntos de $A$}
    \inputsnippet{cpp}{13}{33}{codes/subsets.cpp}
\end{frame}

\begin{frame}[fragile]{Permutações}

    \begin{itemize}
        \item Uma permutação de $N$ elementos $\lbrace a_1, a_2, \ldots, a_N\rbrace$ consiste
            em uma reordenação de seus índices 

        \item Um conjunto $A$ composto por $N$ elementos distintos tem $N!$ permutações distintas

        \item Por exemplo, para $N = 3$, as permutações de $A = \lbrace 1, 2, 3\rbrace$ são:
            $\lbrace 1, 2, 3\rbrace$, $\lbrace 1, 3, 2\rbrace$, $\lbrace 2, 1, 3\rbrace$,
            $\lbrace 2, 3, 1\rbrace$, $\lbrace 3, 1, 2\rbrace$ e $\lbrace 3, 2, 1\rbrace$

        \item É possível listar estas permutações, em aproximadamente 1 segundo, para
            $N = 10$, pois $10! \leq 10^7$

        \item Assim como no caso dos subconjuntos, é possível gerar todas as permutações por
            meio de recursão
        
        \item Contudo, a biblioteca \code{cpp}{algorithm} da linguagem C++ provê duas funções
            para a geração de permutações: \code{cpp}{next_permutation()} e
            \code{cpp}{prev_permutation()}, baseadas em implementações iterativas
    \end{itemize}

\end{frame}

\begin{frame}[fragile]{Implementação iterativa das permutações de $A$}
    \inputsnippet{cpp}{1}{18}{codes/permutacoes.cpp}
\end{frame}

\begin{frame}[fragile]{Combinações}

    \begin{itemize}
        \item Seja $A$ um conjunto de $n$ elementos

        \item Uma combinação de $m$ elementos de $A$ é uma sequência de elementos de
            A $\lbrace a_{i1}, a_{i2}, \ldots, a_{im}\rbrace$ tal que $i_j < i_k$ se $j < k$

        \item O número de combinações de $m$ elementos de $A$ é igual a 
        \[
            C_{n,m} = \binom{n}{m} = \frac{n!}{m!(n - m)!}
        \]

        \item As combinações de $m$ elementos de $A$ podem ser geradas recursivamente, a
            partir de uma modificação na rotina que gera as permutações

        \item Também é possível usar as funções \code{cpp}{prev_permutation()} ou
            \code{cpp}{next_permutation()} para gerar tais combinações
            
    \end{itemize}

\end{frame}

\begin{frame}[fragile]{Implementação iterativa das combinações}
    \inputsnippet{cpp}{1}{16}{codes/combinacoes.cpp}
\end{frame}

\begin{frame}[fragile]{Implementação iterativa das combinações}
    \inputsnippet{cpp}{18}{39}{codes/combinacoes.cpp}
\end{frame}

\begin{frame}[fragile]{Combinações com repetições}

    \begin{itemize}
        \item Seja $A$ um conjunto de $n$ elementos

        \item Uma combinação de $m$ elementos de $A$ com repetição é uma sequência de elementos 
            de A $\lbrace a_{i1}, a_{i2}, \ldots, a_{im}\rbrace$ tais que os índices $i_k$ estão
            em ordem não-decrescente

        \item O número de combinações de $m$ elementos de $A$ com repetição é igual a 
        \[
            CR_{n,m} =\left(\!\!\!\binom{n}{k}\!\!\!\right) = \binom{n + m - 1}{m}
        \]

        \item As combinações de $m$ elementos de $A$ podem ser geradas recursivamente, a
            partir de uma modificação na rotina que gera as combinações

        \item Também é possível gerar tais combinações iterativamente, simulando uma soma
            com vai-um, e saltando os números cuja sequência não obedeça a ordenação
            não-decrescente
 
    \end{itemize}

\end{frame}

\begin{frame}[fragile]{Implementação iterativa das combinações com repetições}
    \inputsnippet{cpp}{1}{20}{codes/combinacoes_com_repeticoes.cpp}
\end{frame}

\begin{frame}[fragile]{Implementação iterativa das combinações com repetições}
    \inputsnippet{cpp}{22}{43}{codes/combinacoes_com_repeticoes.cpp}
\end{frame}




