%\section{LA 3506 -- 4 values whose sum is 0}

\begin{frame}[fragile]{Problema}

The SUM problem can be formulated as follows: given four lists $A, B, C, D$ of integer values, 
compute how many quadruplet $(a, b, c, d)$ belongs to $A\times B\times C\times D$ are such that 
$a + b + c + d = 0$. In the following, we assume that all lists have the same size $n$.

\end{frame}

\begin{frame}[fragile]{Entrada e saída}

\textbf{Input}

The input begins with a single positive integer on a line by itself indicating the number of the 
cases following, each of them as described below. This line is followed by a blank line, and 
there is also a blank line between two consecutive inputs.

The first line of the input file contains the size of the lists $n$ (this value can be as large as 
4000).  We then have $n$ lines containing four integer values (with absolute value as large as 
$2^{28}$) that belong respectively to $A, B, C$ and $D$. 

\vspace{0.1in}

\textbf{Output}

For each test case, your program has to write the number quadruplets whose sum is zero.

The outputs of two consecutive cases will be separated by a blank line.

\end{frame}

\begin{frame}[fragile]{Exemplo de entradas e saídas}

\begin{minipage}[t]{0.45\textwidth}
\textbf{Sample Input}
\begin{verbatim}
1

6
-45 22 42 -16
-41 -27 56 30
-36 53 -37 77
-36 30 -75 -46
26 -38 -10 62
-32 -54 -6 45
\end{verbatim}
\end{minipage}
\begin{minipage}[t]{0.5\textwidth}
\textbf{Sample Output}
\begin{verbatim}
5
\end{verbatim}
\end{minipage}
\end{frame}

\begin{frame}[fragile]{Solução com complexidade $O(N^2\log N)$}

    \begin{itemize}
        \item O produto cartesiano $A\times B\times C\times D$ tem  $4000^4 = 256 \times 10^{12}$,
            o que inviabiliza uma solução \textit{naive} $O(N^4)$

        \item É possível utilizar a técnica \textit{meet in the middle}, observando que
            $a + b = -(c + d)$

        \item Assim, é preciso computar as somas $xs$ e $ys$ dos pares $A\times B$ e $C\times D$,
            respectivamente

        \item As somas registradas em $ys$ devem ser ordenadas, de modo que seja possível 
            utilizar a busca binária
            
        \item Para cada valor $x\in xs$, a resposta será incrementada em $L - R$, onde 
            $[L, R)$ é o intervalo de índices de elementos $y_i$ em $ys$ tais que $y_i = -x$

        \item Este intervalo pode ser computado através da função \code{cpp}{equal_range()}
            da STL da linguagem C++

   \end{itemize}

\end{frame}

\begin{frame}[fragile]{Solução com complexidade $O(N^2\log N)$}
    \inputsnippet{cpp}{1}{20}{codes/LA_3506.cpp}
\end{frame}

\begin{frame}[fragile]{Solução com complexidade $O(N^2\log N)$}
    \inputsnippet{cpp}{22}{38}{codes/LA_3506.cpp}
\end{frame}

\begin{frame}[fragile]{Solução com complexidade $O(N^2\log N)$}
    \inputsnippet{cpp}{40}{55}{codes/LA_3506.cpp}
\end{frame}
