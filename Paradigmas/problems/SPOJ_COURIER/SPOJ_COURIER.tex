%\section{SPOJ COURIER -- The Courier}

\begin{frame}[fragile]{Problema}


Byteland is a scarcely populated country, and residents of different cities seldom communicate with
each other. There is no regular postal service and throughout most of the year a one-man courier
establishment suffices to transport all freight. However, on Christmas Day there is somewhat more
work for the courier than usual, and since he can only transport one parcel at a time on his
bicycle, he finds himself riding back and forth among the cities of Byteland.

The courier needs to schedule a route which would allow him to leave his home city, perform the 
individual orders in arbitrary order (i.e. travel to the city of the sender and transport the 
parcel to the city of the recipient, carrying no more than one parcel at a time), and finally 
return home. All roads are bi-directional, but not all cities are connected by roads directly; some 
pairs of cities may be connected by more than one road. Knowing the lengths of all the roads and 
the errands to be performed, determine the length of the shortest possible cycling route for the 
courier.

\end{frame}

\begin{frame}[fragile]{Entrada e saída}

\textbf{Input}

The input begins with the integer $t$, the number of test cases. Then $t$ test cases follow.

Each test case begins with a line containing three integers: $n\ m\ b$, denoting the number of 
cities in Byteland, the number of roads, and the number of the courier's home city, respectively
($1\leq n\leq 100$, $1\leq b\leq n$, $1\leq m\leq 10000$). The next $m$ lines contain three integers 
each, the $i$-th being $u_i\ v_i\ d_i$, which means that cities $u_i$ and $v_i$ are connected by a 
road of length $d_i$ ($1\leq u_i, v_i\leq 100$, $1\leq d_i\leq 10000$). The following line contains 
integer $z$ -- the number of transport requests the courier has received ($1\leq z\leq 5$). After 
that, $z$ lines with the description of the orders follow. Each consists of three integers, the 
$j$-th being $u_j\ v_j\ b_j$, which signifies that $b_j$ parcels should be transported 
(individually) from city $u_j$ to city $v_j$. The sum of all $b_j$ does not exceed 12.

\end{frame}

\begin{frame}[fragile]{Entrada e saída}

\textbf{Output}

For each test case output a line with a single integer -- the length of the shortest possible 
bicycle route for the courier.

\end{frame}

\begin{frame}[fragile]{Exemplo de entradas e saídas}

\begin{minipage}[t]{0.45\textwidth}
\textbf{Sample Input}
\begin{verbatim}
1
5 7 2
1 2 7
1 3 5
1 5 2
2 4 10
2 5 1
3 4 3
3 5 4
3
1 4 2
5 3 1
5 1 1
\end{verbatim}
\end{minipage}
\begin{minipage}[t]{0.5\textwidth}
\textbf{Sample Output}
\begin{verbatim}
43
\end{verbatim}
\end{minipage}
\end{frame}

\begin{frame}[fragile]{Solução}

    \begin{itemize}
        \item Este é um problema bastante interessante, que pode ser modelado como um TSP, porém
            de forma não óbvia

        \item A travessia será composta pelas $B$ entregas, sendo que cada pacote consiste em
            uma entrega individual

        \item Seja $dp(i, mask)$ a distância mínima para fazer as entregas partindo do vértice
            $i$ e já tendo entregue os pacotes sinalizados pela máscara binária $mask$

        \item O custo das ``arestas'' entre as entregas é dado pela soma da distância entre $i$
            e o vértice $u$ onde está o pacote mais a distância entre $u$ e o destino do pacote
            $v$

        \item As distâncias mínimas entre quaisquer dois vértices pode ser computada por meio
            do algoritmo de Floyd-Warshall

        \item Serão $O(N\times 2^K)$ estados distintos, onde $K = \sum_i b_i$
    \end{itemize}

\end{frame}

\begin{frame}[fragile]{Solução $O(N^3 + K^22^K)$}
    \inputsnippet{cpp}{1}{19}{codes/COURIER.cpp}
\end{frame}

\begin{frame}[fragile]{Solução $O(N^3 + K^22^K)$}
    \inputsnippet{cpp}{21}{40}{codes/COURIER.cpp}
\end{frame}

\begin{frame}[fragile]{Solução $O(N^3 + K^22^K)$}
    \inputsnippet{cpp}{42}{61}{codes/COURIER.cpp}
\end{frame}

\begin{frame}[fragile]{Solução $O(N^3 + K^22^K)$}
    \inputsnippet{cpp}{63}{82}{codes/COURIER.cpp}
\end{frame}

\begin{frame}[fragile]{Solução $O(N^3 + K^22^K)$}
    \inputsnippet{cpp}{84}{105}{codes/COURIER.cpp}
\end{frame}
