\section{Codeforces Round \# 198 (Div. 2) -- Problem D: Bubble Sort Graph}

\begin{frame}[fragile]{Problema}

Iahub recently has learned Bubble Sort, an algorithm that is used to sort a permutation with 
$n$ elements $a_1, a_2, \ldots, a_n$ in ascending order. He is bored of this so simple 
algorithm, so he invents his own graph. The graph (let's call it $G$) initially has $n$ 
vertices and 0 edges. During Bubble Sort execution, edges appear as described in the following 
algorithm (pseudocode).

\end{frame}

\begin{frame}[fragile]{Problema}
    \inputsyntax{cpp}{codes/bubble.cpp}
\end{frame}

\begin{frame}[fragile]{Problema}

For a graph, an independent set is a set of vertices in a graph, no two of which are adjacent 
(so there are no edges between vertices of an independent set). A maximum independent set is an 
independent set which has maximum cardinality. Given the permutation, find the size of the 
maximum independent set of graph $G$, if we use such permutation as the premutation a in 
procedure \texttt{bubbleSortGraph}.

\end{frame}

\begin{frame}[fragile]{Entrada e saída}

\textbf{Input}

The first line of the input contains an integer $n$ ($2\leq n\leq 10^5$). The next line 
contains $n$ distinct integers $a_1, a_2, \ldots, a_n$ ($1\leq a_i\leq n$).

\vspace{0.2in}

\textbf{Output}

Output a single integer -- the answer to the problem.

\end{frame}

\begin{frame}[fragile]{Exemplo de entradas e saídas}

\begin{minipage}[t]{0.45\textwidth}
\textbf{Sample Input}
\begin{verbatim}
3
3 1 2
\end{verbatim}
\end{minipage}
\begin{minipage}[t]{0.5\textwidth}
\textbf{Sample Output}
\begin{verbatim}
2
\end{verbatim}
\end{minipage}
\end{frame}

\begin{frame}[fragile]{Solução com complexidade $O(N\log N)$}

    \begin{itemize}
        \item Gerar o grafo $G$ por meio da execução do código do \textit{bubblesort}
            apresentado leva a um veredito TLE, uma vez que, no pior caso, há $O(N^2)$
            arestas (sequência em ordem decrescente)

        \item Mesmo que fosse possível construir o grafo $G$ em tempo hábil, o maior
            conjunto independente é um problema \textit{NP-Hard}, e como $N\leq 10^5$, novamente
            o veredito seria TLE

        \item O que deve ser observado é que não existirá uma aresta entre $a_i$ e $a_j$ se
            $a_i < a_j$, com $i < j$

        \item Observe que, pela transitividade, se não existe uma aresta entre $a_i$ e $a_j$,
            e também não há aresta entre $a_j$ e $a_k$, não haverá uma aresta entre $a_i$ e 
            $a_k$
   \end{itemize}

\end{frame}


\begin{frame}[fragile]{Solução com complexidade $O(N\log N)$}

    \begin{itemize}
        \item Deste modo, um conjunto independente em $G$ será uma sequência crescente de
            $a = \{ a_1, a_2, \ldots, a_N \}$

        \item A resposta do problema, portanto, será a maior subsequência crescente de $a$

        \item Dados os limites do problema, a implementação quadrática levaria ao TLE

        \item Portanto, o problema deve ser resolvido pela implementação linearítmica da
            LIS
    \end{itemize}

\end{frame}
\begin{frame}[fragile]{Solução com complexidade $O(N\log N)$}
    \inputsnippet{cpp}{1}{21}{codes/340D.cpp}
\end{frame}

\begin{frame}[fragile]{Solução com complexidade $O(N\log N)$}
    \inputsnippet{cpp}{23}{43}{codes/340D.cpp}
\end{frame}
