\section{OJ 13095 -- Tobby and Query}

\begin{frame}[fragile]{Problema}

In his free time Tobby is always searching for interesting things. This time Tobby created the
following problem: given a sequence of $n$ integer numbers, Tobby would like to know how many
different numbers are in the range $[l, r]$ $(r\geq l)$.

\end{frame}

\begin{frame}[fragile]{Entrada e saída}

\textbf{Input}

The input has several test cases. The first line of each test case contains an integer $n$
$(1\leq n\leq 10^5)$, the size of the sequence of numbers. The next line contains $n$ values $a_i$
$(0\leq a_i\leq 9)$, the numbers in the sequence. The next line contains an integer $q$
$(1\leq q\leq 10^4)$, the amount of queries. Then there are $q$ lines, each line contains a query:
two integers $l$ and $r$ $(1\leq l, r\leq n)$.

\vspace{0.2in}

\textbf{Output}

For each test case print $q$ integers, representing the amount of different numbers in the range
$[l, r]$ for each query in the input.

\end{frame}

\begin{frame}[fragile]{Exemplo de entradas e saídas}

\begin{minipage}[t]{0.45\textwidth}
\textbf{Sample Input}
\begin{verbatim}
7
0 2 3 3 7 5 2
3
1 1
2 4
2 7
5
7 7 7 7 7
2
4 5
1 5
\end{verbatim}
\end{minipage}
\begin{minipage}[t]{0.5\textwidth}
\textbf{Sample Output}
\begin{verbatim}
1
2
4
1
1
\end{verbatim}
\end{minipage}
\end{frame}

\begin{frame}[fragile]{Solução $O(N + Q)$}

    \begin{itemize}
        \item Uma forma de responder rapidamente (em $O(1)$) cada uma das consultas é calcular
            as somas dos prefixos $p_d$, onde $d$ representa os 10 dígitos decimais (pois
            $0\leq a_i\leq 9$)

        \item Estas somas podem ser computadas em $O(N)$
        
        \item Assim, a consulta para o intervalo $[L, R]$ pode ser respondida por meio da
            $RSQ(L, R)$ para cada um dos 10 vetores de prefixos:
        \[
            q(L, R) = \sum_{d = 0}^9 \delta(p_d[R] - p_d[L - 1]),
        \]
        onde
        \[
            \delta(x) = \left\{\begin{array}{ll}
                1,& \mbox{se}\ x > 0 \\
                0,& \mbox{caso contrário}
            \end{array}\right.
        \]
   \end{itemize}

\end{frame}

\begin{frame}[fragile]{Solução $O(N + Q)$}
    \inputsnippet{cpp}{1}{20}{codes/13095.cpp}
\end{frame}

\begin{frame}[fragile]{Solução $O(N + Q)$}
    \inputsnippet{cpp}{21}{41}{codes/13095.cpp}
\end{frame}

\begin{frame}[fragile]{Solução $O(N + Q)$}
    \inputsnippet{cpp}{42}{62}{codes/13095.cpp}
\end{frame}
