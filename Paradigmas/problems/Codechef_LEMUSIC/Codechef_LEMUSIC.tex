%\section{Codechef -- Little Elephant and Music}

\begin{frame}[fragile]{Problema}

The Little Elephant from the Zoo of Lviv likes listening to music.

There are $N$ songs, numbered from $1$ to $N$, in his MP3-player. The song $i$ is described by a 
pair of integers $B_i$ and $L_i$ -- the band (represented as integer) that performed that song 
and the length of that song in seconds. The Little Elephant is going to listen all the songs 
exactly once in some order.

The sweetness of the song is equal to the product of the length of that song and the number of 
different bands listened before (including the current playing song).

Help the Little Elephant to find the order that maximizes the total sweetness of all $N$ songs. 
Print that sweetness.
\end{frame}

\begin{frame}[fragile]{Entrada e saída}

\textbf{Input}

The first line of the input contains single integer $T$, denoting the number of test cases. Then 
$T$ test cases follow. The first line of each test case contains single integer $N$, denoting the 
number of the songs. The next $N$ lines describe the songs in the MP3-player. The $i$-th line 
contains two space-sparated integers $B_i$ and $L_i$.

\textbf{Output}

For each test, output the maximum total sweetness.

\textbf{Constraints}

\begin{itemize}
    \item $1 \leq T \leq 5$
    \item $1 \leq N \leq 100000\ (10^5)$
    \item $1 \leq B_i, L_i \leq 1000000000\ (10^9)$
\end{itemize}

\end{frame}

\begin{frame}[fragile]{Exemplo de entradas e saídas}

\begin{minipage}[t]{0.5\textwidth}
\textbf{Sample Input}
\begin{verbatim}
2
3
1 2
2 2
3 2
3
2 3
1 2
2 4
\end{verbatim}
\end{minipage}
\begin{minipage}[t]{0.45\textwidth}
\textbf{Sample Output}
\begin{verbatim}
12
16
\end{verbatim}
\end{minipage}
\end{frame}

\begin{frame}[fragile]{Solução com complexidade $O(N\log N)$}

    \begin{itemize}
        \item Este problema pode ser resolvido por meio de um algoritmo guloso

        \item Considere uma ordenação arbitrária $\lbrace m_1, m_2, \ldots, m_N\rbrace$ das
            músicas

        \item Considere que $j = i + 1$ e que a banda da música $i$ já apareceu ao menos uma vez
            antes de $i$ e que a banda $j$ não apareceu antes de $j$

        \item Se as duas músicas trocarem de posição a contribuição da música $j$ será a mesma,
            pois sua duração não muda e o número de bandas que já apareceram será o mesmo

        \item Contudo, a contribuição da música $i$ aumenta, pois o número de bandas distintas
            é incrementado

        \item Assim, as músicas devem ser ordenadas de tal maneira que as $B$ bandas distintas
            apareçam nas primeiras $B$ posições ao menos uma vez
   \end{itemize}

\end{frame}


\begin{frame}[fragile]{Solução com complexidade $O(N)$}

    \begin{itemize}
        \item Por outro lado, seja $i < j$ índices de duas músicas da mesma banda, com
            $i \leq B$ e a duração de $m_i$ seja maior do que a duração de $m_j$

        \item A troca de ambas músicas de posição melhora a resposta, uma vez que a maior duração
            será multiplicada por $B$ e a menor duração será multiplicada pelo número de
            bandas distintas até $i$, que é um número menor ou igual a $B$

        \item Assim, a ordenação também deve colocar as músicas de menor duração de cada banda
            nas primeiras posições

        \item Por fim, $i < j$ são os índices de duas músicas consecutivas dentre as primeiras 
            $B$ posições, se a duração de $m_i$ for maior do que $m_j$, a troca de posição das
            duas também melhora a resposta

        \item Portanto, as músicas devem ser ordenadas de tal modo que as músicas de menor duração
            de cada banda ocupe as $B$ primeiras posições, em ordem de duração, e as demais
            ocupam as $N - B$ posições, em qualquer ordem 
    \end{itemize}

\end{frame}

\begin{frame}[fragile]{Solução AC com complexidade $O(N\log N)$}
    \inputsnippet{cpp}{1}{16}{codes/LEMUSIC.cpp}
\end{frame}

\begin{frame}[fragile]{Solução AC com complexidade $O(N\log N)$}
    \inputsnippet{cpp}{18}{36}{codes/LEMUSIC.cpp}
\end{frame}

\begin{frame}[fragile]{Solução AC com complexidade $O(N\log N)$}
    \inputsnippet{cpp}{38}{58}{codes/LEMUSIC.cpp}
\end{frame}
