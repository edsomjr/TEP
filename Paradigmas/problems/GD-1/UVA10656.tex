\section{OJ 10656 -- Maximum Sum (II)}

\begin{frame}[fragile]{Problema}

In a given sequence of non-negative integers you will have to find such a sub-sequence in it whose
summation is maximum.

\end{frame}

\begin{frame}[fragile]{Entrada e saída}

\textbf{Input}

The input file contains several input sets. The description of each set is given below:

Each set starts with an integer $N$ $(N < 1000)$ that indicates how many numbers are in that set.
Each of the next $N$ lines contains a single non-negative integer. All these numbers are less 
than 10000.

Input is terminated by a set where $N = 0$. This set should not be processed.

\end{frame}

\begin{frame}[fragile]{Entrada e saída}

\textbf{Output}

For each set of input produce one line of output. This line contains one or more integers which is are
taken from the input sequence and whose summation is maximum. If there is more than one such subsequence print the one that has minimum length. If there is more than one sub-sequence of minimum
length, output the one that occurs first in the given sequence of numbers. A valid sub-sequence must
have a single number in it. Two consecutive numbers in the output are separated by a single space.

\end{frame}


\begin{frame}[fragile]{Exemplo de entradas e saídas}

\begin{minipage}[t]{0.6\textwidth}
\textbf{Sample Input}
\begin{verbatim}
2
3
4
0
\end{verbatim}
\end{minipage}
\begin{minipage}[t]{0.35\textwidth}
\textbf{Sample Output}
\begin{verbatim}
3 4
\end{verbatim}
\end{minipage}
\end{frame}

\begin{frame}[fragile]{Solução com complexidade $O(N)$}

    \begin{itemize}
        \item Como a sequência é composta apenas por números não-negativos, cada elemento $x$
            da sequência ou amplia a soma (caso $x > 0$) ou a mantém (caso $x = 0$)

        \item Assim, a menor sequência máxima é obtida retirando os elementos iguais a zero
            da sequência

        \item Importante notar que os elementos devem ser impressos na mesma ordem da entrada

        \item Há um {\it corner case} a ser tratado: caso a sequência seja inteiramente composta
            por elementos iguais a zero, a saída deve ser um único zero, já que o problema
            estabelece que uma saída válida deve ter no mínimo um número

        \item Como cada elemento é avaliado uma única vez, a complexidade desta solução é $O(N)$
   \end{itemize}

\end{frame}

\begin{frame}[fragile]{Solução com complexidade $O(N)$}
    \inputsnippet{cpp}{1}{21}{10656.cpp}
\end{frame}

\begin{frame}[fragile]{Solução com complexidade $O(N)$}
    \inputsnippet{cpp}{22}{42}{10656.cpp}
\end{frame}
