%\section{SPOJ WACHOVIA -- Wachovia Bank}

\begin{frame}[fragile]{Problema}

Danilo Gheyi is a renowned bank robber. He is known worldwide for accomplishing the most 
profitable bank robbery, in Fortaleza, Ceará. Danilo and his friends dug a tunnel to get into 
the main chest. There were some bags, with different amounts of money or jewelry and weight. 
They had to leave about 50\% of the total value, since the truck couldn't carry all the bags.

Danilo wasn't caught, and to show that he can do it all again, he is planning a robbery to one 
of the safer banks in USA -- the Wachovia Bank. He wants your help to maximize the amount 
stolen, avoiding a huge loss as it happened in Fortaleza.

Write a program that, given the maximum weight the truck is able to carry and the information 
about each bag in the bank, determine the maximum value that Danilo can steal.

\end{frame}

\begin{frame}[fragile]{Entrada e saída}

\textbf{Input}

The input consists of several instances. There is an integer $N$ ($1\leq N\leq 200$) in the 
first line; it stands for the number of instances. The first line of each instance contains two 
integers, $K$ and $M$ ($8\leq K\leq 1000$ and $1\leq M\leq 50$) representing, respectively, the 
maximum weight the truck can handle and the amount of bags in the bank. The next $M$ lines 
describe each bag with two integers $A$ and $B$ ($8\leq A\leq 200$ and $1\leq B\leq 25$): the 
weight and the value of the bag, respectively.

\vspace{0.2in}

\textbf{Output}

For each instance output a sentence ``\texttt{Hey stupid robber, you can get} $P$\texttt{.}'',
and $P$ represents the maximum value Danilo can steal.

\end{frame}

\begin{frame}[fragile]{Exemplo de entradas e saídas}

\begin{scriptsize}
\begin{minipage}[t]{0.45\textwidth}
\textbf{Sample Input}
\begin{verbatim}
3
34 5
178 12
30 1
13 7
34 8
87 6
900 1
900 25
100 10
27 16
131 9
132 17
6 5
6 23
56 21
100 25
1 25
25 25
100 2
\end{verbatim}
\end{minipage}
\begin{minipage}[t]{0.5\textwidth}
\textbf{Sample Output}
\begin{verbatim}
Hey stupid robber, you can get 8.
Hey stupid robber, you can get 25.
Hey stupid robber, you can get 99.
\end{verbatim}
\end{minipage}
\end{scriptsize}

\end{frame}

\begin{frame}[fragile]{Solução $O(KM)$}

    \begin{itemize}
        \item O problema apresentado pode ser modelado como um problema da mochila binária

        \item O caminhão é a mochila do problema, com capacidade $K$

        \item Os elementos da mochila binária são os pertences a serem subtraídos do banco

        \item Os pesos e os valores dos elementos são $A$ e $B$, respectivamente

        \item Estabelecida estas correspondências, basta aplicar a solução de programação
            dinâmica para o problema da mochila, sem alteração alguma

        \item Portanto, a complexidade da solução será $O(KM)$
    \end{itemize}

\end{frame}

\begin{frame}[fragile]{Solução $O(KM)$}
    \inputsnippet{cpp}{1}{20}{codes/WACHOVIA.cpp}
\end{frame}

\begin{frame}[fragile]{Solução $O(KM)$}
    \inputsnippet{cpp}{22}{41}{codes/WACHOVIA.cpp}
\end{frame}

\begin{frame}[fragile]{Solução $O(KM)$}
    \inputsnippet{cpp}{43}{63}{codes/WACHOVIA.cpp}
\end{frame}
