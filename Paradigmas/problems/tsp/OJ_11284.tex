\section{OJ 11284 -- Shopping Trip}

\begin{frame}[fragile]{Problema}

For some reason, Daniel loves to collect and watch operas on DVD. He can find and order all the 
operas he wants from Amazon, and they will even deliver them right to his door, but he can usually 
find a better price at one of his favourite stores. However, with the cost of gas nowadays, it is 
hard to tell whether or not one would actually save money by driving to the stores to purchase the 
DVDs.

Daniel would like to buy some operas today. For each of the operas he wants, he knows exactly
one store that is selling it for a lower cost than the Amazon price. He would like to know if it 
would actually be worth it to go out and buy the operas from the stores.

Daniel only knows the road system connecting his favourite stores, and will only use those roads to
get around. He knows at least one route, if only an indirect one, to every store.

\end{frame}

\begin{frame}[fragile]{Problema}

In his shopping trip, Daniel begins at his house, drives from store to store in any order to 
purchase his operas, then drives back to his house. For any particular opera, he can opt not to 
drive to the store to buy it, since he can just order it from Amazon.

For convenience, Daniel assigned his house the integer 0, and numbered each of his favourite stores
with integers starting at 1. You are given a description of the road system and the exact amount it
would cost for Daniel to drive each road. For each opera Daniel wants, you are given the number of 
the store it is available at, and the amount he would save if he bought that particular opera at 
that store.  Your task is to determine the greatest amount of money Daniel can save by making the 
shopping trip.

\end{frame}

\begin{frame}[fragile]{Entrada e saída}

\textbf{Input}

The first line of input contains a single number indicating the number of scenarios to process. A 
blank line precedes each scenario.

Each scenario begins with line containing two numbers: $N$ ($1\leq N\leq 50$), the number of 
stores, and $M$ ($1\leq M\leq 1000$), the number of roads. The following $M$ lines each contain a 
description of a road.  Each road is described by two integers indicating the house or stores it 
connects, and a real number with two decimal digits indicating the cost in dollars to drive that 
road. All roads are two-way.

The next line in the scenario contains a number $P$ ($1\leq P\leq 12$), the number of opera DVDs 
Daniel wants to buy. For each of the $P$ operas, a line follows containing an integer indicating 
the store number at which the opera is available, and a real number with two decimal digits 
indicating the difference between the Amazon price and the price at that store in dollars.

\end{frame}

\begin{frame}[fragile]{Entrada e saída}

\textbf{Output}

For each scenario in the input, write one line of output indicating the largest amount of money, in
dollars and cents, that Daniel can save by making his shopping trip. Follow the format of the 
sample output; there should always be two digits after the decimal point to indicate the number of 
cents. If Daniel cannot save any money by going to the stores, output a single line saying 
`\texttt{Don’t leave the house}'.

\end{frame}

\begin{frame}[fragile]{Exemplo de entradas e saídas}

\begin{footnotesize}
\begin{minipage}[t]{0.45\textwidth}
\textbf{Sample Input}
\begin{verbatim}
2

4 5
0 1 1.00
1 2 3.00
1 3 2.00
2 4 4.00
3 4 3.25
3
2 1.50
3 7.00
4 9.00

1 1
0 1 1.50
1
1 2.99
\end{verbatim}
\end{minipage}
\begin{minipage}[t]{0.5\textwidth}
\textbf{Sample Output}
\begin{verbatim}
Daniel can save $3.50
Don’t leave the house
\end{verbatim}
\end{minipage}
\end{footnotesize}
\end{frame}

\begin{frame}[fragile]{Solução}

    \begin{itemize}
        \item O problema pode ser modelado como um TSP

        \item A travessia deve ser formada pelas lojas que vendem os DVDs desejados

        \item Como o grafo da entrada não é completo, as distãncias mínimas entre quaisquer duas
            lojas podem ser computada por meio do algoritmo de Floyd-Warshall

        \item Dois pontos importantes: primeiramente, existe a possibilidade de se adquirir
            apenas alguns, ou até mesmo nenhum DVD

        \item O custo de viagem entre as lojas e o ganho na compra dos DVDs devem ter sinais
            opostos, e a escolha destes sinais pode modificar a atualização dos estados
    \end{itemize}

\end{frame}

\begin{frame}[fragile]{Solução}

    \begin{itemize}
        \item Há uma série de cuidados a serem tomados para obter uma solução AC

        \item Embora não fique claro na descrição da entrada, o grafo do problema não é 
            simples, podendo acontecer \textit{loops} e multiarestas

        \item Assim, para cada par de vértices deve ser mantida apenas o menor custo possível

        \item Os \textit{loops} devem ser eliminados, isto é, $dist(u, u) = 0$ para todo 
            $u\in [0, N]$
        
        \item Para evitar erros de precisão, o ideal é trabalhar com múltiplos de centavos,
            postergando a conversão para dólares para o momento da impressão
   \end{itemize}

\end{frame}

\begin{frame}[fragile]{Solução $O(N^3 + P^22^P)$}
    \inputsnippet{cpp}{1}{21}{codes/11284.cpp}
\end{frame}

\begin{frame}[fragile]{Solução $O(N^3 + P^22^P)$}
    \inputsnippet{cpp}{22}{42}{codes/11284.cpp}
\end{frame}

\begin{frame}[fragile]{Solução $O(N^3 + P^22^P)$}
    \inputsnippet{cpp}{43}{63}{codes/11284.cpp}
\end{frame}

\begin{frame}[fragile]{Solução $O(N^3 + P^22^P)$}
    \inputsnippet{cpp}{64}{84}{codes/11284.cpp}
\end{frame}

\begin{frame}[fragile]{Solução $O(N^3 + P^22^P)$}
    \inputsnippet{cpp}{85}{105}{codes/11284.cpp}
\end{frame}
