\section{OJ 12879 -- Golf Bot}

\begin{frame}[fragile]{Problema}

Do you like golf? I hate it. I hate golf so much that I decided to build the ultimate golf robot,
a robot that will never miss a shot. I simply place it over the ball, choose the right direction
and distance and, flawlessly, it will strike the ball across the air and into the hole.  Golf will
never be played again.

Unfortunately, it doesn’t work as planned. So, here I am, standing in the green and preparing my
first strike when I realize that the distance-selector knob built-in doesn’t have all the distance
options! Not everything is lost, as I have 2 shots.

Given my current robot, how many holes will I be able to complete in 2 strokes or less? The ball
must be always on the right line between the tee and the hole.  It isn’t allowed to overstep it
and come back.

\end{frame}

\begin{frame}[fragile]{Entrada e saída}

\textbf{Input}

The input file contains several test cases, each of them as described below.

The first line has one integer: $N$, the number of different distances the Golf Bot can shoot. Each
of the following $N$ lines has one integer, $k_i$, the distance marked in position $i$ of the knob.

Next line has one integer: $M$, the number of holes in this course. Each of the following M lines
has one integer, $d_j$, the distance from Golf Bot to hole $j$.

\textbf{Constraints}:

\begin{itemize}
    \item $1\leq N, M\leq 200 000$
    \item $1\leq k_i , d_j\leq 200 000$
\end{itemize}

\end{frame}

\begin{frame}[fragile]{Entrada e saída}

\textbf{Output}

For each test case, you should output a single integer, the number of holes Golf Bot will be able
to complete. Golf Bot cannot shoot over a hole on purpose and then shoot backwards.

\textbf{Sample Output Explanation}

Golf Bot can shoot 3 different distances (1, 3 and 5) and there are 6 holes in this course at
distances 2, 4, 5, 7, 8 and 9. Golf Bot will be able to put the ball in 4 of these:
\begin{itemize}
    \item The 1st hole, at distance 2, can be reached by striking two times a distance of 1.
    \item The 2nd hole, at distance 4, can be reached by striking with strength 3 and then strength 1 (or
vice-versa).
    \item The 3rd hole can be reached with just one stroke of strength 5.
    \item The 5th hole can be reached with two strikes of strengths 3 and 5.
\end{itemize}

Holes 4 and 6 can never be reached.
\end{frame}

\begin{frame}[fragile]{Exemplo de entradas e saídas}

\begin{minipage}[t]{0.45\textwidth}
\textbf{Sample Input}
\begin{verbatim}
3
1
3
5
6
2
4
5
7
8
9
\end{verbatim}
\end{minipage}
\begin{minipage}[t]{0.5\textwidth}
\textbf{Sample Output}
\begin{verbatim}
4
\end{verbatim}
\end{minipage}
\end{frame}

\begin{frame}[fragile]{Solução $O(K\log K + M)$}

    \begin{itemize}
        \item Para cada distância $d_j$, é preciso verificar se é possível atingí-la com uma
            ou duas tacadas

        \item Com uma tacada basta verificar se $d_j$ é igual a algum dos $k_i$, o que pode ser
            feito em $O(1)$ com uma tabela \textit{hash}
        
        \item Já para duas tacadas é a verificação pode ser feita em $O(N)$, também usando tabelas
            \textit{hash}

        \item Contudo, este algoritmo teria complexidade $O(NM)$, o que resultaria em um veredito
            TLE, uma vez que $N, M\leq 2\times 10^5$
            
        \item A transformada de Fourier permite resolver este problema com complexidade 
            $O(K\log K + M)$, onde $K = \max\{k_1, k_2, \ldots, k_N\}$
   \end{itemize}

\end{frame}

\begin{frame}[fragile]{Solução $O(K\log K + M)$}

    \begin{itemize}
        \item Seja $p(x) = a_0 + a_1x + a_2x^2 + \ldots$ um polinômio tal que $a_0 = 1$ e
            $a_r = 1$, se $r > 0$ e existe ao menos um $k_i = r$, ou $a_i = 0$, caso contrário

        \item Para o exemplo de entrada, $p(x) = 1 + x + x^3 + x^5$

        \item Uma vez que $1 = x^0$, cada monômio de $p(x)$ representa uma tacada possível (ou 
            mesmo não dar tacada, no caso de $x^0$) onde o grau é a distância atingida pelo golpe

        \item As distâncias possíveis de serem atingidas com duas tacadas são dadas pelos monômios
            não nulos do polinômio $q(x) = p^2(x)$, pois
        \[
            q(x) = p^2(x) = (a_0 + a_1x + a_2x^2 + \ldots)(a_0 + a_1x + a_2x^2 + \ldots) 
        \]
        e a distributividade vai associar cada par de monômios possível

        \item Logo, para cada $d_j$ basta verificar se $a_j$ é igual a zero ou não

        \item O produto $p^2(x)$ pode ser obtido com a FFT em $O(K\log K)$, permitindo checar
            cada $d_j$ em $O(1)$
            
    \end{itemize}

\end{frame}

\begin{frame}[fragile]{Solução $O(K\log K + M)$}
    \inputsnippet{cpp}{1}{21}{codes/OJ_12879.cpp}
\end{frame}

\begin{frame}[fragile]{Solução $O(K\log K + M)$}
    \inputsnippet{cpp}{22}{42}{codes/OJ_12879.cpp}
\end{frame}

\begin{frame}[fragile]{Solução $O(K\log K + M)$}
    \inputsnippet{cpp}{43}{63}{codes/OJ_12879.cpp}
\end{frame}

\begin{frame}[fragile]{Solução $O(K\log K + M)$}
    \inputsnippet{cpp}{64}{83}{codes/OJ_12879.cpp}
\end{frame}

\begin{frame}[fragile]{Solução $O(K\log K + M)$}
    \inputsnippet{cpp}{84}{104}{codes/OJ_12879.cpp}
\end{frame}
