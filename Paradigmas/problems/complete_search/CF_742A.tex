\section{Codeforces Round \# 383 (Div. 2) -- Problem A: Arpa's hard exam and Mehrdad's naive cheat}

\begin{frame}[fragile]{Problema}

\textit{There exists an island called Arpa’s land, some beautiful girls live there, as ugly ones 
do.}

Mehrdad wants to become minister of Arpa’s land. Arpa has prepared an exam. Exam has only one question, given $n$, print the last digit of $1378^n$.

Mehrdad has become quite confused and wants you to help him. Please help, although it's a naive cheat.

\end{frame}

\begin{frame}[fragile]{Entrada e saída}

\textbf{Input}

The single line of input contains one integer $n$ ($0\leq n\leq 10^9$).

\textbf{Output}

Print single integer -- the last digit of $1378^n$.

\end{frame}

\begin{frame}[fragile]{Exemplo de entradas e saídas}

\begin{minipage}[t]{0.45\textwidth}
\textbf{Sample Input}
\begin{verbatim}
1

2
\end{verbatim}
\end{minipage}
\begin{minipage}[t]{0.5\textwidth}
\textbf{Sample Output}
\begin{verbatim}
8

4
\end{verbatim}
\end{minipage}
\end{frame}

\begin{frame}[fragile]{Solução com complexidade $O(1)$}

    \begin{itemize}
        \item Observe que a tentativa de se computar o valor exato de $1378^n$ leva a dois
            problemas:
            \begin{enumerate}
                \item a complexidade da solução seria $O(n)$, e $n = 10^{9}$ no pior caso

                \item porém esta complexidade assume que o cálculo de cada produto pode ser
                    feito em $O(1)$, que não é o caso pois tais número crescem exponencialmente,
                    o que leva também a problemas de memória 
            \end{enumerate}

        \item Ainda assim, computar tais valores para os primeiros valores de $n$ (por exemplo,
            $n = 20$), pode revelar padrões na solução que não são óbvios à primeira vista

        \item O primeiro padrão que surge é que os últimos dígitos dos resultados, a partir de
            $n = 1$, formam uma sequência periódica: 
            \[
                8, 2, 4, 6, 8, 2, 4, 6, \ldots
            \] 

        \item Cuidado com o \textit{corner case} $n = 0$: neste caso, a resposta deve ser igual
            a um 

        \item Exceot no caso especial, uma operação de resto da divisão determina o 
            resultado correto, de modo que a solução tem complexidade $O(1)$
   \end{itemize}

\end{frame}

\begin{frame}[fragile]{Solução com complexidade $O(1)$}
    \inputsnippet{haskell}{1}{19}{codes/CF_742A.hs}
\end{frame}
