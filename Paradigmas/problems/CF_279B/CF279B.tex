%\section{Codeforces Round \#171 -- Problem B: Books}

\begin{frame}[fragile]{Problema}

When Valera has got some free time, he goes to the library to read some books. Today he's got $t$
free minutes to read. That's why Valera took $n$ books in the library and for each book he 
estimated the time he is going to need to read it. Let's number the books by integers from $1$ to
$n$. Valera needs $a_i$ minutes to read the $i$-th book.

Valera decided to choose an arbitrary book with number $i$ and read the books one by one, starting
from this book. In other words, he will first read book number $i$, then book number $i + 1$, then
book number $i + 2$ and so on. He continues the process until he either runs out of the free time
or finishes reading the $n$-th book. Valera reads each book up to the end, that is, he doesn't
start reading the book if he doesn't have enough free time to finish reading it.

Print the maximum number of books Valera can read.
\end{frame}

\begin{frame}[fragile]{Entrada e saída}

\textbf{Input}

The first line contains two integers $n$ and $t$ ($1\leq n\leq 10^5; 1\leq t\leq 10^9$) -- the
number of books and the number of free minutes Valera's got. The second line contains a sequence
of $n$ integers $a_1, a_2, \ldots, a_n$ ($1\leq a_i\leq 10^4$), where number a $i$ shows the
number of minutes that the boy needs to read the $i$-th book.

\vspace{0.2in}

\textbf{Output}

Print a single integer -- the maximum number of books Valera can read.

\end{frame}

\begin{frame}[fragile]{Exemplo de entradas e saídas}

\begin{minipage}[t]{0.5\textwidth}
\textbf{Sample Input}
\begin{verbatim}
4 5
3 1 2 1

3 3
2 2 3
\end{verbatim}
\end{minipage}
\begin{minipage}[t]{0.45\textwidth}
\textbf{Sample Output}
\begin{verbatim}
3

1
\end{verbatim}
\end{minipage}
\end{frame}

\begin{frame}[fragile]{Solução com complexidade $O(N)$}

    \begin{itemize}
        \item O problema consiste em avaliar, para cada $i\in [1, N]$, o maior número de
            livros que podem ser lidos começando no $i$-ésimo livro

        \item Uma solução de busca completa avaliaria cada $i$ em $O(N)$, tendo portanto
            complexidade $O(N^2)$

        \item O uso de dois ponteiros reduz esta complexidade para $O(N)$

        \item O ponteiro $L$ inicia no primeiro elemento, e deve ser incrementado sequenciamente
            até o último elemento, representando aqui o $i$ do problema

        \item Já o ponteiro $R$ também inicia no primeiro elemento, e a cada iteração ele será
            o maior dentre $L$ e $R$, pois cada livro será avaliado uma única vez por $R$

   \end{itemize}

\end{frame}

\begin{frame}[fragile]{Solução com complexidade $O(N)$}

    \begin{itemize}
        \item Deve ser mantida uma variável $t$, inicialmente igual a $T$, que mantém o registro
            do tempo ainda disponível para leitura

        \item A cada iteração, enquanto $R$ apontar para um elemento do vetor e houver tempo para
            ler o $R$-ésimo livro, $R$ deve ser incrementado e $t$ atualizado

        \item Ao fim de cada iteração o $R - L$ livros do intervalo $[L, R)$ podem ser lidos

        \item A variável $t$ deve ser atualizada, acrescentado o tempo de leitura do $L$-ésimo
            livro, caso tenha sido decrementado de $t$ previamente

        \item A resposta será o tamanho do maior dentre estes intervalos
   \end{itemize}

\end{frame}

\begin{frame}[fragile]{Solução AC com complexidade $O(N)$}
    \inputsnippet{cpp}{1}{20}{codes/CF279B.cpp}
\end{frame}

\begin{frame}[fragile]{Solução AC com complexidade $O(N)$}
    \inputsnippet{cpp}{22}{42}{codes/CF279B.cpp}
\end{frame}




