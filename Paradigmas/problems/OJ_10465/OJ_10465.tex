%\section{OJ 10465 -- Homer Simpson}

\begin{frame}[fragile]{Problema}

Homer Simpson, a very smart guy, likes eating Krusty-burgers. It takes Homer $m$ minutes to eat a
Krusty-burger. However, there’s a new type of burger in Apu’s Kwik-e-Mart. Homer likes those too.
It takes him $n$ minutes to eat one of these burgers. Given $t$ minutes, you have to find out the
maximum number of burgers Homer can eat without wasting any time. If he must waste time, he can
have beer.

\end{frame}

\begin{frame}[fragile]{Entrada e saída}

\textbf{Input}

Input consists of several test cases. Each test case consists of three integers $m, n, t$ ($0 < m,
n, t < 10000$). Input is terminated by EOF.

\vspace{0.2in}

\textbf{Output}

For each test case, print in a single line the maximum number of burgers Homer can eat without
having beer. If homer must have beer, then also print the time he gets for drinking, separated by
a single space. It is preferable that Homer drinks as little beer as possible.

\end{frame}

\begin{frame}[fragile]{Exemplo de entradas e saídas}

\begin{minipage}[t]{0.45\textwidth}
\textbf{Sample Input}
\begin{verbatim}
3 5 54
3 5 55
\end{verbatim}
\end{minipage}
\begin{minipage}[t]{0.5\textwidth}
\textbf{Sample Output}
\begin{verbatim}
18
17
\end{verbatim}
\end{minipage}
\end{frame}

\begin{frame}[fragile]{Solução $O(T)$}

    \begin{itemize}
        \item Observe que o problema consiste em minimizar o consumo de cerveja e, em segundo lugar,
            maximizar o número de hambúrger

        \item O problema pode ser caracterizado por um único parâmetro: o recurso disponível
            $t$
        
        \item Para cada subproblema $p(t)$ a solução é caracterizada por um par de valores
            $(b, h)$: o tempo mínimo gasto bebendo cervejas e o máximo de hambúrgeres a serem consumidos 

        \item Para manter a ordenação das soluções ótimas e usar a função \texttt{max()} do 
            C++, o valor de $b$ será representado pelo seu simétrico
   \end{itemize}

\end{frame}

\begin{frame}[fragile]{Solução $O(T)$}

    \begin{itemize}
        \item O caso base ocorre quando $t = 0$: neste caso, $p(0) = (0, 0)$
            
        \item Há 3 transições possíveis para o estado $p(t)$:

        \begin{enumerate}
            \item gastar todo o tempo restante com cervejas
            \item comer um hambúrguer durante $m$ minutos
            \item comer um hambúrguer durante $n$ minutos
        \end{enumerate}

        \item Como há $O(T)$ estados distintos e as transições tem custo $O(1)$, uma solução
            baseada em programação dinâmica tem complexidade $O(T)$

        \item A complexidade em memória também é $O(T)$
   \end{itemize}

\end{frame}

\begin{frame}[fragile]{Solução $O(T)$}
    \inputsnippet{cpp}{1}{18}{codes/OJ_10465.cpp}
\end{frame}

\begin{frame}[fragile]{Solução $O(T)$}
    \inputsnippet{cpp}{20}{39}{codes/OJ_10465.cpp}
\end{frame}

\begin{frame}[fragile]{Solução $O(T)$}
    \inputsnippet{cpp}{41}{62}{codes/OJ_10465.cpp}
\end{frame}
