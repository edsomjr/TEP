%\section{OJ 147 -- Dollars}

\begin{frame}[fragile]{Problema}

New Zealand currency consists of $\$100, \$50, \$20, \$10$, and $\$5$ notes and $\$2, \$1, 50c,
20c, 10c$ and $5c$ coins. Write a program that will determine, for any given amount, in how many
ways that amount may be made up. Changing the order of listing does not increase the count. Thus
$20c$ may be made up in $4$ ways: $1\times 20c, 2\times 10c, 10c+2\times 5c$, and $4\times 5c$.

\end{frame}

\begin{frame}[fragile]{Entrada e saída}

\textbf{Input}

Input will consist of a series of real numbers no greater than $\$300.00$ each on a separate line.
Each amount will be valid, that is will be a multiple of $5c$. The file will be terminated by a
line containing zero $(0.00)$.

\vspace{0.2in}

\textbf{Output}

Output will consist of a line for each of the amounts in the input, each line consisting of the
amount of money (with two decimal places and right justified in a field of width $6$), followed by
the number of ways in which that amount may be made up, right justified in a field of width $17$.

\end{frame}

\begin{frame}[fragile]{Exemplo de entradas e saídas}

\begin{minipage}[t]{0.45\textwidth}
\textbf{Sample Input}
\begin{verbatim}
0.20
  2.00
0.00
\end{verbatim}
\end{minipage}
\begin{minipage}[t]{0.5\textwidth}
\textbf{Sample Output}
\begin{verbatim}
   0.20                 4
   2.00               293
\end{verbatim}
\end{minipage}
\end{frame}

\begin{frame}[fragile]{Solução $O(NM)$}

   \begin{itemize}
        \item Este problema é uma variante do problema do troco, onde se deseja obter o número
            de maneiras distintas de se dar um mesmo troco $m$ utilizando-se as moedas disponíveis

        \item Deste modo, é preciso adaptar o algoritmo de programação dinâmica para obter este
            resultado

        \item Importante notar que a ordem não é levada em consideração, de modo que 
            $\{ 0.25, 0.50, 0.25 \}$ e $\{ 0.50, 0.25, 0.25 \}$ são consideradas uma mesma
            maneira de dar o troco de $\$1$

        \item Seja $dp(m, i)$ o número de maneiras distintas de se dar o troco $m$ utilizando-se
            apenas as $i$ menores moedas

        \item São dois casos-base: $dp(0, 0) = 1$ e $dp(m, 0) = 0$, se $m > 0$
   \end{itemize}

\end{frame}

\begin{frame}[fragile]{Solução $O(NM)$}

    \begin{itemize}
        \item São duas transições possíveis: utilizar a $i$-ésima menor moeda ou não, sendo que
            ambas alternativas devem ser totalizadas

        \item Assim,
        \[
            dp(m, i) = dp(m - c_i, i) + dp(m, i - 1),
        \]
        se $c_i \leq m$, ou $dp(m, i) = dp(m, i - 1)$, caso contrário

        \item Se as moedas forem processadas em ordem crescente, é possível deixar ímplicita a
            segunda dimensão do estado, reduzindo o uso de memória e simplificando a implementação

        \item A tabela dos estados pode ser preenchida em $O(NM)$, e após este preenchimento
            cada caso de teste pode ser respondido em $O(1)$, onde $M$ é o maior troco
            possível e $N$ o número de moedas distintas
    \end{itemize}

\end{frame}

\begin{frame}[fragile]{Solução $O(NM)$}
    \inputsnippet{cpp}{1}{18}{codes/147.cpp}
\end{frame}

\begin{frame}[fragile]{Solução $O(NM)$}
    \inputsnippet{cpp}{20}{30}{codes/147.cpp}
\end{frame}
