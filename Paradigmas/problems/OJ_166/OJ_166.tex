%\section{OJ 166 -- Making Change}

\begin{frame}[fragile]{Problema}

Given an amount of money and unlimited (almost) numbers of coins (we will ignore notes for this
problem) we know that an amount of money may be made up in a variety of ways. A more interesting
problem arises when goods are bought and need to be paid for, with the possibility that change may
need to be given. Given the finite resources of most wallets nowadays, we are constrained in the number
of ways in which we can make up an amount to pay for our purchases -- assuming that we can make
up the amount in the first place, but that is another story.

\end{frame}

\begin{frame}[fragile]{Problema}

The problem we will be concerned with will be to minimise the number of coins that change hands
at such a transaction, given that the shopkeeper has an adequate supply of all coins. (The set of
New Zealand coins comprises $5c, 10c, 20c, 50c, \$1$ and $\$2$.) Thus if we need to pay $55c$, and
we do not hold a $50c$ coin, we could pay this as $2\times 20c + 10c + 5c$ to make a total of $4$
coins. If we tender $\$1$ we will receive $45c$ in change which also involves $4$ coins, but if we
tender $\$1.05 (\$1 + 5c)$, we get $50c$ change and the total number of coins that changes hands is
only $3$.

Write a program that will read in the resources available to you and the amount of the purchase
and will determine the minimum number of coins that change hands.

\end{frame}

\begin{frame}[fragile]{Entrada e saída}

\textbf{Input}

Input will consist of a series of lines, each line defining a different situation. Each line will
consist of $6$ integers representing the numbers of coins available to you in the order given
above, followed by a real number representing the value of the transaction, which will always be
less than $\$5.00$. The file will be terminated by six zeroes $(0\ 0\ 0\ 0\ 0\ 0)$. The total value
of the coins will always be sufficient to make up the amount and the amount will always be
achievable, that is it will always be a multiple of $5c$.

\vspace{0.1in}

\textbf{Output}

Output will consist of a series of lines, one for each situation defined in the input. Each line
will consist of the minimum number of coins that change hands right justified in a field $3$
characters wide.

\end{frame}

\begin{frame}[fragile]{Exemplo de entradas e saídas}

\begin{minipage}[t]{0.45\textwidth}
\textbf{Sample Input}
\begin{verbatim}
2 4 2 2 1 0 0.95
2 4 2 0 1 0 0.55
0 0 0 0 0 0
\end{verbatim}
\end{minipage}
\begin{minipage}[t]{0.5\textwidth}
\textbf{Sample Output}
\begin{verbatim}
  2
  3
\end{verbatim}
\end{minipage}
\end{frame}

\begin{frame}[fragile]{Solução $O(NM)$}

   \begin{itemize}
        \item Seja $m \geq C$ a quantidade paga ao caixa, onde $C$ é o valor da conta, em centavos

        \item Defina $G(m, xs)$ como o número mínimo de moedas utilizadas para pagar $m$ usando
            $xs$, onde
        \[
            xs = \{ x_1, x_2, \ldots, x_N \}
        \]
        e $x_i$ é o número de moedas disponíveis do tipo $c_i$
    
        \item Para o cidadão $xs$ é dado na entrada

        \item Para o caixa, $xs = ys$, com $y_i = \infty$ para todo $i\in [1, N]$

        \item Caso não seja possível dar um troco para $m$ usando $xs$, faça $G(m, xs) = \infty$
   \end{itemize}

\end{frame}

\begin{frame}[fragile]{Solução $O(NM)$}

    \begin{itemize}
        \item Quando o cidadão paga $m$ ao caixa, ele recebe $m - C$ de troco

        \item Utilizando a notação descrita, a solução $S$ do problema é dada por
        \[
            S = \min_i\{ G(C + 5i, xs) + G(5i, ys) \},
        \]
        onde $i\in [0, (M - C)/5]$ e $M$ é o maior valor possível para a conta

        \item Como $G(m, xs)$ tem complexidade $O(N)$, a solução proposta tem complexidade 
            $O(NM)$ por caso de teste
    \end{itemize}

\end{frame}

\begin{frame}[fragile]{Solução $O(NM)$}
    \inputsnippet{cpp}{1}{20}{codes/166.cpp}
\end{frame}

\begin{frame}[fragile]{Solução $O(NM)$}
    \inputsnippet{cpp}{22}{41}{codes/166.cpp}
\end{frame}

\begin{frame}[fragile]{Solução $O(NM)$}
    \inputsnippet{cpp}{43}{63}{codes/166.cpp}
\end{frame}
