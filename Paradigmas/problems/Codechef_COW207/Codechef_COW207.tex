%\section{Codechef -- Naruto and Rectangles}

\begin{frame}[fragile]{Problema}

Given $n$ non-negative integers $a_1, a_2, \ldots, a_n$, where each represents a point at 
coordinate $(i, a_i)$. $n$ vertical lines are drawn such that the two endpoints of the line $i$ is
at $(i, a_i)$ and $(i, 0)$.

Naruto wants to draw a rectangle using these lines. The base should be the $x$-axis, while the 
two sides should be any two lines from the above-given lines. The other lines will magically 
disappear. Note that he cannot move the lines. The only action done my Naruto is to draw the top 
line of the rectangle by connecting any two lines.

Now naruto wants to know the maximum possible area of rectangle he can get. Help him find it.
\end{frame}

\begin{frame}[fragile]{Entrada e saída}

\textbf{Input}

\begin{itemize}
    \item First line will contain $T$, number of testcases. Then the testcases follow.
    \item First line of each testcase contains $n$.
    \item Second line contains $n$ space separated integers $a_1, a_2, \ldots, a_n$.
\end{itemize}

\textbf{Output}

For each testcase, output in a single line, the maximum area.

\textbf{Constraints}

\begin{itemize}
    \item $1 \leq T \leq 10$
    \item $2 \leq N \leq 10^5$
    \item $2 \leq a_i \leq 10^3$
\end{itemize}

\end{frame}

\begin{frame}[fragile]{Exemplo de entradas e saídas}

\begin{minipage}[t]{0.5\textwidth}
\textbf{Sample Input}
\begin{verbatim}
1
9
1 8 6 2 5 4 8 3 7
\end{verbatim}
\end{minipage}
\begin{minipage}[t]{0.45\textwidth}
\textbf{Sample Output}
\begin{verbatim}
49
\end{verbatim}
\end{minipage}
\end{frame}

\begin{frame}[fragile]{Solução com complexidade $O(N\log N)$}

    \begin{itemize}
        \item Para cada par $(i, j)$, com $j > i$, o retângulo de maior área que pode ser
            formado terá base $b = (j - i)$ e altura $h = \min(a_i, a_j)$

        \item Como há $O(N^2)$ pares deste tipo, uma solução que verifique todos eles tem
            complexidade $O(N^2)$ e, neste problema, tem veredito TLE

        \item Observe que, para um $i$ fixo, basta verificar apenas dois índices: o elemento $L$
            mais à esquerda de $i$ com $a_L \geq a_i$ e o elemento $R$ mais à direita de $i$ tal
            que $a_R\geq a_i$

        \item Estes dois elementos podem ser identificados, de forma eficiente, por meio de
            ordenação e dois ponteiros

        \item Primeiramente, monte um vetor de pares $(a_i, i)$ e ordene este vetor em ordem
            não-crescente
   \end{itemize}

\end{frame}

\begin{frame}[fragile]{Solução com complexidade $O(N)$}

    \begin{itemize}
        \item Inicialize os ponteiros $L = \infty$ e $R = -\infty$, e a resposta com zero

        \item Para cada par do vetor ordenado, processe o pares $(a, i)$ e $(a_L, L)$, se $L < i$, e os pares
            $(a, i)$ e $(a_R, R)$, se $R > i$

        \item Após uma possível atualização da resposta, ajuste os ponteiros: $L = \min(L, i)$,
            $R = \max(R, i)$

        \item Observe que, com esta estratégia, a altura $a_i$ do próximo elemento a ser processado
            será inferior à altura dos elementos já processados, de modo que ela pode ser 
            combinada, de forma ótima, com qualquer um deles
        
        \item Por isso basta manter o registro, dentre eles, do que está mais à esquerda ($L$)
            e mais à direta ($R$) do vetor

        \item Esta solução tem complexidade $O(N\log N)$, por causa da ordenação
    \end{itemize}

\end{frame}

\begin{frame}[fragile]{Solução AC com complexidade $O(N\log N)$}
    \inputsnippet{cpp}{1}{20}{codes/COW207.cpp}
\end{frame}

\begin{frame}[fragile]{Solução AC com complexidade $O(N\log N)$}
    \inputsnippet{cpp}{22}{40}{codes/COW207.cpp}
\end{frame}

\begin{frame}[fragile]{Solução AC com complexidade $O(N\log N)$}
    \inputsnippet{cpp}{42}{63}{codes/COW207.cpp}
\end{frame}
