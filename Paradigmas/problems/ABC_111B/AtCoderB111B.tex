%\section{AtCoder Beginner Contest 111 -- Problem B: AtCoder Beginner Contest 111}

\begin{frame}[fragile]{Problema}

Kurohashi has never participated in AtCoder Beginner Contest (ABC).

The next ABC to be held is ABC $N$ (the $N$-th ABC ever held). Kurohashi wants to make his debut 
in some ABC $x$ such that all the digits of $x$ in base ten are the same.

What is the earliest ABC where Kurohashi can make his debut?

\end{frame}

\begin{frame}[fragile]{Entrada e saída}

\textbf{Constraints}

\begin{itemize}
    \item $100\leq N\leq 999$
    \item $N$ is an integer.
\end{itemize}

\textbf{Input}

Input is given from Standard Input in the following format:
\begin{align*}
N
\end{align*}

\textbf{Output}

If the earliest ABC where Kurohashi can make his debut is ABC $n$, print $n$.

\end{frame}

\begin{frame}[fragile]{Exemplo de entradas e saídas}

\begin{minipage}[t]{0.5\textwidth}
\textbf{Exemplo de Entrada}
\begin{verbatim}
111

112

750
\end{verbatim}
\end{minipage}
\begin{minipage}[t]{0.45\textwidth}
\textbf{Exemplo de Saída}
\begin{verbatim}
111

222

777
\end{verbatim}
\end{minipage}
\end{frame}

\begin{frame}[fragile]{Solução}

    \begin{itemize}
        \item Há apenas 9 possibilidades para a resposta: $111, 222, 333, 444, 555, 666, 777,
            888$ e $999$

        \item A resposta será o menor dentre entes valores que é maior ou igual a $N$

        \item Tal valor pode ser localizado através de uma busca linear simples

        \item Outra alternativa é utilizar a função \code{c}{lower_bound()} da STL

        \item Em ambos casos, a complexidade da solução é constante, pois há, no máximo, 9
            comparações a serem feitas, independentemente do valor de $N$
    \end{itemize}

\end{frame}

\begin{frame}[fragile]{Solução $O(1)$}
    \inputsnippet{cpp}{1}{15}{codes/111B.cpp}
\end{frame}

\begin{frame}[fragile]{Solução $O(1)$}
    \inputsnippet{cpp}{17}{37}{codes/111B.cpp}
\end{frame}
