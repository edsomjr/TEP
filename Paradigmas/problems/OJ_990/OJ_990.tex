%\section{OJ 990 -- Diving for Gold}

\begin{frame}[fragile]{Problema}

John is a diver and a treasure hunter. He has just found the location of a pirate ship full of 
treasures.  The sofisticated sonar system on board his ship allows him to identify the 
location, depth and quantity of gold in each suken treasure. Unfortunatelly, John forgot to 
bring a GPS device and the chances of ever finding this location again are very slim so he has 
to grab the gold now. To make the situation worse, John has only one compressed air bottle.

John wants to dive with the compressed air bottle to recover as much gold as possible from the
wreck. Write a program John can use to select which treasures he should pick to maximize the 
quantity of gold recovered.

\end{frame}

\begin{frame}[fragile]{Problema}

    The problem has the following restrictions: 

    \begin{itemize}
        \item There are $n$ treasures $\{(d_1, v_1),(d_2, v_2), \ldots, (d_n, v_n)\}$
            represented by the pair (depth, quantity of gold). There are at most 30 treasures.

        \item The air bottle only allows for $t$ seconds under water. $t$ is at most 1000 
            seconds.

        \item In each dive, John can bring the maximum of one treasure at a time.

        \item The descent time is $td_i = w \times d_i$, where $w$ is an integer constant.

        \item The ascent time is $ta_i = 2w \times d_i$, where $w$ is an integer constant.

        \item Due to instrument limitations, all parameters are integer.
    \end{itemize}

\end{frame}

\begin{frame}[fragile]{Entrada e saída}

\textbf{Input}

The input to this program consists of a sequence of integer values. Input contains several test 
cases. The first line of each dataset should contain the values $t$ and $w$. The second line 
contains the number of treasures. Each of the following lines contains the depth $d_i$ and 
quantity of gold $v_i$ of a different treasure.

A blank line separates each test case.

\textbf{Note}:

In this sample input, the bottle of compressed air has a capacity of 200 seconds, the constant 
$w$ has the value 4 and there are 3 treasures, the first one at a depth of 10 meters and worth 
5 coins of gold, the second one at a depth of 10 meters and worth 1 coin of gold, and the third 
one at 7 meters and worth 2 coins of gold.

\end{frame}


\begin{frame}[fragile]{Entrada e saída}
\textbf{Output}

The first line of the output for each dataset should contain the maximum amount of gold that 
John can recover from the wreck. The second line should contain the number of recovered 
treasures. Each of the following lines should contain the depth and amount of gold of each 
recovered treasure. Treasures should be presented in the same order as the input file.

Print a blank line between outputs for different datasets.

\end{frame}

\begin{frame}[fragile]{Exemplo de entradas e saídas}

\begin{minipage}[t]{0.55\textwidth}
\textbf{Sample Input}
\begin{verbatim}
210 4
3
10 5
10 1
7 2
\end{verbatim}
\end{minipage}
\begin{minipage}[t]{0.4\textwidth}
\textbf{Sample Output}
\begin{verbatim}
7
2
10 5
7 2
\end{verbatim}
\end{minipage}
\end{frame}

\begin{frame}[fragile]{Solução $O(TN)$}

   \begin{itemize}
        \item Este é um problema de mochila binária cuja solução demanda não apenas o valor
            ótima, mas também a lista dos elementos selecionados

        \item O tempo máximo de mergulho $T$ é a capacidade da mochila

        \item O valor de cada elemento é $v_i$

        \item Como cada mergulho só permite reaver um único elemento, o ``peso'' do elemento
            é a soma do tempo de mergulho com o tempo de retorno à superfície

        \item Assim,
        \[
            w_i = w\times d_i + 2w\times d_i = 3wd_i
        \]

        \item Para evitar o veredito WA, não se esqueça de incluir uma linha em branco entre
            as saídas de dois casos de teste consecutivos
   \end{itemize}

\end{frame}

\begin{frame}[fragile]{Solução $O(TN)$}
    \inputsnippet{cpp}{1}{20}{codes/990.cpp}
\end{frame}

\begin{frame}[fragile]{Solução $O(TN)$}
    \inputsnippet{cpp}{22}{41}{codes/990.cpp}
\end{frame}

\begin{frame}[fragile]{Solução $O(TN)$}
    \inputsnippet{cpp}{43}{62}{codes/990.cpp}
\end{frame}

\begin{frame}[fragile]{Solução $O(TN)$}
    \inputsnippet{cpp}{64}{84}{codes/990.cpp}
\end{frame}

