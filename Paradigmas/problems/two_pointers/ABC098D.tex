\section{AtCoder Beginner Contest 098 -- Problem D: Xor Sum 2}

\begin{frame}[fragile]{Problema}

There is an integer sequence $A$ of length $N$.

Find the number of the pairs of integers $l$ and $r$ ($1\leq l\leq r\leq N$) that satisfy the 
following condition:

\begin{itemize}
    \item $A_l\ xor\ A_{l + 1}\ xor\ \ldots\ xor\ A_r = A_l + A_{l + 1} + \ldots + A_r$
\end{itemize}

Here, $xor$ denotes the bitwise exclusive OR.
\end{frame}

\begin{frame}[fragile]{Entrada e saída}

\textbf{Constraints}

\begin{itemize}
    \item $1\leq N \leq 2\times 10^5$
    \item $0\leq A_i < 2^{20}$
    \item All values in input are integers.
\end{itemize}

\textbf{Input}

Input is given from Standard Input in the following format:
\begin{align*}
N &\\
A_1 &\ A_2\ \ldots\ A_N
\end{align*}

\textbf{Output}

Print the number of the pairs of integers $l$ and $r$ ($1\leq l\leq r\leq N$) that satisfy the 
condition.

\end{frame}

\begin{frame}[fragile]{Exemplo de entradas e saídas}

\begin{minipage}[t]{0.6\textwidth}
\textbf{Sample Input}
\begin{verbatim}
4
2 5 4 6

9
0 0 0 0 0 0 0 0 0

19
885 8 1 128 83 32 256 206 639 16 4
128 689 32 8 64 885 969 1
\end{verbatim}
\end{minipage}
\begin{minipage}[t]{0.35\textwidth}
\textbf{Sample Output}
\begin{verbatim}
5

45

37
\end{verbatim}
\end{minipage}
\end{frame}

\begin{frame}[fragile]{Solução com complexidade $O(N)$}

    \begin{itemize}
        \item Dados dois inteiros $x$ e $y$, a igualdade $x\ xor\ y = x + y$ só é verdadeira
            se $x$ e $y$ não tiverem nenhum \textit{bit} em comum

        \item Assim, um intervalo de índices $[l, r]$ atenderá o critério do problema somente se,
            para cada par de elementos $A_i, A_j$, com $i, j\in [l, r]$ e $i \neq j$, vale que
            $A_i\ and\ A_j = 0$ (isto é, $A_i$ e $A_j$ não tem \textit{bits} em comum)

        \item Como existem $O(N^2)$ pares de índices $(l, r)$, com $1\leq l\leq r\leq N$, uma
            solução que verifique cada um destes pares teria um veredito TLE

        \item Porém, é possível resolver o problema em $O(N)$, por meio da técnica dos dois
            ponteiros

        \item Para isso, inicie o ponteiro $L$ no primeiro elemento do vetor
   \end{itemize}

\end{frame}

\begin{frame}[fragile]{Solução com complexidade $O(N)$}

    \begin{itemize}
        \item Inicie com zero uma variável $x$, que conterá a disjunção (ou) dos elementos do
            intervalo $[L, R)$

        \item Para cada valor de $L$, inclua $A_L$ em $x$

        \item O ponteiro $R$ deve ser o maior dentre $L$ e o próprio $R$

        \item Enquanto $R$ apontar para um elemento do vetor e $A_R$ não tiver \textit{bits} em
            comum com $x$, inclua $A_R$ em $x$ e incremente $R$

        \item Ao final do processo, qualquer intervalo $[L, r]$, com $r\in [L, R)$, será um
            intervalo válido do problema

        \item Assim, a resposta deve ser acrescida em $R - L$

        \item Após a atualização da resposta, remova o elemento $A_L$ de $x$ e incremente $L$
            (observe que $R - L > 0$ em qualquer iteração)
    \end{itemize}

\end{frame}
\begin{frame}[fragile]{Solução com complexidade $O(N)$}
    \inputsnippet{cpp}{1}{21}{codes/ABC098D.cpp}
\end{frame}

\begin{frame}[fragile]{Solução com complexidade $O(N)$}
    \inputsnippet{cpp}{22}{42}{codes/ABC098D.cpp}
\end{frame}

\begin{frame}[fragile]{Solução com complexidade $O(N)$}
    \inputsnippet{cpp}{43}{63}{codes/ABC098D.cpp}
\end{frame}
