%\section{SPOJ -- Ada and Plants}

\begin{frame}[fragile]{Problema}

Ada the Ladybug has grown many plants. She was trying to grow all plants with equal size. Now she
is wondering about the biggest difference between heights of two plants which are near each other.
Plants are near each other, if there are at most $K$ plants between them.

\end{frame}

\begin{frame}[fragile]{Entrada e saída}

\textbf{Input}

The first line contains $T$, the number of test-cases. The first line of each test-case will
contain $N, K, 1 < N\leq 10^5, 0\leq K\leq 10^5$ where $N$ indicates number of plants.
Next line will contain $N$ integers $0\leq hi\leq 10^9$ indicating height of $i$-th plant.

Sum of all $N$ among all test-cases won't exceed $3\times 10^6$.

\vspace{0.1in}

\textbf{Output}

For each test-case, print exactly one number -- the biggest difference of plants near each other
(biggest $h_i-h_j$ such that $|i-j|-1\leq K$).

\end{frame}

\begin{frame}[fragile]{Exemplo de entradas e saídas}

\begin{minipage}[t]{0.5\textwidth}
\textbf{Sample Input}
\begin{verbatim}
3
5 0
1 2 3 5 6
4 6
1 10 2 9
10 1
1 7 8 9 19 11 21 8 11 0 
\end{verbatim}
\end{minipage}
\begin{minipage}[t]{0.45\textwidth}
\textbf{Sample Output}
\begin{verbatim}
2
9
13
\end{verbatim}
\end{minipage}
\end{frame}

\begin{frame}[fragile]{Solução com complexidade $O(N\log N)$}

    \begin{itemize}
        \item Segundo o texto do problema, quaisquer duas plantas cujos índices estejam em um
            intervalo $[L, R)$, com $R - L = K + 2$, estão próximas o suficiente

        \item A cada intervalo $[L, R)$, a maior diferença ocorrerá entre os elementos com 
            índices neste intervalo e que tenham a maior e a menor altura, respectivamente

        \item Logo, o problema pode ser resolvido usando a técnica de dois ponteiros com janela
            móvel (\textit{sliding window})

        \item Para obter, de forma eficiente, os valores da maior e da menor altura no intervalo,
            há duas formas

        \item Uma delas é manter as alturas dos elementos do intervalo em um \texttt{multiset}:
            os ponteiros \texttt{rbegin()} e \texttt{begin()} apontarão para os elementos
            desejados
   \end{itemize}

\end{frame}

\begin{frame}[fragile]{Solução com complexidade $O(N\log N)$}

    \begin{itemize}
        \item A inserção e remoção de um elemento no \texttt{multiset} é feita em $O(\log K)$

        \item Cada elemento será inserido ou removido do \texttt{multiset} no máximo uma vez, logo
            a solução terá complexidade $O(N\log K)$

        \item Observe que o valor de $K$ pode exceder $N$, de modo que ele deve ser tratado com
            cuidado

        \item Também é possível resolver este problema em $O(N)$: basta usar a estratégia de 
            dupla pilha para manter os valores do menor e do maior elemento do intervalo    

        \item A implementação é mais longa do que a que utiliza o \texttt{multiset}, porém tem
            melhor tempo de execução e usa menos memória
    \end{itemize}

\end{frame}

\begin{frame}[fragile]{Solução AC com complexidade $O(N\log K)$}
    \inputsnippet{cpp}{1}{20}{codes/ADAPLANT.cpp}
\end{frame}

\begin{frame}[fragile]{Solução AC com complexidade $O(N\log K)$}
    \inputsnippet{cpp}{22}{40}{codes/ADAPLANT.cpp}
\end{frame}

\begin{frame}[fragile]{Solução AC com complexidade $O(N\log K)$}
    \inputsnippet{cpp}{42}{63}{codes/ADAPLANT.cpp}
\end{frame}

\begin{frame}[fragile]{Solução AC com complexidade $O(N)$}
    \inputsnippet{cpp}{1}{20}{codes/ADAPLANT2.cpp}
\end{frame}

\begin{frame}[fragile]{Solução AC com complexidade $O(N)$}
    \inputsnippet{cpp}{22}{41}{codes/ADAPLANT2.cpp}
\end{frame}

\begin{frame}[fragile]{Solução AC com complexidade $O(N)$}
    \inputsnippet{cpp}{43}{60}{codes/ADAPLANT2.cpp}
\end{frame}

\begin{frame}[fragile]{Solução AC com complexidade $O(N)$}
    \inputsnippet{cpp}{62}{81}{codes/ADAPLANT2.cpp}
\end{frame}

\begin{frame}[fragile]{Solução AC com complexidade $O(N)$}
    \inputsnippet{cpp}{83}{101}{codes/ADAPLANT2.cpp}
\end{frame}

\begin{frame}[fragile]{Solução AC com complexidade $O(N)$}
    \inputsnippet{cpp}{103}{125}{codes/ADAPLANT2.cpp}
\end{frame}
