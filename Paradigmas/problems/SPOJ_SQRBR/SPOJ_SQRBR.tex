%\section{SPOJ SQRBR -- Square Brackets}

\begin{frame}[fragile]{Problema}

You are given:

\begin{itemize}
    \item a positive integer $n$,
    \item an integer $k$, $1\leq k\leq n$,
    \item an increasing sequence of $k$ integers $0 < s_1 < s_2 < \ldots < s_k \leq 2n$.
\end{itemize}

What is the number of proper bracket expressions of length $2n$ with opening brackets appearing in
positions $s_1, s_2, \ldots,s_k$?

\end{frame}

\begin{frame}[fragile]{Problema}

\textbf{Illustration}

    Several proper bracket expressions:
        \begin{verbatim}
    [[]][[[]][]] 
    [[[][]]][][[]]
        \end{verbatim}
An improper bracket expression:

        \begin{verbatim}
    [[[][]]][]][[]]
        \end{verbatim}
There is exactly one proper expression of length 8 with opening brackets in positions 2, 5 and 7.

\end{frame}

\begin{frame}[fragile]{Problema}

\textbf{Task}

    Write a program which for each data set from a sequence of several data sets:

    \begin{itemize}
        \item reads integers $n, k$ and an increasing sequence of $k$ integers from input,
        \item computes the number of proper bracket expressions of length $2n$ with opening
            brackets appearing at positions $s_1, s_2, \ldots, s_k$,
        \item writes the result to output.
    \end{itemize}

\end{frame}

\begin{frame}[fragile]{Entrada e saída}

\textbf{Input}

The first line of the input file contains one integer $d$, $1 \leq d \leq 10$, which is the number
of data sets. The data sets follow. Each data set occupies two lines of the input file. The first
line contains two integers $n$ and $k$ separated by single space, $1 \leq n \leq 19, 1 \leq k \leq
n$. The second line contains an increasing sequence of $k$ integers from the interval $[1;2n]$
separated by single spaces.

\vspace{0.2in}

\textbf{Output}

The $i$-th line of output should contain one integer -- the number of proper bracket expressions of
length $2n$ with opening brackets appearing at positions $s_1, s_2, \ldots, s_k$.

\end{frame}

\begin{frame}[fragile]{Exemplo de entradas e saídas}

\begin{minipage}[t]{0.45\textwidth}
\textbf{Sample Input}
\begin{verbatim}
5 
1 1 
1 
1 1 
2 
2 1 
1 
3 1 
2 
4 2 
5 7 
\end{verbatim}
\end{minipage}
\begin{minipage}[t]{0.5\textwidth}
\textbf{Sample Output}
\begin{verbatim}
1 
0 
2 
3 
2 
\end{verbatim}
\end{minipage}
\end{frame}

\begin{frame}[fragile]{Solução $O(N^2)$}

    \begin{itemize}
        \item Uma solução de força bruta listaria todas as $2^{2n}$ strings $s_i$ formadas pelos 
            caracteres `\texttt{[}' e `\texttt{]}' e, para cada $i$, identificaria se $s_i$ é uma
            sequência válida e, em caso afirmativo, se os caracteres das posições indicadas na
            entrada são iguais a `\texttt{[}'

        \item A verificação da validade de $s_i$ é feita em $O(N)$, de modo que tal solução teria
            complexidade $O(N2^{2N})$, o que levaria a um veredito TLE

        \item Uma forma de reduzir esta complexidade é construir as sequências válidas
            iterativamente e não recalcular a validade de uma mesma subsequência múltiplas vezes

        \item Isto pode ser feito por meio de programação dinâmica
    \end{itemize}

\end{frame}

\begin{frame}[fragile]{Solução $O(N^2)$}

    \begin{itemize}
        \item Seja $s$ uma string de tamanho $2n$ tal que $s_j = $`\texttt{[}' para toda posição
            $j$ indicada na entrada

        \item Defina $c(i, open)$ como o número de sequências válidas que podem ser formadas 
            a partir do sufixo $s[i, 2n]$ sem modificar os caracteres $s_j$ pré-definidos,
            considerando que restaram $open$ caracteres `\texttt{[}' sem o `\texttt{]}' 
            correspondente no prefixo $s[1, (i-1)]$

        \item Uma vez que os sufixos $s[m, 2n]$ são vazios se $m > 2n$, então $c(m, open) = 0$ se
            $m > 2n$

        \item Atenção, porém, ao caso $m = 2n + 1$: ele trata o primeiro sufixo vazio, e indica
            que todos os caracteres anteriores foram definidos

        \item Assim, se $open = 0$, a sequência definida anteriormente é valida, de modo que
            $c(2n + 1, 0) = 1$

        \item Os demais permanecem iguais a zero, se $m > 2n$

    \end{itemize}

\end{frame}

\begin{frame}[fragile]{Solução $O(N^2)$}

    \begin{itemize}
        \item Para $1\leq i\leq 2n$, há duas transições possíveis:
        \begin{enumerate}
            \item adicionar mais um símbolo `\texttt{[}' na string
            \item adicionar um símbolo `\texttt{]}' na string, se $open > 0$ e se a posição não
                estiver pré-definida
        \end{enumerate}

        \item A primeira transição corresponde a
        \[
            c(i, open) = c(i + 1, open + 1)
        \]

        \item Esta primeira transição sempre pode ser feita

        \item A segunda transição nem sempre pode ser feita, devido as restrições apresentadas

        \item Caso $open > 0$ e $i$ não seja um dos índices pré-definidos com `\texttt{[}', então
        \[
            c(i, open) = c(i + 1, open + 1) + c(i + 1, open - 1)
        \]

        \item Como há $O(N^2)$ estados e as transições são feitas em $O(1)$, a solução tem
            complexidade $O(N^2)$
    \end{itemize}

\end{frame}

\begin{frame}[fragile]{Solução $O(N^2)$}
    \inputsnippet{cpp}{1}{20}{codes/SQRBR.cpp}
\end{frame}

\begin{frame}[fragile]{Solução $O(N^2)$}
    \inputsnippet{cpp}{22}{41}{codes/SQRBR.cpp}
\end{frame}

\begin{frame}[fragile]{Solução $O(N^2)$}
    \inputsnippet{cpp}{43}{63}{codes/SQRBR.cpp}
\end{frame}
