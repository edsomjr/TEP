%\section{Timus 1146 -- Maximum Sum}

\begin{frame}[fragile]{Problema}

Given a 2-dimensional array of positive and negative integers, find the sub-rectangle with the
largest sum. The sum of a rectangle is the sum of all the elements in that rectangle. In this
problem the sub-rectangle with the largest sum is referred to as the maximal sub-rectangle. A
sub-rectangle is any contiguous sub-array of size $1\times 1$ or greater located within the whole
array.

As an example, the maximal sub-rectangle of the array:
    \begin{table}[!ht]
        \begin{tabular}{>{\tt}r>{\tt}r>{\tt}r>{\tt}r}
            0 & -2 & -7 & 0 \\
            \textcolor{blue!60}{9} & \textcolor{blue!60}{2} & -6 & 2 \\
            \textcolor{blue!60}{-4} & \textcolor{blue!60}{1} & -4 & 1 \\
            \textcolor{blue!60}{-1} & \textcolor{blue!60}{8} & 0 & -2 \\
        \end{tabular}
    \end{table}

is in the lower-left-hand corner and has the sum of 15.

\end{frame}

\begin{frame}[fragile]{Entrada e saída}

\textbf{Input}

The input consists of an $N\times N$ array of integers. The input begins with a single positive
integer $N$ on a line by itself indicating the size of the square two dimensional array. This is
followed by $N^2$ integers separated by white-space (newlines and spaces). These $N^2$ integers
make up the array in row-major order (i.e., all numbers on the first row, left-to-right, then all
numbers on the second row, left-to-right, etc.). $N$ may be as large as 100. The numbers in the
array will be in the range $[-127, 127]$.

\vspace{0.2in}

\textbf{Output}

The output is the sum of the maximal sub-rectangle.

\end{frame}

\begin{frame}[fragile]{Exemplo de entradas e saídas}

\begin{minipage}[t]{0.45\textwidth}
\textbf{Sample Input}
\begin{verbatim}
4
0 -2 -7 0
9 2 -6 2
-4 1 -4 1
-1 8 0 -2
\end{verbatim}
\end{minipage}
\begin{minipage}[t]{0.5\textwidth}
\textbf{Sample Output}
\begin{verbatim}
15
\end{verbatim}
\end{minipage}
\end{frame}

\begin{frame}[fragile]{Solução $O(N^3)$}

    \begin{itemize}
        \item Uma solução de força bruta computaria a soma todas as $N^4$ submatrizes, sendo
            que cada soma é feita em $O(N^2)$, de modo que a solução teria complexidade $O(N^6)$

        \item Contudo, o uso de combinado de somas prefixadas e o algoritmo de Kadane permite
            identificar a submatriz de soma máxima com complexidade $O(N^3)$

        \item Para cada par de colunas $(i, j)$, deve ser computado, por meio do algoritmo 
            de Kadane nas somas $p_k(i, j)$, para $1\leq k\leq N$, o intervalo de maior soma,
            onde
            \[
                p_k(i, j) = \sum_{t=i}^j a_{kt}
            \]

        \item Veja que, dados os limites do problema, mesmo nos casos extremos a soma máxima ainda
            pode ser armazenada em variáveis inteiras

    \end{itemize}

\end{frame}

\begin{frame}[fragile]{Solução $O(N^3)$}
    \inputsnippet{cpp}{1}{20}{codes/1146.cpp}
\end{frame}

\begin{frame}[fragile]{Solução $O(N^3)$}
    \inputsnippet{cpp}{22}{40}{codes/1146.cpp}
\end{frame}

\begin{frame}[fragile]{Solução $O(N^3)$}
    \inputsnippet{cpp}{42}{63}{codes/1146.cpp}
\end{frame}
