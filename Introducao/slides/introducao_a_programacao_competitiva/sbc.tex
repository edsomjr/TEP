\section{Maratona de Programação da SBC}

\begin{frame}[fragile]{Maratona de Programação da SBC}

    \begin{itemize}
        \item O Brasil iniciou suas participações no ACM ICPC em 1996, com a I Maratona de
            Programação da SBC
        \item O melhor resultado do Brasil no ACM ICPC é o 13º lugar conquistado pelo 
            Instituto de Matemática e Estatística da USP em 2005
        \item A atual campeã (2024) é a Universidade de São Paulo
        \item A universidade com o maior número de vitórias é a Universidade Federal de
            Pernambuco, com 10 vitórias em 29 edições
    \end{itemize}

\end{frame}

\begin{frame}[fragile]{UnB e a Maratona de Programação}

    \begin{itemize}
        \item A Universidade de Brasília esteve presente na primeira edição do evento, em 1996
        \item O melhor resultado na UnB é uma 47ª posição na Final Mundial de 2019, que ocorreu
            em Moscou

        \item A equipe ``\textit{Rock Lee do Pagode Namora D+}'' foi formada pelos estudantes José Marcos 
            Silva Leite, Luís Braga Gebrim Silva e Thiago Veras

        \item A Faculdade UnB Gama -- FGA -- iniciou suas participações no ano de 2012
        \item O melhor resultado da FGA foi o 13º lugar na Final Brasileira obtido pela equipe 
            Teorema de Offson em 2017
    \end{itemize}

\end{frame}
