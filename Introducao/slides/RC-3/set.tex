\section{std::set}

\subsection{Estruturas}

\begin{frame}[fragile]{Set}

    \metroset{block=fill}
    \begin{block}{Declaração do set}
        \inputsyntax{c++}{setdeclaration.st}
    \end{block}
	
    O set é uma estrutura de dado não linear, normalmente implementada com uma árvore binária vermelha-preta. Como ela é mantida balanceada o set possui boa performance para pesquisas.
         
\end{frame}

\begin{frame}[fragile]{Set}

    \metroset{block=fill}
    \begin{itemize}
        \item Inserção: a inserção vai ter complexidade logarítmica em relação ao tamanho, é necessário achar o melhor lugar para inserir e manter a árvore balanceada. 
        \item Deleção: também possui complexidade logarítmica em relação ao tamanho por causa da necessidade de manter a árvore balanceada.
        \item Pesquisa: achar um valor no set possui uma performance logarítmica, o que é um dos melhores motivos para se utilizar um set.
    \end{itemize}

\end{frame}

\begin{frame}[fragile]{Set}

    \metroset{block=fill}
    \begin{itemize}
        \item Inserção:
        \inputsyntax{c++}{setinsert.st}
        \item Deleção:
        \inputsyntax{c++}{seterase.st}
    \end{itemize}

\end{frame}

\begin{frame}[fragile]{Set}

    \metroset{block=fill}
    \begin{itemize}
        \item Pesquisa:
        \inputsyntax{c++}{setfind.st}
        \item Informações:
        \inputsyntax{c++}{setinfos.st}
    \end{itemize}

\end{frame}

\begin{frame}[fragile]{Set}

    \metroset{block=fill}
    \begin{itemize}
        \item O set não possui elementos repetidos
        \inputsyntax{c++}{setrepeat.st}
    \end{itemize}

\end{frame}

\begin{frame}[fragile]{Set}

    \inputcode{cpp}{set.cpp}

\end{frame}
