\section{std::map}

\subsection{Estruturas}

\begin{frame}[fragile]{Map}

    \begin{block}{Declaração do map}
        \inputsyntax{c++}{mapdeclaration.st}
    \end{block}

    O map é uma estrutura de dado não linear, ela possui pares de chaves e valores. Para cada chave existe um valor.

\end{frame}

\begin{frame}[fragile]{Map}

    Propriedades do container:

    \begin{itemize}
        \item Ordenado: a estrutura do map mantém ordenado.
        \item Chaves únicas: dois elementos (par chave-valor) não possuirão chaves equivalentes.
        \item Alocação dinamica: a estrutura vai alterar sua alocação de acordo com a necessidade.
    \end{itemize}


\end{frame}

\begin{frame}[fragile]{Map}

    \begin{itemize}
        \item Inserção:
        \item Deleção: 
        \item Pesquisa: 
    \end{itemize}

\end{frame}

\begin{frame}[fragile]{Map}

    \begin{itemize}
        \item Inserção:
        \inputsyntax{c++}{mapinsert.st}
        \item Deleção:
        \inputsyntax{c++}{maperase.st}
    \end{itemize}

\end{frame}

\begin{frame}[fragile]{Map}

    \begin{itemize}
        \item Pesquisa:
        \inputsyntax{c++}{mapfind.st}
    \end{itemize}

\end{frame}

\begin{frame}[fragile]{Map}

    \inputsnippet{cpp}{}{18}{map.cpp}

\end{frame}

\begin{frame}[fragile]{Map}

    \inputsnippet{cpp}{20}{}{map.cpp}

\end{frame}
