\section{std::stack}

\subsection{Estruturas}

\begin{frame}[fragile]{Stack}

    \metroset{block=fill}
    \begin{block}{Declaração da Stack}
        \inputsyntax{c++}{stackdeclaration.st}
    \end{block}
	
    No STL, o stack é um container associativo. Ele traz uma estrutura LIFO (last in, first out). É possível inserir elementos no topo da pilha e retirar apenas do topo.
    
\end{frame}

\begin{frame}[fragile]{Stack}

    \metroset{block=fill}
    \begin{itemize}
        \item Inserção: a inserção será sempre constante em relação à sua complexidade de tempo.
        \item Acesso: o acesso ao primeiro e ao último elemento é constante.
        \item Deleção: a interface permite apenas deletar o elemento do topo da fila, complexidade de tempo constante.
    \end{itemize}

\end{frame}

\begin{frame}[fragile]{Stack}

    \metroset{block=fill}
    \begin{itemize}
        \item Inserção:
        \inputsyntax{c++}{stackinsert.st}
        \item Acesso:
        \inputsyntax{c++}{stackaccess.st}
    \end{itemize}

\end{frame}

\begin{frame}[fragile]{Stack}

    \metroset{block=fill}
    \begin{itemize}
        \item Deleção:
        \inputsyntax{c++}{stackerase.st}
        \item Informações:
        \inputsyntax{c++}{stackinfos.st}
    \end{itemize}

\end{frame}

\begin{frame}[fragile]{Stack}

    \inputcode{cpp}{stack.cpp}

\end{frame}
