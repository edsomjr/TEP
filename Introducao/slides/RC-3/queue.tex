\section{std::queue}

\subsection{Estruturas}

\begin{frame}[fragile]{Queue}

    \metroset{block=fill}
    \begin{block}{Declaração da Queue}
        \inputsyntax{c++}{queuedeclaration.st}
    \end{block}
	
    No STL, a queue é um container associativo. Ela traz uma estrutura FIFO (first in, first out). É possível inserir elementos no fim da fila e retirar apenas da frente.
    
\end{frame}

\begin{frame}[fragile]{Queue}

    \metroset{block=fill}
    \begin{itemize}
        \item Inserção: a inserção será sempre constante em relação à sua complexidade de tempo.
        \item Acesso: o acesso ao primeiro e ao último elemento é constante.
        \item Deleção: a interface permite apenas deletar o elemento da frente da fila, complexidade de tempo constante.
    \end{itemize}

\end{frame}

\begin{frame}[fragile]{Queue}

    \metroset{block=fill}
    \begin{itemize}
        \item Inserção:
        \inputsyntax{c++}{queueinsert.st}
        \item Acesso:
        \inputsyntax{c++}{queueaccess.st}
    \end{itemize}

\end{frame}

\begin{frame}[fragile]{Queue}

    \metroset{block=fill}
    \begin{itemize}
        \item Deleção:
        \inputsyntax{c++}{queueerase.st}
        \item Informações:
        \inputsyntax{c++}{queueinfos.st}
    \end{itemize}

\end{frame}

\begin{frame}[fragile]{Queue}

    \inputcode{cpp}{queue.cpp}

\end{frame}
