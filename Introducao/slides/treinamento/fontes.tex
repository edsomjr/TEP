\section{Recursos para prática de programação competitiva}

\begin{frame}[fragile]{Codeforces}

    \begin{itemize}
        \item Codeforces (\texttt{\url{codeforces.com}}) é um juiz online que hospeda competições
            regularmente

        \item É, atualmente, o melhor juiz online para treinamento de alto desempenho

        \item Tem como vantagem a transparência: as soluções dos problemas de todos os
            participantes ficam em aberto para estudo, e também são publicados editoriais
            com comentários sobre as soluções dos problemas

        \item Os contests podem ser feitos posteriormente, através da opção de participação
            virtual

        \item Vale a pena fazer ao menos os contests educacionais virtualmente

    \end{itemize}

\end{frame}

\begin{frame}[fragile]{AtCoder}

    \begin{itemize}
        \item AtCoder (\url{atcoder.jp}) é um juiz online japonês

        \item Ele hospeda, semanalmente, competições em parceria com empresas japonesas

        \item A série ABC (\textit{AtCoder Beginner Contest}) é um excelente material para competidores
            iniciantes

        \item Para competidores mais experientes há a série ARC (\textit{AtCoder Regular Contest}) e 
            também a série AGC (\textit{AtCoder Grand Contest})

        \item O site AtCoder Problems (\url{https://kenkoooo.com/atcoder/#/table/}) é um excelente
            recurso, que permite acompanhar os problemas já feitos e ainda por fazer, além de oferecer
            uma grande gama de estatísticas
     \end{itemize}

\end{frame}

\begin{frame}[fragile]{CD-MOJ}

    \begin{itemize}
        \item O CD-MOJ (\url{https://moj.naquadah.com.br/}) é um juiz online brasileiro desenvolvido
            pelo professor Bruno Ribas

        \item Ele oferece um modo de treino livre aberto a todos

        \item No momento (2025) ele ainda está em desenvolvimento, mas já conta com uma boa base de
            problemas

        \item Estão sendo inseridos, em sua base, todos os problemas da Olimpíada Brasileira de 
            Informática (OBI), com correção parcial, sendo este material um excelente recurso de
            treino e estudo para competidores iniciantes

    \end{itemize}

\end{frame}


\begin{frame}[fragile]{Online Judge}

    \begin{itemize}
        \item O (\textit{\url{onlinejudge.org}}, antigo UVa) é um juiz online com uma imensa base de
            problemas

        \item Já foi a principal plataforma de treino, mas tem sido abandonado em favor do
            Codeforces

        \item Contudo, os problemas do OJ são mais próximos dos problemas do ACM ICPC,
            de forma que um participante de alto nível deve ter contato com estes problemas
            também

        \item A plataforma uHunt (\texttt{\url{uhunt.onlinejudge.org}}) sistematiza a apresentação
            dos problemas do OJ e lista também o progresso do usuário nos problemas listados
            na série de livros \textit{Competitive Programming}

        \item O uHunt também permite a criação de contests virtuais

        \item O Live Archive (\texttt{\url{https://icpcarchive.ecs.baylor.edu/}}) traz os problemas
            da maioria dos eventos do ACM ICPC
    \end{itemize}

\end{frame}

\begin{frame}[fragile]{Livros Didáticos}

    \begin{itemize}
        \item O livro \textit{Competitive Programming 4}, dos irmãos Halim, é a mais conhecida
            referência de programação competitiva

        \item Ele traz uma série de exercícios sugeridos do OJ, a qual está listada no uHunt

        \item O livro \textit{Competitive Programmer's Handbook}, de Antti Laaksonen, é outra
            boa referência em programação competitiva

        \item O PDF do livro é gratuito

        \item Contudo, ele não dá implementação completas dos algoritmos, deixando-as a cargo
            dos leitores

        \item O CSES (\url{https://cses.fi/problemset/}) é um site que acompanha o livro e que traz um
            conjunto de problemas bastante interessante para estudo e treinamento
    \end{itemize}

\end{frame}
