\section{Exemplos de análise de complexidade assintótica}

\begin{frame}[fragile]{Exemplo 1}

	\textbf{Problema:} Implementar uma função que torne maíscula a primeira letra da string
        dada como parâmetro
	\vspace{0.1in}

	\textbf{Solução:} 
        \inputcode{cpp}{codes/capitalize.cpp}
	\vspace{0.2in}

	{\bf Complexidade: } A implementação da função faz uma única atribuição, de modo que, se
        a string tem $n$ caracteres, $f(n) = 1, \forall n$. Assim, o algoritmo tem complexidade
        $O(1)$.

\end{frame} 

\begin{frame}[fragile]{Exemplo 2}

	\textbf{Problema:} Implementar uma função que retorne a string dada, com a primeira letra 
        em maiúsculo
	\vspace{0.1in}

	\textbf{Solução:} 
        \inputcode{cpp}{codes/capitalize2.cpp}
	\vspace{0.2in}

	{\bf Complexidade: } Observe que os $n$ caracteres da string $s$ devem ser copiados em $r$,
        além da atribuição feita na linha 4. Assim, $f(n) = n + 1$, de modo que o algoritmo tem 
        complexidade $O(n)$.

\end{frame} 


\begin{frame}[fragile]{Exemplo 3}

    {\bf Problema:} Implementar uma função que retorne o produto cartesiano dos conjuntos
        de números inteiros $A$ e $B$.
	\vspace{0.1in}

    {\bf Solução:} 
        \inputcode{cpp}{codes/prod.cpp}
\end{frame} 

\begin{frame}[fragile]{Exemplo 3}

    {\bf Complexidade: } Na linha 4 são feitas duas atribuições (variáveis $n$ e $m$, 
        respectivamente). No início do laço externo (linha 6) é feita mais uma atribuição
        (\code{cpp}{i = 0}).

        A cada iteração do laço externo, são feitas 2 atribuições (\code{cpp}{j = 0} e o 
        incremento \code{cpp}{++i}, respectivamente). Como o laço externo é executado $n$ vezes, 
        são mais $2n$ atribuições.

        Em cada iteração do laço interno são feitas mais duas atribuições: o incremento 
        \code{cpp}{++j} e a adição do pair $(i, j)$ ao vetor \code{cpp}{P} por meio do método
        \code{cpp}{push_back()}. Como este método tem complexidade constante, ele equivale a 
        $c$ atribuições. Este laço será iniciado $n$ vezes, e em cada execução são $m$ iterações.
        Assim, serão mais
        \[
            \sum_{i = 0}^n\sum_{j = 0}^m (c + 1) = (c + 1)nm
        \]
        atribuições, de modo que $f(n, m) = 3 + 2n + (c + 1)nm$ é $O(nm)$.
\end{frame}

\begin{frame}[fragile]{Exemplo 4}

    {\bf Problema:} Implementar uma função que retorne todos os pares de naturais $(a, b)$ tais
        que $a < b \leq N$, para um inteiro positivo $N$ dado.
	\vspace{0.1in}

    {\bf Solução:} 
        \inputcode{cpp}{codes/pairs.cpp}
    \vspace{0.2in}

\end{frame}

\begin{frame}[fragile]{Exemplo 4}
    
    \textbf{Complexidade:} No início do laço externo (linha 5) é feita uma atribuição. Este
    laço tem $N$ iterações, e em cada uma delas são feitas mais duas atribuições: o incremento
    \code{cpp}{++a} e a inicialização \code{cpp}{b = i + 1}.

    A cada execução o laço interno itera ($N - i$ vezes), sendo que em cada iteração são duas
    atribuições: o incremento \code{cpp}{++b} e a inserção do par no vetor por meio do
    método \code{cpp}{push_back()} (para fins de simplicação, considere que o custo deste método
    seja o mesmo de uma atribuição). Assim, serão
    \[
        \sum_{i = 1}^N 2(N - i) = 2\sum_{k = 1}^{N - 1} k = 2\left[\frac{N(N - 1)}{2}\right] 
            = N(N - 1)
    \]
    atribuições.  Portanto,
    \[
        f(N) = 1 + 2N + N(N - 1),
    \]
    de modo que o algoritmo tem complexidade $O(N^2)$.
    
\end{frame}

\begin{frame}[fragile]{Exemplo 5}

	{\bf Problema:} Implemente uma função que compute o produto da matriz $A_{n\times m}$ pela
        matriz $B_{m\times p}$.

	\vspace{0.1in}

    {\bf Solução:} 
        \inputcode{cpp}{codes/matrix.cpp}
    \vspace{0.2in}
    
\end{frame}

\begin{frame}[fragile]{Exemplo 5}
	{\bf Complexidade:} No cálculo da complexidade assintótica de um algoritmo não é necessário
    determinar a função $f(n)$ exatamente, e sim determinar o termo dominante.

    A cada execução, o laço externo, o laço intermediário e o laço interno iteram $n, p$ e $m$ 
    vezes, respectivamente. Observe que, em cada execução de um destes laços, o número de 
    atribuições associadas a ele é constante.

    Assim, o algoritmo terá complexidade $O(nmp)$. A título de curiosidade, a função $f(n, m, p)$
    é dada por
    \[
        f(n, m, p) = 4 + 2n + 3np + 2nmp
    \]

\end{frame}

\begin{frame}[fragile]{Exemplo 6}

	{\bf Problema:} Computar um vetor de inteiros $v$ tal que $v_i = 1$, se $i$ é primo, e
        $v_i = 0$, caso contrário, para $i \leq N$.

	\vspace{0.1in}

    {\bf Algoritmo:} 
        \inputcode{cpp}{codes/sieve.cpp}

	\vspace{0.2in}
\end{frame}

\begin{frame}[fragile]{Exemplo 6}

    \textbf{Complexidade:} Na linha 3 é alocado um vetor com $N + 1$ posições, e cada uma
        destas posições é inicializada com o valor $1$, de modo que são $N + 1$ atribuições. A
        linha seguinte faz mais duas atribuições, uma vez que $0$ e $1$ não são primos.

        O laço externo inicia com uma atribuição (\code{cpp}{i = 2}) e itera $N - 1$ vezes,
        incrementando a variável \code{cpp}{i} (\code{cpp}{++i}) em cada uma destas iterações. Já
        a inicialização da variável \code{cpp}{j} (\code{cpp}{j = 2*i}) depende do valor de 
        \code{cpp}{v[i]}: só acontecerá quando \code{cpp}{v[i] = 1}, ou seja, quando \code{cpp}{i}
        for primo.

        Para terminar $f(N)$ precisamente seria necessário determinar o número exato de primos no
        intervalo $[1, N]$ (função $\pi(N)$). Existem boas aproximações para esta quantidade 
        (por exemplo,
            $\pi(N) \approx N/\log N$), porém quanto mais precisa a aproximação, mas sofisticada é a
            função.
\end{frame}

\begin{frame}[fragile]{Exemplo 6}

    Importante lembrar que a notação Big-$O$ fornece uma cota superior.
    Deste modo, o número de execuções do laço interno pode ser majorado, pois $\pi(N) < N$ 
    (pois nem todos os números do intervalo são primos).

    A cada execução do laço interno itera $\lfloor N/i\rfloor$ vezes, fazendo duas atribuições
    (o incremento \code{cpp}{j += i} e a atribuição \code{cpp}{v[j] = 0}) em cada iteração. Assim,
    o número total de atribuições associadas ao laço interno é
    \begin{align*}
        \sum_{p\ \mbox{\tiny primo} }^N 2\left\lfloor \frac{N}{p}\right\rfloor 
        &\leq  \sum_{i = 1}^N 2\left\lfloor \frac{N}{i}\right\rfloor 
        \leq  2\sum_{i = 1}^N \frac{N}{i} \\
        &\leq  2N\sum_{i = 1}^N \frac{1}{i}
        \leq  2N\sum_{i = 1}^\infty \frac{1}{i} \\
        &<  2N\int_{i = 1}^\infty \frac{1}{i} dN = 2N\log N
    \end{align*}
\end{frame}

\begin{frame}[fragile]{Exemplo 6}

    A sequência de aproximações utilizada forneceu uma cota superior para o número de atribuições
    desejado. Embora efetivamente ele seja menor do que a aproximação, ao menos há a garantia
    de que o número total de atribuições feitas pelo laço interno seja sempre menor do que a 
    cota estabelecida.

    Assim, a função $f(n)$ pode ser aproximada por
    \[
        f(n) = 3 + (N + 1) + 2(N - 1) + 2N\log N,
    \]
    de modo que o algoritmo é $O(N\log N)$.

\end{frame}

\begin{frame}[fragile]{Exemplo 7}

	{\bf Problema:} Localizar o índice de uma ocorrência do valor $x$ no vetor ordenado
        $v$, ou $-1$, caso $x$ não esteja presente em $v$.

    {\bf Algoritmo:} 
        \inputcode{cpp}{codes/bs.cpp}

\end{frame}

\begin{frame}[fragile]{Exemplo 7}

    {\bf Complexidade:} Observe que, neste algoritmo, o número total de atribuições depende
    dos valores da entrada.

    No melhor caso, o elemento $x$ se encontra no índice $\lfloor (N - 1)/2\rfloor$, de modo que
    o algoritmo realiza apenas $4$ atribuições:
    \begin{center} \code{cpp}{N = v.size(), a = 0, b = N - 1,  m = a + (b - a)/2}
    \end{center}

    Assim, $f(N) = 3$ e o algoritmo tem complexidade $O(1)$.

    No pior caso, $x$ não está no vetor. Além das 3 atribuições iniciais, o laço itera $k$ vezes, 
    onde $N/2^k \leq 1$, realizando duas atribuições ($m$ e uma dentre as variáveis \code{cpp}{a} e
    \code{cpp}{b}) a cada iteração. Logo $f(N) = 3 + 2\log N$ e o algoritmo tem complexidade 
    $O(\log n)$.

\end{frame}
