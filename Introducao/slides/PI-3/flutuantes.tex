\section{Variáveis em ponto flutuante}

\begin{frame}[fragile]{Variáveis em ponto flutuante}

    \begin{itemize}
        \item Em C e C++ há dois tipos de variável em ponto flutuante: \code{c++}{float} e
            \code{c++}{double}

        \item O tipo \code{c++}{float} representa valores em ponto flutuante com precisão
            simples (7 casas decimais de precisão)

        \item O tipo \code{c++}{double} representa valores em ponto flutuante com precisão
            dupla (15 casas decimais de precisão)

        \item O GCC tem suporte para o tipo \code{c++}{long double}, com 80 \textit{bits} e 
            precisão superior ao tipo \code{c++}{double}

        \item A precisão extra traz custos de memória e também de performance

    \end{itemize}

\end{frame}

\begin{frame}[fragile]{Observações sobre variáveis em ponto flutuante}

    \begin{itemize}
        \item Nem todo valor pode ser representado exatamente em variáveis em ponto
            flutuante:
            \inputcode{c++}{one.cpp}

        \item Multiplicações de valores muito pequenos por valores muito grandes geram erros
            de precisão
            \inputcode{c++}{multi.cpp}

        \item Comparações entre valores flutuantes podem gerar resultados errados por conta
            de erros de precisão:
            \inputcode{c++}{comp.cpp}

    \end{itemize}

\end{frame}

\begin{frame}[fragile]{Observações sobre variáveis em ponto flutuante}

    \begin{itemize}
        \item Se possível, o ideal é evitar o uso de variáveis do tipo ponto flutuante

        \item Em competições, algoritmos corretos podem receber o veredito \texttt{WA} por
            conta de erros de precisão

        \item No cálculo da distância entre dois pontos de coordenadas inteiras, o ideal é 
        trabalhar com o quadrado da distância, evitando a extração da raiz quadrada

        \item No caso de unidades monetárias, melhor trabalhar com múltiplos de centavos, de modo
            que os valores passam a ser todos números inteiros

        \item O mesmo vale para unidades de tempo: melhor trabalhar com a menor unidade inteira
            possível
    \end{itemize}

\end{frame}
