\section{Vetores e Matrizes}

\begin{frame}[fragile]{Definição de matrizes}

	\begin{itemize}
		\item Matrizes são coleções de objetos de um mesmo tipo, referenciadas por um nome 
		comum 
		\item Os elementos de uma matriz são acessados através de índices, que indicam as 
        coordenadas do elemento na matriz

        \item Matrizes unidimensionais são denominadas vetores

		\item Em C/C++, diferentemente da matemática, os índices das matrizes começam em zero, 
        não em 1 (um)

        \item Embora possam ser visualizadas bidimensionalmente, em computadores as matrizes
        são armazenadas de forma linear, uma linha por vez
	\end{itemize}

\end{frame} 

\begin{frame}[fragile]{Declaração de Matrizes} 

    \metroset{block=fill}
    \begin{block}{Sintaxe para declaração de matrizes}
        \inputsyntax{c}{matrizes.st}
    \end{block}
    onde $N_i$ é o número de elementos na $i$-ésima dimensão  

	\begin{itemize}
		\item Apenas a primeira dimensão é obrigatória, sendo as demais opcionais

		\item É possível inicializar um vetor durante sua declaração, deixando a dimensão em 
        aberto (sem preencher) e listando os elementos entre chaves e separados por vírgulas

        \item Exemplo de inicialização de vetor:
            \inputsyntax{c}{ex.st}
	\end{itemize}

\end{frame} 
 
\begin{frame}[fragile]{Acessando os elementos de um vetor}

	\begin{itemize}
		\item Os elementos de um vetor podem ser acessados, tanto para leitura quanto para escrita,
        usando-se o nome do vetor seguido pelo índice do elemento entre colchetes

		\item O primeiro elemento de um vetor de $N$ tem índice 0 (zero)

        \item O último elemento tem índice $N-1$

		\item Indicar como índice um número inteiro fora do intervalo [$0, N-1$] pode resultar, 
        dentre outros, um erro de violação de memória

        \item O acesso de elementos em uma matriz multidimensional é semelhante: basta indicar,
        entre colchetes, o índice do elemento em cada uma das dimensões da matriz
	\end{itemize}
 
\end{frame} 
  
\begin{frame}[fragile]{Exemplo de uso de matrizes}
    \inputsnippet{c}{1}{20}{tabela_esportiva.c}
\end{frame}

\begin{frame}[fragile]{Exemplo de uso de matrizes}
    \inputsnippet{c}{21}{41}{tabela_esportiva.c}
\end{frame}

\subsection{Ponteiros e strings}

\begin{frame}[fragile]{Relação entre ponteiro e vetor}

	\begin{itemize}
		\item O nome de um vetor é um ponteiro para o seu primeiro elemento

		\item Os ponteiros fornecem uma alternativa para o acesso aos elementos de um vetor: se 
        \texttt{\texttt{a}} é um vetor e \texttt{{p}} um ponteiro para o primeiro elemento 
        \texttt{{a}}, as expressões \texttt{{a[i]}} e \texttt{{*(p+i)}} são 
        equivalentes

		\item Em alguns contextos, o acesso via ponteiros pode ser mais rápido que o acesso via 
        índices

        \item No caso de matrizes, a expressão \texttt{{a[i][j]}} é equivalente à expressão
        \texttt{{*(*(p + i) + j)}}

        \item Em uma matriz linearizada, o elemento da posição (\texttt{i, j}) seria acessado
        através da expressão \texttt{a[j + i*M]}, onde \texttt{M} é igual ao número de colunas
        da matriz \texttt{a}
	\end{itemize}

\end{frame}

\begin{frame}[fragile]{Exemplo da relação entre ponteiro e vetor}
    \inputcode{cpp}{relacao_ponteiro_vetor.cpp}
\end{frame}

\begin{frame}[fragile]{Strings}

	\begin{itemize}
		\item Em C, as strings são vetores de caracteres terminados com o caractere 0 (zero)

        \item As operações em strings (atribuição, cópia, comparação, etc) não podem ser
        feitas diretamente

        \item A biblioteca \texttt{string.h} contém funções para manipulação de strings em C

        \item A falta do zero terminador é uma causa comum de \textit{bugs} e falhas de 
        segurança
 
		\item Em C++, strings são objetos de uma classe, embora guardem a notação de acesso aos 
        seus caracteres idêntica a usada em C

		\item Uma string C++ pode receber uma string C em uma atribuição

        \item Para obter uma string C 
		equivalente a uma string C++, basta invocar o método \texttt{c\_str}()
	\end{itemize}

\end{frame} 

\begin{frame}[fragile]{Exemplo de uso de strings em C}
    \inputcode{c}{string.c}
\end{frame}

\begin{frame}[fragile]{Exemplo de uso de strings em C++}
    \inputcode{cpp}{string.cpp}
\end{frame}
