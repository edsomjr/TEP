\section{Encurtamento de Código}

\begin{frame}[fragile]{Encurtamento de Código}

    \begin{itemize}
        \item Em programação competitiva, a velocidade é critério de desempate

        \item Uma maneira de se melhorar o desempenho no quesito velocidade é digitar o mais
            rápido possível

        \item Outra maneira é diminuir a quantidade de caracteres que devem ser digitados no
            código

        \item Esta redução pode ser feita através do uso dos recursos da própria linguagem ou
            através do uso de macros

        \item O uso inteligente de padrões e lógica também pode levar a reduções nos códigos

        \item Como os códigos de competição não visam reuso, não faz sentido usar nomes longos
            e descritivos para variáveis ou evitar o uso de variáveis globais
    \end{itemize}

\end{frame}

\begin{frame}[fragile]{Dica \#1: Renomear tipos}

    Alguns tipos de dados tem nomes longos, que podem ser encurtados através do uso
    da diretiva \code{c++}{using} do C++:
        \inputcode{c++}{using.cpp}

\end{frame}

\begin{frame}[fragile]{Dica \#2: C++ Moderno - \texttt{auto}}

    \begin{itemize}
        \item Alguns recursos do C++ moderno também levam a uma redução do tamanho dos 
        códigos-fontes

        \item A partir do C++11, a palavra reservada \code{c++}{auto} pode ser usada para 
            deduzir o tipo de uma variável a partir de sua inicialização

        \item Em alguns casos, a redução de código é notável:
            \inputcode{c++}{map.cpp}

        \item A inicialização pode ser feita com constantes, retornos de funções ou mesmo
            outras variáveis
            \inputcode{c++}{auto.cpp}
    \end{itemize}

\end{frame}

\begin{frame}[fragile]{Dica \#3: C++ Moderno - \texttt{range for}}

    \begin{itemize}
        \item O padrão C++11 também trouxe o recurso do \textit{range for}

        \item Antes deste padrão, a iteração em um conteiner era feito por meio de iteradores:
            \inputcode{c++}{iter.cpp}

        \item O \textit{range for} simplifica a sintaxe de iteração sobre contêiners,
            elimando o uso explícito de iteradores:
            \inputcode{c++}{rangefor.cpp}
    \end{itemize}

\end{frame}

\begin{frame}[fragile]{Dica \#4: C++ Moderno - \texttt{lambdas}}

    \begin{itemize}
        \item O C++11 também permite o uso de funções \textit{lambda}

        \item \textit{lambdas} são funções anônimas que podem ser declaradas e 
            implementadas localmente

        \item O uso de \textit{lambdas} simplifica o código, permitindo a customização de
            funções que permitem a customização de seu comportamento através do uso de
            ponteiros para funções
            \inputcode{c++}{sort.cpp}
    \end{itemize}

\end{frame}

\begin{frame}[fragile]{Dica \#5: STL}

    \begin{itemize}
        \item A STL (\textit{Standard Template Library}) do C++ é a biblioteca padrão de
            \textit{templates}

        \item Ela oferece a implementação de estruturas de dados e funções parametrizadas

        \item Estas implementações são flexíveis tanto no que diz respeito ao tipo de dado
            que elas agem quanto ao seu conjunto de parâmetros

        \item O uso das estruturas e funções da STL simplificam o código

        \item Além da redução do tamanho, o uso da STL traz confiabilidade ao código, uma
            vez que as implementações do compiladores são robustas e bem testadas

        \item Produzir implementações equivalentes durante uma maratona é inviável
    \end{itemize}

\end{frame}

\begin{frame}[fragile]{Exemplo de uso da STL}
    \inputcode{c++}{stl.cpp}
\end{frame}

\begin{frame}[fragile]{Dica \#6: Impressão de vetores}

    \begin{itemize}
        \item Um padrão comum de respostas de problemas de competição é imprimir um
            vetor de números, separados por espaços entre eles

        \item O código abaixo gera um \textit{Presentation Error}, pois imprime um espaço
            em branco após o último número:
            \inputcode{c++}{pe.cpp}

        \item Além disso, é preciso de uma linha extra, após o \code{c++}{for}, para a 
            quebra de linha

        \item O código abaixo resolve ambos problemas:
            \inputcode{c++}{vecprint.cpp}
    \end{itemize}

\end{frame}

\begin{frame}[fragile]{Dica \#7: Eliminação de \texttt{if}}

    \begin{itemize}
        \item A eliminação de um \code{c++}{if}, além de reduzir o tamanho do código,
            traz como benefício adicional um ganho de performance de execução

        \item O operator \texttt{\%} permite navegar circularmente pelos números
            $0, 1, \ldots, N - 1$ sequencialmente sem a 
            necessidade de um \code{c++}{if} para determinar o momento de se reiniciar o 
            contador
                \inputcode{c++}{mod.cpp} 

        \item É possível alternar entre dois inteiros $A$ e $B$ sem necessidade de um 
            \code{c++}{if}: se $C = A + B$, então o valor de $i$ alterna entre $A$ e $B$ 
            se for atualizado com o valor $C - i$:
                \inputcode{c++}{alt.cpp} 
    \end{itemize}

\end{frame}

\begin{frame}[fragile]{Dica \#8: Definição de constantes importantes}

    \begin{itemize}
        \item Embora constantes também possam ser definidas como macros, usar os recursos
            da própria linguagem permitem um maior nível de verificação e validação da
            parte do compilador

        \item A palavra reservada \code{c++}{const} pode ser usada para este fim

        \item Em geral, as constantes devem ser variáveis globais e receber nome com todos os
            caracteres maiúsculos

        \item Todas as constantes devem receber um valor inicial, uma vez que não é possível
            alterar seus valores em qualquer outro ponto do código

            \inputcode{c++}{const.cpp}

    \end{itemize}

\end{frame}
