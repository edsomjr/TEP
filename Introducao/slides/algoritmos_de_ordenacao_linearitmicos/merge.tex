\section{\it MergeSort}

\begin{frame}[fragile]{\it MergeSort}

    \begin{itemize}
        \item O \textit{MergeSort} é um algoritmo de ordenação antigo, já conhecido por
            John von Neumann em 1945

        \item Ele usa o paradigma dividir-e-conquistar para ordenar os elementos de um
            vetor

        \item Ele divide o vetor em duas metades, ordena cada uma delas e, em seguida,
           funde ambas partes em um todo ordenado

        \item O algoritmo é replicado em cada uma das metades, até que o tamanho de cada
            parte seja trivialmente ordenável 

        \item A complexidade é linearítmica, isto é, $O(N\log N)$, onde $N$ é o número de 
            elementos no vetor
    \end{itemize}

\end{frame}

\begin{frame}[fragile]{Fusão de dois vetores ordenados}

    \begin{itemize}
        \item A divisão do vetor em partes cada vez menores corresponde à etapa 
            \textbf{dividir} do algoritmo

        \item Se $N = 1$, o vetor já está trivialmente ordenado

        \item A fusão de duas partes ordenadas consiste na etapa \textbf{conquistar} do
            paradigma

        \item Esta não é uma etapa trivial, e é linear em relação à soma do número de elementos
            de cada parte

        \item O procedimento consiste em inicializar um ponteiro para o primeiro elemento de 
            cada parte e, sucessivamente, escolher o menor dentre os elementos disponíveis

        \item Este procedimento não pode ser feito \textit{in-place}, gerando um custo de
            memória $O(N)$ adicional ao algoritmo
    \end{itemize}

\end{frame}

\begin{frame}[fragile]{Visualização da rotina de fusão}

    \begin{figure}
        \centering

        \begin{tikzpicture}
            \draw[opacity=0] (-0.7, 5) circle [radius=1pt];
            \draw (0, 5) grid (5, 6);
            \draw (6, 5) grid (10, 6);
            \draw (0, 2) grid (9, 3);

            \node at (0.5, 5.5) { \textcolor{black}{$12$} };
            \node at (1.5, 5.5) { \textcolor{black}{$37$} };
            \node at (2.5, 5.5) { \textcolor{black}{$45$} };
            \node at (3.5, 5.5) { \textcolor{black}{$60$} };
            \node at (4.5, 5.5) { \textcolor{black}{$89$} };

            \node at (6.5, 5.5) { \textcolor{black}{$20$} };
            \node at (7.5, 5.5) { \textcolor{black}{$33$} };
            \node at (8.5, 5.5) { \textcolor{black}{$52$} };
            \node at (9.5, 5.5) { \textcolor{black}{$97$} };

            \draw[opacity=0,->] (0.5, 6.5) node[anchor=south] { $i$ } -- (0.5, 6.25);
            \draw[opacity=0,->] (6.5, 6.5) node[anchor=south] { $j$ } -- (6.5, 6.25);
            \draw[->] (-0.5, 1.5) node[anchor=north] { $k$ } -- (-0.5, 1.75);

        \end{tikzpicture}

    \end{figure}

\end{frame}

\begin{frame}[fragile]{Visualização da rotina de fusão}

    \begin{figure}
        \centering

        \begin{tikzpicture}
            \draw[opacity=0] (-0.7, 5) circle [radius=1pt];
            \draw (0, 5) grid (5, 6);
            \draw (6, 5) grid (10, 6);
            \draw (0, 2) grid (9, 3);

            \node at (0.5, 5.5) { \textcolor{blue}{$12$} };
            \node at (1.5, 5.5) { \textcolor{black}{$37$} };
            \node at (2.5, 5.5) { \textcolor{black}{$45$} };
            \node at (3.5, 5.5) { \textcolor{black}{$60$} };
            \node at (4.5, 5.5) { \textcolor{black}{$89$} };

            \node at (6.5, 5.5) { \textcolor{red}{$20$} };
            \node at (7.5, 5.5) { \textcolor{black}{$33$} };
            \node at (8.5, 5.5) { \textcolor{black}{$52$} };
            \node at (9.5, 5.5) { \textcolor{black}{$97$} };

            \draw[->] (0.5, 6.5) node[anchor=south] { $i$ } -- (0.5, 6.25);
            \draw[->] (6.5, 6.5) node[anchor=south] { $j$ } -- (6.5, 6.25);
            \draw[->] (0.5, 1.5) node[anchor=north] { $k$ } -- (0.5, 1.75);

            \node at (0.5, 2.5) { \textcolor{blue}{$12$} };

        \end{tikzpicture}

    \end{figure}

\end{frame}

\begin{frame}[fragile]{Visualização da rotina de fusão}

    \begin{figure}
        \centering

        \begin{tikzpicture}
            \draw[opacity=0] (-0.7, 5) circle [radius=1pt];
            \draw (0, 5) grid (5, 6);
            \draw (6, 5) grid (10, 6);
            \draw (0, 2) grid (9, 3);

            \node at (0.5, 5.5) { \textcolor{black}{$12$} };
            \node at (1.5, 5.5) { \textcolor{red}{$37$} };
            \node at (2.5, 5.5) { \textcolor{black}{$45$} };
            \node at (3.5, 5.5) { \textcolor{black}{$60$} };
            \node at (4.5, 5.5) { \textcolor{black}{$89$} };

            \node at (6.5, 5.5) { \textcolor{blue}{$20$} };
            \node at (7.5, 5.5) { \textcolor{black}{$33$} };
            \node at (8.5, 5.5) { \textcolor{black}{$52$} };
            \node at (9.5, 5.5) { \textcolor{black}{$97$} };

            \draw[->] (1.5, 6.5) node[anchor=south] { $i$ } -- (1.5, 6.25);
            \draw[->] (6.5, 6.5) node[anchor=south] { $j$ } -- (6.5, 6.25);
            \draw[->] (1.5, 1.5) node[anchor=north] { $k$ } -- (1.5, 1.75);

            \node at (0.5, 2.5) { \textcolor{blue}{$12$} };
            \node at (1.5, 2.5) { \textcolor{blue}{$20$} };

        \end{tikzpicture}

    \end{figure}

\end{frame}

\begin{frame}[fragile]{Visualização da rotina de fusão}

    \begin{figure}
        \centering

        \begin{tikzpicture}
            \draw[opacity=0] (-0.7, 5) circle [radius=1pt];
            \draw (0, 5) grid (5, 6);
            \draw (6, 5) grid (10, 6);
            \draw (0, 2) grid (9, 3);

            \node at (0.5, 5.5) { \textcolor{black}{$12$} };
            \node at (1.5, 5.5) { \textcolor{red}{$37$} };
            \node at (2.5, 5.5) { \textcolor{black}{$45$} };
            \node at (3.5, 5.5) { \textcolor{black}{$60$} };
            \node at (4.5, 5.5) { \textcolor{black}{$89$} };

            \node at (6.5, 5.5) { \textcolor{black}{$20$} };
            \node at (7.5, 5.5) { \textcolor{blue}{$33$} };
            \node at (8.5, 5.5) { \textcolor{black}{$52$} };
            \node at (9.5, 5.5) { \textcolor{black}{$97$} };

            \draw[->] (1.5, 6.5) node[anchor=south] { $i$ } -- (1.5, 6.25);
            \draw[->] (7.5, 6.5) node[anchor=south] { $j$ } -- (7.5, 6.25);
            \draw[->] (2.5, 1.5) node[anchor=north] { $k$ } -- (2.5, 1.75);

            \node at (0.5, 2.5) { \textcolor{blue}{$12$} };
            \node at (1.5, 2.5) { \textcolor{blue}{$20$} };
            \node at (2.5, 2.5) { \textcolor{blue}{$33$} };

        \end{tikzpicture}

    \end{figure}

\end{frame}

\begin{frame}[fragile]{Visualização da rotina de fusão}

    \begin{figure}
        \centering

        \begin{tikzpicture}
            \draw[opacity=0] (-0.7, 5) circle [radius=1pt];
            \draw (0, 5) grid (5, 6);
            \draw (6, 5) grid (10, 6);
            \draw (0, 2) grid (9, 3);

            \node at (0.5, 5.5) { \textcolor{black}{$12$} };
            \node at (1.5, 5.5) { \textcolor{blue}{$37$} };
            \node at (2.5, 5.5) { \textcolor{black}{$45$} };
            \node at (3.5, 5.5) { \textcolor{black}{$60$} };
            \node at (4.5, 5.5) { \textcolor{black}{$89$} };

            \node at (6.5, 5.5) { \textcolor{black}{$20$} };
            \node at (7.5, 5.5) { \textcolor{black}{$33$} };
            \node at (8.5, 5.5) { \textcolor{red}{$52$} };
            \node at (9.5, 5.5) { \textcolor{black}{$97$} };

            \draw[->] (1.5, 6.5) node[anchor=south] { $i$ } -- (1.5, 6.25);
            \draw[->] (8.5, 6.5) node[anchor=south] { $j$ } -- (8.5, 6.25);
            \draw[->] (3.5, 1.5) node[anchor=north] { $k$ } -- (3.5, 1.75);

            \node at (0.5, 2.5) { \textcolor{blue}{$12$} };
            \node at (1.5, 2.5) { \textcolor{blue}{$20$} };
            \node at (2.5, 2.5) { \textcolor{blue}{$33$} };
            \node at (3.5, 2.5) { \textcolor{blue}{$37$} };

        \end{tikzpicture}

    \end{figure}

\end{frame}

\begin{frame}[fragile]{Visualização da rotina de fusão}

    \begin{figure}
        \centering

        \begin{tikzpicture}
            \draw[opacity=0] (-0.7, 5) circle [radius=1pt];
            \draw (0, 5) grid (5, 6);
            \draw (6, 5) grid (10, 6);
            \draw (0, 2) grid (9, 3);

            \node at (0.5, 5.5) { \textcolor{black}{$12$} };
            \node at (1.5, 5.5) { \textcolor{black}{$37$} };
            \node at (2.5, 5.5) { \textcolor{blue}{$45$} };
            \node at (3.5, 5.5) { \textcolor{black}{$60$} };
            \node at (4.5, 5.5) { \textcolor{black}{$89$} };

            \node at (6.5, 5.5) { \textcolor{black}{$20$} };
            \node at (7.5, 5.5) { \textcolor{black}{$33$} };
            \node at (8.5, 5.5) { \textcolor{red}{$52$} };
            \node at (9.5, 5.5) { \textcolor{black}{$97$} };

            \draw[->] (2.5, 6.5) node[anchor=south] { $i$ } -- (2.5, 6.25);
            \draw[->] (8.5, 6.5) node[anchor=south] { $j$ } -- (8.5, 6.25);
            \draw[->] (4.5, 1.5) node[anchor=north] { $k$ } -- (4.5, 1.75);

            \node at (0.5, 2.5) { \textcolor{blue}{$12$} };
            \node at (1.5, 2.5) { \textcolor{blue}{$20$} };
            \node at (2.5, 2.5) { \textcolor{blue}{$33$} };
            \node at (3.5, 2.5) { \textcolor{blue}{$37$} };
            \node at (4.5, 2.5) { \textcolor{blue}{$45$} };

        \end{tikzpicture}

    \end{figure}

\end{frame}

\begin{frame}[fragile]{Visualização da rotina de fusão}

    \begin{figure}
        \centering

        \begin{tikzpicture}
            \draw[opacity=0] (-0.7, 5) circle [radius=1pt];
            \draw (0, 5) grid (5, 6);
            \draw (6, 5) grid (10, 6);
            \draw (0, 2) grid (9, 3);

            \node at (0.5, 5.5) { \textcolor{black}{$12$} };
            \node at (1.5, 5.5) { \textcolor{black}{$37$} };
            \node at (2.5, 5.5) { \textcolor{black}{$45$} };
            \node at (3.5, 5.5) { \textcolor{red}{$60$} };
            \node at (4.5, 5.5) { \textcolor{black}{$89$} };

            \node at (6.5, 5.5) { \textcolor{black}{$20$} };
            \node at (7.5, 5.5) { \textcolor{black}{$33$} };
            \node at (8.5, 5.5) { \textcolor{blue}{$52$} };
            \node at (9.5, 5.5) { \textcolor{black}{$97$} };

            \draw[->] (3.5, 6.5) node[anchor=south] { $i$ } -- (3.5, 6.25);
            \draw[->] (8.5, 6.5) node[anchor=south] { $j$ } -- (8.5, 6.25);
            \draw[->] (5.5, 1.5) node[anchor=north] { $k$ } -- (5.5, 1.75);

            \node at (0.5, 2.5) { \textcolor{blue}{$12$} };
            \node at (1.5, 2.5) { \textcolor{blue}{$20$} };
            \node at (2.5, 2.5) { \textcolor{blue}{$33$} };
            \node at (3.5, 2.5) { \textcolor{blue}{$37$} };
            \node at (4.5, 2.5) { \textcolor{blue}{$45$} };
            \node at (5.5, 2.5) { \textcolor{blue}{$52$} };

        \end{tikzpicture}

    \end{figure}

\end{frame}

\begin{frame}[fragile]{Visualização da rotina de fusão}

    \begin{figure}
        \centering

        \begin{tikzpicture}
            \draw[opacity=0] (-0.7, 5) circle [radius=1pt];
            \draw (0, 5) grid (5, 6);
            \draw (6, 5) grid (10, 6);
            \draw (0, 2) grid (9, 3);

            \node at (0.5, 5.5) { \textcolor{black}{$12$} };
            \node at (1.5, 5.5) { \textcolor{black}{$37$} };
            \node at (2.5, 5.5) { \textcolor{black}{$45$} };
            \node at (3.5, 5.5) { \textcolor{blue}{$60$} };
            \node at (4.5, 5.5) { \textcolor{black}{$89$} };

            \node at (6.5, 5.5) { \textcolor{black}{$20$} };
            \node at (7.5, 5.5) { \textcolor{black}{$33$} };
            \node at (8.5, 5.5) { \textcolor{black}{$52$} };
            \node at (9.5, 5.5) { \textcolor{red}{$97$} };

            \draw[->] (3.5, 6.5) node[anchor=south] { $i$ } -- (3.5, 6.25);
            \draw[->] (9.5, 6.5) node[anchor=south] { $j$ } -- (9.5, 6.25);
            \draw[->] (6.5, 1.5) node[anchor=north] { $k$ } -- (6.5, 1.75);

            \node at (0.5, 2.5) { \textcolor{blue}{$12$} };
            \node at (1.5, 2.5) { \textcolor{blue}{$20$} };
            \node at (2.5, 2.5) { \textcolor{blue}{$33$} };
            \node at (3.5, 2.5) { \textcolor{blue}{$37$} };
            \node at (4.5, 2.5) { \textcolor{blue}{$45$} };
            \node at (5.5, 2.5) { \textcolor{blue}{$52$} };
            \node at (6.5, 2.5) { \textcolor{blue}{$60$} };

        \end{tikzpicture}

    \end{figure}

\end{frame}

\begin{frame}[fragile]{Visualização da rotina de fusão}

    \begin{figure}
        \centering

        \begin{tikzpicture}
            \draw[opacity=0] (-0.7, 5) circle [radius=1pt];
            \draw (0, 5) grid (5, 6);
            \draw (6, 5) grid (10, 6);
            \draw (0, 2) grid (9, 3);

            \node at (0.5, 5.5) { \textcolor{black}{$12$} };
            \node at (1.5, 5.5) { \textcolor{black}{$37$} };
            \node at (2.5, 5.5) { \textcolor{black}{$45$} };
            \node at (3.5, 5.5) { \textcolor{black}{$60$} };
            \node at (4.5, 5.5) { \textcolor{blue}{$89$} };

            \node at (6.5, 5.5) { \textcolor{black}{$20$} };
            \node at (7.5, 5.5) { \textcolor{black}{$33$} };
            \node at (8.5, 5.5) { \textcolor{black}{$52$} };
            \node at (9.5, 5.5) { \textcolor{red}{$97$} };

            \draw[->] (4.5, 6.5) node[anchor=south] { $i$ } -- (4.5, 6.25);
            \draw[->] (9.5, 6.5) node[anchor=south] { $j$ } -- (9.5, 6.25);
            \draw[->] (7.5, 1.5) node[anchor=north] { $k$ } -- (7.5, 1.75);

            \node at (0.5, 2.5) { \textcolor{blue}{$12$} };
            \node at (1.5, 2.5) { \textcolor{blue}{$20$} };
            \node at (2.5, 2.5) { \textcolor{blue}{$33$} };
            \node at (3.5, 2.5) { \textcolor{blue}{$37$} };
            \node at (4.5, 2.5) { \textcolor{blue}{$45$} };
            \node at (5.5, 2.5) { \textcolor{blue}{$52$} };
            \node at (6.5, 2.5) { \textcolor{blue}{$60$} };
            \node at (7.5, 2.5) { \textcolor{blue}{$89$} };

        \end{tikzpicture}

    \end{figure}

\end{frame}

\begin{frame}[fragile]{Visualização da rotina de fusão}

    \begin{figure}
        \centering

        \begin{tikzpicture}
            \draw[opacity=0] (-0.7, 5) circle [radius=1pt];
            \draw (0, 5) grid (5, 6);
            \draw (6, 5) grid (10, 6);
            \draw (0, 2) grid (9, 3);

            \node at (0.5, 5.5) { \textcolor{black}{$12$} };
            \node at (1.5, 5.5) { \textcolor{black}{$37$} };
            \node at (2.5, 5.5) { \textcolor{black}{$45$} };
            \node at (3.5, 5.5) { \textcolor{black}{$60$} };
            \node at (4.5, 5.5) { \textcolor{black}{$89$} };

            \node at (6.5, 5.5) { \textcolor{black}{$20$} };
            \node at (7.5, 5.5) { \textcolor{black}{$33$} };
            \node at (8.5, 5.5) { \textcolor{black}{$52$} };
            \node at (9.5, 5.5) { \textcolor{blue}{$97$} };

            \draw[->] (5.5, 6.5) node[anchor=south] { $i$ } -- (5.5, 6.25);
            \draw[->] (9.5, 6.5) node[anchor=south] { $j$ } -- (9.5, 6.25);
            \draw[->] (8.5, 1.5) node[anchor=north] { $k$ } -- (8.5, 1.75);

            \node at (0.5, 2.5) { \textcolor{blue}{$12$} };
            \node at (1.5, 2.5) { \textcolor{blue}{$20$} };
            \node at (2.5, 2.5) { \textcolor{blue}{$33$} };
            \node at (3.5, 2.5) { \textcolor{blue}{$37$} };
            \node at (4.5, 2.5) { \textcolor{blue}{$45$} };
            \node at (5.5, 2.5) { \textcolor{blue}{$52$} };
            \node at (6.5, 2.5) { \textcolor{blue}{$60$} };
            \node at (7.5, 2.5) { \textcolor{blue}{$89$} };
            \node at (8.5, 2.5) { \textcolor{blue}{$97$} };

        \end{tikzpicture}

    \end{figure}

\end{frame}


\begin{frame}[fragile]{Implementação da rotina de fusão}
    \inputsnippet{cpp}{5}{25}{codes/mergesort.cpp}
\end{frame}

\begin{frame}[fragile]{Visualização do \code{c}{mergesort}}

    \begin{figure}
        \centering

        \begin{tikzpicture}
        \begin{scope}[scale=0.8]
            \draw (0, 5) grid (9, 6);

%            \draw (-1, 3) grid (4, 4);
%            \draw (6, 3) grid (10, 4);
%
%            \foreach \x in {-1,...,1}:
%                \draw ({\x - 0.5}, 1) rectangle ({\x + 0.5}, 2);
%
%            \foreach \x in {3,...,4}:
%                \draw ({\x - 0.5}, 1) rectangle ({\x + 0.5}, 2);
%
%            \foreach \x in {6,...,7}:
%                \draw ({\x - 0.5}, 1) rectangle ({\x + 0.5}, 2);
%
%            \foreach \x in {9,...,10}:
%                \draw ({\x - 0.5}, 1) rectangle ({\x + 0.5}, 2);
%
%            \draw (-2, -1) grid (0, 0);
            \draw[opacity=0] (-2.25, -3) rectangle (-1.25, -2);
%            \draw (-0.75, -3) rectangle (0.25, -2);
%
%            \draw (0.5, -1) rectangle (1.5, 0);
%
%            \draw (2.25, -1) rectangle (3.25, 0);
%            \draw (3.75, -1) rectangle (4.75, 0);
%
%            \draw (5.25, -1) rectangle (6.25, 0);
%            \draw (6.75, -1) rectangle (7.75, 0);
%
%            \draw (8.25, -1) rectangle (9.25, 0);
            \draw[opacity=0] (9.75, -1) rectangle (10.75, 0);

            \node at (0.5, 5.5) { \textcolor{black}{$89$} };
            \node at (1.5, 5.5) { \textcolor{black}{$60$} };
            \node at (2.5, 5.5) { \textcolor{black}{$12$} };
            \node at (3.5, 5.5) { \textcolor{black}{$45$} };
            \node at (4.5, 5.5) { \textcolor{black}{$37$} };
            \node at (5.5, 5.5) { \textcolor{black}{$52$} };
            \node at (6.5, 5.5) { \textcolor{black}{$33$} };
            \node at (7.5, 5.5) { \textcolor{black}{$97$} };
            \node at (8.5, 5.5) { \textcolor{black}{$20$} };

        \end{scope}
        \end{tikzpicture}

    \end{figure}

\end{frame}

\begin{frame}[fragile]{Visualização do \code{c}{mergesort}}

    \begin{figure}
        \centering

        \begin{tikzpicture}
        \begin{scope}[scale=0.8]
            \draw (0, 5) grid (9, 6);

            \draw (-1, 3) grid (4, 4);
%            \draw (6, 3) grid (10, 4);
%
%            \foreach \x in {-1,...,1}:
%                \draw ({\x - 0.5}, 1) rectangle ({\x + 0.5}, 2);
%
%            \foreach \x in {3,...,4}:
%                \draw ({\x - 0.5}, 1) rectangle ({\x + 0.5}, 2);
%
%            \foreach \x in {6,...,7}:
%                \draw ({\x - 0.5}, 1) rectangle ({\x + 0.5}, 2);
%
%            \foreach \x in {9,...,10}:
%                \draw ({\x - 0.5}, 1) rectangle ({\x + 0.5}, 2);
%
%            \draw (-2, -1) grid (0, 0);
            \draw[opacity=0] (-2.25, -3) rectangle (-1.25, -2);
%            \draw (-0.75, -3) rectangle (0.25, -2);
%
%            \draw (0.5, -1) rectangle (1.5, 0);
%
%            \draw (2.25, -1) rectangle (3.25, 0);
%            \draw (3.75, -1) rectangle (4.75, 0);
%
%            \draw (5.25, -1) rectangle (6.25, 0);
%            \draw (6.75, -1) rectangle (7.75, 0);
%
%            \draw (8.25, -1) rectangle (9.25, 0);
            \draw[opacity=0] (9.75, -1) rectangle (10.75, 0);

            \node at (0.5, 5.5) { \textcolor{black}{$89$} };
            \node at (1.5, 5.5) { \textcolor{black}{$60$} };
            \node at (2.5, 5.5) { \textcolor{black}{$12$} };
            \node at (3.5, 5.5) { \textcolor{black}{$45$} };
            \node at (4.5, 5.5) { \textcolor{black}{$37$} };
            \node at (5.5, 5.5) { \textcolor{black}{$52$} };
            \node at (6.5, 5.5) { \textcolor{black}{$33$} };
            \node at (7.5, 5.5) { \textcolor{black}{$97$} };
            \node at (8.5, 5.5) { \textcolor{black}{$20$} };

            \node at (-0.5, 3.5) { \textcolor{green!60!black}{$89$} };
            \node at (0.5, 3.5) { \textcolor{green!60!black}{$60$} };
            \node at (1.5, 3.5) { \textcolor{green!60!black}{$12$} };
            \node at (2.5, 3.5) { \textcolor{green!60!black}{$45$} };
            \node at (3.5, 3.5) { \textcolor{green!60!black}{$37$} };
 
        \end{scope}
        \end{tikzpicture}

    \end{figure}

\end{frame}

\begin{frame}[fragile]{Visualização do \code{c}{mergesort}}

    \begin{figure}
        \centering

        \begin{tikzpicture}
        \begin{scope}[scale=0.8]
            \draw (0, 5) grid (9, 6);

            \draw (-1, 3) grid (4, 4);
%            \draw (6, 3) grid (10, 4);
%
            \foreach \x in {-1,...,1}:
                \draw ({\x - 0.5}, 1) rectangle ({\x + 0.5}, 2);
%
%            \foreach \x in {3,...,4}:
%                \draw ({\x - 0.5}, 1) rectangle ({\x + 0.5}, 2);
%
%            \foreach \x in {6,...,7}:
%                \draw ({\x - 0.5}, 1) rectangle ({\x + 0.5}, 2);
%
%            \foreach \x in {9,...,10}:
%                \draw ({\x - 0.5}, 1) rectangle ({\x + 0.5}, 2);
%
%            \draw (-2, -1) grid (0, 0);
            \draw[opacity=0] (-2.25, -3) rectangle (-1.25, -2);
%            \draw (-0.75, -3) rectangle (0.25, -2);
%
%            \draw (0.5, -1) rectangle (1.5, 0);
%
%            \draw (2.25, -1) rectangle (3.25, 0);
%            \draw (3.75, -1) rectangle (4.75, 0);
%
%            \draw (5.25, -1) rectangle (6.25, 0);
%            \draw (6.75, -1) rectangle (7.75, 0);
%
%            \draw (8.25, -1) rectangle (9.25, 0);
            \draw[opacity=0] (9.75, -1) rectangle (10.75, 0);

            \node at (0.5, 5.5) { \textcolor{black}{$89$} };
            \node at (1.5, 5.5) { \textcolor{black}{$60$} };
            \node at (2.5, 5.5) { \textcolor{black}{$12$} };
            \node at (3.5, 5.5) { \textcolor{black}{$45$} };
            \node at (4.5, 5.5) { \textcolor{black}{$37$} };
            \node at (5.5, 5.5) { \textcolor{black}{$52$} };
            \node at (6.5, 5.5) { \textcolor{black}{$33$} };
            \node at (7.5, 5.5) { \textcolor{black}{$97$} };
            \node at (8.5, 5.5) { \textcolor{black}{$20$} };

            \node at (-0.5, 3.5) { \textcolor{black}{$89$} };
            \node at (0.5, 3.5) { \textcolor{black}{$60$} };
            \node at (1.5, 3.5) { \textcolor{black}{$12$} };
            \node at (2.5, 3.5) { \textcolor{black}{$45$} };
            \node at (3.5, 3.5) { \textcolor{black}{$37$} };
 
            \node at (-1, 1.5) { \textcolor{green!60!black}{$89$} };
            \node at (0, 1.5) { \textcolor{green!60!black}{$60$} };
            \node at (1, 1.5) { \textcolor{green!60!black}{$12$} };
 
        \end{scope}
        \end{tikzpicture}

    \end{figure}

\end{frame}

\begin{frame}[fragile]{Visualização do \code{c}{mergesort}}

    \begin{figure}
        \centering

        \begin{tikzpicture}
        \begin{scope}[scale=0.8]
            \draw (0, 5) grid (9, 6);

            \draw (-1, 3) grid (4, 4);
%            \draw (6, 3) grid (10, 4);
%
            \foreach \x in {-1,...,1}:
                \draw ({\x - 0.5}, 1) rectangle ({\x + 0.5}, 2);
%
%            \foreach \x in {3,...,4}:
%                \draw ({\x - 0.5}, 1) rectangle ({\x + 0.5}, 2);
%
%            \foreach \x in {6,...,7}:
%                \draw ({\x - 0.5}, 1) rectangle ({\x + 0.5}, 2);
%
%            \foreach \x in {9,...,10}:
%                \draw ({\x - 0.5}, 1) rectangle ({\x + 0.5}, 2);
%
            \draw (-2, -1) grid (0, 0);

            \draw[opacity=0] (-2.25, -3) rectangle (-1.25, -2);
%            \draw (-0.75, -3) rectangle (0.25, -2);
%
%            \draw (0.5, -1) rectangle (1.5, 0);
%
%            \draw (2.25, -1) rectangle (3.25, 0);
%            \draw (3.75, -1) rectangle (4.75, 0);
%
%            \draw (5.25, -1) rectangle (6.25, 0);
%            \draw (6.75, -1) rectangle (7.75, 0);
%
%            \draw (8.25, -1) rectangle (9.25, 0);
            \draw[opacity=0] (9.75, -1) rectangle (10.75, 0);

            \node at (0.5, 5.5) { \textcolor{black}{$89$} };
            \node at (1.5, 5.5) { \textcolor{black}{$60$} };
            \node at (2.5, 5.5) { \textcolor{black}{$12$} };
            \node at (3.5, 5.5) { \textcolor{black}{$45$} };
            \node at (4.5, 5.5) { \textcolor{black}{$37$} };
            \node at (5.5, 5.5) { \textcolor{black}{$52$} };
            \node at (6.5, 5.5) { \textcolor{black}{$33$} };
            \node at (7.5, 5.5) { \textcolor{black}{$97$} };
            \node at (8.5, 5.5) { \textcolor{black}{$20$} };

            \node at (-0.5, 3.5) { \textcolor{black}{$89$} };
            \node at (0.5, 3.5) { \textcolor{black}{$60$} };
            \node at (1.5, 3.5) { \textcolor{black}{$12$} };
            \node at (2.5, 3.5) { \textcolor{black}{$45$} };
            \node at (3.5, 3.5) { \textcolor{black}{$37$} };
 
            \node at (-1, 1.5) { \textcolor{black}{$89$} };
            \node at (0, 1.5) { \textcolor{black}{$60$} };
            \node at (1, 1.5) { \textcolor{black}{$12$} };
 
            \node at (-1.5, -0.5) { \textcolor{green!60!black}{$89$} };
            \node at (-0.5, -0.5) { \textcolor{green!60!black}{$60$} };
            %\node at (1.0, -0.5) { \textcolor{blue}{$12$} };
 
        \end{scope}
        \end{tikzpicture}

    \end{figure}

\end{frame}

\begin{frame}[fragile]{Visualização do \code{c}{mergesort}}

    \begin{figure}
        \centering

        \begin{tikzpicture}
        \begin{scope}[scale=0.8]
            \draw (0, 5) grid (9, 6);

            \draw (-1, 3) grid (4, 4);
%            \draw (6, 3) grid (10, 4);
%
            \foreach \x in {-1,...,1}:
                \draw ({\x - 0.5}, 1) rectangle ({\x + 0.5}, 2);
%
%            \foreach \x in {3,...,4}:
%                \draw ({\x - 0.5}, 1) rectangle ({\x + 0.5}, 2);
%
%            \foreach \x in {6,...,7}:
%                \draw ({\x - 0.5}, 1) rectangle ({\x + 0.5}, 2);
%
%            \foreach \x in {9,...,10}:
%                \draw ({\x - 0.5}, 1) rectangle ({\x + 0.5}, 2);
%
            \draw (-2, -1) grid (0, 0);

            \draw[opacity=1] (-2.25, -3) rectangle (-1.25, -2);
            \draw (-0.75, -3) rectangle (0.25, -2);
%
%            \draw (0.5, -1) rectangle (1.5, 0);
%
%            \draw (2.25, -1) rectangle (3.25, 0);
%            \draw (3.75, -1) rectangle (4.75, 0);
%
%            \draw (5.25, -1) rectangle (6.25, 0);
%            \draw (6.75, -1) rectangle (7.75, 0);
%
%            \draw (8.25, -1) rectangle (9.25, 0);
            \draw[opacity=0] (9.75, -1) rectangle (10.75, 0);

            \node at (0.5, 5.5) { \textcolor{black}{$89$} };
            \node at (1.5, 5.5) { \textcolor{black}{$60$} };
            \node at (2.5, 5.5) { \textcolor{black}{$12$} };
            \node at (3.5, 5.5) { \textcolor{black}{$45$} };
            \node at (4.5, 5.5) { \textcolor{black}{$37$} };
            \node at (5.5, 5.5) { \textcolor{black}{$52$} };
            \node at (6.5, 5.5) { \textcolor{black}{$33$} };
            \node at (7.5, 5.5) { \textcolor{black}{$97$} };
            \node at (8.5, 5.5) { \textcolor{black}{$20$} };

            \node at (-0.5, 3.5) { \textcolor{black}{$89$} };
            \node at (0.5, 3.5) { \textcolor{black}{$60$} };
            \node at (1.5, 3.5) { \textcolor{black}{$12$} };
            \node at (2.5, 3.5) { \textcolor{black}{$45$} };
            \node at (3.5, 3.5) { \textcolor{black}{$37$} };
 
            \node at (-1, 1.5) { \textcolor{black}{$89$} };
            \node at (0, 1.5) { \textcolor{black}{$60$} };
            \node at (1, 1.5) { \textcolor{black}{$12$} };
 
            \node at (-1.5, -0.5) { \textcolor{black}{$89$} };
            \node at (-0.5, -0.5) { \textcolor{black}{$60$} };
%            \node at (1.0, -0.5) { \textcolor{blue}{$12$} };
 
            \node at (-1.75, -2.5) { \textcolor{blue}{$89$} };
            \node at (-0.25, -2.5) { \textcolor{blue}{$60$} };
 
        \end{scope}
        \end{tikzpicture}

    \end{figure}

\end{frame}

\begin{frame}[fragile]{Visualização do \code{c}{mergesort}}

    \begin{figure}
        \centering

        \begin{tikzpicture}
        \begin{scope}[scale=0.8]
            \draw (0, 5) grid (9, 6);

            \draw (-1, 3) grid (4, 4);
%            \draw (6, 3) grid (10, 4);
%
            \foreach \x in {-1,...,1}:
                \draw ({\x - 0.5}, 1) rectangle ({\x + 0.5}, 2);
%
%            \foreach \x in {3,...,4}:
%                \draw ({\x - 0.5}, 1) rectangle ({\x + 0.5}, 2);
%
%            \foreach \x in {6,...,7}:
%                \draw ({\x - 0.5}, 1) rectangle ({\x + 0.5}, 2);
%
%            \foreach \x in {9,...,10}:
%                \draw ({\x - 0.5}, 1) rectangle ({\x + 0.5}, 2);
%
            \draw (-2, -1) grid (0, 0);

            \draw[opacity=1] (-2.25, -3) rectangle (-1.25, -2);
            \draw (-0.75, -3) rectangle (0.25, -2);
%
%            \draw (0.5, -1) rectangle (1.5, 0);
%
%            \draw (2.25, -1) rectangle (3.25, 0);
%            \draw (3.75, -1) rectangle (4.75, 0);
%
%            \draw (5.25, -1) rectangle (6.25, 0);
%            \draw (6.75, -1) rectangle (7.75, 0);
%
%            \draw (8.25, -1) rectangle (9.25, 0);
            \draw[opacity=0] (9.75, -1) rectangle (10.75, 0);

            \node at (0.5, 5.5) { \textcolor{black}{$89$} };
            \node at (1.5, 5.5) { \textcolor{black}{$60$} };
            \node at (2.5, 5.5) { \textcolor{black}{$12$} };
            \node at (3.5, 5.5) { \textcolor{black}{$45$} };
            \node at (4.5, 5.5) { \textcolor{black}{$37$} };
            \node at (5.5, 5.5) { \textcolor{black}{$52$} };
            \node at (6.5, 5.5) { \textcolor{black}{$33$} };
            \node at (7.5, 5.5) { \textcolor{black}{$97$} };
            \node at (8.5, 5.5) { \textcolor{black}{$20$} };

            \node at (-0.5, 3.5) { \textcolor{black}{$89$} };
            \node at (0.5, 3.5) { \textcolor{black}{$60$} };
            \node at (1.5, 3.5) { \textcolor{black}{$12$} };
            \node at (2.5, 3.5) { \textcolor{black}{$45$} };
            \node at (3.5, 3.5) { \textcolor{black}{$37$} };
 
            \node at (-1, 1.5) { \textcolor{black}{$89$} };
            \node at (0, 1.5) { \textcolor{black}{$60$} };
            \node at (1, 1.5) { \textcolor{black}{$12$} };
 
            \node at (-1.5, -0.5) { \textcolor{blue}{$60$} };
            \node at (-0.5, -0.5) { \textcolor{blue}{$89$} };
            %\node at (1.0, -0.5) { \textcolor{blue}{$12$} };
 
            \node at (-1.75, -2.5) { \textcolor{red}{$89$} };
            \node at (-0.25, -2.5) { \textcolor{red}{$60$} };
 
        \end{scope}
        \end{tikzpicture}

    \end{figure}

\end{frame}

\begin{frame}[fragile]{Visualização do \code{c}{mergesort}}

    \begin{figure}
        \centering

        \begin{tikzpicture}
        \begin{scope}[scale=0.8]
            \draw (0, 5) grid (9, 6);

            \draw (-1, 3) grid (4, 4);
%            \draw (6, 3) grid (10, 4);
%
            \foreach \x in {-1,...,1}:
                \draw ({\x - 0.5}, 1) rectangle ({\x + 0.5}, 2);
%
%            \foreach \x in {3,...,4}:
%                \draw ({\x - 0.5}, 1) rectangle ({\x + 0.5}, 2);
%
%            \foreach \x in {6,...,7}:
%                \draw ({\x - 0.5}, 1) rectangle ({\x + 0.5}, 2);
%
%            \foreach \x in {9,...,10}:
%                \draw ({\x - 0.5}, 1) rectangle ({\x + 0.5}, 2);
%
            \draw (-2, -1) grid (0, 0);

            \draw[opacity=0] (-2.25, -3) rectangle (-1.25, -2);
%
%            \draw (0.5, -1) rectangle (1.5, 0);
%
%            \draw (2.25, -1) rectangle (3.25, 0);
%            \draw (3.75, -1) rectangle (4.75, 0);
%
%            \draw (5.25, -1) rectangle (6.25, 0);
%            \draw (6.75, -1) rectangle (7.75, 0);
%
%            \draw (8.25, -1) rectangle (9.25, 0);
            \draw[opacity=0] (9.75, -1) rectangle (10.75, 0);

            \node at (0.5, 5.5) { \textcolor{black}{$89$} };
            \node at (1.5, 5.5) { \textcolor{black}{$60$} };
            \node at (2.5, 5.5) { \textcolor{black}{$12$} };
            \node at (3.5, 5.5) { \textcolor{black}{$45$} };
            \node at (4.5, 5.5) { \textcolor{black}{$37$} };
            \node at (5.5, 5.5) { \textcolor{black}{$52$} };
            \node at (6.5, 5.5) { \textcolor{black}{$33$} };
            \node at (7.5, 5.5) { \textcolor{black}{$97$} };
            \node at (8.5, 5.5) { \textcolor{black}{$20$} };

            \node at (-0.5, 3.5) { \textcolor{black}{$89$} };
            \node at (0.5, 3.5) { \textcolor{black}{$60$} };
            \node at (1.5, 3.5) { \textcolor{black}{$12$} };
            \node at (2.5, 3.5) { \textcolor{black}{$45$} };
            \node at (3.5, 3.5) { \textcolor{black}{$37$} };
 
            \node at (-1, 1.5) { \textcolor{black}{$89$} };
            \node at (0, 1.5) { \textcolor{black}{$60$} };
            \node at (1, 1.5) { \textcolor{black}{$12$} };
 
            \node at (-1.5, -0.5) { \textcolor{blue}{$60$} };
            \node at (-0.5, -0.5) { \textcolor{blue}{$89$} };
%            \node at (1.0, -0.5) { \textcolor{blue}{$12$} };
 
        \end{scope}
        \end{tikzpicture}

    \end{figure}

\end{frame}

\begin{frame}[fragile]{Visualização do \code{c}{mergesort}}

    \begin{figure}
        \centering

        \begin{tikzpicture}
        \begin{scope}[scale=0.8]
            \draw (0, 5) grid (9, 6);

            \draw (-1, 3) grid (4, 4);
%            \draw (6, 3) grid (10, 4);
%
            \foreach \x in {-1,...,1}:
                \draw ({\x - 0.5}, 1) rectangle ({\x + 0.5}, 2);
%
%            \foreach \x in {3,...,4}:
%                \draw ({\x - 0.5}, 1) rectangle ({\x + 0.5}, 2);
%
%            \foreach \x in {6,...,7}:
%                \draw ({\x - 0.5}, 1) rectangle ({\x + 0.5}, 2);
%
%            \foreach \x in {9,...,10}:
%                \draw ({\x - 0.5}, 1) rectangle ({\x + 0.5}, 2);
%
            \draw (-2, -1) grid (0, 0);

            \draw[opacity=0] (-2.25, -3) rectangle (-1.25, -2);
%
            \draw (0.5, -1) rectangle (1.5, 0);
%
%            \draw (2.25, -1) rectangle (3.25, 0);
%            \draw (3.75, -1) rectangle (4.75, 0);
%
%            \draw (5.25, -1) rectangle (6.25, 0);
%            \draw (6.75, -1) rectangle (7.75, 0);
%
%            \draw (8.25, -1) rectangle (9.25, 0);
            \draw[opacity=0] (9.75, -1) rectangle (10.75, 0);

            \node at (0.5, 5.5) { \textcolor{black}{$89$} };
            \node at (1.5, 5.5) { \textcolor{black}{$60$} };
            \node at (2.5, 5.5) { \textcolor{black}{$12$} };
            \node at (3.5, 5.5) { \textcolor{black}{$45$} };
            \node at (4.5, 5.5) { \textcolor{black}{$37$} };
            \node at (5.5, 5.5) { \textcolor{black}{$52$} };
            \node at (6.5, 5.5) { \textcolor{black}{$33$} };
            \node at (7.5, 5.5) { \textcolor{black}{$97$} };
            \node at (8.5, 5.5) { \textcolor{black}{$20$} };

            \node at (-0.5, 3.5) { \textcolor{black}{$89$} };
            \node at (0.5, 3.5) { \textcolor{black}{$60$} };
            \node at (1.5, 3.5) { \textcolor{black}{$12$} };
            \node at (2.5, 3.5) { \textcolor{black}{$45$} };
            \node at (3.5, 3.5) { \textcolor{black}{$37$} };
 
            \node at (-1, 1.5) { \textcolor{black}{$89$} };
            \node at (0, 1.5) { \textcolor{black}{$60$} };
            \node at (1, 1.5) { \textcolor{black}{$12$} };
 
            \node at (-1.5, -0.5) { \textcolor{blue}{$60$} };
            \node at (-0.5, -0.5) { \textcolor{blue}{$89$} };
            \node at (1.0, -0.5) { \textcolor{blue}{$12$} };
 
        \end{scope}
        \end{tikzpicture}

    \end{figure}

\end{frame}

\begin{frame}[fragile]{Visualização do \code{c}{mergesort}}

    \begin{figure}
        \centering

        \begin{tikzpicture}
        \begin{scope}[scale=0.8]
            \draw (0, 5) grid (9, 6);

            \draw (-1, 3) grid (4, 4);
%            \draw (6, 3) grid (10, 4);
%
            \foreach \x in {-1,...,1}:
                \draw ({\x - 0.5}, 1) rectangle ({\x + 0.5}, 2);
%
%            \foreach \x in {3,...,4}:
%                \draw ({\x - 0.5}, 1) rectangle ({\x + 0.5}, 2);
%
%            \foreach \x in {6,...,7}:
%                \draw ({\x - 0.5}, 1) rectangle ({\x + 0.5}, 2);
%
%            \foreach \x in {9,...,10}:
%                \draw ({\x - 0.5}, 1) rectangle ({\x + 0.5}, 2);
%
            \draw (-2, -1) grid (0, 0);

            \draw[opacity=0] (-2.25, -3) rectangle (-1.25, -2);
%
            \draw (0.5, -1) rectangle (1.5, 0);
%
%            \draw (2.25, -1) rectangle (3.25, 0);
%            \draw (3.75, -1) rectangle (4.75, 0);
%
%            \draw (5.25, -1) rectangle (6.25, 0);
%            \draw (6.75, -1) rectangle (7.75, 0);
%
%            \draw (8.25, -1) rectangle (9.25, 0);
            \draw[opacity=0] (9.75, -1) rectangle (10.75, 0);

            \node at (0.5, 5.5) { \textcolor{black}{$89$} };
            \node at (1.5, 5.5) { \textcolor{black}{$60$} };
            \node at (2.5, 5.5) { \textcolor{black}{$12$} };
            \node at (3.5, 5.5) { \textcolor{black}{$45$} };
            \node at (4.5, 5.5) { \textcolor{black}{$37$} };
            \node at (5.5, 5.5) { \textcolor{black}{$52$} };
            \node at (6.5, 5.5) { \textcolor{black}{$33$} };
            \node at (7.5, 5.5) { \textcolor{black}{$97$} };
            \node at (8.5, 5.5) { \textcolor{black}{$20$} };

            \node at (-0.5, 3.5) { \textcolor{black}{$89$} };
            \node at (0.5, 3.5) { \textcolor{black}{$60$} };
            \node at (1.5, 3.5) { \textcolor{black}{$12$} };
            \node at (2.5, 3.5) { \textcolor{black}{$45$} };
            \node at (3.5, 3.5) { \textcolor{black}{$37$} };
 
            \node at (-1, 1.5) { \textcolor{blue}{$12$} };
            \node at (0, 1.5) { \textcolor{blue}{$60$} };
            \node at (1, 1.5) { \textcolor{blue}{$89$} };
 
            \node at (-1.5, -0.5) { \textcolor{red}{$60$} };
            \node at (-0.5, -0.5) { \textcolor{red}{$89$} };
            \node at (1.0, -0.5) { \textcolor{red}{$12$} };
 
        \end{scope}
        \end{tikzpicture}

    \end{figure}

\end{frame}

\begin{frame}[fragile]{Visualização do \code{c}{mergesort}}

    \begin{figure}
        \centering

        \begin{tikzpicture}
        \begin{scope}[scale=0.8]
            \draw (0, 5) grid (9, 6);

            \draw (-1, 3) grid (4, 4);
%            \draw (6, 3) grid (10, 4);
%
            \foreach \x in {-1,...,1}:
                \draw ({\x - 0.5}, 1) rectangle ({\x + 0.5}, 2);
%
%            \foreach \x in {3,...,4}:
%                \draw ({\x - 0.5}, 1) rectangle ({\x + 0.5}, 2);
%
%            \foreach \x in {6,...,7}:
%                \draw ({\x - 0.5}, 1) rectangle ({\x + 0.5}, 2);
%
%            \foreach \x in {9,...,10}:
%                \draw ({\x - 0.5}, 1) rectangle ({\x + 0.5}, 2);
%
            \draw[opacity=0] (-2.25, -3) rectangle (-1.25, -2);
%
%            \draw (2.25, -1) rectangle (3.25, 0);
%            \draw (3.75, -1) rectangle (4.75, 0);
%
%            \draw (5.25, -1) rectangle (6.25, 0);
%            \draw (6.75, -1) rectangle (7.75, 0);
%
%            \draw (8.25, -1) rectangle (9.25, 0);
            \draw[opacity=0] (9.75, -1) rectangle (10.75, 0);

            \node at (0.5, 5.5) { \textcolor{black}{$89$} };
            \node at (1.5, 5.5) { \textcolor{black}{$60$} };
            \node at (2.5, 5.5) { \textcolor{black}{$12$} };
            \node at (3.5, 5.5) { \textcolor{black}{$45$} };
            \node at (4.5, 5.5) { \textcolor{black}{$37$} };
            \node at (5.5, 5.5) { \textcolor{black}{$52$} };
            \node at (6.5, 5.5) { \textcolor{black}{$33$} };
            \node at (7.5, 5.5) { \textcolor{black}{$97$} };
            \node at (8.5, 5.5) { \textcolor{black}{$20$} };

            \node at (-0.5, 3.5) { \textcolor{black}{$89$} };
            \node at (0.5, 3.5) { \textcolor{black}{$60$} };
            \node at (1.5, 3.5) { \textcolor{black}{$12$} };
            \node at (2.5, 3.5) { \textcolor{black}{$45$} };
            \node at (3.5, 3.5) { \textcolor{black}{$37$} };
 
            \node at (-1, 1.5) { \textcolor{blue}{$12$} };
            \node at (0, 1.5) { \textcolor{blue}{$60$} };
            \node at (1, 1.5) { \textcolor{blue}{$89$} };
 
        \end{scope}
        \end{tikzpicture}

    \end{figure}

\end{frame}

\begin{frame}[fragile]{Visualização do \code{c}{mergesort}}

    \begin{figure}
        \centering

        \begin{tikzpicture}
        \begin{scope}[scale=0.8]
            \draw (0, 5) grid (9, 6);

            \draw (-1, 3) grid (4, 4);
%            \draw (6, 3) grid (10, 4);
%
            \foreach \x in {-1,...,1}:
                \draw ({\x - 0.5}, 1) rectangle ({\x + 0.5}, 2);
%
            \foreach \x in {3,...,4}:
                \draw ({\x - 0.5}, 1) rectangle ({\x + 0.5}, 2);
%
%            \foreach \x in {6,...,7}:
%                \draw ({\x - 0.5}, 1) rectangle ({\x + 0.5}, 2);
%
%            \foreach \x in {9,...,10}:
%                \draw ({\x - 0.5}, 1) rectangle ({\x + 0.5}, 2);
%
            \draw[opacity=0] (-2.25, -3) rectangle (-1.25, -2);
%
%            \draw (2.25, -1) rectangle (3.25, 0);
%            \draw (3.75, -1) rectangle (4.75, 0);
%
%            \draw (5.25, -1) rectangle (6.25, 0);
%            \draw (6.75, -1) rectangle (7.75, 0);
%
%            \draw (8.25, -1) rectangle (9.25, 0);
            \draw[opacity=0] (9.75, -1) rectangle (10.75, 0);

            \node at (0.5, 5.5) { \textcolor{black}{$89$} };
            \node at (1.5, 5.5) { \textcolor{black}{$60$} };
            \node at (2.5, 5.5) { \textcolor{black}{$12$} };
            \node at (3.5, 5.5) { \textcolor{black}{$45$} };
            \node at (4.5, 5.5) { \textcolor{black}{$37$} };
            \node at (5.5, 5.5) { \textcolor{black}{$52$} };
            \node at (6.5, 5.5) { \textcolor{black}{$33$} };
            \node at (7.5, 5.5) { \textcolor{black}{$97$} };
            \node at (8.5, 5.5) { \textcolor{black}{$20$} };

            \node at (-0.5, 3.5) { \textcolor{black}{$89$} };
            \node at (0.5, 3.5) { \textcolor{black}{$60$} };
            \node at (1.5, 3.5) { \textcolor{black}{$12$} };
            \node at (2.5, 3.5) { \textcolor{black}{$45$} };
            \node at (3.5, 3.5) { \textcolor{black}{$37$} };
 
            \node at (-1, 1.5) { \textcolor{blue}{$12$} };
            \node at (0, 1.5) { \textcolor{blue}{$60$} };
            \node at (1, 1.5) { \textcolor{blue}{$89$} };
 
            \node at (3, 1.5) { \textcolor{green!60!black}{$45$} };
            \node at (4, 1.5) { \textcolor{green!60!black}{$37$} };
 
        \end{scope}
        \end{tikzpicture}

    \end{figure}

\end{frame}

\begin{frame}[fragile]{Visualização do \code{c}{mergesort}}

    \begin{figure}
        \centering

        \begin{tikzpicture}
        \begin{scope}[scale=0.8]
            \draw (0, 5) grid (9, 6);

            \draw (-1, 3) grid (4, 4);
%            \draw (6, 3) grid (10, 4);
%
            \foreach \x in {-1,...,1}:
                \draw ({\x - 0.5}, 1) rectangle ({\x + 0.5}, 2);
%
            \foreach \x in {3,...,4}:
                \draw ({\x - 0.5}, 1) rectangle ({\x + 0.5}, 2);
%
%            \foreach \x in {6,...,7}:
%                \draw ({\x - 0.5}, 1) rectangle ({\x + 0.5}, 2);
%
%            \foreach \x in {9,...,10}:
%                \draw ({\x - 0.5}, 1) rectangle ({\x + 0.5}, 2);
%
            \draw[opacity=0] (-2.25, -3) rectangle (-1.25, -2);
%
            \draw (2.25, -1) rectangle (3.25, 0);
            \draw (3.75, -1) rectangle (4.75, 0);
%
%            \draw (5.25, -1) rectangle (6.25, 0);
%            \draw (6.75, -1) rectangle (7.75, 0);
%
%            \draw (8.25, -1) rectangle (9.25, 0);
            \draw[opacity=0] (9.75, -1) rectangle (10.75, 0);

            \node at (0.5, 5.5) { \textcolor{black}{$89$} };
            \node at (1.5, 5.5) { \textcolor{black}{$60$} };
            \node at (2.5, 5.5) { \textcolor{black}{$12$} };
            \node at (3.5, 5.5) { \textcolor{black}{$45$} };
            \node at (4.5, 5.5) { \textcolor{black}{$37$} };
            \node at (5.5, 5.5) { \textcolor{black}{$52$} };
            \node at (6.5, 5.5) { \textcolor{black}{$33$} };
            \node at (7.5, 5.5) { \textcolor{black}{$97$} };
            \node at (8.5, 5.5) { \textcolor{black}{$20$} };

            \node at (-0.5, 3.5) { \textcolor{black}{$89$} };
            \node at (0.5, 3.5) { \textcolor{black}{$60$} };
            \node at (1.5, 3.5) { \textcolor{black}{$12$} };
            \node at (2.5, 3.5) { \textcolor{black}{$45$} };
            \node at (3.5, 3.5) { \textcolor{black}{$37$} };
 
            \node at (-1, 1.5) { \textcolor{blue}{$12$} };
            \node at (0, 1.5) { \textcolor{blue}{$60$} };
            \node at (1, 1.5) { \textcolor{blue}{$89$} };
 
            \node at (3, 1.5) { \textcolor{black}{$45$} };
            \node at (4, 1.5) { \textcolor{black}{$37$} };
 
            \node at (2.75, -0.5) { \textcolor{blue}{$45$} };
            \node at (4.25, -0.5) { \textcolor{blue}{$37$} };
 
        \end{scope}
        \end{tikzpicture}

    \end{figure}

\end{frame}

\begin{frame}[fragile]{Visualização do \code{c}{mergesort}}

    \begin{figure}
        \centering

        \begin{tikzpicture}
        \begin{scope}[scale=0.8]
            \draw (0, 5) grid (9, 6);

            \draw (-1, 3) grid (4, 4);
%            \draw (6, 3) grid (10, 4);
%
            \foreach \x in {-1,...,1}:
                \draw ({\x - 0.5}, 1) rectangle ({\x + 0.5}, 2);
%
            \foreach \x in {3,...,4}:
                \draw ({\x - 0.5}, 1) rectangle ({\x + 0.5}, 2);
%
%            \foreach \x in {6,...,7}:
%                \draw ({\x - 0.5}, 1) rectangle ({\x + 0.5}, 2);
%
%            \foreach \x in {9,...,10}:
%                \draw ({\x - 0.5}, 1) rectangle ({\x + 0.5}, 2);
%
            \draw[opacity=0] (-2.25, -3) rectangle (-1.25, -2);
%
            \draw (2.25, -1) rectangle (3.25, 0);
            \draw (3.75, -1) rectangle (4.75, 0);
%
%            \draw (5.25, -1) rectangle (6.25, 0);
%            \draw (6.75, -1) rectangle (7.75, 0);
%
%            \draw (8.25, -1) rectangle (9.25, 0);
            \draw[opacity=0] (9.75, -1) rectangle (10.75, 0);

            \node at (0.5, 5.5) { \textcolor{black}{$89$} };
            \node at (1.5, 5.5) { \textcolor{black}{$60$} };
            \node at (2.5, 5.5) { \textcolor{black}{$12$} };
            \node at (3.5, 5.5) { \textcolor{black}{$45$} };
            \node at (4.5, 5.5) { \textcolor{black}{$37$} };
            \node at (5.5, 5.5) { \textcolor{black}{$52$} };
            \node at (6.5, 5.5) { \textcolor{black}{$33$} };
            \node at (7.5, 5.5) { \textcolor{black}{$97$} };
            \node at (8.5, 5.5) { \textcolor{black}{$20$} };

            \node at (-0.5, 3.5) { \textcolor{black}{$89$} };
            \node at (0.5, 3.5) { \textcolor{black}{$60$} };
            \node at (1.5, 3.5) { \textcolor{black}{$12$} };
            \node at (2.5, 3.5) { \textcolor{black}{$45$} };
            \node at (3.5, 3.5) { \textcolor{black}{$37$} };
 
            \node at (-1, 1.5) { \textcolor{blue}{$12$} };
            \node at (0, 1.5) { \textcolor{blue}{$60$} };
            \node at (1, 1.5) { \textcolor{blue}{$89$} };
 
            \node at (3, 1.5) { \textcolor{blue}{$37$} };
            \node at (4, 1.5) { \textcolor{blue}{$45$} };
 
            \node at (2.75, -0.5) { \textcolor{red}{$45$} };
            \node at (4.25, -0.5) { \textcolor{red}{$37$} };
 
        \end{scope}
        \end{tikzpicture}

    \end{figure}

\end{frame}

\begin{frame}[fragile]{Visualização do \code{c}{mergesort}}

    \begin{figure}
        \centering

        \begin{tikzpicture}
        \begin{scope}[scale=0.8]
            \draw (0, 5) grid (9, 6);

            \draw (-1, 3) grid (4, 4);
%            \draw (6, 3) grid (10, 4);
%
            \foreach \x in {-1,...,1}:
                \draw ({\x - 0.5}, 1) rectangle ({\x + 0.5}, 2);
%
            \foreach \x in {3,...,4}:
                \draw ({\x - 0.5}, 1) rectangle ({\x + 0.5}, 2);
%
%            \foreach \x in {6,...,7}:
%                \draw ({\x - 0.5}, 1) rectangle ({\x + 0.5}, 2);
%
%            \foreach \x in {9,...,10}:
%                \draw ({\x - 0.5}, 1) rectangle ({\x + 0.5}, 2);
%
            \draw[opacity=0] (-2.25, -3) rectangle (-1.25, -2);

%            \draw (5.25, -1) rectangle (6.25, 0);
%            \draw (6.75, -1) rectangle (7.75, 0);
%
%            \draw (8.25, -1) rectangle (9.25, 0);
            \draw[opacity=0] (9.75, -1) rectangle (10.75, 0);

            \node at (0.5, 5.5) { \textcolor{black}{$89$} };
            \node at (1.5, 5.5) { \textcolor{black}{$60$} };
            \node at (2.5, 5.5) { \textcolor{black}{$12$} };
            \node at (3.5, 5.5) { \textcolor{black}{$45$} };
            \node at (4.5, 5.5) { \textcolor{black}{$37$} };
            \node at (5.5, 5.5) { \textcolor{black}{$52$} };
            \node at (6.5, 5.5) { \textcolor{black}{$33$} };
            \node at (7.5, 5.5) { \textcolor{black}{$97$} };
            \node at (8.5, 5.5) { \textcolor{black}{$20$} };

            \node at (-0.5, 3.5) { \textcolor{black}{$89$} };
            \node at (0.5, 3.5) { \textcolor{black}{$60$} };
            \node at (1.5, 3.5) { \textcolor{black}{$12$} };
            \node at (2.5, 3.5) { \textcolor{black}{$45$} };
            \node at (3.5, 3.5) { \textcolor{black}{$37$} };
 
            \node at (-1, 1.5) { \textcolor{blue}{$12$} };
            \node at (0, 1.5) { \textcolor{blue}{$60$} };
            \node at (1, 1.5) { \textcolor{blue}{$89$} };
 
            \node at (3, 1.5) { \textcolor{blue}{$37$} };
            \node at (4, 1.5) { \textcolor{blue}{$45$} };
 
        \end{scope}
        \end{tikzpicture}

    \end{figure}

\end{frame}

\begin{frame}[fragile]{Visualização do \code{c}{mergesort}}

    \begin{figure}
        \centering

        \begin{tikzpicture}
        \begin{scope}[scale=0.8]
            \draw (0, 5) grid (9, 6);

            \draw (-1, 3) grid (4, 4);
%            \draw (6, 3) grid (10, 4);
%
            \foreach \x in {-1,...,1}:
                \draw ({\x - 0.5}, 1) rectangle ({\x + 0.5}, 2);
%
            \foreach \x in {3,...,4}:
                \draw ({\x - 0.5}, 1) rectangle ({\x + 0.5}, 2);
%
%            \foreach \x in {6,...,7}:
%                \draw ({\x - 0.5}, 1) rectangle ({\x + 0.5}, 2);
%
%            \foreach \x in {9,...,10}:
%                \draw ({\x - 0.5}, 1) rectangle ({\x + 0.5}, 2);
%
            \draw[opacity=0] (-2.25, -3) rectangle (-1.25, -2);

%            \draw (5.25, -1) rectangle (6.25, 0);
%            \draw (6.75, -1) rectangle (7.75, 0);
%
%            \draw (8.25, -1) rectangle (9.25, 0);
            \draw[opacity=0] (9.75, -1) rectangle (10.75, 0);

            \node at (0.5, 5.5) { \textcolor{black}{$89$} };
            \node at (1.5, 5.5) { \textcolor{black}{$60$} };
            \node at (2.5, 5.5) { \textcolor{black}{$12$} };
            \node at (3.5, 5.5) { \textcolor{black}{$45$} };
            \node at (4.5, 5.5) { \textcolor{black}{$37$} };
            \node at (5.5, 5.5) { \textcolor{black}{$52$} };
            \node at (6.5, 5.5) { \textcolor{black}{$33$} };
            \node at (7.5, 5.5) { \textcolor{black}{$97$} };
            \node at (8.5, 5.5) { \textcolor{black}{$20$} };

            \node at (-0.5, 3.5) { \textcolor{blue}{$12$} };
            \node at (0.5, 3.5) { \textcolor{blue}{$37$} };
            \node at (1.5, 3.5) { \textcolor{blue}{$45$} };
            \node at (2.5, 3.5) { \textcolor{blue}{$60$} };
            \node at (3.5, 3.5) { \textcolor{blue}{$89$} };
 
            \node at (-1, 1.5) { \textcolor{red}{$12$} };
            \node at (0, 1.5) { \textcolor{red}{$60$} };
            \node at (1, 1.5) { \textcolor{red}{$89$} };
 
            \node at (3, 1.5) { \textcolor{red}{$37$} };
            \node at (4, 1.5) { \textcolor{red}{$45$} };
 
        \end{scope}
        \end{tikzpicture}

    \end{figure}

\end{frame}

\begin{frame}[fragile]{Visualização do \code{c}{mergesort}}

    \begin{figure}
        \centering

        \begin{tikzpicture}
        \begin{scope}[scale=0.8]
            \draw (0, 5) grid (9, 6);

            \draw (-1, 3) grid (4, 4);
%            \draw (6, 3) grid (10, 4);
%
%            \foreach \x in {6,...,7}:
%                \draw ({\x - 0.5}, 1) rectangle ({\x + 0.5}, 2);
%
%            \foreach \x in {9,...,10}:
%                \draw ({\x - 0.5}, 1) rectangle ({\x + 0.5}, 2);
%
            \draw[opacity=0] (-2.25, -3) rectangle (-1.25, -2);

%            \draw (5.25, -1) rectangle (6.25, 0);
%            \draw (6.75, -1) rectangle (7.75, 0);
%
%            \draw (8.25, -1) rectangle (9.25, 0);
            \draw[opacity=0] (9.75, -1) rectangle (10.75, 0);

            \node at (0.5, 5.5) { \textcolor{black}{$89$} };
            \node at (1.5, 5.5) { \textcolor{black}{$60$} };
            \node at (2.5, 5.5) { \textcolor{black}{$12$} };
            \node at (3.5, 5.5) { \textcolor{black}{$45$} };
            \node at (4.5, 5.5) { \textcolor{black}{$37$} };
            \node at (5.5, 5.5) { \textcolor{black}{$52$} };
            \node at (6.5, 5.5) { \textcolor{black}{$33$} };
            \node at (7.5, 5.5) { \textcolor{black}{$97$} };
            \node at (8.5, 5.5) { \textcolor{black}{$20$} };

            \node at (-0.5, 3.5) { \textcolor{blue}{$12$} };
            \node at (0.5, 3.5) { \textcolor{blue}{$37$} };
            \node at (1.5, 3.5) { \textcolor{blue}{$45$} };
            \node at (2.5, 3.5) { \textcolor{blue}{$60$} };
            \node at (3.5, 3.5) { \textcolor{blue}{$89$} };

        \end{scope}
        \end{tikzpicture}

    \end{figure}

\end{frame}

\begin{frame}[fragile]{Visualização do \code{c}{mergesort}}

    \begin{figure}
        \centering

        \begin{tikzpicture}
        \begin{scope}[scale=0.8]
            \draw (0, 5) grid (9, 6);

            \draw (-1, 3) grid (4, 4);
            \draw (6, 3) grid (10, 4);
%
%            \foreach \x in {6,...,7}:
%                \draw ({\x - 0.5}, 1) rectangle ({\x + 0.5}, 2);
%
%            \foreach \x in {9,...,10}:
%                \draw ({\x - 0.5}, 1) rectangle ({\x + 0.5}, 2);
%
            \draw[opacity=0] (-2.25, -3) rectangle (-1.25, -2);

%            \draw (5.25, -1) rectangle (6.25, 0);
%            \draw (6.75, -1) rectangle (7.75, 0);
%
%            \draw (8.25, -1) rectangle (9.25, 0);
            \draw[opacity=0] (9.75, -1) rectangle (10.75, 0);

            \node at (0.5, 5.5) { \textcolor{black}{$89$} };
            \node at (1.5, 5.5) { \textcolor{black}{$60$} };
            \node at (2.5, 5.5) { \textcolor{black}{$12$} };
            \node at (3.5, 5.5) { \textcolor{black}{$45$} };
            \node at (4.5, 5.5) { \textcolor{black}{$37$} };
            \node at (5.5, 5.5) { \textcolor{black}{$52$} };
            \node at (6.5, 5.5) { \textcolor{black}{$33$} };
            \node at (7.5, 5.5) { \textcolor{black}{$97$} };
            \node at (8.5, 5.5) { \textcolor{black}{$20$} };

            \node at (-0.5, 3.5) { \textcolor{blue}{$12$} };
            \node at (0.5, 3.5) { \textcolor{blue}{$37$} };
            \node at (1.5, 3.5) { \textcolor{blue}{$45$} };
            \node at (2.5, 3.5) { \textcolor{blue}{$60$} };
            \node at (3.5, 3.5) { \textcolor{blue}{$89$} };

            \node at (6.5, 3.5) { \textcolor{green!60!black}{$52$} };
            \node at (7.5, 3.5) { \textcolor{green!60!black}{$33$} };
            \node at (8.5, 3.5) { \textcolor{green!60!black}{$97$} };
            \node at (9.5, 3.5) { \textcolor{green!60!black}{$20$} };


        \end{scope}
        \end{tikzpicture}

    \end{figure}

\end{frame}

\begin{frame}[fragile]{Visualização do \code{c}{mergesort}}

    \begin{figure}
        \centering

        \begin{tikzpicture}
        \begin{scope}[scale=0.8]
            \draw (0, 5) grid (9, 6);

            \draw (-1, 3) grid (4, 4);
            \draw (6, 3) grid (10, 4);

            \foreach \x in {6,...,7}:
                \draw ({\x - 0.5}, 1) rectangle ({\x + 0.5}, 2);

            %\foreach \x in {9,...,10}:
            %    \draw ({\x - 0.5}, 1) rectangle ({\x + 0.5}, 2);

            \draw[opacity=0] (-2.25, -3) rectangle (-1.25, -2);

%            \draw (5.25, -1) rectangle (6.25, 0);
%            \draw (6.75, -1) rectangle (7.75, 0);
%
%            \draw (8.25, -1) rectangle (9.25, 0);
            \draw[opacity=0] (9.75, -1) rectangle (10.75, 0);

            \node at (0.5, 5.5) { \textcolor{black}{$89$} };
            \node at (1.5, 5.5) { \textcolor{black}{$60$} };
            \node at (2.5, 5.5) { \textcolor{black}{$12$} };
            \node at (3.5, 5.5) { \textcolor{black}{$45$} };
            \node at (4.5, 5.5) { \textcolor{black}{$37$} };
            \node at (5.5, 5.5) { \textcolor{black}{$52$} };
            \node at (6.5, 5.5) { \textcolor{black}{$33$} };
            \node at (7.5, 5.5) { \textcolor{black}{$97$} };
            \node at (8.5, 5.5) { \textcolor{black}{$20$} };

            \node at (-0.5, 3.5) { \textcolor{blue}{$12$} };
            \node at (0.5, 3.5) { \textcolor{blue}{$37$} };
            \node at (1.5, 3.5) { \textcolor{blue}{$45$} };
            \node at (2.5, 3.5) { \textcolor{blue}{$60$} };
            \node at (3.5, 3.5) { \textcolor{blue}{$89$} };

            \node at (6.5, 3.5) { \textcolor{black}{$52$} };
            \node at (7.5, 3.5) { \textcolor{black}{$33$} };
            \node at (8.5, 3.5) { \textcolor{black}{$97$} };
            \node at (9.5, 3.5) { \textcolor{black}{$20$} };

            \node at (6, 1.5) { \textcolor{green!60!black}{$52$} };
            \node at (7, 1.5) { \textcolor{green!60!black}{$33$} };

            %\node at (9, 1.5) { \textcolor{green!60!black}{$97$} };
            %\node at (10, 1.5) { \textcolor{green!60!black}{$20$} };

        \end{scope}
        \end{tikzpicture}

    \end{figure}

\end{frame}

\begin{frame}[fragile]{Visualização do \code{c}{mergesort}}

    \begin{figure}
        \centering

        \begin{tikzpicture}
        \begin{scope}[scale=0.8]
            \draw (0, 5) grid (9, 6);

            \draw (-1, 3) grid (4, 4);
            \draw (6, 3) grid (10, 4);

            \foreach \x in {6,...,7}:
                \draw ({\x - 0.5}, 1) rectangle ({\x + 0.5}, 2);

            %\foreach \x in {9,...,10}:
            %    \draw ({\x - 0.5}, 1) rectangle ({\x + 0.5}, 2);

            \draw[opacity=0] (-2.25, -3) rectangle (-1.25, -2);

            \draw (5.25, -1) rectangle (6.25, 0);
            \draw (6.75, -1) rectangle (7.75, 0);

            %\draw (8.25, -1) rectangle (9.25, 0);
            \draw[opacity=0] (9.75, -1) rectangle (10.75, 0);

            \node at (0.5, 5.5) { \textcolor{black}{$89$} };
            \node at (1.5, 5.5) { \textcolor{black}{$60$} };
            \node at (2.5, 5.5) { \textcolor{black}{$12$} };
            \node at (3.5, 5.5) { \textcolor{black}{$45$} };
            \node at (4.5, 5.5) { \textcolor{black}{$37$} };
            \node at (5.5, 5.5) { \textcolor{black}{$52$} };
            \node at (6.5, 5.5) { \textcolor{black}{$33$} };
            \node at (7.5, 5.5) { \textcolor{black}{$97$} };
            \node at (8.5, 5.5) { \textcolor{black}{$20$} };

            \node at (-0.5, 3.5) { \textcolor{blue}{$12$} };
            \node at (0.5, 3.5) { \textcolor{blue}{$37$} };
            \node at (1.5, 3.5) { \textcolor{blue}{$45$} };
            \node at (2.5, 3.5) { \textcolor{blue}{$60$} };
            \node at (3.5, 3.5) { \textcolor{blue}{$89$} };

            \node at (6.5, 3.5) { \textcolor{black}{$52$} };
            \node at (7.5, 3.5) { \textcolor{black}{$33$} };
            \node at (8.5, 3.5) { \textcolor{black}{$97$} };
            \node at (9.5, 3.5) { \textcolor{black}{$20$} };

            \node at (6, 1.5) { \textcolor{black}{$52$} };
            \node at (7, 1.5) { \textcolor{black}{$33$} };

            %\node at (9, 1.5) { \textcolor{black}{$97$} };
            %\node at (10, 1.5) { \textcolor{black}{$20$} };

            \node at (5.75, -0.5) { \textcolor{blue}{$52$} };
            \node at (7.25, -0.5) { \textcolor{blue}{$33$} };

            %\node at (8.75, -0.5) { \textcolor{blue}{$97$} };
            %\node at (10.25, -0.5) { \textcolor{blue}{$20$} };




        \end{scope}
        \end{tikzpicture}

    \end{figure}

\end{frame}

\begin{frame}[fragile]{Visualização do \code{c}{mergesort}}

    \begin{figure}
        \centering

        \begin{tikzpicture}
        \begin{scope}[scale=0.8]
            \draw (0, 5) grid (9, 6);

            \draw (-1, 3) grid (4, 4);
            \draw (6, 3) grid (10, 4);

            \foreach \x in {6,...,7}:
                \draw ({\x - 0.5}, 1) rectangle ({\x + 0.5}, 2);

            %\foreach \x in {9,...,10}:
            %    \draw ({\x - 0.5}, 1) rectangle ({\x + 0.5}, 2);

            \draw[opacity=0] (-2.25, -3) rectangle (-1.25, -2);

            \draw (5.25, -1) rectangle (6.25, 0);
            \draw (6.75, -1) rectangle (7.75, 0);

            %\draw (8.25, -1) rectangle (9.25, 0);
            \draw[opacity=0] (9.75, -1) rectangle (10.75, 0);

            \node at (0.5, 5.5) { \textcolor{black}{$89$} };
            \node at (1.5, 5.5) { \textcolor{black}{$60$} };
            \node at (2.5, 5.5) { \textcolor{black}{$12$} };
            \node at (3.5, 5.5) { \textcolor{black}{$45$} };
            \node at (4.5, 5.5) { \textcolor{black}{$37$} };
            \node at (5.5, 5.5) { \textcolor{black}{$52$} };
            \node at (6.5, 5.5) { \textcolor{black}{$33$} };
            \node at (7.5, 5.5) { \textcolor{black}{$97$} };
            \node at (8.5, 5.5) { \textcolor{black}{$20$} };

            \node at (-0.5, 3.5) { \textcolor{blue}{$12$} };
            \node at (0.5, 3.5) { \textcolor{blue}{$37$} };
            \node at (1.5, 3.5) { \textcolor{blue}{$45$} };
            \node at (2.5, 3.5) { \textcolor{blue}{$60$} };
            \node at (3.5, 3.5) { \textcolor{blue}{$89$} };

            \node at (6.5, 3.5) { \textcolor{black}{$52$} };
            \node at (7.5, 3.5) { \textcolor{black}{$33$} };
            \node at (8.5, 3.5) { \textcolor{black}{$97$} };
            \node at (9.5, 3.5) { \textcolor{black}{$20$} };

            \node at (6, 1.5) { \textcolor{blue}{$33$} };
            \node at (7, 1.5) { \textcolor{blue}{$52$} };

            %\node at (9, 1.5) { \textcolor{black}{$97$} };
            %\node at (10, 1.5) { \textcolor{black}{$20$} };

            \node at (5.75, -0.5) { \textcolor{red}{$52$} };
            \node at (7.25, -0.5) { \textcolor{red}{$33$} };

            %\node at (8.75, -0.5) { \textcolor{blue}{$97$} };
            %\node at (10.25, -0.5) { \textcolor{blue}{$20$} };

        \end{scope}
        \end{tikzpicture}

    \end{figure}

\end{frame}

\begin{frame}[fragile]{Visualização do \code{c}{mergesort}}

    \begin{figure}
        \centering

        \begin{tikzpicture}
        \begin{scope}[scale=0.8]
            \draw (0, 5) grid (9, 6);

            \draw (-1, 3) grid (4, 4);
            \draw (6, 3) grid (10, 4);

            \foreach \x in {6,...,7}:
                \draw ({\x - 0.5}, 1) rectangle ({\x + 0.5}, 2);

            %\foreach \x in {9,...,10}:
            %    \draw ({\x - 0.5}, 1) rectangle ({\x + 0.5}, 2);

            \draw[opacity=0] (-2.25, -3) rectangle (-1.25, -2);

            %\draw (8.25, -1) rectangle (9.25, 0);
            \draw[opacity=0] (9.75, -1) rectangle (10.75, 0);

            \node at (0.5, 5.5) { \textcolor{black}{$89$} };
            \node at (1.5, 5.5) { \textcolor{black}{$60$} };
            \node at (2.5, 5.5) { \textcolor{black}{$12$} };
            \node at (3.5, 5.5) { \textcolor{black}{$45$} };
            \node at (4.5, 5.5) { \textcolor{black}{$37$} };
            \node at (5.5, 5.5) { \textcolor{black}{$52$} };
            \node at (6.5, 5.5) { \textcolor{black}{$33$} };
            \node at (7.5, 5.5) { \textcolor{black}{$97$} };
            \node at (8.5, 5.5) { \textcolor{black}{$20$} };

            \node at (-0.5, 3.5) { \textcolor{blue}{$12$} };
            \node at (0.5, 3.5) { \textcolor{blue}{$37$} };
            \node at (1.5, 3.5) { \textcolor{blue}{$45$} };
            \node at (2.5, 3.5) { \textcolor{blue}{$60$} };
            \node at (3.5, 3.5) { \textcolor{blue}{$89$} };

            \node at (6.5, 3.5) { \textcolor{black}{$52$} };
            \node at (7.5, 3.5) { \textcolor{black}{$33$} };
            \node at (8.5, 3.5) { \textcolor{black}{$97$} };
            \node at (9.5, 3.5) { \textcolor{black}{$20$} };

            \node at (6, 1.5) { \textcolor{blue}{$33$} };
            \node at (7, 1.5) { \textcolor{blue}{$52$} };

            %\node at (9, 1.5) { \textcolor{black}{$97$} };
            %\node at (10, 1.5) { \textcolor{black}{$20$} };

            %\node at (8.75, -0.5) { \textcolor{blue}{$97$} };
            %\node at (10.25, -0.5) { \textcolor{blue}{$20$} };

        \end{scope}
        \end{tikzpicture}

    \end{figure}

\end{frame}

\begin{frame}[fragile]{Visualização do \code{c}{mergesort}}

    \begin{figure}
        \centering

        \begin{tikzpicture}
        \begin{scope}[scale=0.8]
            \draw (0, 5) grid (9, 6);

            \draw (-1, 3) grid (4, 4);
            \draw (6, 3) grid (10, 4);

            \foreach \x in {6,...,7}:
                \draw ({\x - 0.5}, 1) rectangle ({\x + 0.5}, 2);

            \foreach \x in {9,...,10}:
                \draw ({\x - 0.5}, 1) rectangle ({\x + 0.5}, 2);

            \draw[opacity=0] (-2.25, -3) rectangle (-1.25, -2);

            %\draw (8.25, -1) rectangle (9.25, 0);
            \draw[opacity=0] (9.75, -1) rectangle (10.75, 0);

            \node at (0.5, 5.5) { \textcolor{black}{$89$} };
            \node at (1.5, 5.5) { \textcolor{black}{$60$} };
            \node at (2.5, 5.5) { \textcolor{black}{$12$} };
            \node at (3.5, 5.5) { \textcolor{black}{$45$} };
            \node at (4.5, 5.5) { \textcolor{black}{$37$} };
            \node at (5.5, 5.5) { \textcolor{black}{$52$} };
            \node at (6.5, 5.5) { \textcolor{black}{$33$} };
            \node at (7.5, 5.5) { \textcolor{black}{$97$} };
            \node at (8.5, 5.5) { \textcolor{black}{$20$} };

            \node at (-0.5, 3.5) { \textcolor{blue}{$12$} };
            \node at (0.5, 3.5) { \textcolor{blue}{$37$} };
            \node at (1.5, 3.5) { \textcolor{blue}{$45$} };
            \node at (2.5, 3.5) { \textcolor{blue}{$60$} };
            \node at (3.5, 3.5) { \textcolor{blue}{$89$} };

            \node at (6.5, 3.5) { \textcolor{black}{$52$} };
            \node at (7.5, 3.5) { \textcolor{black}{$33$} };
            \node at (8.5, 3.5) { \textcolor{black}{$97$} };
            \node at (9.5, 3.5) { \textcolor{black}{$20$} };

            \node at (6, 1.5) { \textcolor{blue}{$33$} };
            \node at (7, 1.5) { \textcolor{blue}{$52$} };

            \node at (9, 1.5) { \textcolor{green!60!black}{$97$} };
            \node at (10, 1.5) { \textcolor{green!60!black}{$20$} };

            %\node at (8.75, -0.5) { \textcolor{blue}{$97$} };
            %\node at (10.25, -0.5) { \textcolor{blue}{$20$} };

        \end{scope}
        \end{tikzpicture}

    \end{figure}

\end{frame}

\begin{frame}[fragile]{Visualização do \code{c}{mergesort}}

    \begin{figure}
        \centering

        \begin{tikzpicture}
        \begin{scope}[scale=0.8]
            \draw (0, 5) grid (9, 6);

            \draw (-1, 3) grid (4, 4);
            \draw (6, 3) grid (10, 4);

            \foreach \x in {6,...,7}:
                \draw ({\x - 0.5}, 1) rectangle ({\x + 0.5}, 2);

            \foreach \x in {9,...,10}:
                \draw ({\x - 0.5}, 1) rectangle ({\x + 0.5}, 2);

            \draw[opacity=0] (-2.25, -3) rectangle (-1.25, -2);

            \draw (8.25, -1) rectangle (9.25, 0);
            \draw[opacity=1] (9.75, -1) rectangle (10.75, 0);

            \node at (0.5, 5.5) { \textcolor{black}{$89$} };
            \node at (1.5, 5.5) { \textcolor{black}{$60$} };
            \node at (2.5, 5.5) { \textcolor{black}{$12$} };
            \node at (3.5, 5.5) { \textcolor{black}{$45$} };
            \node at (4.5, 5.5) { \textcolor{black}{$37$} };
            \node at (5.5, 5.5) { \textcolor{black}{$52$} };
            \node at (6.5, 5.5) { \textcolor{black}{$33$} };
            \node at (7.5, 5.5) { \textcolor{black}{$97$} };
            \node at (8.5, 5.5) { \textcolor{black}{$20$} };

            \node at (-0.5, 3.5) { \textcolor{blue}{$12$} };
            \node at (0.5, 3.5) { \textcolor{blue}{$37$} };
            \node at (1.5, 3.5) { \textcolor{blue}{$45$} };
            \node at (2.5, 3.5) { \textcolor{blue}{$60$} };
            \node at (3.5, 3.5) { \textcolor{blue}{$89$} };

            \node at (6.5, 3.5) { \textcolor{black}{$52$} };
            \node at (7.5, 3.5) { \textcolor{black}{$33$} };
            \node at (8.5, 3.5) { \textcolor{black}{$97$} };
            \node at (9.5, 3.5) { \textcolor{black}{$20$} };

            \node at (6, 1.5) { \textcolor{blue}{$33$} };
            \node at (7, 1.5) { \textcolor{blue}{$52$} };

            \node at (9, 1.5) { \textcolor{black}{$97$} };
            \node at (10, 1.5) { \textcolor{black}{$20$} };

            \node at (8.75, -0.5) { \textcolor{blue}{$97$} };
            \node at (10.25, -0.5) { \textcolor{blue}{$20$} };

        \end{scope}
        \end{tikzpicture}

    \end{figure}

\end{frame}

\begin{frame}[fragile]{Visualização do \code{c}{mergesort}}

    \begin{figure}
        \centering

        \begin{tikzpicture}
        \begin{scope}[scale=0.8]
            \draw (0, 5) grid (9, 6);

            \draw (-1, 3) grid (4, 4);
            \draw (6, 3) grid (10, 4);

            \foreach \x in {6,...,7}:
                \draw ({\x - 0.5}, 1) rectangle ({\x + 0.5}, 2);

            \foreach \x in {9,...,10}:
                \draw ({\x - 0.5}, 1) rectangle ({\x + 0.5}, 2);

            \draw[opacity=0] (-2.25, -3) rectangle (-1.25, -2);

            \draw (8.25, -1) rectangle (9.25, 0);
            \draw[opacity=1] (9.75, -1) rectangle (10.75, 0);

            \node at (0.5, 5.5) { \textcolor{black}{$89$} };
            \node at (1.5, 5.5) { \textcolor{black}{$60$} };
            \node at (2.5, 5.5) { \textcolor{black}{$12$} };
            \node at (3.5, 5.5) { \textcolor{black}{$45$} };
            \node at (4.5, 5.5) { \textcolor{black}{$37$} };
            \node at (5.5, 5.5) { \textcolor{black}{$52$} };
            \node at (6.5, 5.5) { \textcolor{black}{$33$} };
            \node at (7.5, 5.5) { \textcolor{black}{$97$} };
            \node at (8.5, 5.5) { \textcolor{black}{$20$} };

            \node at (-0.5, 3.5) { \textcolor{blue}{$12$} };
            \node at (0.5, 3.5) { \textcolor{blue}{$37$} };
            \node at (1.5, 3.5) { \textcolor{blue}{$45$} };
            \node at (2.5, 3.5) { \textcolor{blue}{$60$} };
            \node at (3.5, 3.5) { \textcolor{blue}{$89$} };

            \node at (6.5, 3.5) { \textcolor{black}{$52$} };
            \node at (7.5, 3.5) { \textcolor{black}{$33$} };
            \node at (8.5, 3.5) { \textcolor{black}{$97$} };
            \node at (9.5, 3.5) { \textcolor{black}{$20$} };

            \node at (6, 1.5) { \textcolor{blue}{$33$} };
            \node at (7, 1.5) { \textcolor{blue}{$52$} };

            \node at (9, 1.5) { \textcolor{blue}{$20$} };
            \node at (10, 1.5) { \textcolor{blue}{$97$} };

            \node at (8.75, -0.5) { \textcolor{red}{$97$} };
            \node at (10.25, -0.5) { \textcolor{red}{$20$} };

        \end{scope}
        \end{tikzpicture}

    \end{figure}

\end{frame}

\begin{frame}[fragile]{Visualização do \code{c}{mergesort}}

    \begin{figure}
        \centering

        \begin{tikzpicture}
        \begin{scope}[scale=0.8]
            \draw (0, 5) grid (9, 6);

            \draw (-1, 3) grid (4, 4);
            \draw (6, 3) grid (10, 4);

            \foreach \x in {6,...,7}:
                \draw ({\x - 0.5}, 1) rectangle ({\x + 0.5}, 2);

            \foreach \x in {9,...,10}:
                \draw ({\x - 0.5}, 1) rectangle ({\x + 0.5}, 2);

            \draw[opacity=0] (-2.25, -3) rectangle (-1.25, -2);

            \draw[opacity=0] (9.75, -1) rectangle (10.75, 0);

            \node at (0.5, 5.5) { \textcolor{black}{$89$} };
            \node at (1.5, 5.5) { \textcolor{black}{$60$} };
            \node at (2.5, 5.5) { \textcolor{black}{$12$} };
            \node at (3.5, 5.5) { \textcolor{black}{$45$} };
            \node at (4.5, 5.5) { \textcolor{black}{$37$} };
            \node at (5.5, 5.5) { \textcolor{black}{$52$} };
            \node at (6.5, 5.5) { \textcolor{black}{$33$} };
            \node at (7.5, 5.5) { \textcolor{black}{$97$} };
            \node at (8.5, 5.5) { \textcolor{black}{$20$} };

            \node at (-0.5, 3.5) { \textcolor{blue}{$12$} };
            \node at (0.5, 3.5) { \textcolor{blue}{$37$} };
            \node at (1.5, 3.5) { \textcolor{blue}{$45$} };
            \node at (2.5, 3.5) { \textcolor{blue}{$60$} };
            \node at (3.5, 3.5) { \textcolor{blue}{$89$} };

            \node at (6.5, 3.5) { \textcolor{black}{$52$} };
            \node at (7.5, 3.5) { \textcolor{black}{$33$} };
            \node at (8.5, 3.5) { \textcolor{black}{$97$} };
            \node at (9.5, 3.5) { \textcolor{black}{$20$} };

            \node at (6, 1.5) { \textcolor{blue}{$33$} };
            \node at (7, 1.5) { \textcolor{blue}{$52$} };

            \node at (9, 1.5) { \textcolor{blue}{$20$} };
            \node at (10, 1.5) { \textcolor{blue}{$97$} };

        \end{scope}
        \end{tikzpicture}

    \end{figure}

\end{frame}

\begin{frame}[fragile]{Visualização do \code{c}{mergesort}}

    \begin{figure}
        \centering

        \begin{tikzpicture}
        \begin{scope}[scale=0.8]
            \draw (0, 5) grid (9, 6);

            \draw (-1, 3) grid (4, 4);
            \draw (6, 3) grid (10, 4);

            \foreach \x in {6,...,7}:
                \draw ({\x - 0.5}, 1) rectangle ({\x + 0.5}, 2);

            \foreach \x in {9,...,10}:
                \draw ({\x - 0.5}, 1) rectangle ({\x + 0.5}, 2);

            \draw[opacity=0] (-2.25, -3) rectangle (-1.25, -2);

            \draw[opacity=0] (9.75, -1) rectangle (10.75, 0);

            \node at (0.5, 5.5) { \textcolor{black}{$89$} };
            \node at (1.5, 5.5) { \textcolor{black}{$60$} };
            \node at (2.5, 5.5) { \textcolor{black}{$12$} };
            \node at (3.5, 5.5) { \textcolor{black}{$45$} };
            \node at (4.5, 5.5) { \textcolor{black}{$37$} };
            \node at (5.5, 5.5) { \textcolor{black}{$52$} };
            \node at (6.5, 5.5) { \textcolor{black}{$33$} };
            \node at (7.5, 5.5) { \textcolor{black}{$97$} };
            \node at (8.5, 5.5) { \textcolor{black}{$20$} };

            \node at (-0.5, 3.5) { \textcolor{blue}{$12$} };
            \node at (0.5, 3.5) { \textcolor{blue}{$37$} };
            \node at (1.5, 3.5) { \textcolor{blue}{$45$} };
            \node at (2.5, 3.5) { \textcolor{blue}{$60$} };
            \node at (3.5, 3.5) { \textcolor{blue}{$89$} };

            \node at (6.5, 3.5) { \textcolor{blue}{$20$} };
            \node at (7.5, 3.5) { \textcolor{blue}{$33$} };
            \node at (8.5, 3.5) { \textcolor{blue}{$52$} };
            \node at (9.5, 3.5) { \textcolor{blue}{$97$} };

            \node at (6, 1.5) { \textcolor{red}{$33$} };
            \node at (7, 1.5) { \textcolor{red}{$52$} };

            \node at (9, 1.5) { \textcolor{red}{$20$} };
            \node at (10, 1.5) { \textcolor{red}{$97$} };

        \end{scope}
        \end{tikzpicture}

    \end{figure}

\end{frame}

\begin{frame}[fragile]{Visualização do \code{c}{mergesort}}

    \begin{figure}
        \centering

        \begin{tikzpicture}
        \begin{scope}[scale=0.8]
            \draw (0, 5) grid (9, 6);

            \draw (-1, 3) grid (4, 4);
            \draw (6, 3) grid (10, 4);

            \draw[opacity=0] (-2.25, -3) rectangle (-1.25, -2);
            \draw[opacity=0] (9.75, -1) rectangle (10.75, 0);

            \node at (0.5, 5.5) { \textcolor{black}{$89$} };
            \node at (1.5, 5.5) { \textcolor{black}{$60$} };
            \node at (2.5, 5.5) { \textcolor{black}{$12$} };
            \node at (3.5, 5.5) { \textcolor{black}{$45$} };
            \node at (4.5, 5.5) { \textcolor{black}{$37$} };
            \node at (5.5, 5.5) { \textcolor{black}{$52$} };
            \node at (6.5, 5.5) { \textcolor{black}{$33$} };
            \node at (7.5, 5.5) { \textcolor{black}{$97$} };
            \node at (8.5, 5.5) { \textcolor{black}{$20$} };

            \node at (-0.5, 3.5) { \textcolor{blue}{$12$} };
            \node at (0.5, 3.5) { \textcolor{blue}{$37$} };
            \node at (1.5, 3.5) { \textcolor{blue}{$45$} };
            \node at (2.5, 3.5) { \textcolor{blue}{$60$} };
            \node at (3.5, 3.5) { \textcolor{blue}{$89$} };

            \node at (6.5, 3.5) { \textcolor{blue}{$20$} };
            \node at (7.5, 3.5) { \textcolor{blue}{$33$} };
            \node at (8.5, 3.5) { \textcolor{blue}{$52$} };
            \node at (9.5, 3.5) { \textcolor{blue}{$97$} };

        \end{scope}
        \end{tikzpicture}

    \end{figure}

\end{frame}

\begin{frame}[fragile]{Visualização do \code{c}{mergesort}}

    \begin{figure}
        \centering

        \begin{tikzpicture}
        \begin{scope}[scale=0.8]
            \draw (0, 5) grid (9, 6);

            \draw (-1, 3) grid (4, 4);
            \draw (6, 3) grid (10, 4);

            \draw[opacity=0] (-2.25, -3) rectangle (-1.25, -2);
            \draw[opacity=0] (9.75, -1) rectangle (10.75, 0);

            \node at (0.5, 5.5) { \textcolor{blue}{$12$} };
            \node at (1.5, 5.5) { \textcolor{blue}{$20$} };
            \node at (2.5, 5.5) { \textcolor{blue}{$33$} };
            \node at (3.5, 5.5) { \textcolor{blue}{$37$} };
            \node at (4.5, 5.5) { \textcolor{blue}{$45$} };
            \node at (5.5, 5.5) { \textcolor{blue}{$52$} };
            \node at (6.5, 5.5) { \textcolor{blue}{$60$} };
            \node at (7.5, 5.5) { \textcolor{blue}{$89$} };
            \node at (8.5, 5.5) { \textcolor{blue}{$97$} };

            \node at (-0.5, 3.5) { \textcolor{red}{$12$} };
            \node at (0.5, 3.5) { \textcolor{red}{$37$} };
            \node at (1.5, 3.5) { \textcolor{red}{$45$} };
            \node at (2.5, 3.5) { \textcolor{red}{$60$} };
            \node at (3.5, 3.5) { \textcolor{red}{$89$} };

            \node at (6.5, 3.5) { \textcolor{red}{$20$} };
            \node at (7.5, 3.5) { \textcolor{red}{$33$} };
            \node at (8.5, 3.5) { \textcolor{red}{$52$} };
            \node at (9.5, 3.5) { \textcolor{red}{$97$} };

        \end{scope}
        \end{tikzpicture}

    \end{figure}

\end{frame}

\begin{frame}[fragile]{Visualização do \code{c}{mergesort}}

    \begin{figure}
        \centering

        \begin{tikzpicture}
        \begin{scope}[scale=0.8]
            \draw (0, 5) grid (9, 6);

            \draw[opacity=0] (-2.25, -3) rectangle (-1.25, -2);
            \draw[opacity=0] (9.75, -1) rectangle (10.75, 0);

            \node at (0.5, 5.5) { \textcolor{black}{$12$} };
            \node at (1.5, 5.5) { \textcolor{black}{$20$} };
            \node at (2.5, 5.5) { \textcolor{black}{$33$} };
            \node at (3.5, 5.5) { \textcolor{black}{$37$} };
            \node at (4.5, 5.5) { \textcolor{black}{$45$} };
            \node at (5.5, 5.5) { \textcolor{black}{$52$} };
            \node at (6.5, 5.5) { \textcolor{black}{$60$} };
            \node at (7.5, 5.5) { \textcolor{black}{$89$} };
            \node at (8.5, 5.5) { \textcolor{black}{$97$} };

        \end{scope}
        \end{tikzpicture}

    \end{figure}

\end{frame}


\begin{frame}[fragile]{Implementação do \code{c}{mergesort}}
    \inputsnippet{cpp}{26}{42}{codes/mergesort.cpp}
\end{frame}
