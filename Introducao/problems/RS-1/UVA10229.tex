\section{UVA 10229 -- Modular Fibonacci}

\begin{frame}[fragile]{Problema}

The Fibonacci numbers (0, 1, 1, 2, 3, 5, 8, 13, 21, 34, 55, ...) are defined by the recurrence:
\begin{align*}
F_0 &= 0 \\
F_1 &= 1 \\
F_i &= F_{i-1} + F_{i-2}\ \mbox{for}\ i > 1
\end{align*}

Write a program which calculates $M_n = F_n\ \mbox{mod}\ 2^m$
for given pair of $n$ and $m$. $0 \leq n \leq 2147483647$
and $0 \leq m < 20$. Note that $a\ \mbox{mod}\ b$ gives the remainder when $a$ is divided by $b$.
\end{frame}

\begin{frame}[fragile]{Entrada e saída}

\textbf{Input}

Input consists of several lines specifying a pair of $n$ and $m$.

\vspace{0.5in}

\textbf{Output}

Output should be corresponding $M_n$, one per line.

\end{frame}

\begin{frame}[fragile]{Exemplo de entradas e saídas}

\begin{minipage}[t]{0.5\textwidth}
\textbf{Exemplo de Entrada}
\begin{verbatim}
11 7
11 6
\end{verbatim}
\end{minipage}
\begin{minipage}[t]{0.45\textwidth}
\textbf{Exemplo de Saída}
\begin{verbatim}
89
25
\end{verbatim}
\end{minipage}
\end{frame}

\begin{frame}[fragile]{Solução recursiva}

    \begin{itemize}
        \item A implementação recursiva dos números de Fibonacci trazem uma série de problemas

        \item Primeiramente, a quantidade de chamadas recursivas é demasiadamente alto, o que
            leva ao TLE

        \item Além disso, pode acontecer um estouro na pilha de execução, levando a um RTE

        \item O tipo de variável usada para armazenar os valores pode levar a erros de 
            \textit{overflow}, caso o módulo seja computado apenas ao final

        \item Para computar o valor exato de $F_n$ para $n \geq 100$ é necessário utilizar
            aritmética estendida
   \end{itemize}

\end{frame}

\begin{frame}[fragile]{Solução WA/TLE com complexidade $O(T2^n)$}
    \inputsnippet{cpp}{1}{10}{wa.cpp}
\end{frame}

\begin{frame}[fragile]{Solução WA/TLE com complexidade $O(T2^n)$}
    \inputsnippet{cpp}{11}{31}{wa.cpp}
\end{frame}

\begin{frame}[fragile]{Solução iterativa}

    \begin{itemize}
        \item A implementação iterativa dos números de Fibonacci resolvem dois dos problemas
            apresentados anteriormente 

        \item Primeiramente, não há mais chamadas recursivas

        \item Além disso, não há estouro da pilha de execução

        \item Embora a complexidade seja reduzida para $O(Tn)$, ainda assim a solução não é
            rápida o suficiente para o tamanho máximo de $n$

        \item As operações modulares a cada etapa evitam o problema de \textit{overflow}
    \end{itemize}

\end{frame}

\begin{frame}[fragile]{Solução TLE com complexidade $O(Tn)$}
    \inputsnippet{cpp}{1}{20}{tle.cpp}
\end{frame}

\begin{frame}[fragile]{Solução TLE com complexidade $O(Tn)$}
    \inputsnippet{cpp}{21}{41}{tle.cpp}
\end{frame}

\begin{frame}[fragile]{Forma matricial da recorrência de Fibonacci}

    \begin{itemize}
        \item Para uma solução com complexidade inferior à linear, é preciso resolver a 
            recorrência

        \item Sabemos que $F_{n + 2} = F_{n + 1} + F_{n}$ para $n > 1$, e que $F(0) = 0, F(1) = 1$

        \item Esta relação pode ser representada matricialmente:
        \[
            \begin{bmatrix} F_{n + 2} \\ F_{n + 1}\end{bmatrix} = 
            \begin{bmatrix} 1 & 1 \\ 1 & 0\end{bmatrix}
            \begin{bmatrix} F_{n + 1} \\ F_{n}\end{bmatrix}
        \]

        \item Seja
        \[
            u_n = \begin{bmatrix} F_{n + 1} \\ F_{n}\end{bmatrix},\ \ 
            A = \begin{bmatrix} 1 & 1 \\ 1 & 0\end{bmatrix}\ \ \mbox{e}\ \ 
            u_0 = \begin{bmatrix} 1 \\ 0\end{bmatrix}
        \]

        \item Assim, $u_{n + 1} = Au_n$
    \end{itemize}

\end{frame}

\begin{frame}[fragile]{Fórmula para os termos de Fibonacci sem recorrência}

    \begin{itemize}
        \item Expandindo a recorrência tem-se que
        \begin{align*}
            u_1 &= Au_0 \\
            u_2 &= Au_1 = A(Au_0) = A^2u_0 \\
            u_3 &= Au_2 = A(A^2u_0) = A^3u_0 \\
            & \ldots,
        \end{align*}
        isto é, $u_n = A^nu_0$
        
        \item A expressão acima, que não possui recorrência, 
            permite computar o $n$-ésimo termo da sequência de Fibonacci,
            se conhecido o valor de $A^n$

        \item Esta exponenciação matricial pode ser feita, de forma eficiente, se a matriz
            $A$ possuir autovetores linearmente independentes
    \end{itemize}

\end{frame}

\begin{frame}[fragile]{Autovalores de Fibonacci}

    \begin{itemize}
        \item O vetor $\vec{x}$ é autovetor de $A$ se existir um $\lambda$ real (denominado
            autovalor) tal que $A\vec{x} = \lambda \vec{x}$

        \item Para encontrar os autovalores de $A$, é preciso encontrar os valores $\lambda$
            tais que $(A - \lambda I)\vec{x} = \vec{0}$, o que ocorre quando
        \[
             \det \begin{bmatrix} 1 - \lambda & 1 \\ 1 & -\lambda \end{bmatrix} = 
             \begin{bmatrix} 0 \\ 0\end{bmatrix},
        \] o que leva à expressão
        \[
            p(\lambda) = -\lambda(1 - \lambda) - 1 = 0
        \]

        \item Os zeros do polinômio $p(\lambda) = \lambda^2 - \lambda - 1$ são os autovetores
            de Fibonacci, a saber:
        \[
             \lambda_1 = \frac{1 + \sqrt{5}}{2}\ \ \mbox{e}\ \ 
             \lambda_2 = \frac{1 - \sqrt{5}}{2}
        \]
            
    \end{itemize}

\end{frame}

\begin{frame}[fragile]{Autovetores de Fibonacci}

    \begin{itemize}
        \item Para encontrar os autovetores de Fibonacci, basta levar os autovalores novamente
            na igualdade anterior

        \item Para $\lambda_1$ segue que
        \[
             \begin{bmatrix} 1 - \lambda_1 & 1 \\ 1 & -\lambda_1 \end{bmatrix}
             \begin{bmatrix} x \\ y\end{bmatrix} = 
             \begin{bmatrix} 0 \\ 0\end{bmatrix}
        \]
        cuja solução não trivial é
        \[
             \vec{v}_1 = \begin{bmatrix} \lambda_1 \\ 1\end{bmatrix},
        \]
        pois $\lambda_1 - \lambda_1^2 + 1 = 0$, uma vez que $\lambda_1$ é autovalor   

        \item O mesmo vale para $\lambda_2$, obtendo o segundo autovetor
        \[
             \vec{v}_2 = \begin{bmatrix} \lambda_2 \\ 1\end{bmatrix}
        \]
    \end{itemize}

\end{frame}

\begin{frame}[fragile]{Diagonalização da matrix $A$}

    \begin{itemize}
        \item Como os autovalores $\vec{v}_1$ e $\vec{v}_2$ são linearmente independentes,
            a matriz $A$ é diagonalizável

        \item Seja $S$ a matriz dos autovetores e $\Lambda$ a matriz diagonal dos autovalores
            correspondentes, isto é,
        \[
             S = \begin{bmatrix} \lambda_1 & \lambda_2 \\\ 1 & 1 \end{bmatrix}\ \ \mbox{e}\ \
             \Lambda = \begin{bmatrix} \lambda_1 & 0 \\\ 0 & \lambda_2 \end{bmatrix}
        \]

        \item A definição de autovetor implica que $AS = S\Lambda$, e como $S$ é invertível, segue que
        \[
            A = S\Lambda S^{-1}
        \]
        
            
    \end{itemize}

\end{frame}

\begin{frame}[fragile]{Exponenciação eficiente de matrizes diagonalizadas}

    \begin{itemize}
        \item A expressão $A = S\Lambda S^{-1}$ permite computar $A^n$ de forma eficiente, pois
        \begin{align*}
            A^2 &= (S\Lambda S^{-1})(S\Lambda S^{-1}) = S\Lambda (S^{-1}S)\Lambda S^{-1} = S\Lambda^2 S^{-1} \\
            A^3 &= A^2A = (S\Lambda^2 S^{-1})(S\Lambda S^{-1}) = S\Lambda^2 (S^{-1}S)\Lambda S^{-1} = S\Lambda^3 S^{-1} \\
        &\ldots
        \end{align*}

        \item Assim, $A^n = S\Lambda^n S^{-1}$, com
        \[
            \Lambda^n = \begin{bmatrix} \lambda_1^n & 0 \\\ 0 & \lambda_2^n \end{bmatrix}\ \ \mbox{e}\ \ 
            S^{-1} = \frac{1}{\lambda_1 - \lambda_2}\begin{bmatrix} 1 & -\lambda_2 \\ -1 & \lambda_1 \end{bmatrix}
        \]

        \item A expressão para os números de Fibonacci sem recorrência leva a
        \[
            u_n = A^nu_0 = (S\Lambda^n S^{-1})u_0 = S\Lambda^n (S^{-1})u_0
            = S\Lambda^n \vec{c},
        \]
        onde o vetor $\vec{c}$ é constante
    \end{itemize}

\end{frame}

\begin{frame}[fragile]{Fórmula de Binet}

    \begin{itemize}
        \item De fato,
        \[
            \vec{c} = \frac{1}{\lambda_1 - \lambda_2}\begin{bmatrix} 1 & -\lambda_2 \\ -1 & \lambda_1 \end{bmatrix}
\begin{bmatrix} 1 \\ 0 \end{bmatrix} = 
            \frac{1}{\lambda_1 - \lambda_2}\begin{bmatrix} 1 \\ -1 \end{bmatrix} =
            \frac{1}{\sqrt{5}}\begin{bmatrix} 1 \\ -1 \end{bmatrix}
        \]

        \item Portanto, como $u_n = S\Lambda^n \vec{c}$, 
        \begin{align*}
            \begin{bmatrix} F_{n + 1} \\ F_n \end{bmatrix} &= 
            \frac{1}{\sqrt{5}}\begin{bmatrix} \lambda_1 & \lambda_2 \\\ 1 & 1 \end{bmatrix}
            \begin{bmatrix} \lambda_1^n & 0 \\\ 0 & \lambda_2^n \end{bmatrix}
            \begin{bmatrix} 1 \\ -1 \end{bmatrix} \\
            &= \frac{1}{\sqrt{5}}
            \begin{bmatrix} \lambda_1 & \lambda_2 \\\ 1 & 1 \end{bmatrix}
            \begin{bmatrix} \lambda_1^n \\\ -\lambda_2^n \end{bmatrix} 
            = \frac{1}{\sqrt{5}}
            \begin{bmatrix} \lambda_1^{n + 1} + \lambda_2^{n + 1} \\ \lambda_1^{n} + \lambda_2^{n}   \end{bmatrix}
        \end{align*}

        \item A Fórmula de Binet corresponde à segunda componente do vetor acima, isto é,
        \[
            F_n = \frac{1}{\sqrt{5}}\left[\left(\frac{1 + \sqrt{5}}{2}\right)^{n} + \left(\frac{1 - \sqrt{5}}{2}\right)^{n}\right]
        \]
    \end{itemize}

\end{frame}

\begin{frame}[fragile]{Solução WA usando a fórmula de Binet}
    \inputsnippet{cpp}{1}{11}{binet.cpp}
\end{frame}

\begin{frame}[fragile]{Solução WA usando a fórmula de Binet}
    \inputsnippet{cpp}{12}{32}{binet.cpp}
\end{frame}

\begin{frame}[fragile]{Exponenciação rápida de matrizes}

    \begin{itemize}
        \item Retornando à igualdade $u_{n + 1} = A^nu_0$, é possível computar $A^n$ com 
            complexidade $O(\log n)$ sem o uso de autovetores

        \item Basta observar que
        \[
            A^n = \left\lbrace \begin{array}{ll} A, & \mbox{se}\ n = 1 \\
                A^{\frac{n}{2}}A^{\frac{n}{2}}, & \mbox{se}\ $n$\ \mbox{é par} \\ 
                A^{\frac{n}{2}}A^{\frac{n}{2}}A, & \mbox{caso contrário}
                \end{array}
                \right.
        \]

        \item Vale a igualdade
        \[
            A^n = \begin{bmatrix} F_{n + 1} & F_n \\ F_n & F_{n - 1}\end{bmatrix},
        \]
        a qual pode ser provada por indução

    \end{itemize}
\end{frame}

\begin{frame}[fragile]{Exponenciação rápida de matrizes}

    \begin{itemize}
        \item Para $n = 1$ 
        \[
            A^1 = A = \begin{bmatrix} 1 & 1 \\ 1 & 0\end{bmatrix} = 
            \begin{bmatrix} F_2 & F_1 \\ F_1 & F_0\end{bmatrix}
        \]

        \item Suponha que a afirmativa seja verdadeira para $m$ natural. Para $m+1$ segue que
        \begin{align*}
            A^{m + 1} = A^mA &= \begin{bmatrix} F_{m + 1} & F_m \\ F_m & F_{m - 1}\end{bmatrix}
            \begin{bmatrix} 1 & 1 \\ 1 & 0\end{bmatrix} \\
            &= \begin{bmatrix} F_{m + 1} + F_m & F_{m + 1} \\ F_m  + F_{m - 1}& F_m\end{bmatrix} \\
            &= \begin{bmatrix} F_{m + 2} & F_{m + 1} \\ F_{m + 1} & F_{m}\end{bmatrix} \\
        \end{align*}

        \item Portanto a afirmativa é verdadeira para qualquer $n$ natural
    \end{itemize}

\end{frame}
\begin{frame}[fragile]{Implementação recursiva da exponencição rápida de matrizes}
    \inputcode{cpp}{fast_exp_recursive.cpp}
\end{frame}

\begin{frame}[fragile]{Implementação iterativa da exponencição rápida de matrizes}
    \inputcode{cpp}{fast_exp.cpp}
\end{frame}

\begin{frame}[fragile]{Solução AC com complexidade $O(T\log n)$}
    \inputsnippet{cpp}{1}{21}{matrix.cpp}
\end{frame}

\begin{frame}[fragile]{Solução AC com complexidade $O(T\log n)$}
    \inputsnippet{cpp}{22}{42}{matrix.cpp}
\end{frame}

\begin{frame}[fragile]{Solução AC com complexidade $O(T\log n)$}
    \inputsnippet{cpp}{43}{63}{matrix.cpp}
\end{frame}

\begin{frame}[fragile]{Relação entre números de Fibonacci}

    \begin{itemize}
        \item Da igualdade
        \[
            A^n = \begin{bmatrix} F_{n + 1} & F_n \\ F_n & F_{n - 1}\end{bmatrix} 
        \]
        e do fato de que $A^{m + n} = A^mA^n$ se que
        \[
            A^{n + m} = \begin{bmatrix} F_{m + 1} & F_m \\ F_m & F_{m - 1}\end{bmatrix} 
            \begin{bmatrix} F_{n + 1} & F_n \\ F_n & F_{n - 1}\end{bmatrix} 
        \]

        \item Daí
        \[
            F_{n + m} = F_{m + 1}F_n + F_mF_{n - 1}
        \]
        e
        \[
            F_{n + m - 1} = F_mF_n + F_{m - 1}F_{n - 1}
        \]
    \end{itemize}

\end{frame}

\begin{frame}[fragile]{Relação entre números de Fibonacci}

    \begin{itemize}
        \item Fazendo $n = 2k$ (isto é, $n$ é par), tem-se que
        \begin{align*}
            F_n &= F_{2k} = F_{k + k} = F_{k + 1}F_k + F_kF_{k - 1} \\
            &= (F_k + F_{k - 1})F_k + F_kF_{k - 1} \\
            &= (2F_{k - 1} + F_k)F_k
        \end{align*}

        \item Se $n = 2k - 1$, (isto é, $n$ é ímpar), 
        \[
            F_n = F_{2k - 1} = F_{k + k - 1} = F_{k}^2 + F_{k - 1}^2
        \]

        \item Estas identidades também permitem computar o $n$-ésimo número de Fibonacci com
            complexidade $O(n)$, porém com uma codificação mais curta em relação à exponenciação
            matricial rápida

        \item Note que o número de valores intermediários a serem computados é inferior a
            $4\log n$
    \end{itemize}

\end{frame}

\begin{frame}[fragile]{Solução $O(T\log n)$}
    \inputsnippet{cpp}{1}{21}{cf.cpp}
\end{frame}

\begin{frame}[fragile]{Solução $O(T\log n)$}
    \inputsnippet{cpp}{22}{42}{cf.cpp}
\end{frame}
