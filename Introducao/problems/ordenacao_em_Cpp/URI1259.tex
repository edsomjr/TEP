\section{URI 1259 -- Pares e Ímpares}

\begin{frame}[fragile]{Problema}

Considerando a entrada de valores inteiros não negativos, ordene estes valores segundo o seguinte critério:

\begin{itemize}
    \item Primeiro os Pares
    \item Depois os Ímpares
\end{itemize}

Sendo que deverão ser apresentados os pares em ordem crescente e depois os ímpares em ordem decrescente.

\end{frame}

\begin{frame}[fragile]{Entrada e saída}

\textbf{Entrada}

A primeira linha de entrada contém um único inteiro positivo $N$ $(1 < N < 10^5)$. 
Este é o número de linhas de entrada que vem logo a seguir. As próximas $N$ linhas conterão, cada uma delas, um valor inteiro não negativo.

\vspace{0.1in}

\textbf{Saída}

Apresente todos os valores lidos na entrada segundo a ordem apresentada acima. Cada número deve ser impresso em uma linha, conforme exemplo abaixo.

\end{frame}

\begin{frame}[fragile]{Exemplo de entradas e saídas}

\begin{minipage}[t]{0.5\textwidth}
\textbf{Exemplo de Entrada}
\begin{verbatim}
10
4
32
34
543
3456
654
567
87
6789
98
\end{verbatim}
\end{minipage}
\begin{minipage}[t]{0.45\textwidth}
\textbf{Exemplo de Saída}
\begin{verbatim}
4
32
34
98
654
3456
6789
567
543
87
\end{verbatim}
\end{minipage}
\end{frame}

\begin{frame}[fragile]{Solução com complexidade $O(N\log N)$}

    \begin{itemize}
        \item O problema pode ser resolvido armazenando-se os pares e ímpares em
            vetores distintos, e aplicando a ordenação correspondente em cada um destes
            vetores

        \item Contudo, é possível ordenar todo o vetor de uma só vez, através da
            escrita de um comparador customizado

        \item A paridade de um número $n$ pode ser obtida através do resto da divisão por 2

        \item Se $a$ e $b$ tem paridades distintas, $a$ será menor do que $b$ se for par:
            logo basta comparar as paridades (pois zero significa par, um significa ímpar)

        \item Se $a$ e $b$ tem mesma paridade, a comparação no caso par é $a < b$; no caso 
            ímpar, $a > b$

        \item Por conta da chamada da função \code{c}{sort()}, o algoritmo tem complexidade
            $O(N\log N)$

    \end{itemize}

\end{frame}

\begin{frame}[fragile]{Solução $O(N\log N)$}
    \inputsnippet{cpp}{1}{15}{codes/1259.cpp}
\end{frame}

\begin{frame}[fragile]{Solução $O(N\log N)$}
    \inputsnippet{cpp}{16}{36}{codes/1259.cpp}
\end{frame}
