\section{AtCoder Beginner Contest 106 -- Problem C: To Infinity}

\begin{frame}[fragile]{Problema}

Mr. Infinity has a string $S$ consisting of digits from 1 to 9. Each time the date changes, this 
string changes as follows:

\begin{itemize}
    \item Each occurrence of \textcolor{red}{'2'} in S is replaced with \textcolor{red}{'22'}. Similarly, each \textcolor{red}{'3'} becomes \textcolor{red}{'333'}, \textcolor{red}{'4'} becomes \textcolor{red}{'4444'}, \textcolor{red}{'5'} becomes \textcolor{red}{'55555'}, \textcolor{red}{'6'} becomes \textcolor{red}{'666666'}, \textcolor{red}{'7'} becomes \textcolor{red}{'7777777'}, \textcolor{red}{'8'} becomes \textcolor{red}{'88888888'} and \textcolor{red}{'9'} becomes \textcolor{red}{'999999999'}. \textcolor{red}{'1'} remains as \textcolor{red}{'1'}.
\end{itemize}

For example, if $S$ is \textcolor{red}{'1324'}, it becomes \textcolor{red}{'1333224444'} the next day, and it becomes \textcolor{red}{'133333333322224444444444444444'} the day after next. You are interested in what the string looks like after $5\times 10^{15}$ days. What is the $K$-th character 
from the left in the string after $5\times 10^{15}$ days?

\end{frame}

\begin{frame}[fragile]{Entrada e saída}

\textbf{Constraints}

\begin{itemize}
    \item $2\leq K\leq N\leq 100000$
    \item $A_1, A_2, \ldots, A_N$ is a permutation of $1, 2, \ldots, N$
\end{itemize}

\textbf{Input}

Input is given from Standard Input in the following format:
\begin{align*}
&N\ \ K \\
&A_1\ \ A_2\ \ \ldots\ \ A_N
\end{align*}

\textbf{Output}

Print the $K$-th character from the left in Mr. Infinity's string after $5\times 10^{15}$ days.

\end{frame}

\begin{frame}[fragile]{Exemplo de entradas e saídas}

\begin{minipage}[t]{0.5\textwidth}
\textbf{Exemplo de Entrada}
\begin{verbatim}
1214
4

3
157

299792458
9460730472580800
\end{verbatim}
\end{minipage}
\begin{minipage}[t]{0.45\textwidth}
\textbf{Exemplo de Saída}
\begin{verbatim}
2


3


2
\end{verbatim}
\end{minipage}
\end{frame}

\begin{frame}[fragile]{Solução}

    \begin{itemize}
        \item O primeiro fato a ser observado é, se o número inicial $x$ da string for diferente de 
        1, ele será replicado $x^{5\times 10^{15}}$ vezes, de modo que a resposta será o próprio $x$

        \item O caso especial ocorre quando a string é prefixada por uma sequência de 1s

        \item Se a quantidade de uns for maior ou igual a $K$, a resposta será igual a 1

        \item Caso contrário, a resposta será igual ao primeiro caractere da string diferente de
            1
    \end{itemize}

\end{frame}

\begin{frame}[fragile]{Solução AC}
    \inputsnippet{cpp}{1}{19}{codes/B106C.cpp}
\end{frame}

\begin{frame}[fragile]{Solução AC}
    \inputsnippet{cpp}{20}{40}{codes/B106C.cpp}
\end{frame}
