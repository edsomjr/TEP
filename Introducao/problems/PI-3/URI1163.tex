\section{URI 1163 -- Angry Ducks}

\begin{frame}[fragile]{Problema}

Em uma terra distante existem duas cidades, a Nlogônia onde vivem os Nlogoneses, e Ducklogônia onde vivem seus vizinhos os Duckneses, já à algum tempo estas duas cidades estão em guerra e agora em uma tentativa de ganhar a guerra os Duckneses pretendem atacar a cidade da Nlogônia com um bodoque que atira patos, porem para que não haja erro eles pediram que você construa um programa que dados os valores da altura do bodoque ($h$), os pontos onde inicia ($p1$) e onde termina ($p2$) a cidade da Nlogônia, o ângulo do disparo ($\alpha$) e a velocidade do lançamento, calcule se o projetil atingira o alvo.

Para os cálculos assuma que a aceleração da gravidade é $g$ = 9.80665 e que $\pi$ = 3.14159.
\end{frame}

\begin{frame}[fragile]{Problema}
    \includegraphics[scale=0.25,center]{UOJ_1163.jpg}
\end{frame}

\begin{frame}[fragile]{Entrada e saída}

\textbf{Entrada}

Existem vários casos de teste, cada caso inicia com 1 valor de ponto flutuante $h$ $(1 \leq h \leq 150)$ indicando a altura do bodoque, a próxima linha contem 2 valores inteiros $p1$ e $p2$ $(1 \leq p1, p2 \leq 9999)$ indicando onde inicia e onde termina a Nlogônia, a linha seguinte contem um inteiro $n$ $(1 \leq n \leq 100)$ indicando o numero de tentativas que serão feitas para acertar a Nlogônia, as $n$ linhas seguintes contem dois valores de ponto flutuante com os valores do ângulo $\alpha$ $(1 \leq \alpha \leq 180)$ e a velocidade $V$ $(1 \leq V \leq 150)$ do disparo.

O final do arquivo de entrada é determinado por EOF.
\end{frame}

\begin{frame}[fragile]{Entrada e saída}
\textbf{Saída}

Para cada disparo, seu programa deve imprimir uma única linha no seguinte formato, \lq\lq\texttt{X -> DUCK}” para quando o pato acertar a Nlogônia ou \lq\lq\texttt{X -> NUCK}” quando o pato não acertar a Nlogônia, onde $X$ eh a distancia máxima que o projetil atingiu até chegar ao chão ($Y=0$). $X$ deve ser impresso com 5 casas decimais.

\end{frame}

\begin{frame}[fragile]{Exemplo de entradas e saídas}

\begin{minipage}[t]{0.5\textwidth}
\textbf{Exemplo de Entrada}
\begin{verbatim}
2.1
368 380
3
42.3 60
30 55
44 60.876842
\end{verbatim}
\end{minipage}
\begin{minipage}[t]{0.45\textwidth}
\textbf{Exemplo de Saída}
\begin{verbatim}
367.76208 -> NUCK
270.72675 -> NUCK
379.83781 -> DUCK
\end{verbatim}
\end{minipage}
\end{frame}

\begin{frame}[fragile]{Solução}

    \begin{itemize}
        \item Embora não dito explicitamento no texto, a altura $y$ do pato segue uma trajetória
            quadrática

        \item De acordo com o movimento uniformemente variado, a altura $y$ do pato no instante
            $t$ é igual a
        \[
            y(t) = h + v_yt - \frac{g}{2}{t^2},
        \] onde $h$ é a altura inicial e $v_y$ é a componente $y$ da vetor velocidade, isto é,
        \[
            v_y = v\sin\left(\frac{\alpha\pi}{180}\right)
        \]

        \item Já o movimento em $x$ é linear

        \item De acordo com o movimento uniforme,
        \[
            x(t) = v_xt,
        \] onde $v_x$ é a componente $x$ do vetor velocidade, isto é,
        \[
            v_x = v\cos\left(\frac{\alpha\pi}{180}\right)
        \]

    \end{itemize}

\end{frame}

\begin{frame}[fragile]{Solução}

    \begin{itemize}
        \item Assim, basta usar a fórmula de Báskara para determinar o instante $\hat{t}$ onde
            a altura $y(t)$ é igual a zero:
        \[
            \hat{t} = \frac{-v_y \pm \sqrt{v_y^2 + 2hg}}{2}
        \]

        \item Observe que, dentre as duas soluções possíveis, deve ser escolhida a positiva

        \item A distância percorrida será $x(\hat{t})$ metros

        \item O alvo será atingido se $p1 \leq x(\hat{t})\leq p2$
    \end{itemize}

\end{frame}

\begin{frame}[fragile]{Solução AC}
    \inputsnippet{cpp}{1}{21}{1163.cpp}
\end{frame}

\begin{frame}[fragile]{Solução AC}
    \inputsnippet{cpp}{22}{42}{1163.cpp}
\end{frame}

\begin{frame}[fragile]{Solução AC}
    \inputsnippet{cpp}{43}{63}{1163.cpp}
\end{frame}
