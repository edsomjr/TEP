\section{UVA 100 -- The $3n + 1$ problem}

\begin{frame}[fragile]{Problema}

Problems in Computer Science are often classified as belonging to a certain class of problems (e.g.,
NP, Unsolvable, Recursive). In this problem you will be analyzing a property of an algorithm whose
classification is not known for all possible inputs.

Consider the following algorithm:

\begin{enumerate}
\item input $n$
\item print $n$
\item if $n = 1$ then STOP
\item if $n$ is odd then $n \leftarrow 3n + 1$
\item else $n \leftarrow n/2$
\item GOTO 2
\end{enumerate}

Given the input 22, the following sequence of numbers will be printed
\[
22\ \ 11\ \ 34\ \ 17\ \ 52\ \ 26\ \ 13\ \ 40\ \ 20\ \ 10\ \ 5\ \ 16\ \ 8\ \ 4\ \ 2\ \ 1
\]
\end{frame}

\begin{frame}[fragile]{Problema}
It is conjectured that the algorithm above will terminate (when a 1 is printed) for any integral input
value. Despite the simplicity of the algorithm, it is unknown whether this conjecture is true. It has
been verified, however, for all integers n such that $0 < n < 1,000,000$ (and, in fact, for many more
numbers than this.)

Given an input $n$, it is possible to determine the number of numbers printed before and including
the 1 is printed. For a given $n$ this is called the \textit{cycle-length} of $n$. In the example above, the cycle
length of 22 is 16.

For any two numbers $i$ and $j$ you are to determine the maximum cycle length over all numbers
between and including both $i$ and $j$.
\end{frame}

\begin{frame}[fragile]{Entrada e saída}

\textbf{Input}

The input will consist of a series of pairs of integers $i$ and $j$, one pair of integers per line. All integers will be less than 10,000 and greater than 0.

You should process all pairs of integers and for each pair determine the maximum cycle length over
all integers between and including $i$ and $j$.

You can assume that no operation overflows a 32-bit integer.

\vspace{0.1in}

\textbf{Output}

For each pair of input integers $i$ and $j$ you should output $i, j$, and the maximum cycle length 
for integers between and including $i$ and $j$. These three numbers should be separated by at 
least one space with all three numbers on one line and with one line of output for each line of 
input. The integers $i$ and $j$ must
appear in the output in the same order in which they appeared in the input and should be followed by
the maximum cycle length (on the same line).

\end{frame}

\begin{frame}[fragile]{Exemplo de entradas e saídas}

\begin{minipage}[t]{0.5\textwidth}
\textbf{Exemplo de Entrada}
\begin{verbatim}
1 10
100 200
201 210
900 1000
\end{verbatim}
\end{minipage}
\begin{minipage}[t]{0.45\textwidth}
\textbf{Exemplo de Saída}
\begin{verbatim}
1 10 20
100 200 125
201 210 89
900 1000 174
\end{verbatim}
\end{minipage}
\end{frame}

\begin{frame}[fragile]{Solução}

    \begin{itemize}
        \item A solução para este problema consiste em quatro etapas

        \item A primeira é ler a entrada do problema

        \item A segunda é codificar uma rotina que computa o \textit{cycle-length} de um inteiro
            positivo $n$

        \item A terceira é computar o \textit{cycle-length} de todos os números entre $i$ e $j$

        \item Por fim, produzir a saída correta 

        \item A segunda etapa é o cerne do problema, e efetivamente a implementação é a réplica
            exata do algoritmo dado no problema
   \end{itemize}

\end{frame}

\begin{frame}[fragile]{Solução WA}
    \inputsnippet{cpp}{1}{21}{wa.cpp}
\end{frame}

\begin{frame}[fragile]{Solução WA}
    \inputsnippet{cpp}{22}{42}{wa.cpp}
\end{frame}

\begin{frame}[fragile]{Solução correta}

    \begin{itemize}
        \item A solução anterior, embora aparentemente correta, tem veredito WA!

        \item Importante notar que o problema não está na rotina que computa do
            \textit{cycle-length}

        \item O erro advém de uma leitura sem o rigor devido da entrada: não há, na descrição
            da entrada, a garantia de que $i$ e $j$ estejam em ordem crescente!

        \item Por exemplo, a entrada \texttt{10 1 20} produziria uma saída errada!

        \item Para corrigir este problema é preciso garantir a ordem crescente dos limites
            no laço do intervalo

        \item Porém, é preciso preservar os valores de $i$ e $j$ para produzir a saída na 
            ordem correta
    \end{itemize}

\end{frame}

\begin{frame}[fragile]{Solução AC}
    \inputsnippet{cpp}{1}{21}{ac.cpp}
\end{frame}

\begin{frame}[fragile]{Solução AC}
    \inputsnippet{cpp}{22}{42}{ac.cpp}
\end{frame}
