\section{OJ 10810 -- Ultra-QuickSort}

\begin{frame}[fragile]{Problema}

In this problem, you have to analyze a particular sorting algorithm. The algorithm
processes a sequence of $n$ distinct integers by swapping two adjacent sequence elements until 
the sequence is sorted in ascending order. For the input sequence
\[
9\ 1\ 0\ 5\ 4,
\]
Ultra-QuickSort produces the output
\[
0\ 1\ 4\ 5\ 9.
\]
Your task is to determine how many swap operations Ultra-QuickSort needs to
perform in order to sort a given input sequence.

\end{frame}

\begin{frame}[fragile]{Entrada e saída}

\textbf{Input}

The input contains several test cases. Every test case begins with a line that
contains a single integer $n < 500\ 000$ -- the length of the input sequence. Each of
the the following n lines contains a single integer $0\leq a[i]\leq 999\ 999\ 999$, the $i$-th
input sequence element. Input is terminated by a sequence of length $n = 0$. This
sequence must not be processed.

\textbf{Output}

For every input sequence, your program prints a single line containing an integer
number $op$, the minimum number of swap operations necessary to sort the given
input sequence.

\end{frame}

\begin{frame}[fragile]{Exemplo de entradas e saídas}

\begin{minipage}[t]{0.5\textwidth}
\textbf{Exemplo de Entrada}
\begin{verbatim}
5
9
1
0
5
4
3
1
2
3
0
\end{verbatim}
\end{minipage}
\begin{minipage}[t]{0.45\textwidth}
\textbf{Exemplo de Saída}
\begin{verbatim}
6
0
\end{verbatim}
\end{minipage}
\end{frame}

\begin{frame}[fragile]{Solução $O(N\log N)$}

    \begin{itemize}
        \item Este problema é idêntico ao anterior, com algumas diferenças pontuais

        \item A primeira diferença está na entrada, que termina com $N = 0$, ao invés de EOF

        \item A saída também é diferente, sem nenhuma outra informação a ser impressa além do
            número de inversões

        \item Como o número de elementos do vetor pode chegare a  $10^5$, o algoritmo quadrático
            leva ao TLE

        \item Exceto os pontos apresentados, ambas soluções são idênticas
   \end{itemize}

\end{frame}

\begin{frame}[fragile]{Solução $O(N\log N)$}
    \inputsnippet{cpp}{1}{21}{codes/10810.cpp}
\end{frame}

\begin{frame}[fragile]{Solução $O(N\log N)$}
    \inputsnippet{cpp}{22}{42}{codes/10810.cpp}
\end{frame}

\begin{frame}[fragile]{Solução $O(N\log N)$}
    \inputsnippet{cpp}{43}{61}{codes/10810.cpp}
\end{frame}

\begin{frame}[fragile]{Solução $O(N\log N)$}
    \inputsnippet{cpp}{62}{82}{codes/10810.cpp}
\end{frame}
