\section{Definição de ponto}

\begin{frame}[fragile]{Definição}

    \begin{itemize}
        \item Ponto é um termo primitivo da geometria

        \item No primeiro livro dos elementos, Euclides define ponto como 
            \lq\lq \textit{a point is that which has no part}\rq\rq, que numa tradução livre 
            diz que \lq\lq ponto é aquilo de que nada é parte\rq\rq

        \item Os pontos são elementos adimensionais, isto é, não tem dimensão alguma

        \item Em C/C++, pontos podem ser representados como classes ou estruturas, como
            pares ou como vetores

        \item Cada representação possível tem suas vantagens e desvantagens
   \end{itemize}

\end{frame}

\begin{frame}[fragile]{Representação de pontos como classes}

    \begin{itemize}
        \item Representar um ponto utilizando uma classe ou estrutura tem a vantagem da 
            legibilidade 

        \item A sintaxe para o uso é mais natural

        \item Porém esta representação demanda a implementação dos operadores relacionais 
            para comparações entre pontos

        \item O uso de estruturas simplifica a implementação, uma vez que em competição não há
            necessidade de encapsulamento dos membros

        \item O tipo usado para representar os valores das coordenadas pode ser parametrizado,
            permitindo o uso da mesma implementação seja com inteiros, seja com pontos
            flutuantes
    \end{itemize}

\end{frame}


\begin{frame}[fragile]{Exemplo de implementação de ponto usando uma estrutura}

    \inputcode{c++}{struct.cpp}

\end{frame}


\begin{frame}[fragile]{Representação de pontos como pares}

    \begin{itemize}
        \item A biblioteca padrão do C++ possui o tipo paramêtrico \code{c}{pair}, que pode
            ser usado para representar pontos

        \item Usar pares do C++ tem como vantagem herdar os operadores de comparação dos
            tipos escolhidos

        \item A desvantagem é a notação, que utiliza \texttt{first} e \texttt{second} para
            acessar as duas coordenadas, ao invés de \code{c}{x} e \code{c}{y}, como nas classes e
estruturas 

        \item Esta desvantagem pode ser contornada usando se as seguintes definições de 
            pré-processador:

        \inputcode{c++}{macros.cpp}

        \item Esta solução deve ser utilizada com cuidado, pois pode gerar efeitos colaterias
            indesejados

        \item Além disso, pares podem ser utilizados diretamente para representar pontos tridimensionais
    \end{itemize}

\end{frame}

\begin{frame}[fragile]{Exemplo de implementação de ponto usando pares}
    \inputcode{cpp}{pairs.cpp}
\end{frame}

\begin{frame}[fragile]{Representação de pontos como vetores}

    \begin{itemize}
        \item Representar pontos como vetores bidimensionais permite a travessia de conjuntos
            de pontos em laços por coordenada

        \item Além disso, é a implementação mais curta para pontos multidimensionais

        \item Porém, a legibilidade fica comprometida, uma vez que as coordenadas são acessadas
            por índices

        \item Esta representação não herda a atribuição, e ainda pode gerar 
            confusão com o uso de operadores relacionais
    \end{itemize}

\end{frame}

\begin{frame}[fragile]{Exemplo de implementação de ponto usando vetores}

    \inputcode{c++}{arrays.cpp}

\end{frame}
