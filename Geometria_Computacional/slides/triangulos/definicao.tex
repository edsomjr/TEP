\section{Definição de triângulo}

\begin{frame}[fragile]{Definição}

    \begin{itemize}
        \item Um triângulo é uma figura geométrica fechada composta por três pontos não-colineares, 
            denominados vértices, e três segmentos de retas formados por todos pares possíveis entre 
            estes pontos, denominados arestas ou lados
        \pause

        \item A cada vértice está associado um ângulo, definido pelos dois segmentos de reta
            dos quais o vértice é um dos extremos
        \pause

        \item O triângulo é o mais simples dentre os polígonos, mas possui uma série de
            características e propriedades notáveis
        \pause

    \end{itemize}

    \begin{figure}
        \centering

        \begin{tikzpicture}
            \draw (0, 0) -- (4, 3) -- (5, 1) -- (0, 0);
            \draw[fill] (0, 0) circle [radius=1pt];
            \draw[fill] (4, 3) circle [radius=1pt];
            \draw[fill] (5, 1) circle [radius=1pt];

        \end{tikzpicture}

    \end{figure}
\end{frame}

\begin{frame}[fragile]{Representação de um triângulo}

    \begin{itemize}
        \item Um triângulo pode ser representado pelas coordenadas de seus vértices
        \pause

        \item Outra alternativa é representar o triângulo pelos tamanhos de suas arestas
        \pause

        \item É possível deduzir estes tamanhos a partir da primeira representação
        \pause

        \item Porém há infinitas possibilidades de coordenadas que satisfaçam as três medidas,
            uma vez que translações e rotações preservam tais valores
        \pause

        \item A representação por vértices pode incluir a representação de um triângulo degenerado
            (quando os três pontos são colineares)
        \pause

        \item A representação por medidas inclui mais casos especiais: pode ser que tais
            medidas não formem um triângulo
        \pause

        \item A Desigualdade Triangular diz que, se $a, b, c$ são números reais, eles serão
            medidas dos lados de um triângulo se, e somente se,
        \[
            a \leq b + c,\ \ \ \ b \leq a + c,\ \ \ \ c\leq a + b
        \]
    \end{itemize}

\end{frame}

\begin{frame}[fragile]{Exemplo de representação do triângulo pelos vértices}
    \inputcode{cpp}{codes/triangle.cpp}
\end{frame}
