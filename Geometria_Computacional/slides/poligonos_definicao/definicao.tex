\section{Definição}

\begin{frame}[fragile]{Definição de polígono}

    \begin{itemize}
        \item Polígonos são figuras planas delimitadas por caminhos fechados 
            (o vértice de partida é o vértice de chegada), compostos por segmentos de retas que 
            unem vértices consecutivos
        \pause

        \item Os segmentos que unem os vértices são denominados arestas
        \pause

        \item Embora alguns polígonos especiais (triângulos, quadriláteros) possam ter tratamento
            especial, os algoritmos de polígonos podem ser aplicados igualmente a estes entes
            geométricos
        \pause
    \end{itemize}

    \begin{figure}
        \centering

        \begin{tikzpicture}
            \coordinate (A) at (0, 0);
            \coordinate (B) at (2, 1);
            \coordinate (C) at (3, 4);
            \coordinate (D) at (5, 2);
            \coordinate (E) at (4, 0);

            \draw (A) -- (B) -- (C) -- (D) -- (E) -- (A);

            \coordinate (X) at (6, 1);
            \coordinate (Y) at (9, 3);
            \coordinate (Z) at (9, 1);
            \coordinate (W) at (6, 3);

            \draw (X) -- (Y) -- (Z) -- (W) -- (X);

            \fill (A) circle [radius=1pt];
            \fill (B) circle [radius=1pt];
            \fill (C) circle [radius=1pt];
            \fill (D) circle [radius=1pt];
            \fill (E) circle [radius=1pt];
            \fill (X) circle [radius=1pt];
            \fill (Y) circle [radius=1pt];
            \fill (Z) circle [radius=1pt];
            \fill (W) circle [radius=1pt];

        \end{tikzpicture}
    \end{figure}

\end{frame}

\begin{frame}[fragile]{Representação de polígonos}

    \begin{itemize}
        \item A representação mais comum de um polígono é a listagem de seus vértices, sendo que 
            as arestas ficam subentendidas (há sempre uma aresta unindo dois vértice consecutivos) 
        \pause

        \item Para facilitar a implementação de algumas rotinas, pode ser conveniente inserir, 
            ao final da lista, o ponto de partida
        \pause

        \item É preciso tomar cuidado: ao fazer isso, o 
            número de vértices do polígono passa a ser o número de elementos da lista subtraído 
            de uma unidade    
        \pause

        \inputsyntax{cpp}{codes/poly.cpp}
        \pause

        \item Esta implementação é a mais compacta possível, mas requer atenção a questão do 
            número de vértices, conforme já comentado
        \pause

        \item Uma implementação mais extensa evita os problemas já mencionados
    \end{itemize}

\end{frame}

\begin{frame}[fragile]{Exemplo de implementação de um polígono em C++}
    \inputsnippet{cpp}{1}{19}{codes/polygon.cpp}
\end{frame}

\begin{frame}[fragile]{Polígonos côncavos e convexos}

    \begin{itemize}
        \item Um polígono é dito convexo se, para quaisquer dois pontos $P$ e $Q$ localizados no 
            interior do polígono, o segmento de reta $PQ$ não intercepta nenhuma das arestas do 
            polígono
        \pause

        \item Caso contrário, o polígono é dito côncavo
        \pause

        \item É possível determinar se um polígono é ou não convexo sem recorrer à busca completa,
            isto é, testar todos os possíveis pares de pontos interiores ao polígono
        \pause

        \item A orientação $D$, entre pontos e reta, pode ser utilizada para tal fim
        \pause

        \item Basta checar se, para quaisquer três pontos consecutivos do polígono, eles tem a 
            mesma orientação: ou sempre à esquerda, ou sempre à direita
    \end{itemize}

\end{frame}

\begin{frame}[fragile]{Implementação da rotina de verificação de convexidade}
        \inputsnippet{cpp}{21}{40}{codes/polygon.cpp}
\end{frame}
