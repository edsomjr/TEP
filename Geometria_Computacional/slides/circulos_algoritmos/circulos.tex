\section{Relação entre dois círculos}

\begin{frame}[fragile]{Interseção entre dois círculos}

    \begin{itemize}
        \item Dados dois círculos com centros $C_1, C_2$ e raios $r_1, r_2$, existem cinco 
        cenários possíveis para suas interseções 
        \pause

        \item Seja $d = \mathrm{dist}(C_1, C_2)$. Então:
        \pause

        \begin{enumerate}
            \item se $d > r_1 + r_2$, então os círculos não se interceptam
        \pause
            \item se $d < |r_1 - r_2|$, então também não há interseção, pois um dos círculos 
                (o de menor raio) está contido no outro (o de maior raio)
        \pause
            \item se $d = 0$ e $r_1 = r_2$, então os círculos são idênticos: há infinitos pontos de 
                interseção
        \pause
            \item se $d = r_1 + r_2$, os círculos se interceptam em um único ponto
        \pause
            \item nos demais casos, há dois pontos na interseção entre os círculos
        \end{enumerate}

    \end{itemize}

\end{frame}

\begin{frame}[fragile]{Par de círculos com dois pontos de interseção}

    \begin{itemize}
        \item Seja $C_1 = (x_1, y_1)$ e $C_2 = (x_2, y_2)$
        \pause

        \item As coordenadas dos pontos de interseção $P_1$ e $P_2$ são dadas por
        \begin{align*}
            a &= \frac{r_1^2 - r_2^2 + d^2}{2d}\\
            h &= \sqrt{r_1^2 - a^2} \\
            P &= C_1 + \frac{a}{d}(C_2 - C_1) \\
            P_1 &= \left(P_x + \frac{h}{d}(y_2 - y_1), P_y - \frac{h}{d}(x_2 - x_1)\right) \\
            P_2 &= \left(P_x - \frac{h}{d}(y_2 - y_1), P_y + \frac{h}{d}(x_2 - x_1)\right)
        \end{align*}
        \pause

        \item A justificativa deste resultado segue de desenvolvimento semelhante ao da 
            identificação de um círculo a partir de dois pontos e um raio
        
    \end{itemize}

\end{frame}

\begin{frame}[fragile]{Implementação da interseção entre dois círculos}
    \inputsnippet{cpp}{1}{18}{codes/intersection.cpp}
\end{frame}

\begin{frame}[fragile]{Implementação da interseção entre dois círculos}
    \inputsnippet{cpp}{20}{42}{codes/intersection.cpp}
\end{frame}
