\section{Paralelepípedos}

\begin{frame}[fragile]{Definição e área}

    \begin{itemize}
        \item Um paralelepípedo é uma figura geométrica formada por seis paralelogramos

        \item Em geral, um paralelepípedo é representado por três vetores linearmente 
            independentes $\vec{u}, \vec{v}, \vec{w}$

        \item Caso estes vetores sejam linearmente dependentes, o paralelepípedo resultante
            é degenerado, tendo volume igual a zero

        \item A área é dada pelas somas das áreas de suas faces

        \item A face delimitada pelos vetores $\vec{u}$ e $\vec{v}$ tem área dada por
        \[
            A = |\vec{u} \times \vec{v}|
        \]

        \item O mesmo vale para as outras faces, usando-se os vetores apropriados
    \end{itemize}

\end{frame}

\begin{frame}[fragile]{Volume}

    \begin{itemize}
        \item O volume é igual ao módulo do produto misto 
        \[
            V = |(\vec{u} \times \vec{v}) \cdot \vec{w}|,
        \]
        onde $x\cdot y$ é o produto interno entre os vetores $x$ e $y$

        \item Na prática, o produto misto é equivalente ao determinante 
        \[
            (\vec{u} \times \vec{v}) \cdot \vec{w}
            = \det \begin{bmatrix}
                u_x & u_y & u_z \\
                v_x & v_y & v_z \\
                w_x & w_y & w_z \\
            \end{bmatrix}
        \]

        \item Se conhecidos apenas os comprimentos $a, b, c$ das arestas e os ângulos internos
            $\alpha, \beta, \gamma$ formados entre elas, é possível computar o volume do paralelogramo
                através da expressão:
            \[
                V = abc\sqrt{1 + 2\cos \alpha \cos \beta \cos \gamma - \cos^2 \alpha - \cos^2 \beta - \cos^2 \gamma}
            \]

    \end{itemize}

\end{frame}

\begin{frame}[fragile]{Implementação do cálculo da área e do volume}
    \inputsnippet{cpp}{1}{21}{volume_par.cpp}
\end{frame}

\begin{frame}[fragile]{Implementação do cálculo da área e do volume}
    \inputsnippet{cpp}{22}{42}{volume_par.cpp}
\end{frame}
