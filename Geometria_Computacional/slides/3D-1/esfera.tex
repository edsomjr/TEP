\section{Esfera}

\begin{frame}[fragile]{Definição de esfera}

    \begin{itemize}
        \item A esfera é o conjunto de pontos do espaço tridimensional que são equidistantes
            de um ponto fixo

        \item Este ponto fixo é denominado centro $C$ da esfera

        \item A distância de um ponto da esfera a $C$ é denominada raio $r$

        \item A esfera por ser representa, em coordenadas cartesianas, pela equação abaixo,
onde ($x_0, y_0$) são as coordenadas do centro e $r$ é o raio
        \[
            (x - x_0)^2 + (y - y_0)^2 + (z - z_0)^2 = r^2
        \]

        \item Pode ser útil, porém, utilizar a representação da esfera em coordenadas esféricas, 
            onde $r$ é o raio, $\theta$ um ângulo que varia de 0 a $2\pi$ e $\varphi$ é um ângulo 
            que varia de 0 a $\pi$:
        \[
            \begin{array}{l}
                x = x_0 + r\cos \theta\sin \varphi \\
                y = y_0 + r\sin \theta\sin \varphi \\
                z = z_0 + r\cos \varphi \\
            \end{array}
        \]

    \end{itemize}

\end{frame}

\begin{frame}[fragile]{Área e Volume}

    \begin{itemize}
        \item A área da superfície da esfera é dada por $A = 4\pi r^2$, onde $r$ é o raio
            da esfera

        \item O seu volume é igual a 
        \[
            V = \frac{4}{3}\pi r^3
        \]

        \item Estas expressões podem ser verificadas através das integrais de área e volume 
            em coordenadas esféricas, cujo jacobiano é dado por $r^2\sin(\varphi)$:
        \[
            A = \int_0^\pi \int_0^{2\pi} r^2\sin(\varphi)\ d\theta d\varphi
        \]
        e
        \[
            V = \int_0^R \int_0^\pi \int_0^{2\pi} r^2\sin(\varphi)\ d\theta d\varphi dR
        \]

    \end{itemize}

\end{frame}

\begin{frame}[fragile]{Implementação do cálculo da área e do volume da esfera}
    \inputcode{cpp}{sphere.cpp}
\end{frame}
