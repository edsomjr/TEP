\section{Comparação entre pontos}

\begin{frame}[fragile]{Operadores relacionais}

    \begin{itemize}
        \item Os operadores relacionais que estarão disponíveis dependem da representação
            escolhida
        \pause

        \item Os operadores fundamentais são a igualdade (operador \code{c}{==}) e a 
            desigualdade (operador \code{c}{!=})
        \pause

        \item Os operadores \code{c}{<} e \code{c}{>} também podem ser definidos, embora a 
            semântica destas comparações dependa do contexto e da implementação utilizada
        \pause

        \item Mesmo no caso dos pares, que herda a igualdade, é importante implementá-la caso
            o tipo utilizado para armazenar as coordenadas seja o ponto flutuante, para que
            seja usado o limiar comentado anteriormente
   \end{itemize}

\end{frame}

\begin{frame}[fragile]{Exemplo de implementação da igualdade}
    \inputsnippet{cpp}{1}{19}{codes/equal.cpp}
\end{frame}

\begin{frame}[fragile]{Exemplo de implementação da igualdade}
    \inputsnippet{cpp}{21}{40}{codes/equal.cpp}
\end{frame}
