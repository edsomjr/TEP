\section{Interseção de segmentos de reta}

\begin{frame}[fragile]{Problema}

    \begin{itemize}
        \item O problema da interseção de segmentos de reta consiste em determinar se, em um 
            conjunto $S$ composto por $N$ segmentos de reta, existe um par de segmentos 
            $r,s \in S$ tal que $r\cap s \neq \emptyset$

        \item Uma variante comum é determinar todos os pontos de interseção entre estes
            segmentos

        \item A solução de busca completa testa cada elemento de $S$ contra todos os demais

        \item Como a interseção entre dois segmentos pode ser obtida em $O(1)$ e existem
            $N(N - 1)/2$ pares de segmentos distintos possíveis, esta abordagem tem
            complexidade $O(N^2)$

        \item Existe um algoritmo com menor complexidade para o problema apresentado, e algoritmos 
            sensíveis à entrada para a variante
    \end{itemize}

\end{frame}

\begin{frame}[fragile]{Algoritmo de Shamos-Hoey}

    \begin{itemize}
        \item Shamos e Hoey propuseram, em 1976, um algoritmo capaz de determinar se existe ao
            menos uma interseção entre $N$ segmentos de reta com complexidade $O(N\log N)$ e
            memória $O(N)$

        \item A ideia é ordenar os $N$ segmentos do conjunto $S$ em ordem lexicográfica e manter
            uma árvore binária balanceada $A$ de segmentos ativos

        \item Cada segmento gera dois eventos: o ponto inicial do segmento gera um evento de 
            inclusão de intervalo (1) e o ponto final do segmento um evento de exclusão do 
            intervalo (2)

        \item A fila dos eventos deve ser ordenadas pelo ponto $P = (x_e, y_e)$ que deu origem 
            ao evento

        \item Para cada evento, a árvore de segmentos ativos $A$ deve estar ordenada pela 
            coordenada $y$ dos pontos dos segmentos com coordenada $x = x_e$ 
    \end{itemize}

\end{frame}
