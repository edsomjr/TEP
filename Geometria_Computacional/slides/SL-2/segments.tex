\section{Interseção de segmentos de reta}

\begin{frame}[fragile]{Problema}

    \begin{itemize}
        \item O problema da interseção de segmentos de reta consiste em determinar se, em um 
            conjunto $S$ composto por $N$ segmentos de reta, existe um par de segmentos 
            $r,s \in S$ tal que $r\cap s \neq \emptyset$

        \item Uma variante comum é determinar todos os pontos de interseção entre estes
            segmentos

        \item A solução de busca completa testa cada elemento de $S$ contra todos os demais

        \item Como a interseção entre dois segmentos pode ser obtida em $O(1)$ e existem
            $N(N - 1)/2$ pares de segmentos distintos possíveis, esta abordagem tem
            complexidade $O(N^2)$

        \item Existe um algoritmo com menor complexidade para o problema apresentado, e algoritmos 
            sensíveis à entrada para a variante
    \end{itemize}

\end{frame}

\begin{frame}[fragile]{Algoritmo de Shamos-Hoey}

    \begin{itemize}
        \item Shamos e Hoey propuseram, em 1976, um algoritmo capaz de determinar se existe ao
            menos uma interseção entre $N$ segmentos de reta com complexidade $O(N\log N)$ e
            memória $O(N)$

        \item A ideia é ordenar os $N$ segmentos do conjunto $S$ em ordem lexicográfica e manter
            uma árvore binária balanceada $A$ de segmentos ativos

        \item Cada segmento gera dois eventos: o ponto inicial do segmento gera um evento de 
            inclusão de intervalo (1) e o ponto final do segmento um evento de exclusão do 
            intervalo (2)

        \item A fila dos eventos deve ser ordenadas pelo ponto $P = (x_e, y_e)$ que deu origem 
            ao evento

        \item Para cada evento, a árvore de segmentos ativos $A$ deve estar ordenada pela 
            coordenada $y$ dos pontos dos segmentos com coordenada $x = x_e$ 
    \end{itemize}

\end{frame}

\begin{frame}[fragile]{Algoritmo de Shamos-Hoey}

    \begin{itemize}
        \item Para manter esta ordenação é necessário utilizar uma árvore binária de busca
            balanceada

        \item Uma alternativa é implementar tal árvore (por exemplo, uma árvore \textit{red-black})

        \item Outra alternativa é utilizar um \code{c}{set} da linguagem C++, em conjunto com
            uma variável global que armazene o valor da coordenada $x$ do evento atual e que
            seja utilizada na rotina de comparação

        \item Observe que, uma vez que um segmento $r$ esteja abaixo de um outro segmento $s$ 
            em um ponto $x$, esta relação só mudará para valores maiores do que $x$ 
            caso exista uma interseção ambos

        \item No caso do algoritmo de Shamos-Hoey a existência de interseção é um critério
            de parada, logo não há necessidade de tratar tais casos 
    \end{itemize}

\end{frame}

\begin{frame}[fragile]{Visualização de identificação de interseção de segmentos}

    \begin{figure}
        \centering

        \begin{tikzpicture}
            \coordinate (A) at (1, 1);
            \coordinate (B) at (5, 4);
            \coordinate (C) at (3, 6);
            \coordinate (D) at (7, 4);
            \coordinate (E) at (2, 3);
            \coordinate (F) at (4, 4);
            \coordinate (G) at (5, 2);
            \coordinate (H) at (9, 5);
            \coordinate (I) at (5, 3);
            \coordinate (J) at (8, 2);

            \fill (A) node[anchor=east] { $A$ } circle [radius=1.5pt];
            \fill (B) node[anchor=west] { $B$ } circle [radius=1.5pt];
            \fill (C) node[anchor=east] { $C$ } circle [radius=1.5pt];
            \fill (D) node[anchor=west] { $D$ } circle [radius=1.5pt];
            \fill (E) node[anchor=east] { $E$ } circle [radius=1.5pt];
            \fill (F) node[anchor=west] { $F$ } circle [radius=1.5pt];
            \fill (G) node[anchor=east] { $G$ } circle [radius=1.5pt];
            \fill (H) node[anchor=west] { $H$ } circle [radius=1.5pt];
            \fill (I) node[anchor=east] { $I$ } circle [radius=1.5pt];
            \fill (J) node[anchor=west] { $J$ } circle [radius=1.5pt];

            \draw (A) -- (B);
            \draw (C) -- (D);
            \draw (E) -- (F);
            \draw (G) -- (H);
            \draw (I) -- (J);

        \end{tikzpicture}
    \end{figure}

\end{frame}

\begin{frame}[fragile]{Visualização de identificação de interseção de segmentos}

    \begin{figure}
        \centering

        \begin{tikzpicture}
            \coordinate (A) at (1, 1);
            \coordinate (B) at (5, 4);
            \coordinate (C) at (3, 6);
            \coordinate (D) at (7, 4);
            \coordinate (E) at (2, 3);
            \coordinate (F) at (4, 4);
            \coordinate (G) at (5, 2);
            \coordinate (H) at (9, 5);
            \coordinate (I) at (5, 3);
            \coordinate (J) at (8, 2);

            \draw[thick,blue] (A) -- (B);
            \draw (C) -- (D);
            \draw (E) -- (F);
            \draw (G) -- (H);
            \draw (I) -- (J);

            \fill (A) node[anchor=east] { $A$ } circle [radius=1.5pt];
            \fill (B) node[anchor=west] { $B$ } circle [radius=1.5pt];
            \fill (C) node[anchor=east] { $C$ } circle [radius=1.5pt];
            \fill (D) node[anchor=west] { $D$ } circle [radius=1.5pt];
            \fill (E) node[anchor=east] { $E$ } circle [radius=1.5pt];
            \fill (F) node[anchor=west] { $F$ } circle [radius=1.5pt];
            \fill (G) node[anchor=east] { $G$ } circle [radius=1.5pt];
            \fill (H) node[anchor=west] { $H$ } circle [radius=1.5pt];
            \fill (I) node[anchor=east] { $I$ } circle [radius=1.5pt];
            \fill (J) node[anchor=west] { $J$ } circle [radius=1.5pt];

            \draw[dashed] (1, 0) -- (1, 7);

        \end{tikzpicture}
    \end{figure}

\end{frame}

\begin{frame}[fragile]{Visualização de identificação de interseção de segmentos}

    \begin{figure}
        \centering

        \begin{tikzpicture}
            \coordinate (A) at (1, 1);
            \coordinate (B) at (5, 4);
            \coordinate (C) at (3, 6);
            \coordinate (D) at (7, 4);
            \coordinate (E) at (2, 3);
            \coordinate (F) at (4, 4);
            \coordinate (G) at (5, 2);
            \coordinate (H) at (9, 5);
            \coordinate (I) at (5, 3);
            \coordinate (J) at (8, 2);

            \draw[thick,red] (A) -- (B);
            \draw (C) -- (D);
            \draw[thick,blue] (E) -- (F);
            \draw (G) -- (H);
            \draw (I) -- (J);

            \fill (A) node[anchor=east] { $A$ } circle [radius=1.5pt];
            \fill (B) node[anchor=west] { $B$ } circle [radius=1.5pt];
            \fill (C) node[anchor=east] { $C$ } circle [radius=1.5pt];
            \fill (D) node[anchor=west] { $D$ } circle [radius=1.5pt];
            \fill (E) node[anchor=east] { $E$ } circle [radius=1.5pt];
            \fill (F) node[anchor=west] { $F$ } circle [radius=1.5pt];
            \fill (G) node[anchor=east] { $G$ } circle [radius=1.5pt];
            \fill (H) node[anchor=west] { $H$ } circle [radius=1.5pt];
            \fill (I) node[anchor=east] { $I$ } circle [radius=1.5pt];
            \fill (J) node[anchor=west] { $J$ } circle [radius=1.5pt];

            \draw[dashed] (2, 0) -- (2, 7);

        \end{tikzpicture}
    \end{figure}

\end{frame}

\begin{frame}[fragile]{Visualização de identificação de interseção de segmentos}

    \begin{figure}
        \centering

        \begin{tikzpicture}
            \coordinate (A) at (1, 1);
            \coordinate (B) at (5, 4);
            \coordinate (C) at (3, 6);
            \coordinate (D) at (7, 4);
            \coordinate (E) at (2, 3);
            \coordinate (F) at (4, 4);
            \coordinate (G) at (5, 2);
            \coordinate (H) at (9, 5);
            \coordinate (I) at (5, 3);
            \coordinate (J) at (8, 2);

            \draw[thick] (A) -- (B);
            \draw[thick,blue] (C) -- (D);
            \draw[thick,red] (E) -- (F);
            \draw (G) -- (H);
            \draw (I) -- (J);

            \fill (A) node[anchor=east] { $A$ } circle [radius=1.5pt];
            \fill (B) node[anchor=west] { $B$ } circle [radius=1.5pt];
            \fill (C) node[anchor=east] { $C$ } circle [radius=1.5pt];
            \fill (D) node[anchor=west] { $D$ } circle [radius=1.5pt];
            \fill (E) node[anchor=east] { $E$ } circle [radius=1.5pt];
            \fill (F) node[anchor=west] { $F$ } circle [radius=1.5pt];
            \fill (G) node[anchor=east] { $G$ } circle [radius=1.5pt];
            \fill (H) node[anchor=west] { $H$ } circle [radius=1.5pt];
            \fill (I) node[anchor=east] { $I$ } circle [radius=1.5pt];
            \fill (J) node[anchor=west] { $J$ } circle [radius=1.5pt];

            \draw[dashed] (3, 0) -- (3, 7);

        \end{tikzpicture}
    \end{figure}

\end{frame}

\begin{frame}[fragile]{Visualização de identificação de interseção de segmentos}

    \begin{figure}
        \centering

        \begin{tikzpicture}
            \coordinate (A) at (1, 1);
            \coordinate (B) at (5, 4);
            \coordinate (C) at (3, 6);
            \coordinate (D) at (7, 4);
            \coordinate (E) at (2, 3);
            \coordinate (F) at (4, 4);
            \coordinate (G) at (5, 2);
            \coordinate (H) at (9, 5);
            \coordinate (I) at (5, 3);
            \coordinate (J) at (8, 2);

            \draw[thick,red] (A) -- (B);
            \draw[thick,green!60!black] (C) -- (D);
            \draw[thick,gray!50] (E) -- (F);
            \draw (G) -- (H);
            \draw (I) -- (J);

            \fill (A) node[anchor=east] { $A$ } circle [radius=1.5pt];
            \fill (B) node[anchor=west] { $B$ } circle [radius=1.5pt];
            \fill (C) node[anchor=east] { $C$ } circle [radius=1.5pt];
            \fill (D) node[anchor=west] { $D$ } circle [radius=1.5pt];
            \fill[gray!50] (E) node[anchor=east] { $E$ } circle [radius=1.5pt];
            \fill[gray!50] (F) node[anchor=west] { $F$ } circle [radius=1.5pt];
            \fill (G) node[anchor=east] { $G$ } circle [radius=1.5pt];
            \fill (H) node[anchor=west] { $H$ } circle [radius=1.5pt];
            \fill (I) node[anchor=east] { $I$ } circle [radius=1.5pt];
            \fill (J) node[anchor=west] { $J$ } circle [radius=1.5pt];

            \draw[dashed] (4, 0) -- (4, 7);

        \end{tikzpicture}
    \end{figure}

\end{frame}

\begin{frame}[fragile]{Visualização de identificação de interseção de segmentos}

    \begin{figure}
        \centering

        \begin{tikzpicture}
            \coordinate (A) at (1, 1);
            \coordinate (B) at (5, 4);
            \coordinate (C) at (3, 6);
            \coordinate (D) at (7, 4);
            \coordinate (E) at (2, 3);
            \coordinate (F) at (4, 4);
            \coordinate (G) at (5, 2);
            \coordinate (H) at (9, 5);
            \coordinate (I) at (5, 3);
            \coordinate (J) at (8, 2);

            \draw[thick,green!60!black] (A) -- (B);
            \draw[thick] (C) -- (D);
            \draw[thick,gray!50] (E) -- (F);
            \draw[thick,blue] (G) -- (H);
            \draw (I) -- (J);

            \fill (A) node[anchor=east] { $A$ } circle [radius=1.5pt];
            \fill (B) node[anchor=west] { $B$ } circle [radius=1.5pt];
            \fill (C) node[anchor=east] { $C$ } circle [radius=1.5pt];
            \fill (D) node[anchor=west] { $D$ } circle [radius=1.5pt];
            \fill[gray!50] (E) node[anchor=east] { $E$ } circle [radius=1.5pt];
            \fill[gray!50] (F) node[anchor=west] { $F$ } circle [radius=1.5pt];
            \fill (G) node[anchor=east] { $G$ } circle [radius=1.5pt];
            \fill (H) node[anchor=west] { $H$ } circle [radius=1.5pt];
            \fill (I) node[anchor=east] { $I$ } circle [radius=1.5pt];
            \fill (J) node[anchor=west] { $J$ } circle [radius=1.5pt];

            \draw[dashed] (5, 0) -- (5, 7);

        \end{tikzpicture}
    \end{figure}

\end{frame}

\begin{frame}[fragile]{Visualização de identificação de interseção de segmentos}

    \begin{figure}
        \centering

        \begin{tikzpicture}
            \coordinate (A) at (1, 1);
            \coordinate (B) at (5, 4);
            \coordinate (C) at (3, 6);
            \coordinate (D) at (7, 4);
            \coordinate (E) at (2, 3);
            \coordinate (F) at (4, 4);
            \coordinate (G) at (5, 2);
            \coordinate (H) at (9, 5);
            \coordinate (I) at (5, 3);
            \coordinate (J) at (8, 2);

            \draw[thick,green!60!black] (A) -- (B);
            \draw[thick] (C) -- (D);
            \draw[thick,gray!50] (E) -- (F);
            \draw[thick,red] (G) -- (H);
            \draw[thick,blue] (I) -- (J);

            \fill (A) node[anchor=east] { $A$ } circle [radius=1.5pt];
            \fill (B) node[anchor=west] { $B$ } circle [radius=1.5pt];
            \fill (C) node[anchor=east] { $C$ } circle [radius=1.5pt];
            \fill (D) node[anchor=west] { $D$ } circle [radius=1.5pt];
            \fill[gray!50] (E) node[anchor=east] { $E$ } circle [radius=1.5pt];
            \fill[gray!50] (F) node[anchor=west] { $F$ } circle [radius=1.5pt];
            \fill (G) node[anchor=east] { $G$ } circle [radius=1.5pt];
            \fill (H) node[anchor=west] { $H$ } circle [radius=1.5pt];
            \fill (I) node[anchor=east] { $I$ } circle [radius=1.5pt];
            \fill (J) node[anchor=west] { $J$ } circle [radius=1.5pt];

            \draw[dashed] (5, 0) -- (5, 7);

        \end{tikzpicture}
    \end{figure}

\end{frame}


\begin{frame}[fragile]{Implementação do algoritmo de Shamos-Hoey}
    \inputsnippet{cpp}{1}{17}{shamos_hoey.cpp}
\end{frame}

\begin{frame}[fragile]{Implementação do algoritmo de Shamos-Hoey}
    \inputsnippet{cpp}{18}{38}{shamos_hoey.cpp}
\end{frame}

\begin{frame}[fragile]{Implementação do algoritmo de Shamos-Hoey}
    \inputsnippet{cpp}{40}{60}{shamos_hoey.cpp}
\end{frame}

\begin{frame}[fragile]{Implementação do algoritmo de Shamos-Hoey}
    \inputsnippet{cpp}{61}{80}{shamos_hoey.cpp}
\end{frame}

\begin{frame}[fragile]{Implementação do algoritmo de Shamos-Hoey}
    \inputsnippet{cpp}{81}{97}{shamos_hoey.cpp}
\end{frame}

\begin{frame}[fragile]{Implementação do algoritmo de Shamos-Hoey}
    \inputsnippet{cpp}{98}{118}{shamos_hoey.cpp}
\end{frame}

\begin{frame}[fragile]{Implementação do algoritmo de Shamos-Hoey}
    \inputsnippet{cpp}{120}{140}{shamos_hoey.cpp}
\end{frame}

\begin{frame}[fragile]{Implementação do algoritmo de Shamos-Hoey}
    \inputsnippet{cpp}{141}{161}{shamos_hoey.cpp}
\end{frame}
































































































































