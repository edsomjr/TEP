\section{Cubo}

\begin{frame}[fragile]{Definição de cubo}

    \begin{itemize}
        \item Um cubo é uma região do espaço delimitada por 6 quadrados idênticos, 
            unidos por meio de suas arestas
        \pause

        \item O cubo é um dos cinco poliedros regulares
        \pause

        \item Os demais são o tetraedro, o octaedro, o dodecaedro e o icosaedro, que também são
            denominados Sólidos de Platão
        \pause

        \item O cubo possui 12 arestas, 6 faces e 8 vértices,
            o que pode ser confirmado pela Equação de Euler
        \[
            V + F - A = 2
        \]
        \pause

        \item A menos de sua posição no espaço, um cubo pode ser unicamente determinado pela
            medida $L$ do lado do quadrado que compõe suas faces
        \pause

        \item A medida da diagonal facial, isto é, a diagonal que une dos vértices opostos de uma 
            mesma face, é igual a $L\sqrt{2}$
        \pause

        \item A medida da diagonal espacial, isto é, a diagonal que une dois vértices opostos,
            atravessando o cubo por seu interior, é dada por $L\sqrt{3}$

    \end{itemize}

\end{frame}

\begin{frame}[fragile]{Exemplo de implementação do cubo}
    \inputcode{cpp}{codes/cube.cpp}
\end{frame}

\begin{frame}[fragile]{Área e volume}

    \begin{itemize}
        \item A área da superfície do cubo (a soma das áreas de suas faces) é igual a
            \[
                A = 6L^2,
            \] uma vez que esta área é composta pela soma das áreas de seis quadrados
            idênticos
        \pause

        \item O volume é dado por 
        \[
            V = L^3
        \]
        \pause

        \item Sendo uma expressão cúbica, é preciso tomar cuidado com um possível \textit{overflow} no cálculo do volume
    \end{itemize}

\end{frame}

\begin{frame}[fragile]{Implementação do cálculo da área e do volume}
    \inputcode{cpp}{codes/area_cube.cpp}
\end{frame}

\begin{frame}[fragile]{Relação entre cubo e esfera}

    \begin{itemize}
        \item O cubo tem três esferas associadas
        \pause

        \item A esfera circunscrita, cujo raio é igual a $L(\sqrt{3}/2)$, passa pelos 8 vértices do 
            cubo
        \pause

        \item A esfera inscrita é tangente às 6 faces do cubo
        \pause

        \item O raio da esfera inscrita é igual a $L/2$
        \pause

        \item A esfera tangente às arestas do cubo tem raio igual a $L/\sqrt{2}$

    \end{itemize}

\end{frame}
