\section{Lugares Geométricos}

\begin{frame}[fragile]{Baricentro}

    \begin{itemize}
        \item Um triângulo possui quatro lugares geométricos notáveis

        \item O primeiro deles é o baricentro (centróide ou centro de massa), que é o ponto de 
            interseção entre as três medianas (segmentos de reta que unem um vértice ao ponto médio do lado oposto)

        \item O baricentro divide uma mediana na proporção de 2:1, isto é, ele está a um terço de 
            distância do lado oposto

        \item As coordenadas do baricentro $Q$ podem ser computadas diretamente a partir das 
            coordenadas dos vértices: serão a média aritmética entre as mesmas
        \[
            Q = \left(\frac{x_A, x_B, x_C}{3}, \frac{y_A + y_B + y_C}{3}\right)
        \]

    \end{itemize}

\end{frame}

\begin{frame}[fragile]{Visualização do baricentro}

    \begin{figure}
        \centering

        \begin{tikzpicture}
            \coordinate (A) at (0, 2);
            \coordinate (B) at (3, 6);
            \coordinate (C) at (9, 1);

            \draw (A) -- (B) -- (C) -- (A);

            \draw[fill] (A) circle [radius=2pt] node[left,inner sep = 5pt] { $A$ };
            \draw[fill] (B) circle [radius=2pt] node[above,inner sep = 5pt] { $B$ };
            \draw[fill] (C) circle [radius=2pt] node[right,inner sep = 5pt] { $C$ };

            \draw[dashed] (A) -- (6, 3.5);
            \draw[dashed] (B) -- (4.5, 1.5);
            \draw[dashed] (C) -- (1.5, 4);

            \draw[fill] (4,3) circle [radius=2pt] node[above right,inner sep = 5pt] { $Q$ };
        \end{tikzpicture}
    \end{figure}

\end{frame}

\begin{frame}[fragile]{Implementação da identificação do baricentro}
    \inputcode{cpp}{barycenter.cpp}
\end{frame}

\begin{frame}[fragile]{Ortocentro}

    \begin{itemize}
        \item O ortocentro de um triângulo é o ponto de encontro de suas três alturas

        \item O ortocentro pode mesmo estar fora do triângulo, no caso de um obtusângulo

        \item No caso de um triângulo retângulo, o ortocentro sempre coincide com o vértice 
            oposto à hipotenusa

        \item Para obter as coordenadas do ortocentro, é preciso determinar, inicialmente,
            as retas $r$ e $s$ que contém os segmentos $AB$ e $AC$,
            respectivamente,

        \item Em seguida, é preciso determinar as retas $u$ e $v$ perpendiculares a 
            $r$ e $s$ que passam por $C$ e $B$, respectivamente

        \item O ortocentro $O$ será a interseção entre $u$ e $v$
    \end{itemize}

\end{frame}

\begin{frame}[fragile]{Visualização do ortocentro}

    \begin{figure}
        \centering

        \begin{tikzpicture}
            \coordinate (A) at (0, 0);
            \coordinate (B) at (3, 6);
            \coordinate (C) at (9, 1);
            \coordinate (O) at (3.23529, 3.88235);
 
            \coordinate (BC) at ($(A)!1.29!(O)$);
            \coordinate (AC) at ($(B)!2.62!(O)$);
            \coordinate (AB) at ($(C)!1.18!(O)$);

            \draw (A) -- (B) -- (C) -- (A);

            \draw[fill] (A) circle [radius=2pt] node[left,inner sep = 5pt] { $A$ };
            \draw[fill] (B) circle [radius=2pt] node[above,inner sep = 5pt] { $B$ };
            \draw[fill] (C) circle [radius=2pt] node[right,inner sep = 5pt] { $C$ };
            \draw[fill] (O) circle [radius=2pt] node[right,inner sep = 7.5pt] { $O$ };

            \draw[dashed] (A) -- (BC);
            \draw[dashed] (B) -- (AC);
            \draw[dashed] (C) -- (AB);
        \end{tikzpicture}
    \end{figure}

\end{frame}


\begin{frame}[fragile]{Implementação da identificação do ortocentro}
    \inputsnippet{cpp}{22}{40}{orthocenter.cpp}
\end{frame}

\begin{frame}[fragile]{Incentro}

    \begin{itemize}
        \item O incentro de um triângulo é o ponto de encontro de suas bissetrizes (retas que dividem um ângulo interno na metade)

        \item Além de ser sempre um ponto interior do triângulo, o incentro é o centro do círculo inscrito no triângulo, isto é, o maior círculo que cabe dentro do triângulo e que toca todos os seus três lados 

        \item Os lados do círculo são tangentes ao círculo inscrito

        \item O raio $r$ do círculo inscrito é dado pela razão entre o dobro da área $A$ e o 
            perímetro $P$

        \item As coordenadas do centro $I$ do círculo inscrito são obtidas pela média ponderada 
            das coordenadas $x$ e $y$ pelos comprimentos dos lados opostos:
        \[
            r = \frac{2A}{P}, I_x = \frac{aA_x + bB_x + cC_x}{P}, I_y = \frac{aA_y + bB_y + cC_y}{P}
        \]
    \end{itemize}

\end{frame}

\begin{frame}[fragile]{Visualização do incentro e o círculo inscrito}

    \begin{figure}
        \centering

        \begin{tikzpicture}
            \coordinate (A) at (0, 0);
            \coordinate (B) at (2, 6);
            \coordinate (C) at (8, 3);
            \coordinate (I) at (3.1369140060129013, 3.2552467874317275);
 
            \coordinate (BC) at ($(A)!1.45!(I)$);
            \coordinate (AC) at ($(B)!1.65!(I)$);
            \coordinate (AB) at ($(C)!1.4!(I)$);

            \draw (A) -- (B) -- (C) -- (A);

            \draw[fill] (A) circle [radius=2pt] node[left,inner sep = 5pt] { $A$ };
            \draw[fill] (B) circle [radius=2pt] node[above,inner sep = 5pt] { $B$ };
            \draw[fill] (C) circle [radius=2pt] node[right,inner sep = 5pt] { $C$ };
            \draw[fill] (I) circle [radius=2pt] node[above,inner sep = 5pt] { $I$ };

            \draw[dashed] (A) -- (BC);
            \draw[dashed] (B) -- (AC);
            \draw[dashed] (C) -- (AB);

            \draw (I) circle [radius=1.9465385055021445];
        \end{tikzpicture}
    \end{figure}

\end{frame}

\begin{frame}[fragile]{Implementação do centro e do raio do círculo inscrito}
    \inputcode{cpp}{incenter.cpp}
\end{frame}

\begin{frame}[fragile]{Circuncentro}

    \begin{itemize}
        \item O circuncentro é o ponto de encontro entre as retas bisectoras perpendiculares (isto é, retas perpendiculares aos lados do triângulo que os interceptam nos pontos médios)

        \item O circuncentro é o centro do círculo circunscrito do triângulo, isto é, o círculo 
            que passa pelos três vértices do triângulo

        \item O circuncentro, assim como o ortocentro, pode estar localizado do lado externo do triângulo

        \item Um caso especial interessante é o do triângulo retângulo, onde o circuncentro se 
            localiza no ponto médio da hipotenusa

        \item O raio $R$ do circuncentro é dado pela razão entre o produto das medidas de seus 
            lados $a, b, c$ e o quádruplo de sua área $A$, isto é
        \[
            R = \frac{abc}{4A}
        \]

    \end{itemize}

\end{frame}
\begin{frame}[fragile]{Circuncentro}

    \begin{itemize}
        \item Seja $|P|$ o tamanho do vetor posição de $P$, isto é
        \[
            |P| = \sqrt{P_x^2 + P_y^2}
        \]

        \item As coordenadas do circuncentro são dadas por
        \[
            S_x = \frac{1}{2d} \begin{vmatrix}
                |A|^2 & A_y & 1 \\
                |B|^2 & B_y & 1 \\
                |C|^2 & C_y & 1 \\
            \end{vmatrix}, S_y = \frac{1}{2d} \begin{vmatrix}
                A_x & |A|^2 & 1 \\
                B_x & |B|^2 & 1 \\
                C_x & |C|^2 & 1 \\
            \end{vmatrix}
        \]
        onde
        \[
            d = \begin{vmatrix}
                A_x & A_y & 1 \\
                B_x & B_y & 1 \\
                C_x & C_y & 1 \\
            \end{vmatrix}
        \]

        \item A reta de Euler é uma reta especial associada ao triângulo, que passa pelo baricentro, ortocentro e circuncentro, que estão sempre alinhados
    \end{itemize}

\end{frame}

\begin{frame}[fragile]{Visualização do circuncentro e o círculo circunscrito}

    \begin{figure}
        \centering

        \begin{tikzpicture}[scale=0.7]
            \coordinate (A) at (0, 0);
            \coordinate (B) at (8, 0);
            \coordinate (C) at (8, 6);
            \coordinate (I) at (4, 3);
 
            \coordinate (BC) at ($(A)!0.5!(B)$);
            \coordinate (AC) at ($(B)!0.5!(C)$);
            \coordinate (AB) at ($(A)!0.5!(B)$);

            \draw (A) -- (B) -- (C) -- (A);

            \draw[fill] (A) circle [radius=2pt] node[left,inner sep = 5pt] { $A$ };
            \draw[fill] (B) circle [radius=2pt] node[below,inner sep = 5pt] { $B$ };
            \draw[fill] (C) circle [radius=2pt] node[right,inner sep = 5pt] { $C$ };
            \draw[fill] (I) circle [radius=2pt] node[above,inner sep = 5pt] { $I$ };

            \draw[dashed] (I) -- ++ (306.87:3.9);
            \draw[dashed] (I) -- (AC);
            \draw[dashed] (I) -- (AB);

            \draw (I) circle [radius=5];
        \end{tikzpicture}
    \end{figure}

\end{frame}

\begin{frame}[fragile]{Implementação do centro e raio do círculo circunscrito}
    \inputsnippet{cpp}{1}{18}{circumcircle.cpp}
\end{frame}

\begin{frame}[fragile]{Implementação do centro e raio do círculo circunscrito}
    \inputsnippet{cpp}{19}{39}{circumcircle.cpp}
\end{frame}


