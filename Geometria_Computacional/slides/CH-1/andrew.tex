\section{Cadeia monótona de Andrew}

\begin{frame}[fragile]{Algoritmo de Andrew}

    \begin{itemize}
        \item O algoritmo conhecido como cadeia monótona de Andrew (\textit{Andrew's Monotone
            Chain Algorithm}, no original) é uma alternativa ao algoritmo de Graham para a
            geração do envoltório convexo

        \item Este algoritmo foi proposto por Andrew em 1979

        \item A complexidade é a mesma do algoritmo de Graham: $O(N\log N)$

        \item Ele constrói o envoltório em duas partes: a parte superior (\textit{upper hull})
            e a parte inferior (\textit{lower hull})

        \item Os pontos são ordenados por coordenada $x$ e, em caso de empate, por coordenada $y$
    \end{itemize}

\end{frame}

\begin{frame}[fragile]{Exemplo de ordenação por coordenadas}

    \begin{figure}
        \centering

        \begin{tikzpicture}
            \coordinate (A) at (0, 0);
            \coordinate (B) at (5, 3);
            \coordinate (C) at (8, -2);
            \coordinate (D) at (4, 4);
            \coordinate (E) at (2, 1);
            \coordinate (F) at (2, 5);
            \coordinate (G) at (3, -1);
            \coordinate (H) at (7, 2);
            \coordinate (I) at (5, 0);
            \coordinate (J) at (0, 4);
            \coordinate (K) at (1, -1);
            \coordinate (L) at (7, -2);
            \coordinate (M) at (6, 4);
            \coordinate (N) at (6, 0);
            \coordinate (O) at (1, 3);

            \fill (A) circle [radius=2pt] node[anchor=west] { \footnotesize \tt 1 };
            \fill (B) circle [radius=2pt] node[anchor=west] { \footnotesize \tt 10 };
            \fill (C) circle [radius=2pt] node[anchor=west] { \footnotesize \tt 15 };
            \fill (D) circle [radius=2pt] node[anchor=west] { \footnotesize \tt 8 };
            \fill (E) circle [radius=2pt] node[anchor=west] { \footnotesize \tt 5 };
            \fill (F) circle [radius=2pt] node[anchor=west] { \footnotesize \tt 6 };
            \fill (G) circle [radius=2pt] node[anchor=west] { \footnotesize \tt 7 };
            \fill (H) circle [radius=2pt] node[anchor=west] { \footnotesize \tt 14 };
            \fill (I) circle [radius=2pt] node[anchor=west] { \footnotesize \tt 9 };
            \fill (J) circle [radius=2pt] node[anchor=west] { \footnotesize \tt 2 };
            \fill (K) circle [radius=2pt] node[anchor=west] { \footnotesize \tt 3 };
            \fill (L) circle [radius=2pt] node[anchor=west] { \footnotesize \tt 13 };
            \fill (M) circle [radius=2pt] node[anchor=west] { \footnotesize \tt 12 };
            \fill (N) circle [radius=2pt] node[anchor=west] { \footnotesize \tt 11 };
            \fill (O) circle [radius=2pt] node[anchor=west] { \footnotesize \tt 4 };

        \end{tikzpicture}
    \end{figure}

\end{frame}


\begin{frame}[fragile]{Implementação da rotina de comparação de pontos}
    \inputsnippet{cpp}{1}{21}{andrew.cpp}
\end{frame}

\begin{frame}[fragile]{Geração do envoltório convexo}

    \begin{itemize}
        \item O envoltório convexo é gerado de forma semelhante ao procedimento usado no 
            algoritmo de Graham

        \item O ponto de partida é o ponto mais à esquerda, com menor coordenada $y$

        \item O \textit{lower hull} é gerado empilhando os pontos de acordo com a ordenação,
            desde que o novo ponto e os dois últimos elementos da pilha mantenham a orientação 
            horária, ou que a pilha tenha menos do que dois elementos

        \item Para gerar o \textit{upper hull}, é preciso começar do ponto mais à direita, com
            maior coordenada $y$

        \item A rotina é idêntica à usado no \textit{lower hull}: basta processar os pontos
            do maior para o menor, de acordo com a ordenação

        \item Ao final as duas partes devem ser unidas

        \item O ponto final do \textit{lower hull} deve ser descartado, uma vez que é idêntico
            ao ponto inicial do \textit{upper hull} 

    \end{itemize}

\end{frame}

\begin{frame}[fragile]{Geração do \textit{lower hull}}

    \begin{figure}
        \centering

        \begin{tikzpicture}
            \coordinate (A) at (0, 0);
            \coordinate (B) at (5, 3);
            \coordinate (C) at (8, -2);
            \coordinate (D) at (4, 4);
            \coordinate (E) at (2, 1);
            \coordinate (F) at (2, 5);
            \coordinate (G) at (3, -1);
            \coordinate (H) at (7, 2);
            \coordinate (I) at (5, 0);
            \coordinate (J) at (0, 4);
            \coordinate (K) at (1, -1);
            \coordinate (L) at (7, -2);
            \coordinate (M) at (6, 4);
            \coordinate (N) at (6, 0);
            \coordinate (O) at (1, 3);

            \fill[color=blue] (A) circle [radius=2pt] node[anchor=west] { \footnotesize \tt 1 };
            \fill (B) circle [radius=2pt] node[anchor=west] { \footnotesize \tt 10 };
            \fill (C) circle [radius=2pt] node[anchor=west] { \footnotesize \tt 15 };
            \fill (D) circle [radius=2pt] node[anchor=west] { \footnotesize \tt 8 };
            \fill (E) circle [radius=2pt] node[anchor=west] { \footnotesize \tt 5 };
            \fill (F) circle [radius=2pt] node[anchor=west] { \footnotesize \tt 6 };
            \fill (G) circle [radius=2pt] node[anchor=west] { \footnotesize \tt 7 };
            \fill (H) circle [radius=2pt] node[anchor=west] { \footnotesize \tt 14 };
            \fill (I) circle [radius=2pt] node[anchor=west] { \footnotesize \tt 9 };
            \fill (J) circle [radius=2pt] node[anchor=west] { \footnotesize \tt 2 };
            \fill (K) circle [radius=2pt] node[anchor=west] { \footnotesize \tt 3 };
            \fill (L) circle [radius=2pt] node[anchor=west] { \footnotesize \tt 13 };
            \fill (M) circle [radius=2pt] node[anchor=west] { \footnotesize \tt 12 };
            \fill (N) circle [radius=2pt] node[anchor=west] { \footnotesize \tt 11 };
            \fill (O) circle [radius=2pt] node[anchor=west] { \footnotesize \tt 4 };

        \end{tikzpicture}
    \end{figure}

\end{frame}

\begin{frame}[fragile]{Geração do \textit{lower hull}}

    \begin{figure}
        \centering

        \begin{tikzpicture}
            \coordinate (A) at (0, 0);
            \coordinate (B) at (5, 3);
            \coordinate (C) at (8, -2);
            \coordinate (D) at (4, 4);
            \coordinate (E) at (2, 1);
            \coordinate (F) at (2, 5);
            \coordinate (G) at (3, -1);
            \coordinate (H) at (7, 2);
            \coordinate (I) at (5, 0);
            \coordinate (J) at (0, 4);
            \coordinate (K) at (1, -1);
            \coordinate (L) at (7, -2);
            \coordinate (M) at (6, 4);
            \coordinate (N) at (6, 0);
            \coordinate (O) at (1, 3);

            \fill (A) circle [radius=2pt] node[anchor=west] { \footnotesize \tt 1 };
            \fill (B) circle [radius=2pt] node[anchor=west] { \footnotesize \tt 10 };
            \fill (C) circle [radius=2pt] node[anchor=west] { \footnotesize \tt 15 };
            \fill (D) circle [radius=2pt] node[anchor=west] { \footnotesize \tt 8 };
            \fill (E) circle [radius=2pt] node[anchor=west] { \footnotesize \tt 5 };
            \fill (F) circle [radius=2pt] node[anchor=west] { \footnotesize \tt 6 };
            \fill (G) circle [radius=2pt] node[anchor=west] { \footnotesize \tt 7 };
            \fill (H) circle [radius=2pt] node[anchor=west] { \footnotesize \tt 14 };
            \fill (I) circle [radius=2pt] node[anchor=west] { \footnotesize \tt 9 };
            \fill (J) circle [radius=2pt] node[anchor=west] { \footnotesize \tt 2 };
            \fill (K) circle [radius=2pt] node[anchor=west] { \footnotesize \tt 3 };
            \fill (L) circle [radius=2pt] node[anchor=west] { \footnotesize \tt 13 };
            \fill (M) circle [radius=2pt] node[anchor=west] { \footnotesize \tt 12 };
            \fill (N) circle [radius=2pt] node[anchor=west] { \footnotesize \tt 11 };
            \fill (O) circle [radius=2pt] node[anchor=west] { \footnotesize \tt 4 };

            \draw[blue,thick] (A) -- (J);
        \end{tikzpicture}
    \end{figure}

\end{frame}

\begin{frame}[fragile]{Geração do \textit{lower hull}}

    \begin{figure}
        \centering

        \begin{tikzpicture}
            \coordinate (A) at (0, 0);
            \coordinate (B) at (5, 3);
            \coordinate (C) at (8, -2);
            \coordinate (D) at (4, 4);
            \coordinate (E) at (2, 1);
            \coordinate (F) at (2, 5);
            \coordinate (G) at (3, -1);
            \coordinate (H) at (7, 2);
            \coordinate (I) at (5, 0);
            \coordinate (J) at (0, 4);
            \coordinate (K) at (1, -1);
            \coordinate (L) at (7, -2);
            \coordinate (M) at (6, 4);
            \coordinate (N) at (6, 0);
            \coordinate (O) at (1, 3);

            \fill (A) circle [radius=2pt] node[anchor=west] { \footnotesize \tt 1 };
            \fill (B) circle [radius=2pt] node[anchor=west] { \footnotesize \tt 10 };
            \fill (C) circle [radius=2pt] node[anchor=west] { \footnotesize \tt 15 };
            \fill (D) circle [radius=2pt] node[anchor=west] { \footnotesize \tt 8 };
            \fill (E) circle [radius=2pt] node[anchor=west] { \footnotesize \tt 5 };
            \fill (F) circle [radius=2pt] node[anchor=west] { \footnotesize \tt 6 };
            \fill (G) circle [radius=2pt] node[anchor=west] { \footnotesize \tt 7 };
            \fill (H) circle [radius=2pt] node[anchor=west] { \footnotesize \tt 14 };
            \fill (I) circle [radius=2pt] node[anchor=west] { \footnotesize \tt 9 };
            \fill (J) circle [radius=2pt] node[anchor=west] { \footnotesize \tt 2 };
            \fill (K) circle [radius=2pt] node[anchor=west] { \footnotesize \tt 3 };
            \fill (L) circle [radius=2pt] node[anchor=west] { \footnotesize \tt 13 };
            \fill (M) circle [radius=2pt] node[anchor=west] { \footnotesize \tt 12 };
            \fill (N) circle [radius=2pt] node[anchor=west] { \footnotesize \tt 11 };
            \fill (O) circle [radius=2pt] node[anchor=west] { \footnotesize \tt 4 };

            \draw[blue,thick] (A) -- (J) -- (K);
        \end{tikzpicture}
    \end{figure}

\end{frame}

\begin{frame}[fragile]{Geração do \textit{lower hull}}

    \begin{figure}
        \centering

        \begin{tikzpicture}
            \coordinate (A) at (0, 0);
            \coordinate (B) at (5, 3);
            \coordinate (C) at (8, -2);
            \coordinate (D) at (4, 4);
            \coordinate (E) at (2, 1);
            \coordinate (F) at (2, 5);
            \coordinate (G) at (3, -1);
            \coordinate (H) at (7, 2);
            \coordinate (I) at (5, 0);
            \coordinate (J) at (0, 4);
            \coordinate (K) at (1, -1);
            \coordinate (L) at (7, -2);
            \coordinate (M) at (6, 4);
            \coordinate (N) at (6, 0);
            \coordinate (O) at (1, 3);

            \fill (A) circle [radius=2pt] node[anchor=west] { \footnotesize \tt 1 };
            \fill (B) circle [radius=2pt] node[anchor=west] { \footnotesize \tt 10 };
            \fill (C) circle [radius=2pt] node[anchor=west] { \footnotesize \tt 15 };
            \fill (D) circle [radius=2pt] node[anchor=west] { \footnotesize \tt 8 };
            \fill (E) circle [radius=2pt] node[anchor=west] { \footnotesize \tt 5 };
            \fill (F) circle [radius=2pt] node[anchor=west] { \footnotesize \tt 6 };
            \fill (G) circle [radius=2pt] node[anchor=west] { \footnotesize \tt 7 };
            \fill (H) circle [radius=2pt] node[anchor=west] { \footnotesize \tt 14 };
            \fill (I) circle [radius=2pt] node[anchor=west] { \footnotesize \tt 9 };
            \fill (J) circle [radius=2pt] node[anchor=west] { \footnotesize \tt 2 };
            \fill (K) circle [radius=2pt] node[anchor=west] { \footnotesize \tt 3 };
            \fill (L) circle [radius=2pt] node[anchor=west] { \footnotesize \tt 13 };
            \fill (M) circle [radius=2pt] node[anchor=west] { \footnotesize \tt 12 };
            \fill (N) circle [radius=2pt] node[anchor=west] { \footnotesize \tt 11 };
            \fill (O) circle [radius=2pt] node[anchor=west] { \footnotesize \tt 4 };

            \draw[blue,thick] (A) -- (K);
        \end{tikzpicture}
    \end{figure}

\end{frame}

\begin{frame}[fragile]{Geração do \textit{lower hull}}

    \begin{figure}
        \centering

        \begin{tikzpicture}
            \coordinate (A) at (0, 0);
            \coordinate (B) at (5, 3);
            \coordinate (C) at (8, -2);
            \coordinate (D) at (4, 4);
            \coordinate (E) at (2, 1);
            \coordinate (F) at (2, 5);
            \coordinate (G) at (3, -1);
            \coordinate (H) at (7, 2);
            \coordinate (I) at (5, 0);
            \coordinate (J) at (0, 4);
            \coordinate (K) at (1, -1);
            \coordinate (L) at (7, -2);
            \coordinate (M) at (6, 4);
            \coordinate (N) at (6, 0);
            \coordinate (O) at (1, 3);

            \fill (A) circle [radius=2pt] node[anchor=west] { \footnotesize \tt 1 };
            \fill (B) circle [radius=2pt] node[anchor=west] { \footnotesize \tt 10 };
            \fill (C) circle [radius=2pt] node[anchor=west] { \footnotesize \tt 15 };
            \fill (D) circle [radius=2pt] node[anchor=west] { \footnotesize \tt 8 };
            \fill (E) circle [radius=2pt] node[anchor=west] { \footnotesize \tt 5 };
            \fill (F) circle [radius=2pt] node[anchor=west] { \footnotesize \tt 6 };
            \fill (G) circle [radius=2pt] node[anchor=west] { \footnotesize \tt 7 };
            \fill (H) circle [radius=2pt] node[anchor=west] { \footnotesize \tt 14 };
            \fill (I) circle [radius=2pt] node[anchor=west] { \footnotesize \tt 9 };
            \fill (J) circle [radius=2pt] node[anchor=west] { \footnotesize \tt 2 };
            \fill (K) circle [radius=2pt] node[anchor=west] { \footnotesize \tt 3 };
            \fill (L) circle [radius=2pt] node[anchor=west] { \footnotesize \tt 13 };
            \fill (M) circle [radius=2pt] node[anchor=west] { \footnotesize \tt 12 };
            \fill (N) circle [radius=2pt] node[anchor=west] { \footnotesize \tt 11 };
            \fill (O) circle [radius=2pt] node[anchor=west] { \footnotesize \tt 4 };

            \draw[blue,thick] (A) -- (K) -- (O);
        \end{tikzpicture}
    \end{figure}

\end{frame}

\begin{frame}[fragile]{Geração do \textit{lower hull}}

    \begin{figure}
        \centering

        \begin{tikzpicture}
            \coordinate (A) at (0, 0);
            \coordinate (B) at (5, 3);
            \coordinate (C) at (8, -2);
            \coordinate (D) at (4, 4);
            \coordinate (E) at (2, 1);
            \coordinate (F) at (2, 5);
            \coordinate (G) at (3, -1);
            \coordinate (H) at (7, 2);
            \coordinate (I) at (5, 0);
            \coordinate (J) at (0, 4);
            \coordinate (K) at (1, -1);
            \coordinate (L) at (7, -2);
            \coordinate (M) at (6, 4);
            \coordinate (N) at (6, 0);
            \coordinate (O) at (1, 3);

            \fill (A) circle [radius=2pt] node[anchor=west] { \footnotesize \tt 1 };
            \fill (B) circle [radius=2pt] node[anchor=west] { \footnotesize \tt 10 };
            \fill (C) circle [radius=2pt] node[anchor=west] { \footnotesize \tt 15 };
            \fill (D) circle [radius=2pt] node[anchor=west] { \footnotesize \tt 8 };
            \fill (E) circle [radius=2pt] node[anchor=west] { \footnotesize \tt 5 };
            \fill (F) circle [radius=2pt] node[anchor=west] { \footnotesize \tt 6 };
            \fill (G) circle [radius=2pt] node[anchor=west] { \footnotesize \tt 7 };
            \fill (H) circle [radius=2pt] node[anchor=west] { \footnotesize \tt 14 };
            \fill (I) circle [radius=2pt] node[anchor=west] { \footnotesize \tt 9 };
            \fill (J) circle [radius=2pt] node[anchor=west] { \footnotesize \tt 2 };
            \fill (K) circle [radius=2pt] node[anchor=west] { \footnotesize \tt 3 };
            \fill (L) circle [radius=2pt] node[anchor=west] { \footnotesize \tt 13 };
            \fill (M) circle [radius=2pt] node[anchor=west] { \footnotesize \tt 12 };
            \fill (N) circle [radius=2pt] node[anchor=west] { \footnotesize \tt 11 };
            \fill (O) circle [radius=2pt] node[anchor=west] { \footnotesize \tt 4 };

            \draw (A) -- (K);

            \draw[blue,thick] (K) -- (O) -- (E);
        \end{tikzpicture}
    \end{figure}

\end{frame}

\begin{frame}[fragile]{Geração do \textit{lower hull}}

    \begin{figure}
        \centering

        \begin{tikzpicture}
            \coordinate (A) at (0, 0);
            \coordinate (B) at (5, 3);
            \coordinate (C) at (8, -2);
            \coordinate (D) at (4, 4);
            \coordinate (E) at (2, 1);
            \coordinate (F) at (2, 5);
            \coordinate (G) at (3, -1);
            \coordinate (H) at (7, 2);
            \coordinate (I) at (5, 0);
            \coordinate (J) at (0, 4);
            \coordinate (K) at (1, -1);
            \coordinate (L) at (7, -2);
            \coordinate (M) at (6, 4);
            \coordinate (N) at (6, 0);
            \coordinate (O) at (1, 3);

            \fill (A) circle [radius=2pt] node[anchor=west] { \footnotesize \tt 1 };
            \fill (B) circle [radius=2pt] node[anchor=west] { \footnotesize \tt 10 };
            \fill (C) circle [radius=2pt] node[anchor=west] { \footnotesize \tt 15 };
            \fill (D) circle [radius=2pt] node[anchor=west] { \footnotesize \tt 8 };
            \fill (E) circle [radius=2pt] node[anchor=west] { \footnotesize \tt 5 };
            \fill (F) circle [radius=2pt] node[anchor=west] { \footnotesize \tt 6 };
            \fill (G) circle [radius=2pt] node[anchor=west] { \footnotesize \tt 7 };
            \fill (H) circle [radius=2pt] node[anchor=west] { \footnotesize \tt 14 };
            \fill (I) circle [radius=2pt] node[anchor=west] { \footnotesize \tt 9 };
            \fill (J) circle [radius=2pt] node[anchor=west] { \footnotesize \tt 2 };
            \fill (K) circle [radius=2pt] node[anchor=west] { \footnotesize \tt 3 };
            \fill (L) circle [radius=2pt] node[anchor=west] { \footnotesize \tt 13 };
            \fill (M) circle [radius=2pt] node[anchor=west] { \footnotesize \tt 12 };
            \fill (N) circle [radius=2pt] node[anchor=west] { \footnotesize \tt 11 };
            \fill (O) circle [radius=2pt] node[anchor=west] { \footnotesize \tt 4 };

            \draw (A) -- (K);

            \draw[blue,thick] (A) -- (K) -- (E);
        \end{tikzpicture}
    \end{figure}

\end{frame}

\begin{frame}[fragile]{Geração do \textit{lower hull}}

    \begin{figure}
        \centering

        \begin{tikzpicture}
            \coordinate (A) at (0, 0);
            \coordinate (B) at (5, 3);
            \coordinate (C) at (8, -2);
            \coordinate (D) at (4, 4);
            \coordinate (E) at (2, 1);
            \coordinate (F) at (2, 5);
            \coordinate (G) at (3, -1);
            \coordinate (H) at (7, 2);
            \coordinate (I) at (5, 0);
            \coordinate (J) at (0, 4);
            \coordinate (K) at (1, -1);
            \coordinate (L) at (7, -2);
            \coordinate (M) at (6, 4);
            \coordinate (N) at (6, 0);
            \coordinate (O) at (1, 3);

            \fill (A) circle [radius=2pt] node[anchor=west] { \footnotesize \tt 1 };
            \fill (B) circle [radius=2pt] node[anchor=west] { \footnotesize \tt 10 };
            \fill (C) circle [radius=2pt] node[anchor=west] { \footnotesize \tt 15 };
            \fill (D) circle [radius=2pt] node[anchor=west] { \footnotesize \tt 8 };
            \fill (E) circle [radius=2pt] node[anchor=west] { \footnotesize \tt 5 };
            \fill (F) circle [radius=2pt] node[anchor=west] { \footnotesize \tt 6 };
            \fill (G) circle [radius=2pt] node[anchor=west] { \footnotesize \tt 7 };
            \fill (H) circle [radius=2pt] node[anchor=west] { \footnotesize \tt 14 };
            \fill (I) circle [radius=2pt] node[anchor=west] { \footnotesize \tt 9 };
            \fill (J) circle [radius=2pt] node[anchor=west] { \footnotesize \tt 2 };
            \fill (K) circle [radius=2pt] node[anchor=west] { \footnotesize \tt 3 };
            \fill (L) circle [radius=2pt] node[anchor=west] { \footnotesize \tt 13 };
            \fill (M) circle [radius=2pt] node[anchor=west] { \footnotesize \tt 12 };
            \fill (N) circle [radius=2pt] node[anchor=west] { \footnotesize \tt 11 };
            \fill (O) circle [radius=2pt] node[anchor=west] { \footnotesize \tt 4 };

            \draw (A) -- (K);

            \draw[blue,thick] (K) -- (E) -- (F);
        \end{tikzpicture}
    \end{figure}

\end{frame}

\begin{frame}[fragile]{Geração do \textit{lower hull}}

    \begin{figure}
        \centering

        \begin{tikzpicture}
            \coordinate (A) at (0, 0);
            \coordinate (B) at (5, 3);
            \coordinate (C) at (8, -2);
            \coordinate (D) at (4, 4);
            \coordinate (E) at (2, 1);
            \coordinate (F) at (2, 5);
            \coordinate (G) at (3, -1);
            \coordinate (H) at (7, 2);
            \coordinate (I) at (5, 0);
            \coordinate (J) at (0, 4);
            \coordinate (K) at (1, -1);
            \coordinate (L) at (7, -2);
            \coordinate (M) at (6, 4);
            \coordinate (N) at (6, 0);
            \coordinate (O) at (1, 3);

            \fill (A) circle [radius=2pt] node[anchor=west] { \footnotesize \tt 1 };
            \fill (B) circle [radius=2pt] node[anchor=west] { \footnotesize \tt 10 };
            \fill (C) circle [radius=2pt] node[anchor=west] { \footnotesize \tt 15 };
            \fill (D) circle [radius=2pt] node[anchor=west] { \footnotesize \tt 8 };
            \fill (E) circle [radius=2pt] node[anchor=west] { \footnotesize \tt 5 };
            \fill (F) circle [radius=2pt] node[anchor=west] { \footnotesize \tt 6 };
            \fill (G) circle [radius=2pt] node[anchor=west] { \footnotesize \tt 7 };
            \fill (H) circle [radius=2pt] node[anchor=west] { \footnotesize \tt 14 };
            \fill (I) circle [radius=2pt] node[anchor=west] { \footnotesize \tt 9 };
            \fill (J) circle [radius=2pt] node[anchor=west] { \footnotesize \tt 2 };
            \fill (K) circle [radius=2pt] node[anchor=west] { \footnotesize \tt 3 };
            \fill (L) circle [radius=2pt] node[anchor=west] { \footnotesize \tt 13 };
            \fill (M) circle [radius=2pt] node[anchor=west] { \footnotesize \tt 12 };
            \fill (N) circle [radius=2pt] node[anchor=west] { \footnotesize \tt 11 };
            \fill (O) circle [radius=2pt] node[anchor=west] { \footnotesize \tt 4 };

            \draw (A) -- (K) -- (E);

            \draw[blue,thick] (E) -- (F) -- (G);
        \end{tikzpicture}
    \end{figure}

\end{frame}

\begin{frame}[fragile]{Geração do \textit{lower hull}}

    \begin{figure}
        \centering

        \begin{tikzpicture}
            \coordinate (A) at (0, 0);
            \coordinate (B) at (5, 3);
            \coordinate (C) at (8, -2);
            \coordinate (D) at (4, 4);
            \coordinate (E) at (2, 1);
            \coordinate (F) at (2, 5);
            \coordinate (G) at (3, -1);
            \coordinate (H) at (7, 2);
            \coordinate (I) at (5, 0);
            \coordinate (J) at (0, 4);
            \coordinate (K) at (1, -1);
            \coordinate (L) at (7, -2);
            \coordinate (M) at (6, 4);
            \coordinate (N) at (6, 0);
            \coordinate (O) at (1, 3);

            \fill (A) circle [radius=2pt] node[anchor=west] { \footnotesize \tt 1 };
            \fill (B) circle [radius=2pt] node[anchor=west] { \footnotesize \tt 10 };
            \fill (C) circle [radius=2pt] node[anchor=west] { \footnotesize \tt 15 };
            \fill (D) circle [radius=2pt] node[anchor=west] { \footnotesize \tt 8 };
            \fill (E) circle [radius=2pt] node[anchor=west] { \footnotesize \tt 5 };
            \fill (F) circle [radius=2pt] node[anchor=west] { \footnotesize \tt 6 };
            \fill (G) circle [radius=2pt] node[anchor=west] { \footnotesize \tt 7 };
            \fill (H) circle [radius=2pt] node[anchor=west] { \footnotesize \tt 14 };
            \fill (I) circle [radius=2pt] node[anchor=west] { \footnotesize \tt 9 };
            \fill (J) circle [radius=2pt] node[anchor=west] { \footnotesize \tt 2 };
            \fill (K) circle [radius=2pt] node[anchor=west] { \footnotesize \tt 3 };
            \fill (L) circle [radius=2pt] node[anchor=west] { \footnotesize \tt 13 };
            \fill (M) circle [radius=2pt] node[anchor=west] { \footnotesize \tt 12 };
            \fill (N) circle [radius=2pt] node[anchor=west] { \footnotesize \tt 11 };
            \fill (O) circle [radius=2pt] node[anchor=west] { \footnotesize \tt 4 };

            \draw (A) -- (K) -- (E);

            \draw[blue,thick] (K) -- (E) -- (G);
        \end{tikzpicture}
    \end{figure}

\end{frame}

\begin{frame}[fragile]{Geração do \textit{lower hull}}

    \begin{figure}
        \centering

        \begin{tikzpicture}
            \coordinate (A) at (0, 0);
            \coordinate (B) at (5, 3);
            \coordinate (C) at (8, -2);
            \coordinate (D) at (4, 4);
            \coordinate (E) at (2, 1);
            \coordinate (F) at (2, 5);
            \coordinate (G) at (3, -1);
            \coordinate (H) at (7, 2);
            \coordinate (I) at (5, 0);
            \coordinate (J) at (0, 4);
            \coordinate (K) at (1, -1);
            \coordinate (L) at (7, -2);
            \coordinate (M) at (6, 4);
            \coordinate (N) at (6, 0);
            \coordinate (O) at (1, 3);

            \fill (A) circle [radius=2pt] node[anchor=west] { \footnotesize \tt 1 };
            \fill (B) circle [radius=2pt] node[anchor=west] { \footnotesize \tt 10 };
            \fill (C) circle [radius=2pt] node[anchor=west] { \footnotesize \tt 15 };
            \fill (D) circle [radius=2pt] node[anchor=west] { \footnotesize \tt 8 };
            \fill (E) circle [radius=2pt] node[anchor=west] { \footnotesize \tt 5 };
            \fill (F) circle [radius=2pt] node[anchor=west] { \footnotesize \tt 6 };
            \fill (G) circle [radius=2pt] node[anchor=north] { \footnotesize \tt 7 };
            \fill (H) circle [radius=2pt] node[anchor=west] { \footnotesize \tt 14 };
            \fill (I) circle [radius=2pt] node[anchor=west] { \footnotesize \tt 9 };
            \fill (J) circle [radius=2pt] node[anchor=west] { \footnotesize \tt 2 };
            \fill (K) circle [radius=2pt] node[anchor=north] { \footnotesize \tt 3 };
            \fill (L) circle [radius=2pt] node[anchor=west] { \footnotesize \tt 13 };
            \fill (M) circle [radius=2pt] node[anchor=west] { \footnotesize \tt 12 };
            \fill (N) circle [radius=2pt] node[anchor=west] { \footnotesize \tt 11 };
            \fill (O) circle [radius=2pt] node[anchor=west] { \footnotesize \tt 4 };

            \draw (A) -- (K) -- (G);

            \draw[blue,thick] (A) -- (K) -- (G);
        \end{tikzpicture}
    \end{figure}

\end{frame}

\begin{frame}[fragile]{Geração do \textit{lower hull}}

    \begin{figure}
        \centering

        \begin{tikzpicture}
            \coordinate (A) at (0, 0);
            \coordinate (B) at (5, 3);
            \coordinate (C) at (8, -2);
            \coordinate (D) at (4, 4);
            \coordinate (E) at (2, 1);
            \coordinate (F) at (2, 5);
            \coordinate (G) at (3, -1);
            \coordinate (H) at (7, 2);
            \coordinate (I) at (5, 0);
            \coordinate (J) at (0, 4);
            \coordinate (K) at (1, -1);
            \coordinate (L) at (7, -2);
            \coordinate (M) at (6, 4);
            \coordinate (N) at (6, 0);
            \coordinate (O) at (1, 3);

            \fill (A) circle [radius=2pt] node[anchor=west] { \footnotesize \tt 1 };
            \fill (B) circle [radius=2pt] node[anchor=west] { \footnotesize \tt 10 };
            \fill (C) circle [radius=2pt] node[anchor=west] { \footnotesize \tt 15 };
            \fill (D) circle [radius=2pt] node[anchor=west] { \footnotesize \tt 8 };
            \fill (E) circle [radius=2pt] node[anchor=west] { \footnotesize \tt 5 };
            \fill (F) circle [radius=2pt] node[anchor=west] { \footnotesize \tt 6 };
            \fill (G) circle [radius=2pt] node[anchor=north] { \footnotesize \tt 7 };
            \fill (H) circle [radius=2pt] node[anchor=west] { \footnotesize \tt 14 };
            \fill (I) circle [radius=2pt] node[anchor=west] { \footnotesize \tt 9 };
            \fill (J) circle [radius=2pt] node[anchor=west] { \footnotesize \tt 2 };
            \fill (K) circle [radius=2pt] node[anchor=north] { \footnotesize \tt 3 };
            \fill (L) circle [radius=2pt] node[anchor=west] { \footnotesize \tt 13 };
            \fill (M) circle [radius=2pt] node[anchor=west] { \footnotesize \tt 12 };
            \fill (N) circle [radius=2pt] node[anchor=west] { \footnotesize \tt 11 };
            \fill (O) circle [radius=2pt] node[anchor=west] { \footnotesize \tt 4 };

            \draw (A) -- (K) -- (G);

            \draw[blue,thick] (K) -- (G) -- (D);
        \end{tikzpicture}
    \end{figure}

\end{frame}

\begin{frame}[fragile]{Geração do \textit{lower hull}}

    \begin{figure}
        \centering

        \begin{tikzpicture}
            \coordinate (A) at (0, 0);
            \coordinate (B) at (5, 3);
            \coordinate (C) at (8, -2);
            \coordinate (D) at (4, 4);
            \coordinate (E) at (2, 1);
            \coordinate (F) at (2, 5);
            \coordinate (G) at (3, -1);
            \coordinate (H) at (7, 2);
            \coordinate (I) at (5, 0);
            \coordinate (J) at (0, 4);
            \coordinate (K) at (1, -1);
            \coordinate (L) at (7, -2);
            \coordinate (M) at (6, 4);
            \coordinate (N) at (6, 0);
            \coordinate (O) at (1, 3);

            \fill (A) circle [radius=2pt] node[anchor=west] { \footnotesize \tt 1 };
            \fill (B) circle [radius=2pt] node[anchor=west] { \footnotesize \tt 10 };
            \fill (C) circle [radius=2pt] node[anchor=west] { \footnotesize \tt 15 };
            \fill (D) circle [radius=2pt] node[anchor=west] { \footnotesize \tt 8 };
            \fill (E) circle [radius=2pt] node[anchor=west] { \footnotesize \tt 5 };
            \fill (F) circle [radius=2pt] node[anchor=west] { \footnotesize \tt 6 };
            \fill (G) circle [radius=2pt] node[anchor=north] { \footnotesize \tt 7 };
            \fill (H) circle [radius=2pt] node[anchor=west] { \footnotesize \tt 14 };
            \fill (I) circle [radius=2pt] node[anchor=west] { \footnotesize \tt 9 };
            \fill (J) circle [radius=2pt] node[anchor=west] { \footnotesize \tt 2 };
            \fill (K) circle [radius=2pt] node[anchor=north] { \footnotesize \tt 3 };
            \fill (L) circle [radius=2pt] node[anchor=west] { \footnotesize \tt 13 };
            \fill (M) circle [radius=2pt] node[anchor=west] { \footnotesize \tt 12 };
            \fill (N) circle [radius=2pt] node[anchor=west] { \footnotesize \tt 11 };
            \fill (O) circle [radius=2pt] node[anchor=west] { \footnotesize \tt 4 };

            \draw (A) -- (K) -- (G);

            \draw[blue,thick] (G) -- (D) -- (I);
        \end{tikzpicture}
    \end{figure}

\end{frame}

\begin{frame}[fragile]{Geração do \textit{lower hull}}

    \begin{figure}
        \centering

        \begin{tikzpicture}
            \coordinate (A) at (0, 0);
            \coordinate (B) at (5, 3);
            \coordinate (C) at (8, -2);
            \coordinate (D) at (4, 4);
            \coordinate (E) at (2, 1);
            \coordinate (F) at (2, 5);
            \coordinate (G) at (3, -1);
            \coordinate (H) at (7, 2);
            \coordinate (I) at (5, 0);
            \coordinate (J) at (0, 4);
            \coordinate (K) at (1, -1);
            \coordinate (L) at (7, -2);
            \coordinate (M) at (6, 4);
            \coordinate (N) at (6, 0);
            \coordinate (O) at (1, 3);

            \fill (A) circle [radius=2pt] node[anchor=west] { \footnotesize \tt 1 };
            \fill (B) circle [radius=2pt] node[anchor=west] { \footnotesize \tt 10 };
            \fill (C) circle [radius=2pt] node[anchor=west] { \footnotesize \tt 15 };
            \fill (D) circle [radius=2pt] node[anchor=west] { \footnotesize \tt 8 };
            \fill (E) circle [radius=2pt] node[anchor=west] { \footnotesize \tt 5 };
            \fill (F) circle [radius=2pt] node[anchor=west] { \footnotesize \tt 6 };
            \fill (G) circle [radius=2pt] node[anchor=north] { \footnotesize \tt 7 };
            \fill (H) circle [radius=2pt] node[anchor=west] { \footnotesize \tt 14 };
            \fill (I) circle [radius=2pt] node[anchor=west] { \footnotesize \tt 9 };
            \fill (J) circle [radius=2pt] node[anchor=west] { \footnotesize \tt 2 };
            \fill (K) circle [radius=2pt] node[anchor=north] { \footnotesize \tt 3 };
            \fill (L) circle [radius=2pt] node[anchor=west] { \footnotesize \tt 13 };
            \fill (M) circle [radius=2pt] node[anchor=west] { \footnotesize \tt 12 };
            \fill (N) circle [radius=2pt] node[anchor=west] { \footnotesize \tt 11 };
            \fill (O) circle [radius=2pt] node[anchor=west] { \footnotesize \tt 4 };

            \draw (A) -- (K) -- (G);

            \draw[blue,thick] (K) -- (G) -- (I);
        \end{tikzpicture}
    \end{figure}

\end{frame}

\begin{frame}[fragile]{Geração do \textit{lower hull}}

    \begin{figure}
        \centering

        \begin{tikzpicture}
            \coordinate (A) at (0, 0);
            \coordinate (B) at (5, 3);
            \coordinate (C) at (8, -2);
            \coordinate (D) at (4, 4);
            \coordinate (E) at (2, 1);
            \coordinate (F) at (2, 5);
            \coordinate (G) at (3, -1);
            \coordinate (H) at (7, 2);
            \coordinate (I) at (5, 0);
            \coordinate (J) at (0, 4);
            \coordinate (K) at (1, -1);
            \coordinate (L) at (7, -2);
            \coordinate (M) at (6, 4);
            \coordinate (N) at (6, 0);
            \coordinate (O) at (1, 3);

            \fill (A) circle [radius=2pt] node[anchor=west] { \footnotesize \tt 1 };
            \fill (B) circle [radius=2pt] node[anchor=west] { \footnotesize \tt 10 };
            \fill (C) circle [radius=2pt] node[anchor=west] { \footnotesize \tt 15 };
            \fill (D) circle [radius=2pt] node[anchor=west] { \footnotesize \tt 8 };
            \fill (E) circle [radius=2pt] node[anchor=west] { \footnotesize \tt 5 };
            \fill (F) circle [radius=2pt] node[anchor=west] { \footnotesize \tt 6 };
            \fill (G) circle [radius=2pt] node[anchor=north] { \footnotesize \tt 7 };
            \fill (H) circle [radius=2pt] node[anchor=west] { \footnotesize \tt 14 };
            \fill (I) circle [radius=2pt] node[anchor=west] { \footnotesize \tt 9 };
            \fill (J) circle [radius=2pt] node[anchor=west] { \footnotesize \tt 2 };
            \fill (K) circle [radius=2pt] node[anchor=north] { \footnotesize \tt 3 };
            \fill (L) circle [radius=2pt] node[anchor=west] { \footnotesize \tt 13 };
            \fill (M) circle [radius=2pt] node[anchor=west] { \footnotesize \tt 12 };
            \fill (N) circle [radius=2pt] node[anchor=west] { \footnotesize \tt 11 };
            \fill (O) circle [radius=2pt] node[anchor=west] { \footnotesize \tt 4 };

            \draw (A) -- (K) -- (G);

            \draw[blue,thick] (G) -- (I) -- (B);
        \end{tikzpicture}
    \end{figure}

\end{frame}

\begin{frame}[fragile]{Geração do \textit{lower hull}}

    \begin{figure}
        \centering

        \begin{tikzpicture}
            \coordinate (A) at (0, 0);
            \coordinate (B) at (5, 3);
            \coordinate (C) at (8, -2);
            \coordinate (D) at (4, 4);
            \coordinate (E) at (2, 1);
            \coordinate (F) at (2, 5);
            \coordinate (G) at (3, -1);
            \coordinate (H) at (7, 2);
            \coordinate (I) at (5, 0);
            \coordinate (J) at (0, 4);
            \coordinate (K) at (1, -1);
            \coordinate (L) at (7, -2);
            \coordinate (M) at (6, 4);
            \coordinate (N) at (6, 0);
            \coordinate (O) at (1, 3);

            \fill (A) circle [radius=2pt] node[anchor=west] { \footnotesize \tt 1 };
            \fill (B) circle [radius=2pt] node[anchor=west] { \footnotesize \tt 10 };
            \fill (C) circle [radius=2pt] node[anchor=west] { \footnotesize \tt 15 };
            \fill (D) circle [radius=2pt] node[anchor=west] { \footnotesize \tt 8 };
            \fill (E) circle [radius=2pt] node[anchor=west] { \footnotesize \tt 5 };
            \fill (F) circle [radius=2pt] node[anchor=west] { \footnotesize \tt 6 };
            \fill (G) circle [radius=2pt] node[anchor=north] { \footnotesize \tt 7 };
            \fill (H) circle [radius=2pt] node[anchor=west] { \footnotesize \tt 14 };
            \fill (I) circle [radius=2pt] node[anchor=west] { \footnotesize \tt 9 };
            \fill (J) circle [radius=2pt] node[anchor=west] { \footnotesize \tt 2 };
            \fill (K) circle [radius=2pt] node[anchor=north] { \footnotesize \tt 3 };
            \fill (L) circle [radius=2pt] node[anchor=west] { \footnotesize \tt 13 };
            \fill (M) circle [radius=2pt] node[anchor=west] { \footnotesize \tt 12 };
            \fill (N) circle [radius=2pt] node[anchor=west] { \footnotesize \tt 11 };
            \fill (O) circle [radius=2pt] node[anchor=west] { \footnotesize \tt 4 };

            \draw (A) -- (K) -- (G) -- (I);

            \draw[blue,thick] (I) -- (B) -- (N);
        \end{tikzpicture}
    \end{figure}

\end{frame}

\begin{frame}[fragile]{Geração do \textit{lower hull}}

    \begin{figure}
        \centering

        \begin{tikzpicture}
            \coordinate (A) at (0, 0);
            \coordinate (B) at (5, 3);
            \coordinate (C) at (8, -2);
            \coordinate (D) at (4, 4);
            \coordinate (E) at (2, 1);
            \coordinate (F) at (2, 5);
            \coordinate (G) at (3, -1);
            \coordinate (H) at (7, 2);
            \coordinate (I) at (5, 0);
            \coordinate (J) at (0, 4);
            \coordinate (K) at (1, -1);
            \coordinate (L) at (7, -2);
            \coordinate (M) at (6, 4);
            \coordinate (N) at (6, 0);
            \coordinate (O) at (1, 3);

            \fill (A) circle [radius=2pt] node[anchor=west] { \footnotesize \tt 1 };
            \fill (B) circle [radius=2pt] node[anchor=west] { \footnotesize \tt 10 };
            \fill (C) circle [radius=2pt] node[anchor=west] { \footnotesize \tt 15 };
            \fill (D) circle [radius=2pt] node[anchor=west] { \footnotesize \tt 8 };
            \fill (E) circle [radius=2pt] node[anchor=west] { \footnotesize \tt 5 };
            \fill (F) circle [radius=2pt] node[anchor=west] { \footnotesize \tt 6 };
            \fill (G) circle [radius=2pt] node[anchor=north] { \footnotesize \tt 7 };
            \fill (H) circle [radius=2pt] node[anchor=west] { \footnotesize \tt 14 };
            \fill (I) circle [radius=2pt] node[anchor=south] { \footnotesize \tt 9 };
            \fill (J) circle [radius=2pt] node[anchor=west] { \footnotesize \tt 2 };
            \fill (K) circle [radius=2pt] node[anchor=north] { \footnotesize \tt 3 };
            \fill (L) circle [radius=2pt] node[anchor=west] { \footnotesize \tt 13 };
            \fill (M) circle [radius=2pt] node[anchor=west] { \footnotesize \tt 12 };
            \fill (N) circle [radius=2pt] node[anchor=west] { \footnotesize \tt 11 };
            \fill (O) circle [radius=2pt] node[anchor=west] { \footnotesize \tt 4 };

            \draw (A) -- (K) -- (G) -- (I);

            \draw[blue,thick] (G) -- (I) -- (N);
        \end{tikzpicture}
    \end{figure}

\end{frame}

\begin{frame}[fragile]{Geração do \textit{lower hull}}

    \begin{figure}
        \centering

        \begin{tikzpicture}
            \coordinate (A) at (0, 0);
            \coordinate (B) at (5, 3);
            \coordinate (C) at (8, -2);
            \coordinate (D) at (4, 4);
            \coordinate (E) at (2, 1);
            \coordinate (F) at (2, 5);
            \coordinate (G) at (3, -1);
            \coordinate (H) at (7, 2);
            \coordinate (I) at (5, 0);
            \coordinate (J) at (0, 4);
            \coordinate (K) at (1, -1);
            \coordinate (L) at (7, -2);
            \coordinate (M) at (6, 4);
            \coordinate (N) at (6, 0);
            \coordinate (O) at (1, 3);

            \fill (A) circle [radius=2pt] node[anchor=west] { \footnotesize \tt 1 };
            \fill (B) circle [radius=2pt] node[anchor=west] { \footnotesize \tt 10 };
            \fill (C) circle [radius=2pt] node[anchor=west] { \footnotesize \tt 15 };
            \fill (D) circle [radius=2pt] node[anchor=west] { \footnotesize \tt 8 };
            \fill (E) circle [radius=2pt] node[anchor=west] { \footnotesize \tt 5 };
            \fill (F) circle [radius=2pt] node[anchor=west] { \footnotesize \tt 6 };
            \fill (G) circle [radius=2pt] node[anchor=north] { \footnotesize \tt 7 };
            \fill (H) circle [radius=2pt] node[anchor=west] { \footnotesize \tt 14 };
            \fill (I) circle [radius=2pt] node[anchor=west] { \footnotesize \tt 9 };
            \fill (J) circle [radius=2pt] node[anchor=west] { \footnotesize \tt 2 };
            \fill (K) circle [radius=2pt] node[anchor=north] { \footnotesize \tt 3 };
            \fill (L) circle [radius=2pt] node[anchor=west] { \footnotesize \tt 13 };
            \fill (M) circle [radius=2pt] node[anchor=west] { \footnotesize \tt 12 };
            \fill (N) circle [radius=2pt] node[anchor=west] { \footnotesize \tt 11 };
            \fill (O) circle [radius=2pt] node[anchor=west] { \footnotesize \tt 4 };

            \draw (A) -- (K) -- (G);

            \draw[blue,thick] (K) -- (G) -- (N);
        \end{tikzpicture}
    \end{figure}

\end{frame}

\begin{frame}[fragile]{Geração do \textit{lower hull}}

    \begin{figure}
        \centering

        \begin{tikzpicture}
            \coordinate (A) at (0, 0);
            \coordinate (B) at (5, 3);
            \coordinate (C) at (8, -2);
            \coordinate (D) at (4, 4);
            \coordinate (E) at (2, 1);
            \coordinate (F) at (2, 5);
            \coordinate (G) at (3, -1);
            \coordinate (H) at (7, 2);
            \coordinate (I) at (5, 0);
            \coordinate (J) at (0, 4);
            \coordinate (K) at (1, -1);
            \coordinate (L) at (7, -2);
            \coordinate (M) at (6, 4);
            \coordinate (N) at (6, 0);
            \coordinate (O) at (1, 3);

            \fill (A) circle [radius=2pt] node[anchor=west] { \footnotesize \tt 1 };
            \fill (B) circle [radius=2pt] node[anchor=west] { \footnotesize \tt 10 };
            \fill (C) circle [radius=2pt] node[anchor=west] { \footnotesize \tt 15 };
            \fill (D) circle [radius=2pt] node[anchor=west] { \footnotesize \tt 8 };
            \fill (E) circle [radius=2pt] node[anchor=west] { \footnotesize \tt 5 };
            \fill (F) circle [radius=2pt] node[anchor=west] { \footnotesize \tt 6 };
            \fill (G) circle [radius=2pt] node[anchor=north] { \footnotesize \tt 7 };
            \fill (H) circle [radius=2pt] node[anchor=west] { \footnotesize \tt 14 };
            \fill (I) circle [radius=2pt] node[anchor=west] { \footnotesize \tt 9 };
            \fill (J) circle [radius=2pt] node[anchor=west] { \footnotesize \tt 2 };
            \fill (K) circle [radius=2pt] node[anchor=north] { \footnotesize \tt 3 };
            \fill (L) circle [radius=2pt] node[anchor=west] { \footnotesize \tt 13 };
            \fill (M) circle [radius=2pt] node[anchor=west] { \footnotesize \tt 12 };
            \fill (N) circle [radius=2pt] node[anchor=west] { \footnotesize \tt 11 };
            \fill (O) circle [radius=2pt] node[anchor=west] { \footnotesize \tt 4 };

            \draw (A) -- (K) -- (G);

            \draw[blue,thick] (G) -- (N) -- (M);
        \end{tikzpicture}
    \end{figure}

\end{frame}

\begin{frame}[fragile]{Geração do \textit{lower hull}}

    \begin{figure}
        \centering

        \begin{tikzpicture}
            \coordinate (A) at (0, 0);
            \coordinate (B) at (5, 3);
            \coordinate (C) at (8, -2);
            \coordinate (D) at (4, 4);
            \coordinate (E) at (2, 1);
            \coordinate (F) at (2, 5);
            \coordinate (G) at (3, -1);
            \coordinate (H) at (7, 2);
            \coordinate (I) at (5, 0);
            \coordinate (J) at (0, 4);
            \coordinate (K) at (1, -1);
            \coordinate (L) at (7, -2);
            \coordinate (M) at (6, 4);
            \coordinate (N) at (6, 0);
            \coordinate (O) at (1, 3);

            \fill (A) circle [radius=2pt] node[anchor=west] { \footnotesize \tt 1 };
            \fill (B) circle [radius=2pt] node[anchor=west] { \footnotesize \tt 10 };
            \fill (C) circle [radius=2pt] node[anchor=west] { \footnotesize \tt 15 };
            \fill (D) circle [radius=2pt] node[anchor=west] { \footnotesize \tt 8 };
            \fill (E) circle [radius=2pt] node[anchor=west] { \footnotesize \tt 5 };
            \fill (F) circle [radius=2pt] node[anchor=west] { \footnotesize \tt 6 };
            \fill (G) circle [radius=2pt] node[anchor=north] { \footnotesize \tt 7 };
            \fill (H) circle [radius=2pt] node[anchor=west] { \footnotesize \tt 14 };
            \fill (I) circle [radius=2pt] node[anchor=west] { \footnotesize \tt 9 };
            \fill (J) circle [radius=2pt] node[anchor=west] { \footnotesize \tt 2 };
            \fill (K) circle [radius=2pt] node[anchor=north] { \footnotesize \tt 3 };
            \fill (L) circle [radius=2pt] node[anchor=west] { \footnotesize \tt 13 };
            \fill (M) circle [radius=2pt] node[anchor=west] { \footnotesize \tt 12 };
            \fill (N) circle [radius=2pt] node[anchor=west] { \footnotesize \tt 11 };
            \fill (O) circle [radius=2pt] node[anchor=west] { \footnotesize \tt 4 };

            \draw (A) -- (K) -- (G) -- (N);

            \draw[blue,thick] (N) -- (M) -- (L);
        \end{tikzpicture}
    \end{figure}

\end{frame}

\begin{frame}[fragile]{Geração do \textit{lower hull}}

    \begin{figure}
        \centering

        \begin{tikzpicture}
            \coordinate (A) at (0, 0);
            \coordinate (B) at (5, 3);
            \coordinate (C) at (8, -2);
            \coordinate (D) at (4, 4);
            \coordinate (E) at (2, 1);
            \coordinate (F) at (2, 5);
            \coordinate (G) at (3, -1);
            \coordinate (H) at (7, 2);
            \coordinate (I) at (5, 0);
            \coordinate (J) at (0, 4);
            \coordinate (K) at (1, -1);
            \coordinate (L) at (7, -2);
            \coordinate (M) at (6, 4);
            \coordinate (N) at (6, 0);
            \coordinate (O) at (1, 3);

            \fill (A) circle [radius=2pt] node[anchor=west] { \footnotesize \tt 1 };
            \fill (B) circle [radius=2pt] node[anchor=west] { \footnotesize \tt 10 };
            \fill (C) circle [radius=2pt] node[anchor=west] { \footnotesize \tt 15 };
            \fill (D) circle [radius=2pt] node[anchor=west] { \footnotesize \tt 8 };
            \fill (E) circle [radius=2pt] node[anchor=west] { \footnotesize \tt 5 };
            \fill (F) circle [radius=2pt] node[anchor=west] { \footnotesize \tt 6 };
            \fill (G) circle [radius=2pt] node[anchor=north] { \footnotesize \tt 7 };
            \fill (H) circle [radius=2pt] node[anchor=west] { \footnotesize \tt 14 };
            \fill (I) circle [radius=2pt] node[anchor=west] { \footnotesize \tt 9 };
            \fill (J) circle [radius=2pt] node[anchor=west] { \footnotesize \tt 2 };
            \fill (K) circle [radius=2pt] node[anchor=north] { \footnotesize \tt 3 };
            \fill (L) circle [radius=2pt] node[anchor=west] { \footnotesize \tt 13 };
            \fill (M) circle [radius=2pt] node[anchor=west] { \footnotesize \tt 12 };
            \fill (N) circle [radius=2pt] node[anchor=west] { \footnotesize \tt 11 };
            \fill (O) circle [radius=2pt] node[anchor=west] { \footnotesize \tt 4 };

            \draw (A) -- (K) -- (G) -- (N);

            \draw[blue,thick] (G) -- (N) -- (L);
        \end{tikzpicture}
    \end{figure}

\end{frame}

\begin{frame}[fragile]{Geração do \textit{lower hull}}

    \begin{figure}
        \centering

        \begin{tikzpicture}
            \coordinate (A) at (0, 0);
            \coordinate (B) at (5, 3);
            \coordinate (C) at (8, -2);
            \coordinate (D) at (4, 4);
            \coordinate (E) at (2, 1);
            \coordinate (F) at (2, 5);
            \coordinate (G) at (3, -1);
            \coordinate (H) at (7, 2);
            \coordinate (I) at (5, 0);
            \coordinate (J) at (0, 4);
            \coordinate (K) at (1, -1);
            \coordinate (L) at (7, -2);
            \coordinate (M) at (6, 4);
            \coordinate (N) at (6, 0);
            \coordinate (O) at (1, 3);

            \fill (A) circle [radius=2pt] node[anchor=west] { \footnotesize \tt 1 };
            \fill (B) circle [radius=2pt] node[anchor=west] { \footnotesize \tt 10 };
            \fill (C) circle [radius=2pt] node[anchor=west] { \footnotesize \tt 15 };
            \fill (D) circle [radius=2pt] node[anchor=west] { \footnotesize \tt 8 };
            \fill (E) circle [radius=2pt] node[anchor=west] { \footnotesize \tt 5 };
            \fill (F) circle [radius=2pt] node[anchor=west] { \footnotesize \tt 6 };
            \fill (G) circle [radius=2pt] node[anchor=north] { \footnotesize \tt 7 };
            \fill (H) circle [radius=2pt] node[anchor=west] { \footnotesize \tt 14 };
            \fill (I) circle [radius=2pt] node[anchor=west] { \footnotesize \tt 9 };
            \fill (J) circle [radius=2pt] node[anchor=west] { \footnotesize \tt 2 };
            \fill (K) circle [radius=2pt] node[anchor=north] { \footnotesize \tt 3 };
            \fill (L) circle [radius=2pt] node[anchor=west] { \footnotesize \tt 13 };
            \fill (M) circle [radius=2pt] node[anchor=west] { \footnotesize \tt 12 };
            \fill (N) circle [radius=2pt] node[anchor=west] { \footnotesize \tt 11 };
            \fill (O) circle [radius=2pt] node[anchor=west] { \footnotesize \tt 4 };

            \draw (A) -- (K) -- (G);

            \draw[blue,thick] (K) -- (G) -- (L);
        \end{tikzpicture}
    \end{figure}

\end{frame}

\begin{frame}[fragile]{Geração do \textit{lower hull}}

    \begin{figure}
        \centering

        \begin{tikzpicture}
            \coordinate (A) at (0, 0);
            \coordinate (B) at (5, 3);
            \coordinate (C) at (8, -2);
            \coordinate (D) at (4, 4);
            \coordinate (E) at (2, 1);
            \coordinate (F) at (2, 5);
            \coordinate (G) at (3, -1);
            \coordinate (H) at (7, 2);
            \coordinate (I) at (5, 0);
            \coordinate (J) at (0, 4);
            \coordinate (K) at (1, -1);
            \coordinate (L) at (7, -2);
            \coordinate (M) at (6, 4);
            \coordinate (N) at (6, 0);
            \coordinate (O) at (1, 3);

            \fill (A) circle [radius=2pt] node[anchor=west] { \footnotesize \tt 1 };
            \fill (B) circle [radius=2pt] node[anchor=west] { \footnotesize \tt 10 };
            \fill (C) circle [radius=2pt] node[anchor=west] { \footnotesize \tt 15 };
            \fill (D) circle [radius=2pt] node[anchor=west] { \footnotesize \tt 8 };
            \fill (E) circle [radius=2pt] node[anchor=west] { \footnotesize \tt 5 };
            \fill (F) circle [radius=2pt] node[anchor=west] { \footnotesize \tt 6 };
            \fill (G) circle [radius=2pt] node[anchor=south] { \footnotesize \tt 7 };
            \fill (H) circle [radius=2pt] node[anchor=west] { \footnotesize \tt 14 };
            \fill (I) circle [radius=2pt] node[anchor=west] { \footnotesize \tt 9 };
            \fill (J) circle [radius=2pt] node[anchor=west] { \footnotesize \tt 2 };
            \fill (K) circle [radius=2pt] node[anchor=north] { \footnotesize \tt 3 };
            \fill (L) circle [radius=2pt] node[anchor=west] { \footnotesize \tt 13 };
            \fill (M) circle [radius=2pt] node[anchor=west] { \footnotesize \tt 12 };
            \fill (N) circle [radius=2pt] node[anchor=west] { \footnotesize \tt 11 };
            \fill (O) circle [radius=2pt] node[anchor=west] { \footnotesize \tt 4 };

            \draw (A) -- (K) -- (L);

            \draw[blue,thick] (A) -- (K) -- (L);
        \end{tikzpicture}
    \end{figure}

\end{frame}

\begin{frame}[fragile]{Geração do \textit{lower hull}}

    \begin{figure}
        \centering

        \begin{tikzpicture}
            \coordinate (A) at (0, 0);
            \coordinate (B) at (5, 3);
            \coordinate (C) at (8, -2);
            \coordinate (D) at (4, 4);
            \coordinate (E) at (2, 1);
            \coordinate (F) at (2, 5);
            \coordinate (G) at (3, -1);
            \coordinate (H) at (7, 2);
            \coordinate (I) at (5, 0);
            \coordinate (J) at (0, 4);
            \coordinate (K) at (1, -1);
            \coordinate (L) at (7, -2);
            \coordinate (M) at (6, 4);
            \coordinate (N) at (6, 0);
            \coordinate (O) at (1, 3);

            \fill (A) circle [radius=2pt] node[anchor=west] { \footnotesize \tt 1 };
            \fill (B) circle [radius=2pt] node[anchor=west] { \footnotesize \tt 10 };
            \fill (C) circle [radius=2pt] node[anchor=west] { \footnotesize \tt 15 };
            \fill (D) circle [radius=2pt] node[anchor=west] { \footnotesize \tt 8 };
            \fill (E) circle [radius=2pt] node[anchor=west] { \footnotesize \tt 5 };
            \fill (F) circle [radius=2pt] node[anchor=west] { \footnotesize \tt 6 };
            \fill (G) circle [radius=2pt] node[anchor=south] { \footnotesize \tt 7 };
            \fill (H) circle [radius=2pt] node[anchor=west] { \footnotesize \tt 14 };
            \fill (I) circle [radius=2pt] node[anchor=west] { \footnotesize \tt 9 };
            \fill (J) circle [radius=2pt] node[anchor=west] { \footnotesize \tt 2 };
            \fill (K) circle [radius=2pt] node[anchor=north] { \footnotesize \tt 3 };
            \fill (L) circle [radius=2pt] node[anchor=west] { \footnotesize \tt 13 };
            \fill (M) circle [radius=2pt] node[anchor=west] { \footnotesize \tt 12 };
            \fill (N) circle [radius=2pt] node[anchor=west] { \footnotesize \tt 11 };
            \fill (O) circle [radius=2pt] node[anchor=west] { \footnotesize \tt 4 };

            \draw (A) -- (K) -- (L);

            \draw[blue,thick] (K) -- (L) -- (H);
        \end{tikzpicture}
    \end{figure}

\end{frame}

\begin{frame}[fragile]{Geração do \textit{lower hull}}

    \begin{figure}
        \centering

        \begin{tikzpicture}
            \coordinate (A) at (0, 0);
            \coordinate (B) at (5, 3);
            \coordinate (C) at (8, -2);
            \coordinate (D) at (4, 4);
            \coordinate (E) at (2, 1);
            \coordinate (F) at (2, 5);
            \coordinate (G) at (3, -1);
            \coordinate (H) at (7, 2);
            \coordinate (I) at (5, 0);
            \coordinate (J) at (0, 4);
            \coordinate (K) at (1, -1);
            \coordinate (L) at (7, -2);
            \coordinate (M) at (6, 4);
            \coordinate (N) at (6, 0);
            \coordinate (O) at (1, 3);

            \fill (A) circle [radius=2pt] node[anchor=west] { \footnotesize \tt 1 };
            \fill (B) circle [radius=2pt] node[anchor=west] { \footnotesize \tt 10 };
            \fill (C) circle [radius=2pt] node[anchor=west] { \footnotesize \tt 15 };
            \fill (D) circle [radius=2pt] node[anchor=west] { \footnotesize \tt 8 };
            \fill (E) circle [radius=2pt] node[anchor=west] { \footnotesize \tt 5 };
            \fill (F) circle [radius=2pt] node[anchor=west] { \footnotesize \tt 6 };
            \fill (G) circle [radius=2pt] node[anchor=south] { \footnotesize \tt 7 };
            \fill (H) circle [radius=2pt] node[anchor=west] { \footnotesize \tt 14 };
            \fill (I) circle [radius=2pt] node[anchor=west] { \footnotesize \tt 9 };
            \fill (J) circle [radius=2pt] node[anchor=west] { \footnotesize \tt 2 };
            \fill (K) circle [radius=2pt] node[anchor=north] { \footnotesize \tt 3 };
            \fill (L) circle [radius=2pt] node[anchor=west] { \footnotesize \tt 13 };
            \fill (M) circle [radius=2pt] node[anchor=west] { \footnotesize \tt 12 };
            \fill (N) circle [radius=2pt] node[anchor=west] { \footnotesize \tt 11 };
            \fill (O) circle [radius=2pt] node[anchor=west] { \footnotesize \tt 4 };

            \draw (A) -- (K) -- (L);

            \draw[blue,thick] (L) -- (H) -- (C);
        \end{tikzpicture}
    \end{figure}

\end{frame}

\begin{frame}[fragile]{Geração do \textit{lower hull}}

    \begin{figure}
        \centering

        \begin{tikzpicture}
            \coordinate (A) at (0, 0);
            \coordinate (B) at (5, 3);
            \coordinate (C) at (8, -2);
            \coordinate (D) at (4, 4);
            \coordinate (E) at (2, 1);
            \coordinate (F) at (2, 5);
            \coordinate (G) at (3, -1);
            \coordinate (H) at (7, 2);
            \coordinate (I) at (5, 0);
            \coordinate (J) at (0, 4);
            \coordinate (K) at (1, -1);
            \coordinate (L) at (7, -2);
            \coordinate (M) at (6, 4);
            \coordinate (N) at (6, 0);
            \coordinate (O) at (1, 3);

            \fill (A) circle [radius=2pt] node[anchor=west] { \footnotesize \tt 1 };
            \fill (B) circle [radius=2pt] node[anchor=west] { \footnotesize \tt 10 };
            \fill (C) circle [radius=2pt] node[anchor=west] { \footnotesize \tt 15 };
            \fill (D) circle [radius=2pt] node[anchor=west] { \footnotesize \tt 8 };
            \fill (E) circle [radius=2pt] node[anchor=west] { \footnotesize \tt 5 };
            \fill (F) circle [radius=2pt] node[anchor=west] { \footnotesize \tt 6 };
            \fill (G) circle [radius=2pt] node[anchor=south] { \footnotesize \tt 7 };
            \fill (H) circle [radius=2pt] node[anchor=west] { \footnotesize \tt 14 };
            \fill (I) circle [radius=2pt] node[anchor=west] { \footnotesize \tt 9 };
            \fill (J) circle [radius=2pt] node[anchor=west] { \footnotesize \tt 2 };
            \fill (K) circle [radius=2pt] node[anchor=north] { \footnotesize \tt 3 };
            \fill (L) circle [radius=2pt] node[anchor=south] { \footnotesize \tt 13 };
            \fill (M) circle [radius=2pt] node[anchor=west] { \footnotesize \tt 12 };
            \fill (N) circle [radius=2pt] node[anchor=west] { \footnotesize \tt 11 };
            \fill (O) circle [radius=2pt] node[anchor=west] { \footnotesize \tt 4 };

            \draw (A) -- (K) -- (L);

            \draw[blue,thick] (K) -- (L) -- (C);
        \end{tikzpicture}
    \end{figure}

\end{frame}

\begin{frame}[fragile]{Geração do \textit{lower hull}}

    \begin{figure}
        \centering

        \begin{tikzpicture}
            \coordinate (A) at (0, 0);
            \coordinate (B) at (5, 3);
            \coordinate (C) at (8, -2);
            \coordinate (D) at (4, 4);
            \coordinate (E) at (2, 1);
            \coordinate (F) at (2, 5);
            \coordinate (G) at (3, -1);
            \coordinate (H) at (7, 2);
            \coordinate (I) at (5, 0);
            \coordinate (J) at (0, 4);
            \coordinate (K) at (1, -1);
            \coordinate (L) at (7, -2);
            \coordinate (M) at (6, 4);
            \coordinate (N) at (6, 0);
            \coordinate (O) at (1, 3);

            \fill (A) circle [radius=2pt] node[anchor=west] { \footnotesize \tt 1 };
            \fill (B) circle [radius=2pt] node[anchor=west] { \footnotesize \tt 10 };
            \fill (C) circle [radius=2pt] node[anchor=west] { \footnotesize \tt 15 };
            \fill (D) circle [radius=2pt] node[anchor=west] { \footnotesize \tt 8 };
            \fill (E) circle [radius=2pt] node[anchor=west] { \footnotesize \tt 5 };
            \fill (F) circle [radius=2pt] node[anchor=west] { \footnotesize \tt 6 };
            \fill (G) circle [radius=2pt] node[anchor=south] { \footnotesize \tt 7 };
            \fill (H) circle [radius=2pt] node[anchor=west] { \footnotesize \tt 14 };
            \fill (I) circle [radius=2pt] node[anchor=west] { \footnotesize \tt 9 };
            \fill (J) circle [radius=2pt] node[anchor=west] { \footnotesize \tt 2 };
            \fill (K) circle [radius=2pt] node[anchor=north] { \footnotesize \tt 3 };
            \fill (L) circle [radius=2pt] node[anchor=south] { \footnotesize \tt 13 };
            \fill (M) circle [radius=2pt] node[anchor=west] { \footnotesize \tt 12 };
            \fill (N) circle [radius=2pt] node[anchor=west] { \footnotesize \tt 11 };
            \fill (O) circle [radius=2pt] node[anchor=west] { \footnotesize \tt 4 };

            \draw (A) -- (K) -- (L) -- (C);

        \end{tikzpicture}
    \end{figure}

\end{frame}


\begin{frame}[fragile]{Implementação da geração do envoltório convexo}
    \inputsnippet{cpp}{23}{40}{andrew.cpp}
\end{frame}

\begin{frame}[fragile]{Implementação da geração do envoltório convexo}
    \inputsnippet{cpp}{42}{61}{andrew.cpp}
\end{frame}




