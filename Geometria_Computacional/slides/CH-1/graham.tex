\section{Algoritmo de Graham}

\begin{frame}[fragile]{Algoritmo de Graham}

    \begin{itemize}
        \item O algoritmo de Graham (\textit{Graham Scan}, no original), foi proposto por
            Ronald Graham em 1972

        \item Ele iniciamente ordena todos os $N$ pontos de $P$ de acordo com o ângulo que eles 
            foram com um ponto pivô fixado previamente

        \item A escolha padrão para o pivô é o ponto de menor coordenada $y$ 

        \item Caso exista mais de um ponto com coordenada $y$ mínima, escolhe-se o de maior 
            coordenada $x$ dentre eles

        \item Se $P$ é armazenado em um vetor, o algoritmo pode ser simplificado movendo-se o pivô 
            para a primeira posição
    \end{itemize}

\end{frame}

\begin{frame}[fragile]{Implementação da escolha do pivô}
    \inputsnippet{cpp}{34}{51}{graham.cpp}
\end{frame}

\begin{frame}[fragile]{Ordenação dos pontos de acordo com o ângulo}

    \begin{itemize}
        \item Para realizar a ordenação dos pontos é preciso definir um operador booleano que
        receba dois pontos $P$ e $Q$ e retorne verdadeiro se $P$ antecede $Q$ de
        acordo com a ordenação proposta

        \item Como é necessário o conhecimento do pivô para tal ordenação, há três possibilidades 
            para a implementação deste operador:

        \begin{enumerate}
            \item implementar o operator \code{c}{<} da classe \code{c}{Point}, tornando o pivô um 
                membro da classe para que o operador tenha acesso a ele;
            \item tornar o pivô uma variável global;
            \item usar uma função lambda no terceiro parâmetro da função \code{c}{sort()}, 
                capturando o pivô por referência ou cópia
        \end{enumerate}

        \item O ângulo que o vetor diferença entre o vetor-posição do pivô e o vetor posição de um
            ponto do conjunto $P$ faz com o eixo-$x$ positivo pode ser obtido através da função
            \code{c}{atan2()} da biblioteca \code{c}{math.h} da linguagem C/C++

    \end{itemize}

\end{frame}

\documentclass[12pt]{article}
\usepackage{amsmath}

\pagestyle{empty}

\begin{document}

\[
    \cos \theta = \frac{u \cdot v}{|u||v|} = \frac{u_xv_x + u_yv_y}{(\sqrt{u_x^2 + u_y^2})(\sqrt{v_x^2 + v_y^2})}
\]

\end{document}


\begin{frame}[fragile]{Implementação da rotina de ordenação dos pontos}
    \inputsnippet{cpp}{52}{69}{graham.cpp}
\end{frame}

\begin{frame}[fragile]{Identificação do envoltório convexo}

    \begin{itemize}
        \item Após a ordenação dos pontos, o algoritmo procede empilhando
            três pontos de $P$: inicialmente os pontos cujos índices são $n - 1, 0$ e $1$

        \item  O invariante a ser mantido é que os três elementos do topo de pilha estão em 
            sentido anti-horário ($D > 0$)

        \item Para cada um dos demais pontos $Q_i$ de $P$, com $i = 2, 3, \ldots, n - 1$, 
            verifica-se se este ponto mantem o sentido anti-horário com os dois elementos do topo 
            da pilha

        \item Em caso afirmativo, o ponto é inserido na pilha 

        \item Caso contrário, remove-se o topo da pilha e se verifica o invariante para
            $Q_i$ novamente

        \item Como cada ponto é ou inserido ou removido uma única vez, este processo tem 
            complexidade $O(N)$, e o algoritmo como um todo tem complexidade $O(N\log N)$, 
            devido à ordenação
    \end{itemize}

\end{frame}

\begin{frame}[fragile]{Visualização do algoritmo de Graham}

    \begin{figure}
        \centering

        \begin{tikzpicture}
            \coordinate (A) at (0, 0);
            \coordinate (B) at (5, 3);
            \coordinate (C) at (8, -2);
            \coordinate (D) at (4, 4);
            \coordinate (E) at (2, 1);
            \coordinate (F) at (2, 5);
            \coordinate (G) at (3, -1);
            \coordinate (H) at (7, 2);
            \coordinate (I) at (5, 0);
            \coordinate (J) at (0, 4);
            \coordinate (K) at (1, -1);
            \coordinate (L) at (7, -2);
            \coordinate (M) at (6, 4);
            \coordinate (N) at (6, 0);
            \coordinate (O) at (1, 3);

            \fill (A) circle [radius=2pt] node[anchor=west] { \footnotesize \tt 11 };
            \fill (B) circle [radius=2pt] node[anchor=west] { \footnotesize \tt 3 };
            \fill (C) circle [radius=2pt] node[anchor=west] { \hspace{0.01in} pivô };
            \fill (D) circle [radius=2pt] node[anchor=west] { \footnotesize \tt 4 };
            \fill (E) circle [radius=2pt] node[anchor=west] { \footnotesize \tt 10 };
            \fill (F) circle [radius=2pt] node[anchor=west] { \footnotesize \tt 5 };
            \fill (G) circle [radius=2pt] node[anchor=west] { \footnotesize \tt 12 };
            \fill (H) circle [radius=2pt] node[anchor=west] { \footnotesize \tt 1 };
            \fill (I) circle [radius=2pt] node[anchor=west] { \footnotesize \tt 9 };
            \fill (J) circle [radius=2pt] node[anchor=west] { \footnotesize \tt 7 };
            \fill (K) circle [radius=2pt] node[anchor=west] { \footnotesize \tt 13 };
            \fill (L) circle [radius=2pt] node[anchor=west] { \footnotesize \tt 14 };
            \fill (M) circle [radius=2pt] node[anchor=west] { \footnotesize \tt 2 };
            \fill (N) circle [radius=2pt] node[anchor=west] { \footnotesize \tt 6 };
            \fill (O) circle [radius=2pt] node[anchor=west] { \footnotesize \tt 8 };

        \end{tikzpicture}
    \end{figure}

\end{frame}

\begin{frame}[fragile]{Visualização do algoritmo de Graham}

    \begin{figure}
        \centering

        \begin{tikzpicture}
            \coordinate (A) at (0, 0);
            \coordinate (B) at (5, 3);
            \coordinate (C) at (8, -2);
            \coordinate (D) at (4, 4);
            \coordinate (E) at (2, 1);
            \coordinate (F) at (2, 5);
            \coordinate (G) at (3, -1);
            \coordinate (H) at (7, 2);
            \coordinate (I) at (5, 0);
            \coordinate (J) at (0, 4);
            \coordinate (K) at (1, -1);
            \coordinate (L) at (7, -2);
            \coordinate (M) at (6, 4);
            \coordinate (N) at (6, 0);
            \coordinate (O) at (1, 3);

            \fill (A) circle [radius=2pt] node[anchor=west] { \footnotesize \tt 11 };
            \fill (B) circle [radius=2pt] node[anchor=west] { \footnotesize \tt 3 };
            \fill (C) circle [radius=2pt] node[anchor=west] { \hspace{0.01in} pivô };
            \fill (D) circle [radius=2pt] node[anchor=west] { \footnotesize \tt 4 };
            \fill (E) circle [radius=2pt] node[anchor=west] { \footnotesize \tt 10 };
            \fill (F) circle [radius=2pt] node[anchor=west] { \footnotesize \tt 5 };
            \fill (G) circle [radius=2pt] node[anchor=west] { \footnotesize \tt 12 };
            \fill (H) circle [radius=2pt] node[anchor=west] { \footnotesize \tt 1 };
            \fill (I) circle [radius=2pt] node[anchor=west] { \footnotesize \tt 9 };
            \fill (J) circle [radius=2pt] node[anchor=west] { \footnotesize \tt 7 };
            \fill (K) circle [radius=2pt] node[anchor=west] { \footnotesize \tt 13 };
            \fill (L) circle [radius=2pt] node[anchor=north] { \footnotesize \tt 14 };
            \fill (M) circle [radius=2pt] node[anchor=west] { \footnotesize \tt 2 };
            \fill (N) circle [radius=2pt] node[anchor=west] { \footnotesize \tt 6 };
            \fill (O) circle [radius=2pt] node[anchor=west] { \footnotesize \tt 8 };

            \draw (L) -- (C) -- (H);

        \end{tikzpicture}
    \end{figure}

\end{frame}

\begin{frame}[fragile]{Visualização do algoritmo de Graham}

    \begin{figure}
        \centering

        \begin{tikzpicture}
            \coordinate (A) at (0, 0);
            \coordinate (B) at (5, 3);
            \coordinate (C) at (8, -2);
            \coordinate (D) at (4, 4);
            \coordinate (E) at (2, 1);
            \coordinate (F) at (2, 5);
            \coordinate (G) at (3, -1);
            \coordinate (H) at (7, 2);
            \coordinate (I) at (5, 0);
            \coordinate (J) at (0, 4);
            \coordinate (K) at (1, -1);
            \coordinate (L) at (7, -2);
            \coordinate (M) at (6, 4);
            \coordinate (N) at (6, 0);
            \coordinate (O) at (1, 3);

            \fill (A) circle [radius=2pt] node[anchor=west] { \footnotesize \tt 11 };
            \fill (B) circle [radius=2pt] node[anchor=west] { \footnotesize \tt 3 };
            \fill (C) circle [radius=2pt] node[anchor=west] { \hspace{0.01in} pivô };
            \fill (D) circle [radius=2pt] node[anchor=west] { \footnotesize \tt 4 };
            \fill (E) circle [radius=2pt] node[anchor=west] { \footnotesize \tt 10 };
            \fill (F) circle [radius=2pt] node[anchor=west] { \footnotesize \tt 5 };
            \fill (G) circle [radius=2pt] node[anchor=west] { \footnotesize \tt 12 };
            \fill (H) circle [radius=2pt] node[anchor=west] { \footnotesize \tt 1 };
            \fill (I) circle [radius=2pt] node[anchor=west] { \footnotesize \tt 9 };
            \fill (J) circle [radius=2pt] node[anchor=west] { \footnotesize \tt 7 };
            \fill (K) circle [radius=2pt] node[anchor=west] { \footnotesize \tt 13 };
            \fill (L) circle [radius=2pt] node[anchor=north] { \footnotesize \tt 14 };
            \fill (M) circle [radius=2pt] node[anchor=west] { \footnotesize \tt 2 };
            \fill (N) circle [radius=2pt] node[anchor=west] { \footnotesize \tt 6 };
            \fill (O) circle [radius=2pt] node[anchor=west] { \footnotesize \tt 8 };

            \draw (L) -- (C) -- (H);
            \draw[blue,thick] (C) -- (H) -- (M);

        \end{tikzpicture}
    \end{figure}

\end{frame}

\begin{frame}[fragile]{Visualização do algoritmo de Graham}

    \begin{figure}
        \centering

        \begin{tikzpicture}
            \coordinate (A) at (0, 0);
            \coordinate (B) at (5, 3);
            \coordinate (C) at (8, -2);
            \coordinate (D) at (4, 4);
            \coordinate (E) at (2, 1);
            \coordinate (F) at (2, 5);
            \coordinate (G) at (3, -1);
            \coordinate (H) at (7, 2);
            \coordinate (I) at (5, 0);
            \coordinate (J) at (0, 4);
            \coordinate (K) at (1, -1);
            \coordinate (L) at (7, -2);
            \coordinate (M) at (6, 4);
            \coordinate (N) at (6, 0);
            \coordinate (O) at (1, 3);

            \fill (A) circle [radius=2pt] node[anchor=west] { \footnotesize \tt 11 };
            \fill (B) circle [radius=2pt] node[anchor=west] { \footnotesize \tt 3 };
            \fill (C) circle [radius=2pt] node[anchor=west] { \hspace{0.01in} pivô };
            \fill (D) circle [radius=2pt] node[anchor=west] { \footnotesize \tt 4 };
            \fill (E) circle [radius=2pt] node[anchor=west] { \footnotesize \tt 10 };
            \fill (F) circle [radius=2pt] node[anchor=west] { \footnotesize \tt 5 };
            \fill (G) circle [radius=2pt] node[anchor=west] { \footnotesize \tt 12 };
            \fill (H) circle [radius=2pt] node[anchor=west] { \footnotesize \tt 1 };
            \fill (I) circle [radius=2pt] node[anchor=west] { \footnotesize \tt 9 };
            \fill (J) circle [radius=2pt] node[anchor=west] { \footnotesize \tt 7 };
            \fill (K) circle [radius=2pt] node[anchor=west] { \footnotesize \tt 13 };
            \fill (L) circle [radius=2pt] node[anchor=north] { \footnotesize \tt 14 };
            \fill (M) circle [radius=2pt] node[anchor=west] { \footnotesize \tt 2 };
            \fill (N) circle [radius=2pt] node[anchor=west] { \footnotesize \tt 6 };
            \fill (O) circle [radius=2pt] node[anchor=west] { \footnotesize \tt 8 };

            \draw (L) -- (C) -- (H);
            \draw[blue,thick] (H) -- (M) -- (B);

        \end{tikzpicture}
    \end{figure}

\end{frame}

\begin{frame}[fragile]{Visualização do algoritmo de Graham}

    \begin{figure}
        \centering

        \begin{tikzpicture}
            \coordinate (A) at (0, 0);
            \coordinate (B) at (5, 3);
            \coordinate (C) at (8, -2);
            \coordinate (D) at (4, 4);
            \coordinate (E) at (2, 1);
            \coordinate (F) at (2, 5);
            \coordinate (G) at (3, -1);
            \coordinate (H) at (7, 2);
            \coordinate (I) at (5, 0);
            \coordinate (J) at (0, 4);
            \coordinate (K) at (1, -1);
            \coordinate (L) at (7, -2);
            \coordinate (M) at (6, 4);
            \coordinate (N) at (6, 0);
            \coordinate (O) at (1, 3);

            \fill (A) circle [radius=2pt] node[anchor=west] { \footnotesize \tt 11 };
            \fill (B) circle [radius=2pt] node[anchor=west] { \footnotesize \tt 3 };
            \fill (C) circle [radius=2pt] node[anchor=west] { \hspace{0.01in} pivô };
            \fill (D) circle [radius=2pt] node[anchor=west] { \footnotesize \tt 4 };
            \fill (E) circle [radius=2pt] node[anchor=west] { \footnotesize \tt 10 };
            \fill (F) circle [radius=2pt] node[anchor=west] { \footnotesize \tt 5 };
            \fill (G) circle [radius=2pt] node[anchor=west] { \footnotesize \tt 12 };
            \fill (H) circle [radius=2pt] node[anchor=west] { \footnotesize \tt 1 };
            \fill (I) circle [radius=2pt] node[anchor=west] { \footnotesize \tt 9 };
            \fill (J) circle [radius=2pt] node[anchor=west] { \footnotesize \tt 7 };
            \fill (K) circle [radius=2pt] node[anchor=west] { \footnotesize \tt 13 };
            \fill (L) circle [radius=2pt] node[anchor=north] { \footnotesize \tt 14 };
            \fill (M) circle [radius=2pt] node[anchor=west] { \footnotesize \tt 2 };
            \fill (N) circle [radius=2pt] node[anchor=west] { \footnotesize \tt 6 };
            \fill (O) circle [radius=2pt] node[anchor=west] { \footnotesize \tt 8 };

            \draw (L) -- (C) -- (H) -- (M);
            %\draw[blue,thick] (H) -- (M) -- (B);
            \draw[blue,thick] (M) -- (B) -- (D);

        \end{tikzpicture}
    \end{figure}

\end{frame}

\begin{frame}[fragile]{Visualização do algoritmo de Graham}

    \begin{figure}
        \centering

        \begin{tikzpicture}
            \coordinate (A) at (0, 0);
            \coordinate (B) at (5, 3);
            \coordinate (C) at (8, -2);
            \coordinate (D) at (4, 4);
            \coordinate (E) at (2, 1);
            \coordinate (F) at (2, 5);
            \coordinate (G) at (3, -1);
            \coordinate (H) at (7, 2);
            \coordinate (I) at (5, 0);
            \coordinate (J) at (0, 4);
            \coordinate (K) at (1, -1);
            \coordinate (L) at (7, -2);
            \coordinate (M) at (6, 4);
            \coordinate (N) at (6, 0);
            \coordinate (O) at (1, 3);

            \fill (A) circle [radius=2pt] node[anchor=west] { \footnotesize \tt 11 };
            \fill (B) circle [radius=2pt] node[anchor=west] { \footnotesize \tt 3 };
            \fill (C) circle [radius=2pt] node[anchor=west] { \hspace{0.01in} pivô };
            \fill (D) circle [radius=2pt] node[anchor=north] { \footnotesize \tt 4 };
            \fill (E) circle [radius=2pt] node[anchor=west] { \footnotesize \tt 10 };
            \fill (F) circle [radius=2pt] node[anchor=west] { \footnotesize \tt 5 };
            \fill (G) circle [radius=2pt] node[anchor=west] { \footnotesize \tt 12 };
            \fill (H) circle [radius=2pt] node[anchor=west] { \footnotesize \tt 1 };
            \fill (I) circle [radius=2pt] node[anchor=west] { \footnotesize \tt 9 };
            \fill (J) circle [radius=2pt] node[anchor=west] { \footnotesize \tt 7 };
            \fill (K) circle [radius=2pt] node[anchor=west] { \footnotesize \tt 13 };
            \fill (L) circle [radius=2pt] node[anchor=north] { \footnotesize \tt 14 };
            \fill (M) circle [radius=2pt] node[anchor=west] { \footnotesize \tt 2 };
            \fill (N) circle [radius=2pt] node[anchor=west] { \footnotesize \tt 6 };
            \fill (O) circle [radius=2pt] node[anchor=west] { \footnotesize \tt 8 };

            \draw (L) -- (C) -- (H) -- (M);
            \draw[blue,thick] (H) -- (M) -- (D);

        \end{tikzpicture}
    \end{figure}

\end{frame}

\begin{frame}[fragile]{Visualização do algoritmo de Graham}

    \begin{figure}
        \centering

        \begin{tikzpicture}
            \coordinate (A) at (0, 0);
            \coordinate (B) at (5, 3);
            \coordinate (C) at (8, -2);
            \coordinate (D) at (4, 4);
            \coordinate (E) at (2, 1);
            \coordinate (F) at (2, 5);
            \coordinate (G) at (3, -1);
            \coordinate (H) at (7, 2);
            \coordinate (I) at (5, 0);
            \coordinate (J) at (0, 4);
            \coordinate (K) at (1, -1);
            \coordinate (L) at (7, -2);
            \coordinate (M) at (6, 4);
            \coordinate (N) at (6, 0);
            \coordinate (O) at (1, 3);

            \fill (A) circle [radius=2pt] node[anchor=west] { \footnotesize \tt 11 };
            \fill (B) circle [radius=2pt] node[anchor=west] { \footnotesize \tt 3 };
            \fill (C) circle [radius=2pt] node[anchor=west] { \hspace{0.01in} pivô };
            \fill (D) circle [radius=2pt] node[anchor=north] { \footnotesize \tt 4 };
            \fill (E) circle [radius=2pt] node[anchor=west] { \footnotesize \tt 10 };
            \fill (F) circle [radius=2pt] node[anchor=west] { \footnotesize \tt 5 };
            \fill (G) circle [radius=2pt] node[anchor=west] { \footnotesize \tt 12 };
            \fill (H) circle [radius=2pt] node[anchor=west] { \footnotesize \tt 1 };
            \fill (I) circle [radius=2pt] node[anchor=west] { \footnotesize \tt 9 };
            \fill (J) circle [radius=2pt] node[anchor=west] { \footnotesize \tt 7 };
            \fill (K) circle [radius=2pt] node[anchor=west] { \footnotesize \tt 13 };
            \fill (L) circle [radius=2pt] node[anchor=north] { \footnotesize \tt 14 };
            \fill (M) circle [radius=2pt] node[anchor=west] { \footnotesize \tt 2 };
            \fill (N) circle [radius=2pt] node[anchor=west] { \footnotesize \tt 6 };
            \fill (O) circle [radius=2pt] node[anchor=west] { \footnotesize \tt 8 };

            \draw (L) -- (C) -- (H) -- (M);
            \draw[blue,thick] (M) -- (D) -- (F);

        \end{tikzpicture}
    \end{figure}

\end{frame}

\begin{frame}[fragile]{Visualização do algoritmo de Graham}

    \begin{figure}
        \centering

        \begin{tikzpicture}
            \coordinate (A) at (0, 0);
            \coordinate (B) at (5, 3);
            \coordinate (C) at (8, -2);
            \coordinate (D) at (4, 4);
            \coordinate (E) at (2, 1);
            \coordinate (F) at (2, 5);
            \coordinate (G) at (3, -1);
            \coordinate (H) at (7, 2);
            \coordinate (I) at (5, 0);
            \coordinate (J) at (0, 4);
            \coordinate (K) at (1, -1);
            \coordinate (L) at (7, -2);
            \coordinate (M) at (6, 4);
            \coordinate (N) at (6, 0);
            \coordinate (O) at (1, 3);

            \fill (A) circle [radius=2pt] node[anchor=west] { \footnotesize \tt 11 };
            \fill (B) circle [radius=2pt] node[anchor=west] { \footnotesize \tt 3 };
            \fill (C) circle [radius=2pt] node[anchor=west] { \hspace{0.01in} pivô };
            \fill (D) circle [radius=2pt] node[anchor=north] { \footnotesize \tt 4 };
            \fill (E) circle [radius=2pt] node[anchor=west] { \footnotesize \tt 10 };
            \fill (F) circle [radius=2pt] node[anchor=north] { \footnotesize \tt 5 };
            \fill (G) circle [radius=2pt] node[anchor=west] { \footnotesize \tt 12 };
            \fill (H) circle [radius=2pt] node[anchor=west] { \footnotesize \tt 1 };
            \fill (I) circle [radius=2pt] node[anchor=west] { \footnotesize \tt 9 };
            \fill (J) circle [radius=2pt] node[anchor=west] { \footnotesize \tt 7 };
            \fill (K) circle [radius=2pt] node[anchor=west] { \footnotesize \tt 13 };
            \fill (L) circle [radius=2pt] node[anchor=north] { \footnotesize \tt 14 };
            \fill (M) circle [radius=2pt] node[anchor=west] { \footnotesize \tt 2 };
            \fill (N) circle [radius=2pt] node[anchor=west] { \footnotesize \tt 6 };
            \fill (O) circle [radius=2pt] node[anchor=west] { \footnotesize \tt 8 };

            \draw (L) -- (C) -- (H) -- (M);
            \draw[blue,thick] (H) -- (M) -- (F);

        \end{tikzpicture}
    \end{figure}

\end{frame}

\begin{frame}[fragile]{Visualização do algoritmo de Graham}

    \begin{figure}
        \centering

        \begin{tikzpicture}
            \coordinate (A) at (0, 0);
            \coordinate (B) at (5, 3);
            \coordinate (C) at (8, -2);
            \coordinate (D) at (4, 4);
            \coordinate (E) at (2, 1);
            \coordinate (F) at (2, 5);
            \coordinate (G) at (3, -1);
            \coordinate (H) at (7, 2);
            \coordinate (I) at (5, 0);
            \coordinate (J) at (0, 4);
            \coordinate (K) at (1, -1);
            \coordinate (L) at (7, -2);
            \coordinate (M) at (6, 4);
            \coordinate (N) at (6, 0);
            \coordinate (O) at (1, 3);

            \fill (A) circle [radius=2pt] node[anchor=west] { \footnotesize \tt 11 };
            \fill (B) circle [radius=2pt] node[anchor=west] { \footnotesize \tt 3 };
            \fill (C) circle [radius=2pt] node[anchor=west] { \hspace{0.01in} pivô };
            \fill (D) circle [radius=2pt] node[anchor=north] { \footnotesize \tt 4 };
            \fill (E) circle [radius=2pt] node[anchor=west] { \footnotesize \tt 10 };
            \fill (F) circle [radius=2pt] node[anchor=north] { \footnotesize \tt 5 };
            \fill (G) circle [radius=2pt] node[anchor=west] { \footnotesize \tt 12 };
            \fill (H) circle [radius=2pt] node[anchor=west] { \footnotesize \tt 1 };
            \fill (I) circle [radius=2pt] node[anchor=west] { \footnotesize \tt 9 };
            \fill (J) circle [radius=2pt] node[anchor=west] { \footnotesize \tt 7 };
            \fill (K) circle [radius=2pt] node[anchor=west] { \footnotesize \tt 13 };
            \fill (L) circle [radius=2pt] node[anchor=north] { \footnotesize \tt 14 };
            \fill (M) circle [radius=2pt] node[anchor=west] { \footnotesize \tt 2 };
            \fill (N) circle [radius=2pt] node[anchor=west] { \footnotesize \tt 6 };
            \fill (O) circle [radius=2pt] node[anchor=west] { \footnotesize \tt 8 };

            \draw (L) -- (C) -- (H) -- (M) -- (F);
            \draw[blue,thick] (M) -- (F) -- (N);

        \end{tikzpicture}
    \end{figure}

\end{frame}
 
\begin{frame}[fragile]{Visualização do algoritmo de Graham}

    \begin{figure}
        \centering

        \begin{tikzpicture}
            \coordinate (A) at (0, 0);
            \coordinate (B) at (5, 3);
            \coordinate (C) at (8, -2);
            \coordinate (D) at (4, 4);
            \coordinate (E) at (2, 1);
            \coordinate (F) at (2, 5);
            \coordinate (G) at (3, -1);
            \coordinate (H) at (7, 2);
            \coordinate (I) at (5, 0);
            \coordinate (J) at (0, 4);
            \coordinate (K) at (1, -1);
            \coordinate (L) at (7, -2);
            \coordinate (M) at (6, 4);
            \coordinate (N) at (6, 0);
            \coordinate (O) at (1, 3);

            \fill (A) circle [radius=2pt] node[anchor=west] { \footnotesize \tt 11 };
            \fill (B) circle [radius=2pt] node[anchor=west] { \footnotesize \tt 3 };
            \fill (C) circle [radius=2pt] node[anchor=west] { \hspace{0.01in} pivô };
            \fill (D) circle [radius=2pt] node[anchor=north] { \footnotesize \tt 4 };
            \fill (E) circle [radius=2pt] node[anchor=west] { \footnotesize \tt 10 };
            \fill (F) circle [radius=2pt] node[anchor=north] { \footnotesize \tt 5 };
            \fill (G) circle [radius=2pt] node[anchor=west] { \footnotesize \tt 12 };
            \fill (H) circle [radius=2pt] node[anchor=west] { \footnotesize \tt 1 };
            \fill (I) circle [radius=2pt] node[anchor=west] { \footnotesize \tt 9 };
            \fill (J) circle [radius=2pt] node[anchor=west] { \footnotesize \tt 7 };
            \fill (K) circle [radius=2pt] node[anchor=west] { \footnotesize \tt 13 };
            \fill (L) circle [radius=2pt] node[anchor=north] { \footnotesize \tt 14 };
            \fill (M) circle [radius=2pt] node[anchor=west] { \footnotesize \tt 2 };
            \fill (N) circle [radius=2pt] node[anchor=west] { \footnotesize \tt 6 };
            \fill (O) circle [radius=2pt] node[anchor=west] { \footnotesize \tt 8 };

            \draw (L) -- (C) -- (H) -- (M) -- (F);
            \draw[blue,thick] (F) -- (N) -- (J);

        \end{tikzpicture}
    \end{figure}

\end{frame}
 
\begin{frame}[fragile]{Visualização do algoritmo de Graham}

    \begin{figure}
        \centering

        \begin{tikzpicture}
            \coordinate (A) at (0, 0);
            \coordinate (B) at (5, 3);
            \coordinate (C) at (8, -2);
            \coordinate (D) at (4, 4);
            \coordinate (E) at (2, 1);
            \coordinate (F) at (2, 5);
            \coordinate (G) at (3, -1);
            \coordinate (H) at (7, 2);
            \coordinate (I) at (5, 0);
            \coordinate (J) at (0, 4);
            \coordinate (K) at (1, -1);
            \coordinate (L) at (7, -2);
            \coordinate (M) at (6, 4);
            \coordinate (N) at (6, 0);
            \coordinate (O) at (1, 3);

            \fill (A) circle [radius=2pt] node[anchor=west] { \footnotesize \tt 11 };
            \fill (B) circle [radius=2pt] node[anchor=west] { \footnotesize \tt 3 };
            \fill (C) circle [radius=2pt] node[anchor=west] { \hspace{0.01in} pivô };
            \fill (D) circle [radius=2pt] node[anchor=north] { \footnotesize \tt 4 };
            \fill (E) circle [radius=2pt] node[anchor=west] { \footnotesize \tt 10 };
            \fill (F) circle [radius=2pt] node[anchor=north] { \footnotesize \tt 5 };
            \fill (G) circle [radius=2pt] node[anchor=west] { \footnotesize \tt 12 };
            \fill (H) circle [radius=2pt] node[anchor=west] { \footnotesize \tt 1 };
            \fill (I) circle [radius=2pt] node[anchor=west] { \footnotesize \tt 9 };
            \fill (J) circle [radius=2pt] node[anchor=east] { \footnotesize \tt 7 };
            \fill (K) circle [radius=2pt] node[anchor=west] { \footnotesize \tt 13 };
            \fill (L) circle [radius=2pt] node[anchor=north] { \footnotesize \tt 14 };
            \fill (M) circle [radius=2pt] node[anchor=west] { \footnotesize \tt 2 };
            \fill (N) circle [radius=2pt] node[anchor=west] { \footnotesize \tt 6 };
            \fill (O) circle [radius=2pt] node[anchor=west] { \footnotesize \tt 8 };

            \draw (L) -- (C) -- (H) -- (M) -- (F);
            \draw[blue,thick] (M) -- (F) -- (J);

        \end{tikzpicture}
    \end{figure}

\end{frame}

\begin{frame}[fragile]{Visualização do algoritmo de Graham}

    \begin{figure}
        \centering

        \begin{tikzpicture}
            \coordinate (A) at (0, 0);
            \coordinate (B) at (5, 3);
            \coordinate (C) at (8, -2);
            \coordinate (D) at (4, 4);
            \coordinate (E) at (2, 1);
            \coordinate (F) at (2, 5);
            \coordinate (G) at (3, -1);
            \coordinate (H) at (7, 2);
            \coordinate (I) at (5, 0);
            \coordinate (J) at (0, 4);
            \coordinate (K) at (1, -1);
            \coordinate (L) at (7, -2);
            \coordinate (M) at (6, 4);
            \coordinate (N) at (6, 0);
            \coordinate (O) at (1, 3);

            \fill (A) circle [radius=2pt] node[anchor=west] { \footnotesize \tt 11 };
            \fill (B) circle [radius=2pt] node[anchor=west] { \footnotesize \tt 3 };
            \fill (C) circle [radius=2pt] node[anchor=west] { \hspace{0.01in} pivô };
            \fill (D) circle [radius=2pt] node[anchor=north] { \footnotesize \tt 4 };
            \fill (E) circle [radius=2pt] node[anchor=west] { \footnotesize \tt 10 };
            \fill (F) circle [radius=2pt] node[anchor=north] { \footnotesize \tt 5 };
            \fill (G) circle [radius=2pt] node[anchor=west] { \footnotesize \tt 12 };
            \fill (H) circle [radius=2pt] node[anchor=west] { \footnotesize \tt 1 };
            \fill (I) circle [radius=2pt] node[anchor=west] { \footnotesize \tt 9 };
            \fill (J) circle [radius=2pt] node[anchor=east] { \footnotesize \tt 7 };
            \fill (K) circle [radius=2pt] node[anchor=west] { \footnotesize \tt 13 };
            \fill (L) circle [radius=2pt] node[anchor=north] { \footnotesize \tt 14 };
            \fill (M) circle [radius=2pt] node[anchor=west] { \footnotesize \tt 2 };
            \fill (N) circle [radius=2pt] node[anchor=west] { \footnotesize \tt 6 };
            \fill (O) circle [radius=2pt] node[anchor=west] { \footnotesize \tt 8 };

            \draw (L) -- (C) -- (H) -- (M) -- (F) -- (J);
            \draw[blue,thick] (F) -- (J) -- (O);

        \end{tikzpicture}
    \end{figure}

\end{frame}

\begin{frame}[fragile]{Visualização do algoritmo de Graham}

    \begin{figure}
        \centering

        \begin{tikzpicture}
            \coordinate (A) at (0, 0);
            \coordinate (B) at (5, 3);
            \coordinate (C) at (8, -2);
            \coordinate (D) at (4, 4);
            \coordinate (E) at (2, 1);
            \coordinate (F) at (2, 5);
            \coordinate (G) at (3, -1);
            \coordinate (H) at (7, 2);
            \coordinate (I) at (5, 0);
            \coordinate (J) at (0, 4);
            \coordinate (K) at (1, -1);
            \coordinate (L) at (7, -2);
            \coordinate (M) at (6, 4);
            \coordinate (N) at (6, 0);
            \coordinate (O) at (1, 3);

            \fill (A) circle [radius=2pt] node[anchor=west] { \footnotesize \tt 11 };
            \fill (B) circle [radius=2pt] node[anchor=west] { \footnotesize \tt 3 };
            \fill (C) circle [radius=2pt] node[anchor=west] { \hspace{0.01in} pivô };
            \fill (D) circle [radius=2pt] node[anchor=north] { \footnotesize \tt 4 };
            \fill (E) circle [radius=2pt] node[anchor=west] { \footnotesize \tt 10 };
            \fill (F) circle [radius=2pt] node[anchor=north] { \footnotesize \tt 5 };
            \fill (G) circle [radius=2pt] node[anchor=west] { \footnotesize \tt 12 };
            \fill (H) circle [radius=2pt] node[anchor=west] { \footnotesize \tt 1 };
            \fill (I) circle [radius=2pt] node[anchor=west] { \footnotesize \tt 9 };
            \fill (J) circle [radius=2pt] node[anchor=east] { \footnotesize \tt 7 };
            \fill (K) circle [radius=2pt] node[anchor=west] { \footnotesize \tt 13 };
            \fill (L) circle [radius=2pt] node[anchor=north] { \footnotesize \tt 14 };
            \fill (M) circle [radius=2pt] node[anchor=west] { \footnotesize \tt 2 };
            \fill (N) circle [radius=2pt] node[anchor=west] { \footnotesize \tt 6 };
            \fill (O) circle [radius=2pt] node[anchor=west] { \footnotesize \tt 8 };

            \draw (L) -- (C) -- (H) -- (M) -- (F) -- (J);
            \draw[blue,thick] (J) -- (O) -- (I);

        \end{tikzpicture}
    \end{figure}

\end{frame}

\begin{frame}[fragile]{Visualização do algoritmo de Graham}

    \begin{figure}
        \centering

        \begin{tikzpicture}
            \coordinate (A) at (0, 0);
            \coordinate (B) at (5, 3);
            \coordinate (C) at (8, -2);
            \coordinate (D) at (4, 4);
            \coordinate (E) at (2, 1);
            \coordinate (F) at (2, 5);
            \coordinate (G) at (3, -1);
            \coordinate (H) at (7, 2);
            \coordinate (I) at (5, 0);
            \coordinate (J) at (0, 4);
            \coordinate (K) at (1, -1);
            \coordinate (L) at (7, -2);
            \coordinate (M) at (6, 4);
            \coordinate (N) at (6, 0);
            \coordinate (O) at (1, 3);

            \fill (A) circle [radius=2pt] node[anchor=west] { \footnotesize \tt 11 };
            \fill (B) circle [radius=2pt] node[anchor=west] { \footnotesize \tt 3 };
            \fill (C) circle [radius=2pt] node[anchor=west] { \hspace{0.01in} pivô };
            \fill (D) circle [radius=2pt] node[anchor=north] { \footnotesize \tt 4 };
            \fill (E) circle [radius=2pt] node[anchor=north] { \footnotesize \tt 10 };
            \fill (F) circle [radius=2pt] node[anchor=north] { \footnotesize \tt 5 };
            \fill (G) circle [radius=2pt] node[anchor=west] { \footnotesize \tt 12 };
            \fill (H) circle [radius=2pt] node[anchor=west] { \footnotesize \tt 1 };
            \fill (I) circle [radius=2pt] node[anchor=west] { \footnotesize \tt 9 };
            \fill (J) circle [radius=2pt] node[anchor=east] { \footnotesize \tt 7 };
            \fill (K) circle [radius=2pt] node[anchor=west] { \footnotesize \tt 13 };
            \fill (L) circle [radius=2pt] node[anchor=north] { \footnotesize \tt 14 };
            \fill (M) circle [radius=2pt] node[anchor=west] { \footnotesize \tt 2 };
            \fill (N) circle [radius=2pt] node[anchor=west] { \footnotesize \tt 6 };
            \fill (O) circle [radius=2pt] node[anchor=west] { \footnotesize \tt 8 };

            \draw (L) -- (C) -- (H) -- (M) -- (F) -- (J) -- (O);
            \draw[blue,thick] (O) -- (I) -- (E);

        \end{tikzpicture}
    \end{figure}

\end{frame}

\begin{frame}[fragile]{Visualização do algoritmo de Graham}

    \begin{figure}
        \centering

        \begin{tikzpicture}
            \coordinate (A) at (0, 0);
            \coordinate (B) at (5, 3);
            \coordinate (C) at (8, -2);
            \coordinate (D) at (4, 4);
            \coordinate (E) at (2, 1);
            \coordinate (F) at (2, 5);
            \coordinate (G) at (3, -1);
            \coordinate (H) at (7, 2);
            \coordinate (I) at (5, 0);
            \coordinate (J) at (0, 4);
            \coordinate (K) at (1, -1);
            \coordinate (L) at (7, -2);
            \coordinate (M) at (6, 4);
            \coordinate (N) at (6, 0);
            \coordinate (O) at (1, 3);

            \fill (A) circle [radius=2pt] node[anchor=west] { \footnotesize \tt 11 };
            \fill (B) circle [radius=2pt] node[anchor=west] { \footnotesize \tt 3 };
            \fill (C) circle [radius=2pt] node[anchor=west] { \hspace{0.01in} pivô };
            \fill (D) circle [radius=2pt] node[anchor=north] { \footnotesize \tt 4 };
            \fill (E) circle [radius=2pt] node[anchor=north] { \footnotesize \tt 10 };
            \fill (F) circle [radius=2pt] node[anchor=north] { \footnotesize \tt 5 };
            \fill (G) circle [radius=2pt] node[anchor=west] { \footnotesize \tt 12 };
            \fill (H) circle [radius=2pt] node[anchor=west] { \footnotesize \tt 1 };
            \fill (I) circle [radius=2pt] node[anchor=west] { \footnotesize \tt 9 };
            \fill (J) circle [radius=2pt] node[anchor=east] { \footnotesize \tt 7 };
            \fill (K) circle [radius=2pt] node[anchor=west] { \footnotesize \tt 13 };
            \fill (L) circle [radius=2pt] node[anchor=north] { \footnotesize \tt 14 };
            \fill (M) circle [radius=2pt] node[anchor=west] { \footnotesize \tt 2 };
            \fill (N) circle [radius=2pt] node[anchor=west] { \footnotesize \tt 6 };
            \fill (O) circle [radius=2pt] node[anchor=west] { \footnotesize \tt 8 };

            \draw (L) -- (C) -- (H) -- (M) -- (F) -- (J) -- (O);
            \draw[blue,thick] (J) -- (O) -- (E);

        \end{tikzpicture}
    \end{figure}

\end{frame}

\begin{frame}[fragile]{Visualização do algoritmo de Graham}

    \begin{figure}
        \centering

        \begin{tikzpicture}
            \coordinate (A) at (0, 0);
            \coordinate (B) at (5, 3);
            \coordinate (C) at (8, -2);
            \coordinate (D) at (4, 4);
            \coordinate (E) at (2, 1);
            \coordinate (F) at (2, 5);
            \coordinate (G) at (3, -1);
            \coordinate (H) at (7, 2);
            \coordinate (I) at (5, 0);
            \coordinate (J) at (0, 4);
            \coordinate (K) at (1, -1);
            \coordinate (L) at (7, -2);
            \coordinate (M) at (6, 4);
            \coordinate (N) at (6, 0);
            \coordinate (O) at (1, 3);

            \fill (A) circle [radius=2pt] node[anchor=west] { \footnotesize \tt 11 };
            \fill (B) circle [radius=2pt] node[anchor=west] { \footnotesize \tt 3 };
            \fill (C) circle [radius=2pt] node[anchor=west] { \hspace{0.01in} pivô };
            \fill (D) circle [radius=2pt] node[anchor=north] { \footnotesize \tt 4 };
            \fill (E) circle [radius=2pt] node[anchor=north] { \footnotesize \tt 10 };
            \fill (F) circle [radius=2pt] node[anchor=north] { \footnotesize \tt 5 };
            \fill (G) circle [radius=2pt] node[anchor=west] { \footnotesize \tt 12 };
            \fill (H) circle [radius=2pt] node[anchor=west] { \footnotesize \tt 1 };
            \fill (I) circle [radius=2pt] node[anchor=west] { \footnotesize \tt 9 };
            \fill (J) circle [radius=2pt] node[anchor=east] { \footnotesize \tt 7 };
            \fill (K) circle [radius=2pt] node[anchor=west] { \footnotesize \tt 13 };
            \fill (L) circle [radius=2pt] node[anchor=north] { \footnotesize \tt 14 };
            \fill (M) circle [radius=2pt] node[anchor=west] { \footnotesize \tt 2 };
            \fill (N) circle [radius=2pt] node[anchor=west] { \footnotesize \tt 6 };
            \fill (O) circle [radius=2pt] node[anchor=west] { \footnotesize \tt 8 };

            \draw (L) -- (C) -- (H) -- (M) -- (F) -- (J);
            \draw[blue,thick] (F) -- (J) -- (E);

        \end{tikzpicture}
    \end{figure}

\end{frame}

\begin{frame}[fragile]{Visualização do algoritmo de Graham}

    \begin{figure}
        \centering

        \begin{tikzpicture}
            \coordinate (A) at (0, 0);
            \coordinate (B) at (5, 3);
            \coordinate (C) at (8, -2);
            \coordinate (D) at (4, 4);
            \coordinate (E) at (2, 1);
            \coordinate (F) at (2, 5);
            \coordinate (G) at (3, -1);
            \coordinate (H) at (7, 2);
            \coordinate (I) at (5, 0);
            \coordinate (J) at (0, 4);
            \coordinate (K) at (1, -1);
            \coordinate (L) at (7, -2);
            \coordinate (M) at (6, 4);
            \coordinate (N) at (6, 0);
            \coordinate (O) at (1, 3);

            \fill (A) circle [radius=2pt] node[anchor=west] { \footnotesize \tt 11 };
            \fill (B) circle [radius=2pt] node[anchor=west] { \footnotesize \tt 3 };
            \fill (C) circle [radius=2pt] node[anchor=west] { \hspace{0.01in} pivô };
            \fill (D) circle [radius=2pt] node[anchor=north] { \footnotesize \tt 4 };
            \fill (E) circle [radius=2pt] node[anchor=north] { \footnotesize \tt 10 };
            \fill (F) circle [radius=2pt] node[anchor=north] { \footnotesize \tt 5 };
            \fill (G) circle [radius=2pt] node[anchor=west] { \footnotesize \tt 12 };
            \fill (H) circle [radius=2pt] node[anchor=west] { \footnotesize \tt 1 };
            \fill (I) circle [radius=2pt] node[anchor=west] { \footnotesize \tt 9 };
            \fill (J) circle [radius=2pt] node[anchor=east] { \footnotesize \tt 7 };
            \fill (K) circle [radius=2pt] node[anchor=west] { \footnotesize \tt 13 };
            \fill (L) circle [radius=2pt] node[anchor=north] { \footnotesize \tt 14 };
            \fill (M) circle [radius=2pt] node[anchor=west] { \footnotesize \tt 2 };
            \fill (N) circle [radius=2pt] node[anchor=west] { \footnotesize \tt 6 };
            \fill (O) circle [radius=2pt] node[anchor=west] { \footnotesize \tt 8 };

            \draw (L) -- (C) -- (H) -- (M) -- (F) -- (J);
            \draw[blue,thick] (J) -- (E) -- (A);

        \end{tikzpicture}
    \end{figure}

\end{frame}

\begin{frame}[fragile]{Visualização do algoritmo de Graham}

    \begin{figure}
        \centering

        \begin{tikzpicture}
            \coordinate (A) at (0, 0);
            \coordinate (B) at (5, 3);
            \coordinate (C) at (8, -2);
            \coordinate (D) at (4, 4);
            \coordinate (E) at (2, 1);
            \coordinate (F) at (2, 5);
            \coordinate (G) at (3, -1);
            \coordinate (H) at (7, 2);
            \coordinate (I) at (5, 0);
            \coordinate (J) at (0, 4);
            \coordinate (K) at (1, -1);
            \coordinate (L) at (7, -2);
            \coordinate (M) at (6, 4);
            \coordinate (N) at (6, 0);
            \coordinate (O) at (1, 3);

            \fill (A) circle [radius=2pt] node[anchor=west] { \footnotesize \tt 11 };
            \fill (B) circle [radius=2pt] node[anchor=west] { \footnotesize \tt 3 };
            \fill (C) circle [radius=2pt] node[anchor=west] { \hspace{0.01in} pivô };
            \fill (D) circle [radius=2pt] node[anchor=north] { \footnotesize \tt 4 };
            \fill (E) circle [radius=2pt] node[anchor=north] { \footnotesize \tt 10 };
            \fill (F) circle [radius=2pt] node[anchor=north] { \footnotesize \tt 5 };
            \fill (G) circle [radius=2pt] node[anchor=west] { \footnotesize \tt 12 };
            \fill (H) circle [radius=2pt] node[anchor=west] { \footnotesize \tt 1 };
            \fill (I) circle [radius=2pt] node[anchor=west] { \footnotesize \tt 9 };
            \fill (J) circle [radius=2pt] node[anchor=east] { \footnotesize \tt 7 };
            \fill (K) circle [radius=2pt] node[anchor=west] { \footnotesize \tt 13 };
            \fill (L) circle [radius=2pt] node[anchor=north] { \footnotesize \tt 14 };
            \fill (M) circle [radius=2pt] node[anchor=west] { \footnotesize \tt 2 };
            \fill (N) circle [radius=2pt] node[anchor=west] { \footnotesize \tt 6 };
            \fill (O) circle [radius=2pt] node[anchor=west] { \footnotesize \tt 8 };

            \draw (L) -- (C) -- (H) -- (M) -- (F) -- (J);
            \draw[blue,thick] (F) -- (J) -- (A);

        \end{tikzpicture}
    \end{figure}

\end{frame}

\begin{frame}[fragile]{Visualização do algoritmo de Graham}

    \begin{figure}
        \centering

        \begin{tikzpicture}
            \coordinate (A) at (0, 0);
            \coordinate (B) at (5, 3);
            \coordinate (C) at (8, -2);
            \coordinate (D) at (4, 4);
            \coordinate (E) at (2, 1);
            \coordinate (F) at (2, 5);
            \coordinate (G) at (3, -1);
            \coordinate (H) at (7, 2);
            \coordinate (I) at (5, 0);
            \coordinate (J) at (0, 4);
            \coordinate (K) at (1, -1);
            \coordinate (L) at (7, -2);
            \coordinate (M) at (6, 4);
            \coordinate (N) at (6, 0);
            \coordinate (O) at (1, 3);

            \fill (A) circle [radius=2pt] node[anchor=west] { \footnotesize \tt 11 };
            \fill (B) circle [radius=2pt] node[anchor=west] { \footnotesize \tt 3 };
            \fill (C) circle [radius=2pt] node[anchor=west] { \hspace{0.01in} pivô };
            \fill (D) circle [radius=2pt] node[anchor=north] { \footnotesize \tt 4 };
            \fill (E) circle [radius=2pt] node[anchor=north] { \footnotesize \tt 10 };
            \fill (F) circle [radius=2pt] node[anchor=north] { \footnotesize \tt 5 };
            \fill (G) circle [radius=2pt] node[anchor=west] { \footnotesize \tt 12 };
            \fill (H) circle [radius=2pt] node[anchor=west] { \footnotesize \tt 1 };
            \fill (I) circle [radius=2pt] node[anchor=west] { \footnotesize \tt 9 };
            \fill (J) circle [radius=2pt] node[anchor=east] { \footnotesize \tt 7 };
            \fill (K) circle [radius=2pt] node[anchor=west] { \footnotesize \tt 13 };
            \fill (L) circle [radius=2pt] node[anchor=north] { \footnotesize \tt 14 };
            \fill (M) circle [radius=2pt] node[anchor=west] { \footnotesize \tt 2 };
            \fill (N) circle [radius=2pt] node[anchor=west] { \footnotesize \tt 6 };
            \fill (O) circle [radius=2pt] node[anchor=west] { \footnotesize \tt 8 };

            \draw (L) -- (C) -- (H) -- (M) -- (F) -- (J);
            \draw[blue,thick] (J) -- (A) -- (G);

        \end{tikzpicture}
    \end{figure}

\end{frame}

\begin{frame}[fragile]{Visualização do algoritmo de Graham}

    \begin{figure}
        \centering

        \begin{tikzpicture}
            \coordinate (A) at (0, 0);
            \coordinate (B) at (5, 3);
            \coordinate (C) at (8, -2);
            \coordinate (D) at (4, 4);
            \coordinate (E) at (2, 1);
            \coordinate (F) at (2, 5);
            \coordinate (G) at (3, -1);
            \coordinate (H) at (7, 2);
            \coordinate (I) at (5, 0);
            \coordinate (J) at (0, 4);
            \coordinate (K) at (1, -1);
            \coordinate (L) at (7, -2);
            \coordinate (M) at (6, 4);
            \coordinate (N) at (6, 0);
            \coordinate (O) at (1, 3);

            \fill (A) circle [radius=2pt] node[anchor=east] { \footnotesize \tt 11 };
            \fill (B) circle [radius=2pt] node[anchor=west] { \footnotesize \tt 3 };
            \fill (C) circle [radius=2pt] node[anchor=west] { \hspace{0.01in} pivô };
            \fill (D) circle [radius=2pt] node[anchor=north] { \footnotesize \tt 4 };
            \fill (E) circle [radius=2pt] node[anchor=north] { \footnotesize \tt 10 };
            \fill (F) circle [radius=2pt] node[anchor=north] { \footnotesize \tt 5 };
            \fill (G) circle [radius=2pt] node[anchor=west] { \footnotesize \tt 12 };
            \fill (H) circle [radius=2pt] node[anchor=west] { \footnotesize \tt 1 };
            \fill (I) circle [radius=2pt] node[anchor=west] { \footnotesize \tt 9 };
            \fill (J) circle [radius=2pt] node[anchor=east] { \footnotesize \tt 7 };
            \fill (K) circle [radius=2pt] node[anchor=north] { \footnotesize \tt 13 };
            \fill (L) circle [radius=2pt] node[anchor=north] { \footnotesize \tt 14 };
            \fill (M) circle [radius=2pt] node[anchor=west] { \footnotesize \tt 2 };
            \fill (N) circle [radius=2pt] node[anchor=west] { \footnotesize \tt 6 };
            \fill (O) circle [radius=2pt] node[anchor=west] { \footnotesize \tt 8 };

            \draw (L) -- (C) -- (H) -- (M) -- (F) -- (J) -- (A);
            \draw[blue,thick] (A) -- (G) -- (K);

        \end{tikzpicture}
    \end{figure}

\end{frame}

\begin{frame}[fragile]{Visualização do algoritmo de Graham}

    \begin{figure}
        \centering

        \begin{tikzpicture}
            \coordinate (A) at (0, 0);
            \coordinate (B) at (5, 3);
            \coordinate (C) at (8, -2);
            \coordinate (D) at (4, 4);
            \coordinate (E) at (2, 1);
            \coordinate (F) at (2, 5);
            \coordinate (G) at (3, -1);
            \coordinate (H) at (7, 2);
            \coordinate (I) at (5, 0);
            \coordinate (J) at (0, 4);
            \coordinate (K) at (1, -1);
            \coordinate (L) at (7, -2);
            \coordinate (M) at (6, 4);
            \coordinate (N) at (6, 0);
            \coordinate (O) at (1, 3);

            \fill (A) circle [radius=2pt] node[anchor=east] { \footnotesize \tt 11 };
            \fill (B) circle [radius=2pt] node[anchor=west] { \footnotesize \tt 3 };
            \fill (C) circle [radius=2pt] node[anchor=west] { \hspace{0.01in} pivô };
            \fill (D) circle [radius=2pt] node[anchor=north] { \footnotesize \tt 4 };
            \fill (E) circle [radius=2pt] node[anchor=north] { \footnotesize \tt 10 };
            \fill (F) circle [radius=2pt] node[anchor=north] { \footnotesize \tt 5 };
            \fill (G) circle [radius=2pt] node[anchor=west] { \footnotesize \tt 12 };
            \fill (H) circle [radius=2pt] node[anchor=west] { \footnotesize \tt 1 };
            \fill (I) circle [radius=2pt] node[anchor=west] { \footnotesize \tt 9 };
            \fill (J) circle [radius=2pt] node[anchor=east] { \footnotesize \tt 7 };
            \fill (K) circle [radius=2pt] node[anchor=north] { \footnotesize \tt 13 };
            \fill (L) circle [radius=2pt] node[anchor=north] { \footnotesize \tt 14 };
            \fill (M) circle [radius=2pt] node[anchor=west] { \footnotesize \tt 2 };
            \fill (N) circle [radius=2pt] node[anchor=west] { \footnotesize \tt 6 };
            \fill (O) circle [radius=2pt] node[anchor=west] { \footnotesize \tt 8 };

            \draw (L) -- (C) -- (H) -- (M) -- (F) -- (J) -- (A);
            \draw[blue,thick] (J) -- (A) -- (K);

        \end{tikzpicture}
    \end{figure}

\end{frame}

\begin{frame}[fragile]{Visualização do algoritmo de Graham}

    \begin{figure}
        \centering

        \begin{tikzpicture}
            \coordinate (A) at (0, 0);
            \coordinate (B) at (5, 3);
            \coordinate (C) at (8, -2);
            \coordinate (D) at (4, 4);
            \coordinate (E) at (2, 1);
            \coordinate (F) at (2, 5);
            \coordinate (G) at (3, -1);
            \coordinate (H) at (7, 2);
            \coordinate (I) at (5, 0);
            \coordinate (J) at (0, 4);
            \coordinate (K) at (1, -1);
            \coordinate (L) at (7, -2);
            \coordinate (M) at (6, 4);
            \coordinate (N) at (6, 0);
            \coordinate (O) at (1, 3);

            \fill (A) circle [radius=2pt] node[anchor=east] { \footnotesize \tt 11 };
            \fill (B) circle [radius=2pt] node[anchor=west] { \footnotesize \tt 3 };
            \fill (C) circle [radius=2pt] node[anchor=west] { \hspace{0.01in} pivô };
            \fill (D) circle [radius=2pt] node[anchor=north] { \footnotesize \tt 4 };
            \fill (E) circle [radius=2pt] node[anchor=north] { \footnotesize \tt 10 };
            \fill (F) circle [radius=2pt] node[anchor=north] { \footnotesize \tt 5 };
            \fill (G) circle [radius=2pt] node[anchor=west] { \footnotesize \tt 12 };
            \fill (H) circle [radius=2pt] node[anchor=west] { \footnotesize \tt 1 };
            \fill (I) circle [radius=2pt] node[anchor=west] { \footnotesize \tt 9 };
            \fill (J) circle [radius=2pt] node[anchor=east] { \footnotesize \tt 7 };
            \fill (K) circle [radius=2pt] node[anchor=north] { \footnotesize \tt 13 };
            \fill (L) circle [radius=2pt] node[anchor=north] { \footnotesize \tt 14 };
            \fill (M) circle [radius=2pt] node[anchor=west] { \footnotesize \tt 2 };
            \fill (N) circle [radius=2pt] node[anchor=west] { \footnotesize \tt 6 };
            \fill (O) circle [radius=2pt] node[anchor=west] { \footnotesize \tt 8 };

            \draw (L) -- (C) -- (H) -- (M) -- (F) -- (J) -- (A);
            \draw[blue,thick] (A) -- (K) -- (L);

        \end{tikzpicture}
    \end{figure}

\end{frame}

\begin{frame}[fragile]{Visualização do algoritmo de Graham}

    \begin{figure}
        \centering

        \begin{tikzpicture}
            \coordinate (A) at (0, 0);
            \coordinate (B) at (5, 3);
            \coordinate (C) at (8, -2);
            \coordinate (D) at (4, 4);
            \coordinate (E) at (2, 1);
            \coordinate (F) at (2, 5);
            \coordinate (G) at (3, -1);
            \coordinate (H) at (7, 2);
            \coordinate (I) at (5, 0);
            \coordinate (J) at (0, 4);
            \coordinate (K) at (1, -1);
            \coordinate (L) at (7, -2);
            \coordinate (M) at (6, 4);
            \coordinate (N) at (6, 0);
            \coordinate (O) at (1, 3);

            \fill (A) circle [radius=2pt] node[anchor=east] { \footnotesize \tt 11 };
            \fill (B) circle [radius=2pt] node[anchor=west] { \footnotesize \tt 3 };
            \fill (C) circle [radius=2pt] node[anchor=west] { \hspace{0.01in} pivô };
            \fill (D) circle [radius=2pt] node[anchor=north] { \footnotesize \tt 4 };
            \fill (E) circle [radius=2pt] node[anchor=north] { \footnotesize \tt 10 };
            \fill (F) circle [radius=2pt] node[anchor=north] { \footnotesize \tt 5 };
            \fill (G) circle [radius=2pt] node[anchor=west] { \footnotesize \tt 12 };
            \fill (H) circle [radius=2pt] node[anchor=west] { \footnotesize \tt 1 };
            \fill (I) circle [radius=2pt] node[anchor=west] { \footnotesize \tt 9 };
            \fill (J) circle [radius=2pt] node[anchor=east] { \footnotesize \tt 7 };
            \fill (K) circle [radius=2pt] node[anchor=north] { \footnotesize \tt 13 };
            \fill (L) circle [radius=2pt] node[anchor=north] { \footnotesize \tt 14 };
            \fill (M) circle [radius=2pt] node[anchor=west] { \footnotesize \tt 2 };
            \fill (N) circle [radius=2pt] node[anchor=west] { \footnotesize \tt 6 };
            \fill (O) circle [radius=2pt] node[anchor=west] { \footnotesize \tt 8 };

            \draw (L) -- (C) -- (H) -- (M) -- (F) -- (J) -- (A) -- (K) -- (L);

        \end{tikzpicture}
    \end{figure}

\end{frame}


\begin{frame}[fragile]{Implementação da rotina de envoltório convexo}
    \inputsnippet{cpp}{70}{88}{graham.cpp}
\end{frame}

\begin{frame}[fragile]{Implementação da rotina de envoltório convexo}
    \inputsnippet{cpp}{89}{103}{graham.cpp}
\end{frame}




