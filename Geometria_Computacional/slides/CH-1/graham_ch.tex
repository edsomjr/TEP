\begin{frame}[fragile]{Visualização do algoritmo de Graham}

    \begin{figure}
        \centering

        \begin{tikzpicture}
            \coordinate (A) at (0, 0);
            \coordinate (B) at (5, 3);
            \coordinate (C) at (8, -2);
            \coordinate (D) at (4, 4);
            \coordinate (E) at (2, 1);
            \coordinate (F) at (2, 5);
            \coordinate (G) at (3, -1);
            \coordinate (H) at (7, 2);
            \coordinate (I) at (5, 0);
            \coordinate (J) at (0, 4);
            \coordinate (K) at (1, -1);
            \coordinate (L) at (7, -2);
            \coordinate (M) at (6, 4);
            \coordinate (N) at (6, 0);
            \coordinate (O) at (1, 3);

            \fill (A) circle [radius=2pt] node[anchor=west] { \footnotesize \tt 11 };
            \fill (B) circle [radius=2pt] node[anchor=west] { \footnotesize \tt 3 };
            \fill (C) circle [radius=2pt] node[anchor=west] { \hspace{0.01in} pivô };
            \fill (D) circle [radius=2pt] node[anchor=west] { \footnotesize \tt 4 };
            \fill (E) circle [radius=2pt] node[anchor=west] { \footnotesize \tt 10 };
            \fill (F) circle [radius=2pt] node[anchor=west] { \footnotesize \tt 5 };
            \fill (G) circle [radius=2pt] node[anchor=west] { \footnotesize \tt 12 };
            \fill (H) circle [radius=2pt] node[anchor=west] { \footnotesize \tt 1 };
            \fill (I) circle [radius=2pt] node[anchor=west] { \footnotesize \tt 9 };
            \fill (J) circle [radius=2pt] node[anchor=west] { \footnotesize \tt 7 };
            \fill (K) circle [radius=2pt] node[anchor=west] { \footnotesize \tt 13 };
            \fill (L) circle [radius=2pt] node[anchor=west] { \footnotesize \tt 14 };
            \fill (M) circle [radius=2pt] node[anchor=west] { \footnotesize \tt 2 };
            \fill (N) circle [radius=2pt] node[anchor=west] { \footnotesize \tt 6 };
            \fill (O) circle [radius=2pt] node[anchor=west] { \footnotesize \tt 8 };

        \end{tikzpicture}
    \end{figure}

\end{frame}

\begin{frame}[fragile]{Visualização do algoritmo de Graham}

    \begin{figure}
        \centering

        \begin{tikzpicture}
            \coordinate (A) at (0, 0);
            \coordinate (B) at (5, 3);
            \coordinate (C) at (8, -2);
            \coordinate (D) at (4, 4);
            \coordinate (E) at (2, 1);
            \coordinate (F) at (2, 5);
            \coordinate (G) at (3, -1);
            \coordinate (H) at (7, 2);
            \coordinate (I) at (5, 0);
            \coordinate (J) at (0, 4);
            \coordinate (K) at (1, -1);
            \coordinate (L) at (7, -2);
            \coordinate (M) at (6, 4);
            \coordinate (N) at (6, 0);
            \coordinate (O) at (1, 3);

            \fill (A) circle [radius=2pt] node[anchor=west] { \footnotesize \tt 11 };
            \fill (B) circle [radius=2pt] node[anchor=west] { \footnotesize \tt 3 };
            \fill (C) circle [radius=2pt] node[anchor=west] { \hspace{0.01in} pivô };
            \fill (D) circle [radius=2pt] node[anchor=west] { \footnotesize \tt 4 };
            \fill (E) circle [radius=2pt] node[anchor=west] { \footnotesize \tt 10 };
            \fill (F) circle [radius=2pt] node[anchor=west] { \footnotesize \tt 5 };
            \fill (G) circle [radius=2pt] node[anchor=west] { \footnotesize \tt 12 };
            \fill (H) circle [radius=2pt] node[anchor=west] { \footnotesize \tt 1 };
            \fill (I) circle [radius=2pt] node[anchor=west] { \footnotesize \tt 9 };
            \fill (J) circle [radius=2pt] node[anchor=west] { \footnotesize \tt 7 };
            \fill (K) circle [radius=2pt] node[anchor=west] { \footnotesize \tt 13 };
            \fill (L) circle [radius=2pt] node[anchor=north] { \footnotesize \tt 14 };
            \fill (M) circle [radius=2pt] node[anchor=west] { \footnotesize \tt 2 };
            \fill (N) circle [radius=2pt] node[anchor=west] { \footnotesize \tt 6 };
            \fill (O) circle [radius=2pt] node[anchor=west] { \footnotesize \tt 8 };

            \draw (L) -- (C) -- (H);

        \end{tikzpicture}
    \end{figure}

\end{frame}

\begin{frame}[fragile]{Visualização do algoritmo de Graham}

    \begin{figure}
        \centering

        \begin{tikzpicture}
            \coordinate (A) at (0, 0);
            \coordinate (B) at (5, 3);
            \coordinate (C) at (8, -2);
            \coordinate (D) at (4, 4);
            \coordinate (E) at (2, 1);
            \coordinate (F) at (2, 5);
            \coordinate (G) at (3, -1);
            \coordinate (H) at (7, 2);
            \coordinate (I) at (5, 0);
            \coordinate (J) at (0, 4);
            \coordinate (K) at (1, -1);
            \coordinate (L) at (7, -2);
            \coordinate (M) at (6, 4);
            \coordinate (N) at (6, 0);
            \coordinate (O) at (1, 3);

            \fill (A) circle [radius=2pt] node[anchor=west] { \footnotesize \tt 11 };
            \fill (B) circle [radius=2pt] node[anchor=west] { \footnotesize \tt 3 };
            \fill (C) circle [radius=2pt] node[anchor=west] { \hspace{0.01in} pivô };
            \fill (D) circle [radius=2pt] node[anchor=west] { \footnotesize \tt 4 };
            \fill (E) circle [radius=2pt] node[anchor=west] { \footnotesize \tt 10 };
            \fill (F) circle [radius=2pt] node[anchor=west] { \footnotesize \tt 5 };
            \fill (G) circle [radius=2pt] node[anchor=west] { \footnotesize \tt 12 };
            \fill (H) circle [radius=2pt] node[anchor=west] { \footnotesize \tt 1 };
            \fill (I) circle [radius=2pt] node[anchor=west] { \footnotesize \tt 9 };
            \fill (J) circle [radius=2pt] node[anchor=west] { \footnotesize \tt 7 };
            \fill (K) circle [radius=2pt] node[anchor=west] { \footnotesize \tt 13 };
            \fill (L) circle [radius=2pt] node[anchor=north] { \footnotesize \tt 14 };
            \fill (M) circle [radius=2pt] node[anchor=west] { \footnotesize \tt 2 };
            \fill (N) circle [radius=2pt] node[anchor=west] { \footnotesize \tt 6 };
            \fill (O) circle [radius=2pt] node[anchor=west] { \footnotesize \tt 8 };

            \draw (L) -- (C) -- (H);
            \draw[blue,thick] (C) -- (H) -- (M);

        \end{tikzpicture}
    \end{figure}

\end{frame}

\begin{frame}[fragile]{Visualização do algoritmo de Graham}

    \begin{figure}
        \centering

        \begin{tikzpicture}
            \coordinate (A) at (0, 0);
            \coordinate (B) at (5, 3);
            \coordinate (C) at (8, -2);
            \coordinate (D) at (4, 4);
            \coordinate (E) at (2, 1);
            \coordinate (F) at (2, 5);
            \coordinate (G) at (3, -1);
            \coordinate (H) at (7, 2);
            \coordinate (I) at (5, 0);
            \coordinate (J) at (0, 4);
            \coordinate (K) at (1, -1);
            \coordinate (L) at (7, -2);
            \coordinate (M) at (6, 4);
            \coordinate (N) at (6, 0);
            \coordinate (O) at (1, 3);

            \fill (A) circle [radius=2pt] node[anchor=west] { \footnotesize \tt 11 };
            \fill (B) circle [radius=2pt] node[anchor=west] { \footnotesize \tt 3 };
            \fill (C) circle [radius=2pt] node[anchor=west] { \hspace{0.01in} pivô };
            \fill (D) circle [radius=2pt] node[anchor=west] { \footnotesize \tt 4 };
            \fill (E) circle [radius=2pt] node[anchor=west] { \footnotesize \tt 10 };
            \fill (F) circle [radius=2pt] node[anchor=west] { \footnotesize \tt 5 };
            \fill (G) circle [radius=2pt] node[anchor=west] { \footnotesize \tt 12 };
            \fill (H) circle [radius=2pt] node[anchor=west] { \footnotesize \tt 1 };
            \fill (I) circle [radius=2pt] node[anchor=west] { \footnotesize \tt 9 };
            \fill (J) circle [radius=2pt] node[anchor=west] { \footnotesize \tt 7 };
            \fill (K) circle [radius=2pt] node[anchor=west] { \footnotesize \tt 13 };
            \fill (L) circle [radius=2pt] node[anchor=north] { \footnotesize \tt 14 };
            \fill (M) circle [radius=2pt] node[anchor=west] { \footnotesize \tt 2 };
            \fill (N) circle [radius=2pt] node[anchor=west] { \footnotesize \tt 6 };
            \fill (O) circle [radius=2pt] node[anchor=west] { \footnotesize \tt 8 };

            \draw (L) -- (C) -- (H);
            \draw[blue,thick] (H) -- (M) -- (B);

        \end{tikzpicture}
    \end{figure}

\end{frame}

\begin{frame}[fragile]{Visualização do algoritmo de Graham}

    \begin{figure}
        \centering

        \begin{tikzpicture}
            \coordinate (A) at (0, 0);
            \coordinate (B) at (5, 3);
            \coordinate (C) at (8, -2);
            \coordinate (D) at (4, 4);
            \coordinate (E) at (2, 1);
            \coordinate (F) at (2, 5);
            \coordinate (G) at (3, -1);
            \coordinate (H) at (7, 2);
            \coordinate (I) at (5, 0);
            \coordinate (J) at (0, 4);
            \coordinate (K) at (1, -1);
            \coordinate (L) at (7, -2);
            \coordinate (M) at (6, 4);
            \coordinate (N) at (6, 0);
            \coordinate (O) at (1, 3);

            \fill (A) circle [radius=2pt] node[anchor=west] { \footnotesize \tt 11 };
            \fill (B) circle [radius=2pt] node[anchor=west] { \footnotesize \tt 3 };
            \fill (C) circle [radius=2pt] node[anchor=west] { \hspace{0.01in} pivô };
            \fill (D) circle [radius=2pt] node[anchor=west] { \footnotesize \tt 4 };
            \fill (E) circle [radius=2pt] node[anchor=west] { \footnotesize \tt 10 };
            \fill (F) circle [radius=2pt] node[anchor=west] { \footnotesize \tt 5 };
            \fill (G) circle [radius=2pt] node[anchor=west] { \footnotesize \tt 12 };
            \fill (H) circle [radius=2pt] node[anchor=west] { \footnotesize \tt 1 };
            \fill (I) circle [radius=2pt] node[anchor=west] { \footnotesize \tt 9 };
            \fill (J) circle [radius=2pt] node[anchor=west] { \footnotesize \tt 7 };
            \fill (K) circle [radius=2pt] node[anchor=west] { \footnotesize \tt 13 };
            \fill (L) circle [radius=2pt] node[anchor=north] { \footnotesize \tt 14 };
            \fill (M) circle [radius=2pt] node[anchor=west] { \footnotesize \tt 2 };
            \fill (N) circle [radius=2pt] node[anchor=west] { \footnotesize \tt 6 };
            \fill (O) circle [radius=2pt] node[anchor=west] { \footnotesize \tt 8 };

            \draw (L) -- (C) -- (H) -- (M);
            %\draw[blue,thick] (H) -- (M) -- (B);
            \draw[blue,thick] (M) -- (B) -- (D);

        \end{tikzpicture}
    \end{figure}

\end{frame}

\begin{frame}[fragile]{Visualização do algoritmo de Graham}

    \begin{figure}
        \centering

        \begin{tikzpicture}
            \coordinate (A) at (0, 0);
            \coordinate (B) at (5, 3);
            \coordinate (C) at (8, -2);
            \coordinate (D) at (4, 4);
            \coordinate (E) at (2, 1);
            \coordinate (F) at (2, 5);
            \coordinate (G) at (3, -1);
            \coordinate (H) at (7, 2);
            \coordinate (I) at (5, 0);
            \coordinate (J) at (0, 4);
            \coordinate (K) at (1, -1);
            \coordinate (L) at (7, -2);
            \coordinate (M) at (6, 4);
            \coordinate (N) at (6, 0);
            \coordinate (O) at (1, 3);

            \fill (A) circle [radius=2pt] node[anchor=west] { \footnotesize \tt 11 };
            \fill (B) circle [radius=2pt] node[anchor=west] { \footnotesize \tt 3 };
            \fill (C) circle [radius=2pt] node[anchor=west] { \hspace{0.01in} pivô };
            \fill (D) circle [radius=2pt] node[anchor=north] { \footnotesize \tt 4 };
            \fill (E) circle [radius=2pt] node[anchor=west] { \footnotesize \tt 10 };
            \fill (F) circle [radius=2pt] node[anchor=west] { \footnotesize \tt 5 };
            \fill (G) circle [radius=2pt] node[anchor=west] { \footnotesize \tt 12 };
            \fill (H) circle [radius=2pt] node[anchor=west] { \footnotesize \tt 1 };
            \fill (I) circle [radius=2pt] node[anchor=west] { \footnotesize \tt 9 };
            \fill (J) circle [radius=2pt] node[anchor=west] { \footnotesize \tt 7 };
            \fill (K) circle [radius=2pt] node[anchor=west] { \footnotesize \tt 13 };
            \fill (L) circle [radius=2pt] node[anchor=north] { \footnotesize \tt 14 };
            \fill (M) circle [radius=2pt] node[anchor=west] { \footnotesize \tt 2 };
            \fill (N) circle [radius=2pt] node[anchor=west] { \footnotesize \tt 6 };
            \fill (O) circle [radius=2pt] node[anchor=west] { \footnotesize \tt 8 };

            \draw (L) -- (C) -- (H) -- (M);
            \draw[blue,thick] (H) -- (M) -- (D);

        \end{tikzpicture}
    \end{figure}

\end{frame}

\begin{frame}[fragile]{Visualização do algoritmo de Graham}

    \begin{figure}
        \centering

        \begin{tikzpicture}
            \coordinate (A) at (0, 0);
            \coordinate (B) at (5, 3);
            \coordinate (C) at (8, -2);
            \coordinate (D) at (4, 4);
            \coordinate (E) at (2, 1);
            \coordinate (F) at (2, 5);
            \coordinate (G) at (3, -1);
            \coordinate (H) at (7, 2);
            \coordinate (I) at (5, 0);
            \coordinate (J) at (0, 4);
            \coordinate (K) at (1, -1);
            \coordinate (L) at (7, -2);
            \coordinate (M) at (6, 4);
            \coordinate (N) at (6, 0);
            \coordinate (O) at (1, 3);

            \fill (A) circle [radius=2pt] node[anchor=west] { \footnotesize \tt 11 };
            \fill (B) circle [radius=2pt] node[anchor=west] { \footnotesize \tt 3 };
            \fill (C) circle [radius=2pt] node[anchor=west] { \hspace{0.01in} pivô };
            \fill (D) circle [radius=2pt] node[anchor=north] { \footnotesize \tt 4 };
            \fill (E) circle [radius=2pt] node[anchor=west] { \footnotesize \tt 10 };
            \fill (F) circle [radius=2pt] node[anchor=west] { \footnotesize \tt 5 };
            \fill (G) circle [radius=2pt] node[anchor=west] { \footnotesize \tt 12 };
            \fill (H) circle [radius=2pt] node[anchor=west] { \footnotesize \tt 1 };
            \fill (I) circle [radius=2pt] node[anchor=west] { \footnotesize \tt 9 };
            \fill (J) circle [radius=2pt] node[anchor=west] { \footnotesize \tt 7 };
            \fill (K) circle [radius=2pt] node[anchor=west] { \footnotesize \tt 13 };
            \fill (L) circle [radius=2pt] node[anchor=north] { \footnotesize \tt 14 };
            \fill (M) circle [radius=2pt] node[anchor=west] { \footnotesize \tt 2 };
            \fill (N) circle [radius=2pt] node[anchor=west] { \footnotesize \tt 6 };
            \fill (O) circle [radius=2pt] node[anchor=west] { \footnotesize \tt 8 };

            \draw (L) -- (C) -- (H) -- (M);
            \draw[blue,thick] (M) -- (D) -- (F);

        \end{tikzpicture}
    \end{figure}

\end{frame}

\begin{frame}[fragile]{Visualização do algoritmo de Graham}

    \begin{figure}
        \centering

        \begin{tikzpicture}
            \coordinate (A) at (0, 0);
            \coordinate (B) at (5, 3);
            \coordinate (C) at (8, -2);
            \coordinate (D) at (4, 4);
            \coordinate (E) at (2, 1);
            \coordinate (F) at (2, 5);
            \coordinate (G) at (3, -1);
            \coordinate (H) at (7, 2);
            \coordinate (I) at (5, 0);
            \coordinate (J) at (0, 4);
            \coordinate (K) at (1, -1);
            \coordinate (L) at (7, -2);
            \coordinate (M) at (6, 4);
            \coordinate (N) at (6, 0);
            \coordinate (O) at (1, 3);

            \fill (A) circle [radius=2pt] node[anchor=west] { \footnotesize \tt 11 };
            \fill (B) circle [radius=2pt] node[anchor=west] { \footnotesize \tt 3 };
            \fill (C) circle [radius=2pt] node[anchor=west] { \hspace{0.01in} pivô };
            \fill (D) circle [radius=2pt] node[anchor=north] { \footnotesize \tt 4 };
            \fill (E) circle [radius=2pt] node[anchor=west] { \footnotesize \tt 10 };
            \fill (F) circle [radius=2pt] node[anchor=north] { \footnotesize \tt 5 };
            \fill (G) circle [radius=2pt] node[anchor=west] { \footnotesize \tt 12 };
            \fill (H) circle [radius=2pt] node[anchor=west] { \footnotesize \tt 1 };
            \fill (I) circle [radius=2pt] node[anchor=west] { \footnotesize \tt 9 };
            \fill (J) circle [radius=2pt] node[anchor=west] { \footnotesize \tt 7 };
            \fill (K) circle [radius=2pt] node[anchor=west] { \footnotesize \tt 13 };
            \fill (L) circle [radius=2pt] node[anchor=north] { \footnotesize \tt 14 };
            \fill (M) circle [radius=2pt] node[anchor=west] { \footnotesize \tt 2 };
            \fill (N) circle [radius=2pt] node[anchor=west] { \footnotesize \tt 6 };
            \fill (O) circle [radius=2pt] node[anchor=west] { \footnotesize \tt 8 };

            \draw (L) -- (C) -- (H) -- (M);
            \draw[blue,thick] (H) -- (M) -- (F);

        \end{tikzpicture}
    \end{figure}

\end{frame}

\begin{frame}[fragile]{Visualização do algoritmo de Graham}

    \begin{figure}
        \centering

        \begin{tikzpicture}
            \coordinate (A) at (0, 0);
            \coordinate (B) at (5, 3);
            \coordinate (C) at (8, -2);
            \coordinate (D) at (4, 4);
            \coordinate (E) at (2, 1);
            \coordinate (F) at (2, 5);
            \coordinate (G) at (3, -1);
            \coordinate (H) at (7, 2);
            \coordinate (I) at (5, 0);
            \coordinate (J) at (0, 4);
            \coordinate (K) at (1, -1);
            \coordinate (L) at (7, -2);
            \coordinate (M) at (6, 4);
            \coordinate (N) at (6, 0);
            \coordinate (O) at (1, 3);

            \fill (A) circle [radius=2pt] node[anchor=west] { \footnotesize \tt 11 };
            \fill (B) circle [radius=2pt] node[anchor=west] { \footnotesize \tt 3 };
            \fill (C) circle [radius=2pt] node[anchor=west] { \hspace{0.01in} pivô };
            \fill (D) circle [radius=2pt] node[anchor=north] { \footnotesize \tt 4 };
            \fill (E) circle [radius=2pt] node[anchor=west] { \footnotesize \tt 10 };
            \fill (F) circle [radius=2pt] node[anchor=north] { \footnotesize \tt 5 };
            \fill (G) circle [radius=2pt] node[anchor=west] { \footnotesize \tt 12 };
            \fill (H) circle [radius=2pt] node[anchor=west] { \footnotesize \tt 1 };
            \fill (I) circle [radius=2pt] node[anchor=west] { \footnotesize \tt 9 };
            \fill (J) circle [radius=2pt] node[anchor=west] { \footnotesize \tt 7 };
            \fill (K) circle [radius=2pt] node[anchor=west] { \footnotesize \tt 13 };
            \fill (L) circle [radius=2pt] node[anchor=north] { \footnotesize \tt 14 };
            \fill (M) circle [radius=2pt] node[anchor=west] { \footnotesize \tt 2 };
            \fill (N) circle [radius=2pt] node[anchor=west] { \footnotesize \tt 6 };
            \fill (O) circle [radius=2pt] node[anchor=west] { \footnotesize \tt 8 };

            \draw (L) -- (C) -- (H) -- (M) -- (F);
            \draw[blue,thick] (M) -- (F) -- (N);

        \end{tikzpicture}
    \end{figure}

\end{frame}
 
\begin{frame}[fragile]{Visualização do algoritmo de Graham}

    \begin{figure}
        \centering

        \begin{tikzpicture}
            \coordinate (A) at (0, 0);
            \coordinate (B) at (5, 3);
            \coordinate (C) at (8, -2);
            \coordinate (D) at (4, 4);
            \coordinate (E) at (2, 1);
            \coordinate (F) at (2, 5);
            \coordinate (G) at (3, -1);
            \coordinate (H) at (7, 2);
            \coordinate (I) at (5, 0);
            \coordinate (J) at (0, 4);
            \coordinate (K) at (1, -1);
            \coordinate (L) at (7, -2);
            \coordinate (M) at (6, 4);
            \coordinate (N) at (6, 0);
            \coordinate (O) at (1, 3);

            \fill (A) circle [radius=2pt] node[anchor=west] { \footnotesize \tt 11 };
            \fill (B) circle [radius=2pt] node[anchor=west] { \footnotesize \tt 3 };
            \fill (C) circle [radius=2pt] node[anchor=west] { \hspace{0.01in} pivô };
            \fill (D) circle [radius=2pt] node[anchor=north] { \footnotesize \tt 4 };
            \fill (E) circle [radius=2pt] node[anchor=west] { \footnotesize \tt 10 };
            \fill (F) circle [radius=2pt] node[anchor=north] { \footnotesize \tt 5 };
            \fill (G) circle [radius=2pt] node[anchor=west] { \footnotesize \tt 12 };
            \fill (H) circle [radius=2pt] node[anchor=west] { \footnotesize \tt 1 };
            \fill (I) circle [radius=2pt] node[anchor=west] { \footnotesize \tt 9 };
            \fill (J) circle [radius=2pt] node[anchor=west] { \footnotesize \tt 7 };
            \fill (K) circle [radius=2pt] node[anchor=west] { \footnotesize \tt 13 };
            \fill (L) circle [radius=2pt] node[anchor=north] { \footnotesize \tt 14 };
            \fill (M) circle [radius=2pt] node[anchor=west] { \footnotesize \tt 2 };
            \fill (N) circle [radius=2pt] node[anchor=west] { \footnotesize \tt 6 };
            \fill (O) circle [radius=2pt] node[anchor=west] { \footnotesize \tt 8 };

            \draw (L) -- (C) -- (H) -- (M) -- (F);
            \draw[blue,thick] (F) -- (N) -- (J);

        \end{tikzpicture}
    \end{figure}

\end{frame}
 
\begin{frame}[fragile]{Visualização do algoritmo de Graham}

    \begin{figure}
        \centering

        \begin{tikzpicture}
            \coordinate (A) at (0, 0);
            \coordinate (B) at (5, 3);
            \coordinate (C) at (8, -2);
            \coordinate (D) at (4, 4);
            \coordinate (E) at (2, 1);
            \coordinate (F) at (2, 5);
            \coordinate (G) at (3, -1);
            \coordinate (H) at (7, 2);
            \coordinate (I) at (5, 0);
            \coordinate (J) at (0, 4);
            \coordinate (K) at (1, -1);
            \coordinate (L) at (7, -2);
            \coordinate (M) at (6, 4);
            \coordinate (N) at (6, 0);
            \coordinate (O) at (1, 3);

            \fill (A) circle [radius=2pt] node[anchor=west] { \footnotesize \tt 11 };
            \fill (B) circle [radius=2pt] node[anchor=west] { \footnotesize \tt 3 };
            \fill (C) circle [radius=2pt] node[anchor=west] { \hspace{0.01in} pivô };
            \fill (D) circle [radius=2pt] node[anchor=north] { \footnotesize \tt 4 };
            \fill (E) circle [radius=2pt] node[anchor=west] { \footnotesize \tt 10 };
            \fill (F) circle [radius=2pt] node[anchor=north] { \footnotesize \tt 5 };
            \fill (G) circle [radius=2pt] node[anchor=west] { \footnotesize \tt 12 };
            \fill (H) circle [radius=2pt] node[anchor=west] { \footnotesize \tt 1 };
            \fill (I) circle [radius=2pt] node[anchor=west] { \footnotesize \tt 9 };
            \fill (J) circle [radius=2pt] node[anchor=east] { \footnotesize \tt 7 };
            \fill (K) circle [radius=2pt] node[anchor=west] { \footnotesize \tt 13 };
            \fill (L) circle [radius=2pt] node[anchor=north] { \footnotesize \tt 14 };
            \fill (M) circle [radius=2pt] node[anchor=west] { \footnotesize \tt 2 };
            \fill (N) circle [radius=2pt] node[anchor=west] { \footnotesize \tt 6 };
            \fill (O) circle [radius=2pt] node[anchor=west] { \footnotesize \tt 8 };

            \draw (L) -- (C) -- (H) -- (M) -- (F);
            \draw[blue,thick] (M) -- (F) -- (J);

        \end{tikzpicture}
    \end{figure}

\end{frame}

\begin{frame}[fragile]{Visualização do algoritmo de Graham}

    \begin{figure}
        \centering

        \begin{tikzpicture}
            \coordinate (A) at (0, 0);
            \coordinate (B) at (5, 3);
            \coordinate (C) at (8, -2);
            \coordinate (D) at (4, 4);
            \coordinate (E) at (2, 1);
            \coordinate (F) at (2, 5);
            \coordinate (G) at (3, -1);
            \coordinate (H) at (7, 2);
            \coordinate (I) at (5, 0);
            \coordinate (J) at (0, 4);
            \coordinate (K) at (1, -1);
            \coordinate (L) at (7, -2);
            \coordinate (M) at (6, 4);
            \coordinate (N) at (6, 0);
            \coordinate (O) at (1, 3);

            \fill (A) circle [radius=2pt] node[anchor=west] { \footnotesize \tt 11 };
            \fill (B) circle [radius=2pt] node[anchor=west] { \footnotesize \tt 3 };
            \fill (C) circle [radius=2pt] node[anchor=west] { \hspace{0.01in} pivô };
            \fill (D) circle [radius=2pt] node[anchor=north] { \footnotesize \tt 4 };
            \fill (E) circle [radius=2pt] node[anchor=west] { \footnotesize \tt 10 };
            \fill (F) circle [radius=2pt] node[anchor=north] { \footnotesize \tt 5 };
            \fill (G) circle [radius=2pt] node[anchor=west] { \footnotesize \tt 12 };
            \fill (H) circle [radius=2pt] node[anchor=west] { \footnotesize \tt 1 };
            \fill (I) circle [radius=2pt] node[anchor=west] { \footnotesize \tt 9 };
            \fill (J) circle [radius=2pt] node[anchor=east] { \footnotesize \tt 7 };
            \fill (K) circle [radius=2pt] node[anchor=west] { \footnotesize \tt 13 };
            \fill (L) circle [radius=2pt] node[anchor=north] { \footnotesize \tt 14 };
            \fill (M) circle [radius=2pt] node[anchor=west] { \footnotesize \tt 2 };
            \fill (N) circle [radius=2pt] node[anchor=west] { \footnotesize \tt 6 };
            \fill (O) circle [radius=2pt] node[anchor=west] { \footnotesize \tt 8 };

            \draw (L) -- (C) -- (H) -- (M) -- (F) -- (J);
            \draw[blue,thick] (F) -- (J) -- (O);

        \end{tikzpicture}
    \end{figure}

\end{frame}

\begin{frame}[fragile]{Visualização do algoritmo de Graham}

    \begin{figure}
        \centering

        \begin{tikzpicture}
            \coordinate (A) at (0, 0);
            \coordinate (B) at (5, 3);
            \coordinate (C) at (8, -2);
            \coordinate (D) at (4, 4);
            \coordinate (E) at (2, 1);
            \coordinate (F) at (2, 5);
            \coordinate (G) at (3, -1);
            \coordinate (H) at (7, 2);
            \coordinate (I) at (5, 0);
            \coordinate (J) at (0, 4);
            \coordinate (K) at (1, -1);
            \coordinate (L) at (7, -2);
            \coordinate (M) at (6, 4);
            \coordinate (N) at (6, 0);
            \coordinate (O) at (1, 3);

            \fill (A) circle [radius=2pt] node[anchor=west] { \footnotesize \tt 11 };
            \fill (B) circle [radius=2pt] node[anchor=west] { \footnotesize \tt 3 };
            \fill (C) circle [radius=2pt] node[anchor=west] { \hspace{0.01in} pivô };
            \fill (D) circle [radius=2pt] node[anchor=north] { \footnotesize \tt 4 };
            \fill (E) circle [radius=2pt] node[anchor=west] { \footnotesize \tt 10 };
            \fill (F) circle [radius=2pt] node[anchor=north] { \footnotesize \tt 5 };
            \fill (G) circle [radius=2pt] node[anchor=west] { \footnotesize \tt 12 };
            \fill (H) circle [radius=2pt] node[anchor=west] { \footnotesize \tt 1 };
            \fill (I) circle [radius=2pt] node[anchor=west] { \footnotesize \tt 9 };
            \fill (J) circle [radius=2pt] node[anchor=east] { \footnotesize \tt 7 };
            \fill (K) circle [radius=2pt] node[anchor=west] { \footnotesize \tt 13 };
            \fill (L) circle [radius=2pt] node[anchor=north] { \footnotesize \tt 14 };
            \fill (M) circle [radius=2pt] node[anchor=west] { \footnotesize \tt 2 };
            \fill (N) circle [radius=2pt] node[anchor=west] { \footnotesize \tt 6 };
            \fill (O) circle [radius=2pt] node[anchor=west] { \footnotesize \tt 8 };

            \draw (L) -- (C) -- (H) -- (M) -- (F) -- (J);
            \draw[blue,thick] (J) -- (O) -- (I);

        \end{tikzpicture}
    \end{figure}

\end{frame}

\begin{frame}[fragile]{Visualização do algoritmo de Graham}

    \begin{figure}
        \centering

        \begin{tikzpicture}
            \coordinate (A) at (0, 0);
            \coordinate (B) at (5, 3);
            \coordinate (C) at (8, -2);
            \coordinate (D) at (4, 4);
            \coordinate (E) at (2, 1);
            \coordinate (F) at (2, 5);
            \coordinate (G) at (3, -1);
            \coordinate (H) at (7, 2);
            \coordinate (I) at (5, 0);
            \coordinate (J) at (0, 4);
            \coordinate (K) at (1, -1);
            \coordinate (L) at (7, -2);
            \coordinate (M) at (6, 4);
            \coordinate (N) at (6, 0);
            \coordinate (O) at (1, 3);

            \fill (A) circle [radius=2pt] node[anchor=west] { \footnotesize \tt 11 };
            \fill (B) circle [radius=2pt] node[anchor=west] { \footnotesize \tt 3 };
            \fill (C) circle [radius=2pt] node[anchor=west] { \hspace{0.01in} pivô };
            \fill (D) circle [radius=2pt] node[anchor=north] { \footnotesize \tt 4 };
            \fill (E) circle [radius=2pt] node[anchor=north] { \footnotesize \tt 10 };
            \fill (F) circle [radius=2pt] node[anchor=north] { \footnotesize \tt 5 };
            \fill (G) circle [radius=2pt] node[anchor=west] { \footnotesize \tt 12 };
            \fill (H) circle [radius=2pt] node[anchor=west] { \footnotesize \tt 1 };
            \fill (I) circle [radius=2pt] node[anchor=west] { \footnotesize \tt 9 };
            \fill (J) circle [radius=2pt] node[anchor=east] { \footnotesize \tt 7 };
            \fill (K) circle [radius=2pt] node[anchor=west] { \footnotesize \tt 13 };
            \fill (L) circle [radius=2pt] node[anchor=north] { \footnotesize \tt 14 };
            \fill (M) circle [radius=2pt] node[anchor=west] { \footnotesize \tt 2 };
            \fill (N) circle [radius=2pt] node[anchor=west] { \footnotesize \tt 6 };
            \fill (O) circle [radius=2pt] node[anchor=west] { \footnotesize \tt 8 };

            \draw (L) -- (C) -- (H) -- (M) -- (F) -- (J) -- (O);
            \draw[blue,thick] (O) -- (I) -- (E);

        \end{tikzpicture}
    \end{figure}

\end{frame}

\begin{frame}[fragile]{Visualização do algoritmo de Graham}

    \begin{figure}
        \centering

        \begin{tikzpicture}
            \coordinate (A) at (0, 0);
            \coordinate (B) at (5, 3);
            \coordinate (C) at (8, -2);
            \coordinate (D) at (4, 4);
            \coordinate (E) at (2, 1);
            \coordinate (F) at (2, 5);
            \coordinate (G) at (3, -1);
            \coordinate (H) at (7, 2);
            \coordinate (I) at (5, 0);
            \coordinate (J) at (0, 4);
            \coordinate (K) at (1, -1);
            \coordinate (L) at (7, -2);
            \coordinate (M) at (6, 4);
            \coordinate (N) at (6, 0);
            \coordinate (O) at (1, 3);

            \fill (A) circle [radius=2pt] node[anchor=west] { \footnotesize \tt 11 };
            \fill (B) circle [radius=2pt] node[anchor=west] { \footnotesize \tt 3 };
            \fill (C) circle [radius=2pt] node[anchor=west] { \hspace{0.01in} pivô };
            \fill (D) circle [radius=2pt] node[anchor=north] { \footnotesize \tt 4 };
            \fill (E) circle [radius=2pt] node[anchor=north] { \footnotesize \tt 10 };
            \fill (F) circle [radius=2pt] node[anchor=north] { \footnotesize \tt 5 };
            \fill (G) circle [radius=2pt] node[anchor=west] { \footnotesize \tt 12 };
            \fill (H) circle [radius=2pt] node[anchor=west] { \footnotesize \tt 1 };
            \fill (I) circle [radius=2pt] node[anchor=west] { \footnotesize \tt 9 };
            \fill (J) circle [radius=2pt] node[anchor=east] { \footnotesize \tt 7 };
            \fill (K) circle [radius=2pt] node[anchor=west] { \footnotesize \tt 13 };
            \fill (L) circle [radius=2pt] node[anchor=north] { \footnotesize \tt 14 };
            \fill (M) circle [radius=2pt] node[anchor=west] { \footnotesize \tt 2 };
            \fill (N) circle [radius=2pt] node[anchor=west] { \footnotesize \tt 6 };
            \fill (O) circle [radius=2pt] node[anchor=west] { \footnotesize \tt 8 };

            \draw (L) -- (C) -- (H) -- (M) -- (F) -- (J) -- (O);
            \draw[blue,thick] (J) -- (O) -- (E);

        \end{tikzpicture}
    \end{figure}

\end{frame}

\begin{frame}[fragile]{Visualização do algoritmo de Graham}

    \begin{figure}
        \centering

        \begin{tikzpicture}
            \coordinate (A) at (0, 0);
            \coordinate (B) at (5, 3);
            \coordinate (C) at (8, -2);
            \coordinate (D) at (4, 4);
            \coordinate (E) at (2, 1);
            \coordinate (F) at (2, 5);
            \coordinate (G) at (3, -1);
            \coordinate (H) at (7, 2);
            \coordinate (I) at (5, 0);
            \coordinate (J) at (0, 4);
            \coordinate (K) at (1, -1);
            \coordinate (L) at (7, -2);
            \coordinate (M) at (6, 4);
            \coordinate (N) at (6, 0);
            \coordinate (O) at (1, 3);

            \fill (A) circle [radius=2pt] node[anchor=west] { \footnotesize \tt 11 };
            \fill (B) circle [radius=2pt] node[anchor=west] { \footnotesize \tt 3 };
            \fill (C) circle [radius=2pt] node[anchor=west] { \hspace{0.01in} pivô };
            \fill (D) circle [radius=2pt] node[anchor=north] { \footnotesize \tt 4 };
            \fill (E) circle [radius=2pt] node[anchor=north] { \footnotesize \tt 10 };
            \fill (F) circle [radius=2pt] node[anchor=north] { \footnotesize \tt 5 };
            \fill (G) circle [radius=2pt] node[anchor=west] { \footnotesize \tt 12 };
            \fill (H) circle [radius=2pt] node[anchor=west] { \footnotesize \tt 1 };
            \fill (I) circle [radius=2pt] node[anchor=west] { \footnotesize \tt 9 };
            \fill (J) circle [radius=2pt] node[anchor=east] { \footnotesize \tt 7 };
            \fill (K) circle [radius=2pt] node[anchor=west] { \footnotesize \tt 13 };
            \fill (L) circle [radius=2pt] node[anchor=north] { \footnotesize \tt 14 };
            \fill (M) circle [radius=2pt] node[anchor=west] { \footnotesize \tt 2 };
            \fill (N) circle [radius=2pt] node[anchor=west] { \footnotesize \tt 6 };
            \fill (O) circle [radius=2pt] node[anchor=west] { \footnotesize \tt 8 };

            \draw (L) -- (C) -- (H) -- (M) -- (F) -- (J);
            \draw[blue,thick] (F) -- (J) -- (E);

        \end{tikzpicture}
    \end{figure}

\end{frame}

\begin{frame}[fragile]{Visualização do algoritmo de Graham}

    \begin{figure}
        \centering

        \begin{tikzpicture}
            \coordinate (A) at (0, 0);
            \coordinate (B) at (5, 3);
            \coordinate (C) at (8, -2);
            \coordinate (D) at (4, 4);
            \coordinate (E) at (2, 1);
            \coordinate (F) at (2, 5);
            \coordinate (G) at (3, -1);
            \coordinate (H) at (7, 2);
            \coordinate (I) at (5, 0);
            \coordinate (J) at (0, 4);
            \coordinate (K) at (1, -1);
            \coordinate (L) at (7, -2);
            \coordinate (M) at (6, 4);
            \coordinate (N) at (6, 0);
            \coordinate (O) at (1, 3);

            \fill (A) circle [radius=2pt] node[anchor=west] { \footnotesize \tt 11 };
            \fill (B) circle [radius=2pt] node[anchor=west] { \footnotesize \tt 3 };
            \fill (C) circle [radius=2pt] node[anchor=west] { \hspace{0.01in} pivô };
            \fill (D) circle [radius=2pt] node[anchor=north] { \footnotesize \tt 4 };
            \fill (E) circle [radius=2pt] node[anchor=north] { \footnotesize \tt 10 };
            \fill (F) circle [radius=2pt] node[anchor=north] { \footnotesize \tt 5 };
            \fill (G) circle [radius=2pt] node[anchor=west] { \footnotesize \tt 12 };
            \fill (H) circle [radius=2pt] node[anchor=west] { \footnotesize \tt 1 };
            \fill (I) circle [radius=2pt] node[anchor=west] { \footnotesize \tt 9 };
            \fill (J) circle [radius=2pt] node[anchor=east] { \footnotesize \tt 7 };
            \fill (K) circle [radius=2pt] node[anchor=west] { \footnotesize \tt 13 };
            \fill (L) circle [radius=2pt] node[anchor=north] { \footnotesize \tt 14 };
            \fill (M) circle [radius=2pt] node[anchor=west] { \footnotesize \tt 2 };
            \fill (N) circle [radius=2pt] node[anchor=west] { \footnotesize \tt 6 };
            \fill (O) circle [radius=2pt] node[anchor=west] { \footnotesize \tt 8 };

            \draw (L) -- (C) -- (H) -- (M) -- (F) -- (J);
            \draw[blue,thick] (J) -- (E) -- (A);

        \end{tikzpicture}
    \end{figure}

\end{frame}

\begin{frame}[fragile]{Visualização do algoritmo de Graham}

    \begin{figure}
        \centering

        \begin{tikzpicture}
            \coordinate (A) at (0, 0);
            \coordinate (B) at (5, 3);
            \coordinate (C) at (8, -2);
            \coordinate (D) at (4, 4);
            \coordinate (E) at (2, 1);
            \coordinate (F) at (2, 5);
            \coordinate (G) at (3, -1);
            \coordinate (H) at (7, 2);
            \coordinate (I) at (5, 0);
            \coordinate (J) at (0, 4);
            \coordinate (K) at (1, -1);
            \coordinate (L) at (7, -2);
            \coordinate (M) at (6, 4);
            \coordinate (N) at (6, 0);
            \coordinate (O) at (1, 3);

            \fill (A) circle [radius=2pt] node[anchor=west] { \footnotesize \tt 11 };
            \fill (B) circle [radius=2pt] node[anchor=west] { \footnotesize \tt 3 };
            \fill (C) circle [radius=2pt] node[anchor=west] { \hspace{0.01in} pivô };
            \fill (D) circle [radius=2pt] node[anchor=north] { \footnotesize \tt 4 };
            \fill (E) circle [radius=2pt] node[anchor=north] { \footnotesize \tt 10 };
            \fill (F) circle [radius=2pt] node[anchor=north] { \footnotesize \tt 5 };
            \fill (G) circle [radius=2pt] node[anchor=west] { \footnotesize \tt 12 };
            \fill (H) circle [radius=2pt] node[anchor=west] { \footnotesize \tt 1 };
            \fill (I) circle [radius=2pt] node[anchor=west] { \footnotesize \tt 9 };
            \fill (J) circle [radius=2pt] node[anchor=east] { \footnotesize \tt 7 };
            \fill (K) circle [radius=2pt] node[anchor=west] { \footnotesize \tt 13 };
            \fill (L) circle [radius=2pt] node[anchor=north] { \footnotesize \tt 14 };
            \fill (M) circle [radius=2pt] node[anchor=west] { \footnotesize \tt 2 };
            \fill (N) circle [radius=2pt] node[anchor=west] { \footnotesize \tt 6 };
            \fill (O) circle [radius=2pt] node[anchor=west] { \footnotesize \tt 8 };

            \draw (L) -- (C) -- (H) -- (M) -- (F) -- (J);
            \draw[blue,thick] (F) -- (J) -- (A);

        \end{tikzpicture}
    \end{figure}

\end{frame}

\begin{frame}[fragile]{Visualização do algoritmo de Graham}

    \begin{figure}
        \centering

        \begin{tikzpicture}
            \coordinate (A) at (0, 0);
            \coordinate (B) at (5, 3);
            \coordinate (C) at (8, -2);
            \coordinate (D) at (4, 4);
            \coordinate (E) at (2, 1);
            \coordinate (F) at (2, 5);
            \coordinate (G) at (3, -1);
            \coordinate (H) at (7, 2);
            \coordinate (I) at (5, 0);
            \coordinate (J) at (0, 4);
            \coordinate (K) at (1, -1);
            \coordinate (L) at (7, -2);
            \coordinate (M) at (6, 4);
            \coordinate (N) at (6, 0);
            \coordinate (O) at (1, 3);

            \fill (A) circle [radius=2pt] node[anchor=west] { \footnotesize \tt 11 };
            \fill (B) circle [radius=2pt] node[anchor=west] { \footnotesize \tt 3 };
            \fill (C) circle [radius=2pt] node[anchor=west] { \hspace{0.01in} pivô };
            \fill (D) circle [radius=2pt] node[anchor=north] { \footnotesize \tt 4 };
            \fill (E) circle [radius=2pt] node[anchor=north] { \footnotesize \tt 10 };
            \fill (F) circle [radius=2pt] node[anchor=north] { \footnotesize \tt 5 };
            \fill (G) circle [radius=2pt] node[anchor=west] { \footnotesize \tt 12 };
            \fill (H) circle [radius=2pt] node[anchor=west] { \footnotesize \tt 1 };
            \fill (I) circle [radius=2pt] node[anchor=west] { \footnotesize \tt 9 };
            \fill (J) circle [radius=2pt] node[anchor=east] { \footnotesize \tt 7 };
            \fill (K) circle [radius=2pt] node[anchor=west] { \footnotesize \tt 13 };
            \fill (L) circle [radius=2pt] node[anchor=north] { \footnotesize \tt 14 };
            \fill (M) circle [radius=2pt] node[anchor=west] { \footnotesize \tt 2 };
            \fill (N) circle [radius=2pt] node[anchor=west] { \footnotesize \tt 6 };
            \fill (O) circle [radius=2pt] node[anchor=west] { \footnotesize \tt 8 };

            \draw (L) -- (C) -- (H) -- (M) -- (F) -- (J);
            \draw[blue,thick] (J) -- (A) -- (G);

        \end{tikzpicture}
    \end{figure}

\end{frame}

\begin{frame}[fragile]{Visualização do algoritmo de Graham}

    \begin{figure}
        \centering

        \begin{tikzpicture}
            \coordinate (A) at (0, 0);
            \coordinate (B) at (5, 3);
            \coordinate (C) at (8, -2);
            \coordinate (D) at (4, 4);
            \coordinate (E) at (2, 1);
            \coordinate (F) at (2, 5);
            \coordinate (G) at (3, -1);
            \coordinate (H) at (7, 2);
            \coordinate (I) at (5, 0);
            \coordinate (J) at (0, 4);
            \coordinate (K) at (1, -1);
            \coordinate (L) at (7, -2);
            \coordinate (M) at (6, 4);
            \coordinate (N) at (6, 0);
            \coordinate (O) at (1, 3);

            \fill (A) circle [radius=2pt] node[anchor=east] { \footnotesize \tt 11 };
            \fill (B) circle [radius=2pt] node[anchor=west] { \footnotesize \tt 3 };
            \fill (C) circle [radius=2pt] node[anchor=west] { \hspace{0.01in} pivô };
            \fill (D) circle [radius=2pt] node[anchor=north] { \footnotesize \tt 4 };
            \fill (E) circle [radius=2pt] node[anchor=north] { \footnotesize \tt 10 };
            \fill (F) circle [radius=2pt] node[anchor=north] { \footnotesize \tt 5 };
            \fill (G) circle [radius=2pt] node[anchor=west] { \footnotesize \tt 12 };
            \fill (H) circle [radius=2pt] node[anchor=west] { \footnotesize \tt 1 };
            \fill (I) circle [radius=2pt] node[anchor=west] { \footnotesize \tt 9 };
            \fill (J) circle [radius=2pt] node[anchor=east] { \footnotesize \tt 7 };
            \fill (K) circle [radius=2pt] node[anchor=north] { \footnotesize \tt 13 };
            \fill (L) circle [radius=2pt] node[anchor=north] { \footnotesize \tt 14 };
            \fill (M) circle [radius=2pt] node[anchor=west] { \footnotesize \tt 2 };
            \fill (N) circle [radius=2pt] node[anchor=west] { \footnotesize \tt 6 };
            \fill (O) circle [radius=2pt] node[anchor=west] { \footnotesize \tt 8 };

            \draw (L) -- (C) -- (H) -- (M) -- (F) -- (J) -- (A);
            \draw[blue,thick] (A) -- (G) -- (K);

        \end{tikzpicture}
    \end{figure}

\end{frame}

\begin{frame}[fragile]{Visualização do algoritmo de Graham}

    \begin{figure}
        \centering

        \begin{tikzpicture}
            \coordinate (A) at (0, 0);
            \coordinate (B) at (5, 3);
            \coordinate (C) at (8, -2);
            \coordinate (D) at (4, 4);
            \coordinate (E) at (2, 1);
            \coordinate (F) at (2, 5);
            \coordinate (G) at (3, -1);
            \coordinate (H) at (7, 2);
            \coordinate (I) at (5, 0);
            \coordinate (J) at (0, 4);
            \coordinate (K) at (1, -1);
            \coordinate (L) at (7, -2);
            \coordinate (M) at (6, 4);
            \coordinate (N) at (6, 0);
            \coordinate (O) at (1, 3);

            \fill (A) circle [radius=2pt] node[anchor=east] { \footnotesize \tt 11 };
            \fill (B) circle [radius=2pt] node[anchor=west] { \footnotesize \tt 3 };
            \fill (C) circle [radius=2pt] node[anchor=west] { \hspace{0.01in} pivô };
            \fill (D) circle [radius=2pt] node[anchor=north] { \footnotesize \tt 4 };
            \fill (E) circle [radius=2pt] node[anchor=north] { \footnotesize \tt 10 };
            \fill (F) circle [radius=2pt] node[anchor=north] { \footnotesize \tt 5 };
            \fill (G) circle [radius=2pt] node[anchor=west] { \footnotesize \tt 12 };
            \fill (H) circle [radius=2pt] node[anchor=west] { \footnotesize \tt 1 };
            \fill (I) circle [radius=2pt] node[anchor=west] { \footnotesize \tt 9 };
            \fill (J) circle [radius=2pt] node[anchor=east] { \footnotesize \tt 7 };
            \fill (K) circle [radius=2pt] node[anchor=north] { \footnotesize \tt 13 };
            \fill (L) circle [radius=2pt] node[anchor=north] { \footnotesize \tt 14 };
            \fill (M) circle [radius=2pt] node[anchor=west] { \footnotesize \tt 2 };
            \fill (N) circle [radius=2pt] node[anchor=west] { \footnotesize \tt 6 };
            \fill (O) circle [radius=2pt] node[anchor=west] { \footnotesize \tt 8 };

            \draw (L) -- (C) -- (H) -- (M) -- (F) -- (J) -- (A);
            \draw[blue,thick] (J) -- (A) -- (K);

        \end{tikzpicture}
    \end{figure}

\end{frame}

\begin{frame}[fragile]{Visualização do algoritmo de Graham}

    \begin{figure}
        \centering

        \begin{tikzpicture}
            \coordinate (A) at (0, 0);
            \coordinate (B) at (5, 3);
            \coordinate (C) at (8, -2);
            \coordinate (D) at (4, 4);
            \coordinate (E) at (2, 1);
            \coordinate (F) at (2, 5);
            \coordinate (G) at (3, -1);
            \coordinate (H) at (7, 2);
            \coordinate (I) at (5, 0);
            \coordinate (J) at (0, 4);
            \coordinate (K) at (1, -1);
            \coordinate (L) at (7, -2);
            \coordinate (M) at (6, 4);
            \coordinate (N) at (6, 0);
            \coordinate (O) at (1, 3);

            \fill (A) circle [radius=2pt] node[anchor=east] { \footnotesize \tt 11 };
            \fill (B) circle [radius=2pt] node[anchor=west] { \footnotesize \tt 3 };
            \fill (C) circle [radius=2pt] node[anchor=west] { \hspace{0.01in} pivô };
            \fill (D) circle [radius=2pt] node[anchor=north] { \footnotesize \tt 4 };
            \fill (E) circle [radius=2pt] node[anchor=north] { \footnotesize \tt 10 };
            \fill (F) circle [radius=2pt] node[anchor=north] { \footnotesize \tt 5 };
            \fill (G) circle [radius=2pt] node[anchor=west] { \footnotesize \tt 12 };
            \fill (H) circle [radius=2pt] node[anchor=west] { \footnotesize \tt 1 };
            \fill (I) circle [radius=2pt] node[anchor=west] { \footnotesize \tt 9 };
            \fill (J) circle [radius=2pt] node[anchor=east] { \footnotesize \tt 7 };
            \fill (K) circle [radius=2pt] node[anchor=north] { \footnotesize \tt 13 };
            \fill (L) circle [radius=2pt] node[anchor=north] { \footnotesize \tt 14 };
            \fill (M) circle [radius=2pt] node[anchor=west] { \footnotesize \tt 2 };
            \fill (N) circle [radius=2pt] node[anchor=west] { \footnotesize \tt 6 };
            \fill (O) circle [radius=2pt] node[anchor=west] { \footnotesize \tt 8 };

            \draw (L) -- (C) -- (H) -- (M) -- (F) -- (J) -- (A);
            \draw[blue,thick] (A) -- (K) -- (L);

        \end{tikzpicture}
    \end{figure}

\end{frame}

\begin{frame}[fragile]{Visualização do algoritmo de Graham}

    \begin{figure}
        \centering

        \begin{tikzpicture}
            \coordinate (A) at (0, 0);
            \coordinate (B) at (5, 3);
            \coordinate (C) at (8, -2);
            \coordinate (D) at (4, 4);
            \coordinate (E) at (2, 1);
            \coordinate (F) at (2, 5);
            \coordinate (G) at (3, -1);
            \coordinate (H) at (7, 2);
            \coordinate (I) at (5, 0);
            \coordinate (J) at (0, 4);
            \coordinate (K) at (1, -1);
            \coordinate (L) at (7, -2);
            \coordinate (M) at (6, 4);
            \coordinate (N) at (6, 0);
            \coordinate (O) at (1, 3);

            \fill (A) circle [radius=2pt] node[anchor=east] { \footnotesize \tt 11 };
            \fill (B) circle [radius=2pt] node[anchor=west] { \footnotesize \tt 3 };
            \fill (C) circle [radius=2pt] node[anchor=west] { \hspace{0.01in} pivô };
            \fill (D) circle [radius=2pt] node[anchor=north] { \footnotesize \tt 4 };
            \fill (E) circle [radius=2pt] node[anchor=north] { \footnotesize \tt 10 };
            \fill (F) circle [radius=2pt] node[anchor=north] { \footnotesize \tt 5 };
            \fill (G) circle [radius=2pt] node[anchor=west] { \footnotesize \tt 12 };
            \fill (H) circle [radius=2pt] node[anchor=west] { \footnotesize \tt 1 };
            \fill (I) circle [radius=2pt] node[anchor=west] { \footnotesize \tt 9 };
            \fill (J) circle [radius=2pt] node[anchor=east] { \footnotesize \tt 7 };
            \fill (K) circle [radius=2pt] node[anchor=north] { \footnotesize \tt 13 };
            \fill (L) circle [radius=2pt] node[anchor=north] { \footnotesize \tt 14 };
            \fill (M) circle [radius=2pt] node[anchor=west] { \footnotesize \tt 2 };
            \fill (N) circle [radius=2pt] node[anchor=west] { \footnotesize \tt 6 };
            \fill (O) circle [radius=2pt] node[anchor=west] { \footnotesize \tt 8 };

            \draw (L) -- (C) -- (H) -- (M) -- (F) -- (J) -- (A) -- (K) -- (L);

        \end{tikzpicture}
    \end{figure}

\end{frame}
