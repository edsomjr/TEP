\section{Geometria Computacional}

\begin{frame}[fragile]{Geometria Computacional e Programação Competitiva}

    \begin{itemize}
        \item A Geometria Computacional é uma área de estudos recente (anos 70), porém é um tópico 
        frequente em maratonas de programação
        \pause

        \item Em geral, os competidores devem postergar a solução de problemas desta área para o meio ou o fim do \textit{contest}, pois
        \pause

        \begin{enumerate}
            \item estes problemas possuem muitos \textit{corner cases}, os quais devem ser tratados com atenção
        \pause
            \item erros de precisão inerentes às variáveis em ponto flutuante podem levar ao WA
        \pause
            \item estes problemas envolvem operações triviais quando feitas com caneta e lápis, mas que possuem implementações sofisticadas
        \pause
            \item estes problemas apresentam soluções que podem ter codificações longas e tediosas
        \end{enumerate}
        \pause

        \item Para atacar tais problemas, o competidor tem que se preparar previamente registrando, em suas anotações, as fórmulas básicas e implementações testadas dos algoritmos clássicos da geometria computacional
    \end{itemize}

\end{frame}

\begin{frame}[fragile]{Observações sobre problemas de Geometria Computacional}

    \begin{itemize}
        \item trate todos os \textit{corner cases} ou use implementações que os evitem;
        \pause

        \item evite o uso de variáveis do tipo ponto flutuante sempre que possível;
        \pause

        \item se for necessário o uso de variáveis em ponto flutuante, defina um limiar $\varepsilon$ 
        (por exemplo, $\varepsilon = 10^{-9}$) para o teste de igualdades:
        \pause

        \begin{enumerate}[i.]
            \item $a = b \ \mbox{se, e somente se, }\ |a - b| < \varepsilon$
        \pause
            \item $a \leq 0 \ \mbox{se, e somente se, }\ a < \varepsilon$
        \pause
            \item $a \geq 0 \ \mbox{se, e somente se, }\ a > -\varepsilon$
        \end{enumerate}
        \pause

        \inputcode{c++}{codes/equals.cpp}

    \end{itemize}

\end{frame}

\begin{frame}[fragile]{Observações sobre problemas de Geometria Computacional}

    \begin{itemize}
        \item caso seja necessário usar variáveis do tipo ponto flutuante, utilize 
            \code{c++}{double} ao invés de \code{c++}{float}
        \pause

        \item se a precisão do tipo \code{c++}{float} (32 \textit{bits}) e do \code{c++}{double} 
        (64 \textit{bits}) não forem suficientes, utilize o tipo \code{c++}{long double}
        (80 \textit{bits})
        \pause

        \item tome cuidado com a impressão do zero em ponto flutuante: em determinados casos, 
        a saída ser prefixada pelo o sinal negativo
        \pause

        \inputcode{c++}{codes/negative_zero.cpp}
    \end{itemize}

\end{frame}

\begin{frame}[fragile]{Abordagens para um problema de Geometria Computacional}

    \begin{itemize}
        \item Há três abordagens possíveis na busca da solução de um problema de Geometria
        Computacional: Geometria Analítica, Geometria Plana e Álgebra Linear
        \pause

        \item Na Geometria Analítica:
        \pause

        \begin{enumerate}
            \item as figuras geométricas são localizadas no espaço através das coordenadas de seus pontos ou vértices
        \pause
            \item são usadas equações para representar figuras e relações
        \pause
            \item novas relações podem ser deduzidas através da combinação e manipulação destas expressões
        \end {enumerate}
        \pause

        \item Na Geometria Plana: 
        \pause

        \begin{enumerate}
            \item as figuras são descritas por suas propriedades
        \pause
            \item a posição absoluta no espaço não é importante, apenas a distância relativa entre duas ou mais figuras
        \pause
            \item as relações são descobertas através de simetrias e semelhanças
        \end{enumerate}
    \end{itemize}

\end{frame}

\begin{frame}[fragile]{Abordagens para um problema de Geometria Computacional}

    \begin{itemize}
 
        \item Na Álgebra Linear: 
        \pause
    
        \begin{itemize}
            \item segmentos de retas são interpretados como vetores
        \pause
            \item transformações como rotações e translações podem simplificar problemas ao deslocar os entes geométricos para uma nova origem
        \pause
            \item transformações lineares são representadas como matrizes, e sua composição equivale a uma multiplicação matricial
        \pause
        \end{itemize}

        \item A Geometria Analítica é a mais comum entre as três abordagens
        \pause

        \item A Geometria Plana é a abordagem menos comum, mas pode simplificar os problemas quando bem utilizada
        \pause
        \item A Álgebra Linear é útil para descobrir relações que seriam trabalhosas de se verificar utilizando apenas a Geometria Analítica
    \end{itemize}

\end{frame}
