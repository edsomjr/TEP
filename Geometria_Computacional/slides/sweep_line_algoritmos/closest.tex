\section{Par de pontos mais próximo}

\begin{frame}[fragile]{Par de pontos mais próximo}

    \begin{itemize}
        \item Dado um conjunto $S$ de $N$ de pontos no plano bidimensional, o problema de 
            encontrar o par de pontos mais próximo consiste em encontrar dois pontos $P, Q\in S$
            tal que
            \[
                \dist(P, Q) = \min\lbrace \dist(P_i, P_j)\rbrace,\ \ \forall P_i\in S\ \ \ \mbox{com}\ \ \ i \neq j
            \]
        \pause

        \item Uma abordagem de busca completa consiste em computar as distância entre todos
            os pares de pontos possível, tendo complexidade $O(N^2)$
        \pause

        \item Contudo, o problema pode ser resolvido em $O(N\log N)$ através do \textit{sweep line}
        \pause

        \item Os pontos deve ser ordenados em ordem lexicográfica
    \end{itemize}

\end{frame}

\begin{frame}[fragile]{Par de pontos mais próximo}

    \begin{itemize}
 
        \item Seja $d = \dist(P_1, P_2)$
        \pause
 
        \item Agora, para todos pontos $P_3, P_4, \ldots, P_N$, deve-se computar todos os pontos
            vizinhos de $P_i = (x, y)$ tais que as coordenadas $x$ estejam no intervalo
            $[x - d, x]$ e que as coordenadas $y$ estejam no intervalo $[y - d, y + d]$
        \pause

        \item Estes pontos podem ser identificados mantendo-se um conjunto de pontos cujas 
            coordenadas estejam entre $[x - d, x]$, ordenado em ordem crescente de coordenada
            $y$
        \pause

        \item Caso a distância de $P_i$ para algum destes pontos seja inferior a $d$, o valor de
            $d$ é atualizado e a varredura continua com este novo valor
        \pause

        \item O ponto principal é que existem, no máximo, $O(1)$ pontos neste retângulo, o que
            faz com que a complexidade do algoritmo seja $O(N\log N)$, por conta da ordenação

    \end{itemize}

\end{frame}

\begin{frame}[fragile]{Visualização de identificação do par de pontos mais próximo}

    \begin{figure}
        \centering

        \begin{tikzpicture}
            \node[anchor=west] at (-2, 7) { Par mais próximo: \texttt{-} };

            \coordinate (A) at (2, 4);
            \coordinate (B) at (5, 3);
            \coordinate (C) at (8, 1);
            \coordinate (D) at (3, 6);
            \coordinate (E) at (1, 1);
            \coordinate (F) at (4, 4);
            \coordinate (G) at (8, 5);
            \coordinate (H) at (6, 2);
            \coordinate (I) at (1, 6);
            \coordinate (J) at (7, 4);
            \coordinate (K) at (8, 6);
            \coordinate (L) at (3, 2);
            \coordinate (M) at (6, 5);
            \coordinate (N) at (-1, 4);

            \fill (A) node[anchor=north] { $A$ } circle [radius=1.5pt];
            \fill (B) node[anchor=north] { $B$ } circle [radius=1.5pt];
            \fill (C) node[anchor=north] { $C$ } circle [radius=1.5pt];
            \fill (D) node[anchor=north] { $D$ } circle [radius=1.5pt];
            \fill (E) node[anchor=north] { $E$ } circle [radius=1.5pt];
            \fill (F) node[anchor=north] { $F$ } circle [radius=1.5pt];
            \fill (G) node[anchor=north] { $G$ } circle [radius=1.5pt];
            \fill (H) node[anchor=north] { $H$ } circle [radius=1.5pt];
            \fill (I) node[anchor=north] { $I$ } circle [radius=1.5pt];
            \fill (J) node[anchor=north] { $J$ } circle [radius=1.5pt];
            \fill (K) node[anchor=north] { $K$ } circle [radius=1.5pt];
            \fill (L) node[anchor=north] { $L$ } circle [radius=1.5pt];
            \fill (M) node[anchor=north] { $M$ } circle [radius=1.5pt];
            \fill (N) node[anchor=north] { $N$ } circle [radius=1.5pt];

        \end{tikzpicture}
    \end{figure}

\end{frame}

\begin{frame}[fragile]{Visualização de identificação do par de pontos mais próximo}

    \begin{figure}
        \centering

        \begin{tikzpicture}
            \node[anchor=west] at (-2, 7) { Par inicial, $\dist(N, E) = \mathbf{3.605551}$ };

            \coordinate (A) at (2, 4);
            \coordinate (B) at (5, 3);
            \coordinate (C) at (8, 1);
            \coordinate (D) at (3, 6);
            \coordinate (E) at (1, 1);
            \coordinate (F) at (4, 4);
            \coordinate (G) at (8, 5);
            \coordinate (H) at (6, 2);
            \coordinate (I) at (1, 6);
            \coordinate (J) at (7, 4);
            \coordinate (K) at (8, 6);
            \coordinate (L) at (3, 2);
            \coordinate (M) at (6, 5);
            \coordinate (N) at (-1, 4);

            \fill (A) node[anchor=north] { $A$ } circle [radius=1.5pt];
            \fill (B) node[anchor=north] { $B$ } circle [radius=1.5pt];
            \fill (C) node[anchor=north] { $C$ } circle [radius=1.5pt];
            \fill (D) node[anchor=north] { $D$ } circle [radius=1.5pt];
            \fill (E) node[anchor=north] { $E$ } circle [radius=1.5pt];
            \fill (F) node[anchor=north] { $F$ } circle [radius=1.5pt];
            \fill (G) node[anchor=north] { $G$ } circle [radius=1.5pt];
            \fill (H) node[anchor=north] { $H$ } circle [radius=1.5pt];
            \fill (I) node[anchor=north] { $I$ } circle [radius=1.5pt];
            \fill (J) node[anchor=north] { $J$ } circle [radius=1.5pt];
            \fill (K) node[anchor=north] { $K$ } circle [radius=1.5pt];
            \fill (L) node[anchor=north] { $L$ } circle [radius=1.5pt];
            \fill (M) node[anchor=north] { $M$ } circle [radius=1.5pt];
            \fill (N) node[anchor=south] { $N$ } circle [radius=1.5pt];

            \draw[thick] (N) -- (E);

        \end{tikzpicture}
    \end{figure}

\end{frame}

\begin{frame}[fragile]{Visualização de identificação do par de pontos mais próximo}

    \begin{figure}
        \centering

        \begin{tikzpicture}
            %\node[anchor=west] at (-2, 7) { Par inicial, $\dist(N, E) = \mathbf{3.605551}$ };
            \node[anchor=west] at (2, 9) { Avaliação do ponto $I$ };

            \coordinate (A) at (2, 4);
            \coordinate (B) at (5, 3);
            \coordinate (C) at (8, 1);
            \coordinate (D) at (3, 6);
            \coordinate (E) at (1, 1);
            \coordinate (F) at (4, 4);
            \coordinate (G) at (8, 5);
            \coordinate (H) at (6, 2);
            \coordinate (I) at (1, 6);
            \coordinate (J) at (7, 4);
            \coordinate (K) at (8, 6);
            \coordinate (L) at (3, 2);
            \coordinate (M) at (6, 5);
            \coordinate (N) at (-1, 4);

            \fill (A) node[anchor=north] { $A$ } circle [radius=1.5pt];
            \fill (B) node[anchor=north] { $B$ } circle [radius=1.5pt];
            \fill (C) node[anchor=north] { $C$ } circle [radius=1.5pt];
            \fill (D) node[anchor=north] { $D$ } circle [radius=1.5pt];
            \fill[color=gray] (E) node[anchor=north] { $E$ } circle [radius=1.5pt];
            \fill (F) node[anchor=north] { $F$ } circle [radius=1.5pt];
            \fill (G) node[anchor=north] { $G$ } circle [radius=1.5pt];
            \fill (H) node[anchor=north] { $H$ } circle [radius=1.5pt];
            \fill (I) node[anchor=west] { $I$ } circle [radius=1.5pt];
            \fill (J) node[anchor=north] { $J$ } circle [radius=1.5pt];
            \fill (K) node[anchor=north] { $K$ } circle [radius=1.5pt];
            \fill (L) node[anchor=north] { $L$ } circle [radius=1.5pt];
            \fill (M) node[anchor=north] { $M$ } circle [radius=1.5pt];
            \fill[color=gray] (N) node[anchor=south] { $N$ } circle [radius=1.5pt];

            \draw[dashed] ($(I) + (-3.605551, -3.605551)$) rectangle ($(I) + (0, 3.605551)$);

        \end{tikzpicture}
    \end{figure}

\end{frame}

\begin{frame}[fragile]{Visualização de identificação do par de pontos mais próximo}

    \begin{figure}
        \centering

        \begin{tikzpicture}
            \node[anchor=west] at (2, 8) { $\dist(I, N) = \mathbf{2.828427} < \dist(N, E) = 3.605551$ };
            %\node[anchor=west] at (2, 9) { Avaliação do ponto $I$ };
            %\node[anchor=west] at (-2, 7) { Distância: \tt \textcolor{blue}{2.828427} };

            \coordinate (A) at (2, 4);
            \coordinate (B) at (5, 3);
            \coordinate (C) at (8, 1);
            \coordinate (D) at (3, 6);
            \coordinate (E) at (1, 1);
            \coordinate (F) at (4, 4);
            \coordinate (G) at (8, 5);
            \coordinate (H) at (6, 2);
            \coordinate (I) at (1, 6);
            \coordinate (J) at (7, 4);
            \coordinate (K) at (8, 6);
            \coordinate (L) at (3, 2);
            \coordinate (M) at (6, 5);
            \coordinate (N) at (-1, 4);

            \fill (A) node[anchor=north] { $A$ } circle [radius=1.5pt];
            \fill (B) node[anchor=north] { $B$ } circle [radius=1.5pt];
            \fill (C) node[anchor=north] { $C$ } circle [radius=1.5pt];
            \fill (D) node[anchor=north] { $D$ } circle [radius=1.5pt];
            \fill[color=gray]  (E) node[anchor=north] { $E$ } circle [radius=1.5pt];
            \fill (F) node[anchor=north] { $F$ } circle [radius=1.5pt];
            \fill (G) node[anchor=north] { $G$ } circle [radius=1.5pt];
            \fill (H) node[anchor=north] { $H$ } circle [radius=1.5pt];
            \fill (I) node[anchor=west] { $I$ } circle [radius=1.5pt];
            \fill (J) node[anchor=north] { $J$ } circle [radius=1.5pt];
            \fill (K) node[anchor=north] { $K$ } circle [radius=1.5pt];
            \fill (L) node[anchor=north] { $L$ } circle [radius=1.5pt];
            \fill (M) node[anchor=north] { $M$ } circle [radius=1.5pt];
            \fill [color=gray] (N) node[anchor=south] { $N$ } circle [radius=1.5pt];

            \draw[thick] (N) -- (I);
            \draw[dashed] ($(I) + (-2.828427, -2.828427)$) rectangle ($(I) + (0, 2.828427)$);

        \end{tikzpicture}
    \end{figure}

\end{frame}

\begin{frame}[fragile]{Visualização de identificação do par de pontos mais próximo}

    \begin{figure}
        \centering

        \begin{tikzpicture}

            %\node[anchor=west] at (2, 8) { $\dist(I, N) = \mathbf{2.828427} < \dist(N, E) = 3.605551$ };
            \node[anchor=west] at (3, 7) { Avaliação do ponto $A$ };

            \coordinate (A) at (2, 4);
            \coordinate (B) at (5, 3);
            \coordinate (C) at (8, 1);
            \coordinate (D) at (3, 6);
            \coordinate (E) at (1, 1);
            \coordinate (F) at (4, 4);
            \coordinate (G) at (8, 5);
            \coordinate (H) at (6, 2);
            \coordinate (I) at (1, 6);
            \coordinate (J) at (7, 4);
            \coordinate (K) at (8, 6);
            \coordinate (L) at (3, 2);
            \coordinate (M) at (6, 5);
            \coordinate (N) at (-1, 4);

            \fill (A) node[anchor=west] { $A$ } circle [radius=1.5pt];
            \fill (B) node[anchor=north] { $B$ } circle [radius=1.5pt];
            \fill (C) node[anchor=north] { $C$ } circle [radius=1.5pt];
            \fill (D) node[anchor=north] { $D$ } circle [radius=1.5pt];
            \fill[color=gray]  (E) node[anchor=north] { $E$ } circle [radius=1.5pt];
            \fill (F) node[anchor=north] { $F$ } circle [radius=1.5pt];
            \fill (G) node[anchor=north] { $G$ } circle [radius=1.5pt];
            \fill (H) node[anchor=north] { $H$ } circle [radius=1.5pt];
            \fill[color=gray]  (I) node[anchor=west] { $I$ } circle [radius=1.5pt];
            \fill (J) node[anchor=north] { $J$ } circle [radius=1.5pt];
            \fill (K) node[anchor=north] { $K$ } circle [radius=1.5pt];
            \fill (L) node[anchor=north] { $L$ } circle [radius=1.5pt];
            \fill (M) node[anchor=north] { $M$ } circle [radius=1.5pt];
            \fill[color=gray]  (N) node[anchor=south] { $N$ } circle [radius=1.5pt];

            \draw[dashed] ($(A) + (-2.828427, -2.828427)$) rectangle ($(A) + (0, 2.828427)$);

        \end{tikzpicture}
    \end{figure}

\end{frame}

\begin{frame}[fragile]{Visualização de identificação do par de pontos mais próximo}

    \begin{figure}
        \centering

        \begin{tikzpicture}

            \node[anchor=west] at (0, 7) { $\dist(A, I) = \mathbf{2.236068} < \dist(I, N) = 2.828427$ };
            %\node[anchor=west] at (3, 7) { Avaliação do ponto $A$ };

            \coordinate (A) at (2, 4);
            \coordinate (B) at (5, 3);
            \coordinate (C) at (8, 1);
            \coordinate (D) at (3, 6);
            \coordinate (E) at (1, 1);
            \coordinate (F) at (4, 4);
            \coordinate (G) at (8, 5);
            \coordinate (H) at (6, 2);
            \coordinate (I) at (1, 6);
            \coordinate (J) at (7, 4);
            \coordinate (K) at (8, 6);
            \coordinate (L) at (3, 2);
            \coordinate (M) at (6, 5);
            \coordinate (N) at (-1, 4);

            \fill (A) node[anchor=west] { $A$ } circle [radius=1.5pt];
            \fill (B) node[anchor=north] { $B$ } circle [radius=1.5pt];
            \fill (C) node[anchor=north] { $C$ } circle [radius=1.5pt];
            \fill (D) node[anchor=north] { $D$ } circle [radius=1.5pt];
            \fill[color=gray]  (E) node[anchor=north] { $E$ } circle [radius=1.5pt];
            \fill (F) node[anchor=north] { $F$ } circle [radius=1.5pt];
            \fill (G) node[anchor=north] { $G$ } circle [radius=1.5pt];
            \fill (H) node[anchor=north] { $H$ } circle [radius=1.5pt];
            \fill[color=gray]  (I) node[anchor=west] { $I$ } circle [radius=1.5pt];
            \fill (J) node[anchor=north] { $J$ } circle [radius=1.5pt];
            \fill (K) node[anchor=north] { $K$ } circle [radius=1.5pt];
            \fill (L) node[anchor=north] { $L$ } circle [radius=1.5pt];
            \fill (M) node[anchor=north] { $M$ } circle [radius=1.5pt];
            \fill[color=gray]  (N) node[anchor=south] { $N$ } circle [radius=1.5pt];

            \draw[thick] (A) -- (I);
            \draw[dashed] ($(A) + (-2.236068, -2.236068)$) rectangle ($(A) + (0, 2.236068)$);

        \end{tikzpicture}
    \end{figure}

\end{frame}

\begin{frame}[fragile]{Visualização de identificação do par de pontos mais próximo}

    \begin{figure}
        \centering

        \begin{tikzpicture}

            %\node[anchor=west] at (0, 7) { $\dist(A, I) = \mathbf{2.236068} < \dist(I, N) = 2.828427$ };
            \node[anchor=west] at (0, 7) { Avaliação do ponto $L$ };

            \coordinate (A) at (2, 4);
            \coordinate (B) at (5, 3);
            \coordinate (C) at (8, 1);
            \coordinate (D) at (3, 6);
            \coordinate (E) at (1, 1);
            \coordinate (F) at (4, 4);
            \coordinate (G) at (8, 5);
            \coordinate (H) at (6, 2);
            \coordinate (I) at (1, 6);
            \coordinate (J) at (7, 4);
            \coordinate (K) at (8, 6);
            \coordinate (L) at (3, 2);
            \coordinate (M) at (6, 5);
            \coordinate (N) at (-1, 4);

            \fill[color=gray]  (A) node[anchor=west] { $A$ } circle [radius=1.5pt];
            \fill (B) node[anchor=north] { $B$ } circle [radius=1.5pt];
            \fill (C) node[anchor=north] { $C$ } circle [radius=1.5pt];
            \fill (D) node[anchor=north] { $D$ } circle [radius=1.5pt];
            \fill[color=gray]  (E) node[anchor=north] { $E$ } circle [radius=1.5pt];
            \fill (F) node[anchor=north] { $F$ } circle [radius=1.5pt];
            \fill (G) node[anchor=north] { $G$ } circle [radius=1.5pt];
            \fill (H) node[anchor=north] { $H$ } circle [radius=1.5pt];
            \fill[color=gray]  (I) node[anchor=west] { $I$ } circle [radius=1.5pt];
            \fill (J) node[anchor=north] { $J$ } circle [radius=1.5pt];
            \fill (K) node[anchor=north] { $K$ } circle [radius=1.5pt];
            \fill (L) node[anchor=west] { $L$ } circle [radius=1.5pt];
            \fill (M) node[anchor=north] { $M$ } circle [radius=1.5pt];
            \fill[color=gray]  (N) node[anchor=south] { $N$ } circle [radius=1.5pt];

            \draw[dashed] ($(L) + (-2.236068, -2.236068)$) rectangle ($(L) + (0, 2.236068)$);

        \end{tikzpicture}
    \end{figure}

\end{frame}

\begin{frame}[fragile]{Visualização de identificação do par de pontos mais próximo}

    \begin{figure}
        \centering

        \begin{tikzpicture}

            \node[anchor=west] at (0, 7) { $\dist(L, E) = \dist(A, I) = \mathbf{2.236068}$ };
            %\node[anchor=west] at (0, 7) { Avaliação do ponto $L$ };

            \coordinate (A) at (2, 4);
            \coordinate (B) at (5, 3);
            \coordinate (C) at (8, 1);
            \coordinate (D) at (3, 6);
            \coordinate (E) at (1, 1);
            \coordinate (F) at (4, 4);
            \coordinate (G) at (8, 5);
            \coordinate (H) at (6, 2);
            \coordinate (I) at (1, 6);
            \coordinate (J) at (7, 4);
            \coordinate (K) at (8, 6);
            \coordinate (L) at (3, 2);
            \coordinate (M) at (6, 5);
            \coordinate (N) at (-1, 4);

            \fill[color=gray]  (A) node[anchor=west] { $A$ } circle [radius=1.5pt];
            \fill (B) node[anchor=north] { $B$ } circle [radius=1.5pt];
            \fill (C) node[anchor=north] { $C$ } circle [radius=1.5pt];
            \fill (D) node[anchor=north] { $D$ } circle [radius=1.5pt];
            \fill[color=gray]  (E) node[anchor=north] { $E$ } circle [radius=1.5pt];
            \fill (F) node[anchor=north] { $F$ } circle [radius=1.5pt];
            \fill (G) node[anchor=north] { $G$ } circle [radius=1.5pt];
            \fill (H) node[anchor=north] { $H$ } circle [radius=1.5pt];
            \fill[color=gray]  (I) node[anchor=west] { $I$ } circle [radius=1.5pt];
            \fill (J) node[anchor=north] { $J$ } circle [radius=1.5pt];
            \fill (K) node[anchor=north] { $K$ } circle [radius=1.5pt];
            \fill (L) node[anchor=west] { $L$ } circle [radius=1.5pt];
            \fill (M) node[anchor=north] { $M$ } circle [radius=1.5pt];
            \fill[color=gray]  (N) node[anchor=south] { $N$ } circle [radius=1.5pt];

            \draw[thick] (L) -- (E);
            \draw[dashed] ($(L) + (-2.236068, -2.236068)$) rectangle ($(L) + (0, 2.236068)$);

        \end{tikzpicture}
    \end{figure}

\end{frame}

\begin{frame}[fragile]{Visualização de identificação do par de pontos mais próximo}

    \begin{figure}
        \centering

        \begin{tikzpicture}
            \node[anchor=west] at (0, 7) { $\dist(L, A) = \dist(A, I) = \mathbf{2.236068}$ };
            %\node[anchor=west] at (0, 7) { Avaliação do ponto $L$ };

            \coordinate (A) at (2, 4);
            \coordinate (B) at (5, 3);
            \coordinate (C) at (8, 1);
            \coordinate (D) at (3, 6);
            \coordinate (E) at (1, 1);
            \coordinate (F) at (4, 4);
            \coordinate (G) at (8, 5);
            \coordinate (H) at (6, 2);
            \coordinate (I) at (1, 6);
            \coordinate (J) at (7, 4);
            \coordinate (K) at (8, 6);
            \coordinate (L) at (3, 2);
            \coordinate (M) at (6, 5);
            \coordinate (N) at (-1, 4);

            \fill[color=gray]  (A) node[anchor=west] { $A$ } circle [radius=1.5pt];
            \fill (B) node[anchor=north] { $B$ } circle [radius=1.5pt];
            \fill (C) node[anchor=north] { $C$ } circle [radius=1.5pt];
            \fill (D) node[anchor=north] { $D$ } circle [radius=1.5pt];
            \fill[color=gray]  (E) node[anchor=north] { $E$ } circle [radius=1.5pt];
            \fill (F) node[anchor=north] { $F$ } circle [radius=1.5pt];
            \fill (G) node[anchor=north] { $G$ } circle [radius=1.5pt];
            \fill (H) node[anchor=north] { $H$ } circle [radius=1.5pt];
            \fill[color=gray]  (I) node[anchor=west] { $I$ } circle [radius=1.5pt];
            \fill (J) node[anchor=north] { $J$ } circle [radius=1.5pt];
            \fill (K) node[anchor=north] { $K$ } circle [radius=1.5pt];
            \fill (L) node[anchor=west] { $L$ } circle [radius=1.5pt];
            \fill (M) node[anchor=north] { $M$ } circle [radius=1.5pt];
            \fill[color=gray]  (N) node[anchor=south] { $N$ } circle [radius=1.5pt];

            \draw[thick] (A) -- (L);
            \draw[dashed] ($(L) + (-2.236068, -2.236068)$) rectangle ($(L) + (0, 2.236068)$);

        \end{tikzpicture}
    \end{figure}

\end{frame}

\begin{frame}[fragile]{Visualização de identificação do par de pontos mais próximo}

    \begin{figure}
        \centering

        \begin{tikzpicture}
            %\node[anchor=west] at (0, 7) { $\dist(L, A) = \dist(A, I) = \mathbf{2.236068}$ };
            \node[anchor=west] at (4, 7) { Avaliação do ponto $D$ };

            \coordinate (A) at (2, 4);
            \coordinate (B) at (5, 3);
            \coordinate (C) at (8, 1);
            \coordinate (D) at (3, 6);
            \coordinate (E) at (1, 1);
            \coordinate (F) at (4, 4);
            \coordinate (G) at (8, 5);
            \coordinate (H) at (6, 2);
            \coordinate (I) at (1, 6);
            \coordinate (J) at (7, 4);
            \coordinate (K) at (8, 6);
            \coordinate (L) at (3, 2);
            \coordinate (M) at (6, 5);
            \coordinate (N) at (-1, 4);

            \fill[color=gray]  (A) node[anchor=west] { $A$ } circle [radius=1.5pt];
            \fill (B) node[anchor=north] { $B$ } circle [radius=1.5pt];
            \fill (C) node[anchor=north] { $C$ } circle [radius=1.5pt];
            \fill (D) node[anchor=west] { $D$ } circle [radius=1.5pt];
            \fill[color=gray]  (E) node[anchor=north] { $E$ } circle [radius=1.5pt];
            \fill (F) node[anchor=north] { $F$ } circle [radius=1.5pt];
            \fill (G) node[anchor=north] { $G$ } circle [radius=1.5pt];
            \fill (H) node[anchor=north] { $H$ } circle [radius=1.5pt];
            \fill[color=gray]  (I) node[anchor=west] { $I$ } circle [radius=1.5pt];
            \fill (J) node[anchor=north] { $J$ } circle [radius=1.5pt];
            \fill (K) node[anchor=north] { $K$ } circle [radius=1.5pt];
            \fill[color=gray]  (L) node[anchor=west] { $L$ } circle [radius=1.5pt];
            \fill (M) node[anchor=north] { $M$ } circle [radius=1.5pt];
            \fill[color=gray]  (N) node[anchor=south] { $N$ } circle [radius=1.5pt];

            \draw[dashed] ($(D) + (-2.236068, -2.236068)$) rectangle ($(D) + (0, 2.236068)$);

        \end{tikzpicture}
    \end{figure}

\end{frame}

\begin{frame}[fragile]{Visualização de identificação do par de pontos mais próximo}

    \begin{figure}
        \centering

        \begin{tikzpicture}
            \node[anchor=west] at (3.5, 7) { $\dist(D, A) = \dist(A, I) = \mathbf{2.236068}$ };
            %\node[anchor=west] at (4, 7) { Avaliação do ponto $D$ };

            \coordinate (A) at (2, 4);
            \coordinate (B) at (5, 3);
            \coordinate (C) at (8, 1);
            \coordinate (D) at (3, 6);
            \coordinate (E) at (1, 1);
            \coordinate (F) at (4, 4);
            \coordinate (G) at (8, 5);
            \coordinate (H) at (6, 2);
            \coordinate (I) at (1, 6);
            \coordinate (J) at (7, 4);
            \coordinate (K) at (8, 6);
            \coordinate (L) at (3, 2);
            \coordinate (M) at (6, 5);
            \coordinate (N) at (-1, 4);

            \fill[color=gray]  (A) node[anchor=west] { $A$ } circle [radius=1.5pt];
            \fill (B) node[anchor=north] { $B$ } circle [radius=1.5pt];
            \fill (C) node[anchor=north] { $C$ } circle [radius=1.5pt];
            \fill (D) node[anchor=west] { $D$ } circle [radius=1.5pt];
            \fill[color=gray]  (E) node[anchor=north] { $E$ } circle [radius=1.5pt];
            \fill (F) node[anchor=north] { $F$ } circle [radius=1.5pt];
            \fill (G) node[anchor=north] { $G$ } circle [radius=1.5pt];
            \fill (H) node[anchor=north] { $H$ } circle [radius=1.5pt];
            \fill[color=gray]  (I) node[anchor=west] { $I$ } circle [radius=1.5pt];
            \fill (J) node[anchor=north] { $J$ } circle [radius=1.5pt];
            \fill (K) node[anchor=north] { $K$ } circle [radius=1.5pt];
            \fill[color=gray]  (L) node[anchor=west] { $L$ } circle [radius=1.5pt];
            \fill (M) node[anchor=north] { $M$ } circle [radius=1.5pt];
            \fill[color=gray]  (N) node[anchor=south] { $N$ } circle [radius=1.5pt];

            \draw[thick] (A) -- (D);
            \draw[dashed] ($(D) + (-2.236068, -2.236068)$) rectangle ($(D) + (0, 2.236068)$);

        \end{tikzpicture}
    \end{figure}

\end{frame}

\begin{frame}[fragile]{Visualização de identificação do par de pontos mais próximo}

    \begin{figure}
        \centering

        \begin{tikzpicture}
            \node[anchor=west] at (3.5, 7) { $\dist(D, I) = \mathbf{2} < \dist(A, I) = 2.236068$ };
            %\node[anchor=west] at (4, 7) { Avaliação do ponto $D$ };

            \coordinate (A) at (2, 4);
            \coordinate (B) at (5, 3);
            \coordinate (C) at (8, 1);
            \coordinate (D) at (3, 6);
            \coordinate (E) at (1, 1);
            \coordinate (F) at (4, 4);
            \coordinate (G) at (8, 5);
            \coordinate (H) at (6, 2);
            \coordinate (I) at (1, 6);
            \coordinate (J) at (7, 4);
            \coordinate (K) at (8, 6);
            \coordinate (L) at (3, 2);
            \coordinate (M) at (6, 5);
            \coordinate (N) at (-1, 4);

            \fill[color=gray]  (A) node[anchor=north] { $A$ } circle [radius=1.5pt];
            \fill (B) node[anchor=north] { $B$ } circle [radius=1.5pt];
            \fill (C) node[anchor=north] { $C$ } circle [radius=1.5pt];
            \fill (D) node[anchor=west] { $D$ } circle [radius=1.5pt];
            \fill[color=gray]  (E) node[anchor=north] { $E$ } circle [radius=1.5pt];
            \fill (F) node[anchor=north] { $F$ } circle [radius=1.5pt];
            \fill (G) node[anchor=north] { $G$ } circle [radius=1.5pt];
            \fill (H) node[anchor=north] { $H$ } circle [radius=1.5pt];
            \fill[color=gray]  (I) node[anchor=east] { $I$ } circle [radius=1.5pt];
            \fill (J) node[anchor=north] { $J$ } circle [radius=1.5pt];
            \fill (K) node[anchor=north] { $K$ } circle [radius=1.5pt];
            \fill[color=gray]  (L) node[anchor=west] { $L$ } circle [radius=1.5pt];
            \fill (M) node[anchor=north] { $M$ } circle [radius=1.5pt];
            \fill[color=gray]  (N) node[anchor=south] { $N$ } circle [radius=1.5pt];

            \draw[thick] (D) -- (I);
            \draw[dashed] ($(D) + (-2.000000, -2.000000)$) rectangle ($(D) + (0, 2.000000)$);

        \end{tikzpicture}
    \end{figure}

\end{frame}

\begin{frame}[fragile]{Visualização de identificação do par de pontos mais próximo}

    \begin{figure}
        \centering

        \begin{tikzpicture}
            %\node[anchor=west] at (3.5, 7) { $\dist(D, I) = \mathbf{2} < \dist(A, I) = 2.236068$ };
            \node[anchor=west] at (0, 7) { Avaliação do ponto $F$ };

            \coordinate (A) at (2, 4);
            \coordinate (B) at (5, 3);
            \coordinate (C) at (8, 1);
            \coordinate (D) at (3, 6);
            \coordinate (E) at (1, 1);
            \coordinate (F) at (4, 4);
            \coordinate (G) at (8, 5);
            \coordinate (H) at (6, 2);
            \coordinate (I) at (1, 6);
            \coordinate (J) at (7, 4);
            \coordinate (K) at (8, 6);
            \coordinate (L) at (3, 2);
            \coordinate (M) at (6, 5);
            \coordinate (N) at (-1, 4);

            \fill[color=gray]  (A) node[anchor=east] { $A$ } circle [radius=1.5pt];
            \fill (B) node[anchor=north] { $B$ } circle [radius=1.5pt];
            \fill (C) node[anchor=north] { $C$ } circle [radius=1.5pt];
            \fill[color=gray]  (D) node[anchor=south] { $D$ } circle [radius=1.5pt];
            \fill[color=gray]  (E) node[anchor=north] { $E$ } circle [radius=1.5pt];
            \fill (F) node[anchor=west] { $F$ } circle [radius=1.5pt];
            \fill (G) node[anchor=north] { $G$ } circle [radius=1.5pt];
            \fill (H) node[anchor=north] { $H$ } circle [radius=1.5pt];
            \fill[color=gray]  (I) node[anchor=east] { $I$ } circle [radius=1.5pt];
            \fill (J) node[anchor=north] { $J$ } circle [radius=1.5pt];
            \fill (K) node[anchor=north] { $K$ } circle [radius=1.5pt];
            \fill[color=gray]  (L) node[anchor=north] { $L$ } circle [radius=1.5pt];
            \fill (M) node[anchor=north] { $M$ } circle [radius=1.5pt];
            \fill[color=gray]  (N) node[anchor=south] { $N$ } circle [radius=1.5pt];

            \draw[dashed] ($(F) + (-2.000000, -2.000000)$) rectangle ($(F) + (0, 2.000000)$);

        \end{tikzpicture}
    \end{figure}

\end{frame}

\begin{frame}[fragile]{Visualização de identificação do par de pontos mais próximo}

    \begin{figure}
        \centering

        \begin{tikzpicture}
            \node[anchor=west] at (0, 7) { $\dist(F, L) = 2.236068 > \dist(D, I) = \mathbf{2}$ };
            %\node[anchor=west] at (0, 7) { Avaliação do ponto $F$ };

            \coordinate (A) at (2, 4);
            \coordinate (B) at (5, 3);
            \coordinate (C) at (8, 1);
            \coordinate (D) at (3, 6);
            \coordinate (E) at (1, 1);
            \coordinate (F) at (4, 4);
            \coordinate (G) at (8, 5);
            \coordinate (H) at (6, 2);
            \coordinate (I) at (1, 6);
            \coordinate (J) at (7, 4);
            \coordinate (K) at (8, 6);
            \coordinate (L) at (3, 2);
            \coordinate (M) at (6, 5);
            \coordinate (N) at (-1, 4);

            \fill[color=gray]  (A) node[anchor=east] { $A$ } circle [radius=1.5pt];
            \fill (B) node[anchor=north] { $B$ } circle [radius=1.5pt];
            \fill (C) node[anchor=north] { $C$ } circle [radius=1.5pt];
            \fill[color=gray]  (D) node[anchor=south] { $D$ } circle [radius=1.5pt];
            \fill[color=gray]  (E) node[anchor=north] { $E$ } circle [radius=1.5pt];
            \fill (F) node[anchor=west] { $F$ } circle [radius=1.5pt];
            \fill (G) node[anchor=north] { $G$ } circle [radius=1.5pt];
            \fill (H) node[anchor=north] { $H$ } circle [radius=1.5pt];
            \fill[color=gray]  (I) node[anchor=east] { $I$ } circle [radius=1.5pt];
            \fill (J) node[anchor=north] { $J$ } circle [radius=1.5pt];
            \fill (K) node[anchor=north] { $K$ } circle [radius=1.5pt];
            \fill[color=gray]  (L) node[anchor=north] { $L$ } circle [radius=1.5pt];
            \fill (M) node[anchor=north] { $M$ } circle [radius=1.5pt];
            \fill[color=gray]  (N) node[anchor=south] { $N$ } circle [radius=1.5pt];

            \draw[thick] (F) -- (L);
            \draw[dashed] ($(F) + (-2.000000, -2.000000)$) rectangle ($(F) + (0, 2.000000)$);

        \end{tikzpicture}
    \end{figure}

\end{frame}

\begin{frame}[fragile]{Visualização de identificação do par de pontos mais próximo}

    \begin{figure}
        \centering

        \begin{tikzpicture}
            \node[anchor=west] at (0, 7) { $\dist(F, A) =\dist(D, I) = \mathbf{2}$ };
            %\node[anchor=west] at (0, 7) { $\dist(F, L) = 2.236068 > \dist(D, I) = \mathbf{2}$ };
            %\node[anchor=west] at (0, 7) { Avaliação do ponto $F$ };

            \coordinate (A) at (2, 4);
            \coordinate (B) at (5, 3);
            \coordinate (C) at (8, 1);
            \coordinate (D) at (3, 6);
            \coordinate (E) at (1, 1);
            \coordinate (F) at (4, 4);
            \coordinate (G) at (8, 5);
            \coordinate (H) at (6, 2);
            \coordinate (I) at (1, 6);
            \coordinate (J) at (7, 4);
            \coordinate (K) at (8, 6);
            \coordinate (L) at (3, 2);
            \coordinate (M) at (6, 5);
            \coordinate (N) at (-1, 4);

            \fill[color=gray]  (A) node[anchor=east] { $A$ } circle [radius=1.5pt];
            \fill (B) node[anchor=north] { $B$ } circle [radius=1.5pt];
            \fill (C) node[anchor=north] { $C$ } circle [radius=1.5pt];
            \fill[color=gray]  (D) node[anchor=south] { $D$ } circle [radius=1.5pt];
            \fill[color=gray]  (E) node[anchor=north] { $E$ } circle [radius=1.5pt];
            \fill (F) node[anchor=west] { $F$ } circle [radius=1.5pt];
            \fill (G) node[anchor=north] { $G$ } circle [radius=1.5pt];
            \fill (H) node[anchor=north] { $H$ } circle [radius=1.5pt];
            \fill[color=gray]  (I) node[anchor=east] { $I$ } circle [radius=1.5pt];
            \fill (J) node[anchor=north] { $J$ } circle [radius=1.5pt];
            \fill (K) node[anchor=north] { $K$ } circle [radius=1.5pt];
            \fill[color=gray]  (L) node[anchor=north] { $L$ } circle [radius=1.5pt];
            \fill (M) node[anchor=north] { $M$ } circle [radius=1.5pt];
            \fill[color=gray]  (N) node[anchor=south] { $N$ } circle [radius=1.5pt];

            \draw[thick] (F) -- (A);
            \draw[dashed] ($(F) + (-2.000000, -2.000000)$) rectangle ($(F) + (0, 2.000000)$);

        \end{tikzpicture}
    \end{figure}

\end{frame}

\begin{frame}[fragile]{Visualização de identificação do par de pontos mais próximo}

    \begin{figure}
        \centering

        \begin{tikzpicture}
            \node[anchor=west] at (0, 7) { $\dist(F, D) = 2.236068 > \dist(D, I) = \mathbf{2}$ };
            %\node[anchor=west] at (0, 7) { Avaliação do ponto $F$ };

            \coordinate (A) at (2, 4);
            \coordinate (B) at (5, 3);
            \coordinate (C) at (8, 1);
            \coordinate (D) at (3, 6);
            \coordinate (E) at (1, 1);
            \coordinate (F) at (4, 4);
            \coordinate (G) at (8, 5);
            \coordinate (H) at (6, 2);
            \coordinate (I) at (1, 6);
            \coordinate (J) at (7, 4);
            \coordinate (K) at (8, 6);
            \coordinate (L) at (3, 2);
            \coordinate (M) at (6, 5);
            \coordinate (N) at (-1, 4);

            \fill[color=gray]  (A) node[anchor=east] { $A$ } circle [radius=1.5pt];
            \fill (B) node[anchor=north] { $B$ } circle [radius=1.5pt];
            \fill (C) node[anchor=north] { $C$ } circle [radius=1.5pt];
            \fill[color=gray]  (D) node[anchor=south] { $D$ } circle [radius=1.5pt];
            \fill[color=gray]  (E) node[anchor=north] { $E$ } circle [radius=1.5pt];
            \fill (F) node[anchor=west] { $F$ } circle [radius=1.5pt];
            \fill (G) node[anchor=north] { $G$ } circle [radius=1.5pt];
            \fill (H) node[anchor=north] { $H$ } circle [radius=1.5pt];
            \fill[color=gray]  (I) node[anchor=east] { $I$ } circle [radius=1.5pt];
            \fill (J) node[anchor=north] { $J$ } circle [radius=1.5pt];
            \fill (K) node[anchor=north] { $K$ } circle [radius=1.5pt];
            \fill[color=gray]  (L) node[anchor=north] { $L$ } circle [radius=1.5pt];
            \fill (M) node[anchor=north] { $M$ } circle [radius=1.5pt];
            \fill[color=gray]  (N) node[anchor=south] { $N$ } circle [radius=1.5pt];

            \draw[thick] (F) -- (D);
            \draw[dashed] ($(F) + (-2.000000, -2.000000)$) rectangle ($(F) + (0, 2.000000)$);

        \end{tikzpicture}
    \end{figure}

\end{frame}

\begin{frame}[fragile]{Visualização de identificação do par de pontos mais próximo}

    \begin{figure}
        \centering

        \begin{tikzpicture}
            %\node[anchor=west] at (0, 7) { $\dist(F, D) = 2.236068 > \dist(D, I) = \mathbf{2}$ };
            \node[anchor=west] at (0, 7) { Avaliação do ponto $B$ };

            \coordinate (A) at (2, 4);
            \coordinate (B) at (5, 3);
            \coordinate (C) at (8, 1);
            \coordinate (D) at (3, 6);
            \coordinate (E) at (1, 1);
            \coordinate (F) at (4, 4);
            \coordinate (G) at (8, 5);
            \coordinate (H) at (6, 2);
            \coordinate (I) at (1, 6);
            \coordinate (J) at (7, 4);
            \coordinate (K) at (8, 6);
            \coordinate (L) at (3, 2);
            \coordinate (M) at (6, 5);
            \coordinate (N) at (-1, 4);

            \fill[color=gray]  (A) node[anchor=east] { $A$ } circle [radius=1.5pt];
            \fill (B) node[anchor=west] { $B$ } circle [radius=1.5pt];
            \fill (C) node[anchor=north] { $C$ } circle [radius=1.5pt];
            \fill[color=gray]  (D) node[anchor=south] { $D$ } circle [radius=1.5pt];
            \fill[color=gray]  (E) node[anchor=north] { $E$ } circle [radius=1.5pt];
            \fill[color=gray]  (F) node[anchor=west] { $F$ } circle [radius=1.5pt];
            \fill (G) node[anchor=north] { $G$ } circle [radius=1.5pt];
            \fill (H) node[anchor=north] { $H$ } circle [radius=1.5pt];
            \fill[color=gray]  (I) node[anchor=east] { $I$ } circle [radius=1.5pt];
            \fill (J) node[anchor=north] { $J$ } circle [radius=1.5pt];
            \fill (K) node[anchor=north] { $K$ } circle [radius=1.5pt];
            \fill[color=gray]  (L) node[anchor=east] { $L$ } circle [radius=1.5pt];
            \fill (M) node[anchor=north] { $M$ } circle [radius=1.5pt];
            \fill[color=gray]  (N) node[anchor=south] { $N$ } circle [radius=1.5pt];

            \draw[dashed] ($(B) + (-2.000000, -2.000000)$) rectangle ($(B) + (0, 2.000000)$);

        \end{tikzpicture}
    \end{figure}

\end{frame}

\begin{frame}[fragile]{Visualização de identificação do par de pontos mais próximo}

    \begin{figure}
        \centering

        \begin{tikzpicture}
            \node[anchor=west] at (0, 7) { $\dist(B, L) = 2.236068 > \dist(D, I) = \mathbf{2}$ };
            %\node[anchor=west] at (0, 7) { Avaliação do ponto $B$ };

            \coordinate (A) at (2, 4);
            \coordinate (B) at (5, 3);
            \coordinate (C) at (8, 1);
            \coordinate (D) at (3, 6);
            \coordinate (E) at (1, 1);
            \coordinate (F) at (4, 4);
            \coordinate (G) at (8, 5);
            \coordinate (H) at (6, 2);
            \coordinate (I) at (1, 6);
            \coordinate (J) at (7, 4);
            \coordinate (K) at (8, 6);
            \coordinate (L) at (3, 2);
            \coordinate (M) at (6, 5);
            \coordinate (N) at (-1, 4);

            \fill[color=gray]  (A) node[anchor=east] { $A$ } circle [radius=1.5pt];
            \fill (B) node[anchor=west] { $B$ } circle [radius=1.5pt];
            \fill (C) node[anchor=north] { $C$ } circle [radius=1.5pt];
            \fill[color=gray]  (D) node[anchor=south] { $D$ } circle [radius=1.5pt];
            \fill[color=gray]  (E) node[anchor=north] { $E$ } circle [radius=1.5pt];
            \fill[color=gray]  (F) node[anchor=west] { $F$ } circle [radius=1.5pt];
            \fill (G) node[anchor=north] { $G$ } circle [radius=1.5pt];
            \fill (H) node[anchor=north] { $H$ } circle [radius=1.5pt];
            \fill[color=gray]  (I) node[anchor=east] { $I$ } circle [radius=1.5pt];
            \fill (J) node[anchor=north] { $J$ } circle [radius=1.5pt];
            \fill (K) node[anchor=north] { $K$ } circle [radius=1.5pt];
            \fill[color=gray]  (L) node[anchor=east] { $L$ } circle [radius=1.5pt];
            \fill (M) node[anchor=north] { $M$ } circle [radius=1.5pt];
            \fill[color=gray]  (N) node[anchor=south] { $N$ } circle [radius=1.5pt];

            \draw[thick] (B) -- (L);
            \draw[dashed] ($(B) + (-2.000000, -2.000000)$) rectangle ($(B) + (0, 2.000000)$);

        \end{tikzpicture}
    \end{figure}

\end{frame}

\begin{frame}[fragile]{Visualização de identificação do par de pontos mais próximo}

    \begin{figure}
        \centering

        \begin{tikzpicture}
            \node[anchor=west] at (0, 7) { $\dist(B, F) = \mathbf{1.414213} < \dist(D, I) = 2$ };
            %\node[anchor=west] at (0, 7) { Avaliação do ponto $B$ };

            \coordinate (A) at (2, 4);
            \coordinate (B) at (5, 3);
            \coordinate (C) at (8, 1);
            \coordinate (D) at (3, 6);
            \coordinate (E) at (1, 1);
            \coordinate (F) at (4, 4);
            \coordinate (G) at (8, 5);
            \coordinate (H) at (6, 2);
            \coordinate (I) at (1, 6);
            \coordinate (J) at (7, 4);
            \coordinate (K) at (8, 6);
            \coordinate (L) at (3, 2);
            \coordinate (M) at (6, 5);
            \coordinate (N) at (-1, 4);

            \fill[color=gray]  (A) node[anchor=east] { $A$ } circle [radius=1.5pt];
            \fill (B) node[anchor=west] { $B$ } circle [radius=1.5pt];
            \fill (C) node[anchor=north] { $C$ } circle [radius=1.5pt];
            \fill[color=gray]  (D) node[anchor=south] { $D$ } circle [radius=1.5pt];
            \fill[color=gray]  (E) node[anchor=north] { $E$ } circle [radius=1.5pt];
            \fill[color=gray]  (F) node[anchor=west] { $F$ } circle [radius=1.5pt];
            \fill (G) node[anchor=north] { $G$ } circle [radius=1.5pt];
            \fill (H) node[anchor=north] { $H$ } circle [radius=1.5pt];
            \fill[color=gray]  (I) node[anchor=east] { $I$ } circle [radius=1.5pt];
            \fill (J) node[anchor=north] { $J$ } circle [radius=1.5pt];
            \fill (K) node[anchor=north] { $K$ } circle [radius=1.5pt];
            \fill[color=gray]  (L) node[anchor=east] { $L$ } circle [radius=1.5pt];
            \fill (M) node[anchor=north] { $M$ } circle [radius=1.5pt];
            \fill[color=gray]  (N) node[anchor=south] { $N$ } circle [radius=1.5pt];

            \draw[thick] (B) -- (F);
            \draw[dashed] ($(B) + (-1.414213, -1.414213)$) rectangle ($(B) + (0, 1.414213)$);

        \end{tikzpicture}
    \end{figure}

\end{frame}

\begin{frame}[fragile]{Visualização de identificação do par de pontos mais próximo}

    \begin{figure}
        \centering

        \begin{tikzpicture}
            %\node[anchor=west] at (0, 7) { $\dist(B, F) = \mathbf{1.414213} > \dist(D, I) = 2$ };
            \node[anchor=west] at (0, 7) { Avaliação do ponto $H$ };

            \coordinate (A) at (2, 4);
            \coordinate (B) at (5, 3);
            \coordinate (C) at (8, 1);
            \coordinate (D) at (3, 6);
            \coordinate (E) at (1, 1);
            \coordinate (F) at (4, 4);
            \coordinate (G) at (8, 5);
            \coordinate (H) at (6, 2);
            \coordinate (I) at (1, 6);
            \coordinate (J) at (7, 4);
            \coordinate (K) at (8, 6);
            \coordinate (L) at (3, 2);
            \coordinate (M) at (6, 5);
            \coordinate (N) at (-1, 4);

            \fill[color=gray]  (A) node[anchor=east] { $A$ } circle [radius=1.5pt];
            \fill[color=gray]  (B) node[anchor=west] { $B$ } circle [radius=1.5pt];
            \fill (C) node[anchor=north] { $C$ } circle [radius=1.5pt];
            \fill[color=gray]  (D) node[anchor=south] { $D$ } circle [radius=1.5pt];
            \fill[color=gray]  (E) node[anchor=north] { $E$ } circle [radius=1.5pt];
            \fill[color=gray]  (F) node[anchor=west] { $F$ } circle [radius=1.5pt];
            \fill (G) node[anchor=north] { $G$ } circle [radius=1.5pt];
            \fill (H) node[anchor=west] { $H$ } circle [radius=1.5pt];
            \fill[color=gray]  (I) node[anchor=east] { $I$ } circle [radius=1.5pt];
            \fill (J) node[anchor=north] { $J$ } circle [radius=1.5pt];
            \fill (K) node[anchor=north] { $K$ } circle [radius=1.5pt];
            \fill[color=gray]  (L) node[anchor=east] { $L$ } circle [radius=1.5pt];
            \fill (M) node[anchor=north] { $M$ } circle [radius=1.5pt];
            \fill[color=gray]  (N) node[anchor=south] { $N$ } circle [radius=1.5pt];

            \draw[dashed] ($(H) + (-1.414213, -1.414213)$) rectangle ($(H) + (0, 1.414213)$);

        \end{tikzpicture}
    \end{figure}

\end{frame}

\begin{frame}[fragile]{Visualização de identificação do par de pontos mais próximo}

    \begin{figure}
        \centering

        \begin{tikzpicture}
            \node[anchor=west] at (0, 7) { $\dist(H, B) = \dist(B, F) = \mathbf{1.414213}$ };
            %\node[anchor=west] at (0, 7) { Avaliação do ponto $H$ };

            \coordinate (A) at (2, 4);
            \coordinate (B) at (5, 3);
            \coordinate (C) at (8, 1);
            \coordinate (D) at (3, 6);
            \coordinate (E) at (1, 1);
            \coordinate (F) at (4, 4);
            \coordinate (G) at (8, 5);
            \coordinate (H) at (6, 2);
            \coordinate (I) at (1, 6);
            \coordinate (J) at (7, 4);
            \coordinate (K) at (8, 6);
            \coordinate (L) at (3, 2);
            \coordinate (M) at (6, 5);
            \coordinate (N) at (-1, 4);

            \fill[color=gray]  (A) node[anchor=east] { $A$ } circle [radius=1.5pt];
            \fill[color=gray]  (B) node[anchor=west] { $B$ } circle [radius=1.5pt];
            \fill (C) node[anchor=north] { $C$ } circle [radius=1.5pt];
            \fill[color=gray]  (D) node[anchor=south] { $D$ } circle [radius=1.5pt];
            \fill[color=gray]  (E) node[anchor=north] { $E$ } circle [radius=1.5pt];
            \fill[color=gray]  (F) node[anchor=west] { $F$ } circle [radius=1.5pt];
            \fill (G) node[anchor=north] { $G$ } circle [radius=1.5pt];
            \fill (H) node[anchor=west] { $H$ } circle [radius=1.5pt];
            \fill[color=gray]  (I) node[anchor=east] { $I$ } circle [radius=1.5pt];
            \fill (J) node[anchor=north] { $J$ } circle [radius=1.5pt];
            \fill (K) node[anchor=north] { $K$ } circle [radius=1.5pt];
            \fill[color=gray]  (L) node[anchor=east] { $L$ } circle [radius=1.5pt];
            \fill (M) node[anchor=north] { $M$ } circle [radius=1.5pt];
            \fill[color=gray]  (N) node[anchor=south] { $N$ } circle [radius=1.5pt];

            \draw[thick] (B) -- (H);
            \draw[dashed] ($(H) + (-1.414213, -1.414213)$) rectangle ($(H) + (0, 1.414213)$);

        \end{tikzpicture}
    \end{figure}

\end{frame}

\begin{frame}[fragile]{Visualização de identificação do par de pontos mais próximo}

    \begin{figure}
        \centering

        \begin{tikzpicture}
            %\node[anchor=west] at (0, 7) { $\dist(H, B) = \dist(B, F) = \mathbf{1.414213}$ };
            \node[anchor=west] at (0, 7) { Avaliação do ponto $M$ };

            \coordinate (A) at (2, 4);
            \coordinate (B) at (5, 3);
            \coordinate (C) at (8, 1);
            \coordinate (D) at (3, 6);
            \coordinate (E) at (1, 1);
            \coordinate (F) at (4, 4);
            \coordinate (G) at (8, 5);
            \coordinate (H) at (6, 2);
            \coordinate (I) at (1, 6);
            \coordinate (J) at (7, 4);
            \coordinate (K) at (8, 6);
            \coordinate (L) at (3, 2);
            \coordinate (M) at (6, 5);
            \coordinate (N) at (-1, 4);

            \fill[color=gray]  (A) node[anchor=east] { $A$ } circle [radius=1.5pt];
            \fill[color=gray]  (B) node[anchor=west] { $B$ } circle [radius=1.5pt];
            \fill (C) node[anchor=north] { $C$ } circle [radius=1.5pt];
            \fill[color=gray]  (D) node[anchor=south] { $D$ } circle [radius=1.5pt];
            \fill[color=gray]  (E) node[anchor=north] { $E$ } circle [radius=1.5pt];
            \fill[color=gray]  (F) node[anchor=west] { $F$ } circle [radius=1.5pt];
            \fill (G) node[anchor=north] { $G$ } circle [radius=1.5pt];
            \fill[color=gray]  (H) node[anchor=west] { $H$ } circle [radius=1.5pt];
            \fill[color=gray]  (I) node[anchor=east] { $I$ } circle [radius=1.5pt];
            \fill (J) node[anchor=north] { $J$ } circle [radius=1.5pt];
            \fill (K) node[anchor=north] { $K$ } circle [radius=1.5pt];
            \fill[color=gray]  (L) node[anchor=east] { $L$ } circle [radius=1.5pt];
            \fill (M) node[anchor=west] { $M$ } circle [radius=1.5pt];
            \fill[color=gray]  (N) node[anchor=south] { $N$ } circle [radius=1.5pt];

            \draw[dashed] ($(M) + (-1.414213, -1.414213)$) rectangle ($(M) + (0, 1.414213)$);

        \end{tikzpicture}
    \end{figure}

\end{frame}

\begin{frame}[fragile]{Visualização de identificação do par de pontos mais próximo}

    \begin{figure}
        \centering

        \begin{tikzpicture}
            %\node[anchor=west] at (0, 7) { $\dist(H, B) = \dist(B, F) = \mathbf{1.414213}$ };
            \node[anchor=west] at (0, 7) { Avaliação do ponto $J$ };

            \coordinate (A) at (2, 4);
            \coordinate (B) at (5, 3);
            \coordinate (C) at (8, 1);
            \coordinate (D) at (3, 6);
            \coordinate (E) at (1, 1);
            \coordinate (F) at (4, 4);
            \coordinate (G) at (8, 5);
            \coordinate (H) at (6, 2);
            \coordinate (I) at (1, 6);
            \coordinate (J) at (7, 4);
            \coordinate (K) at (8, 6);
            \coordinate (L) at (3, 2);
            \coordinate (M) at (6, 5);
            \coordinate (N) at (-1, 4);

            \fill[color=gray]  (A) node[anchor=east] { $A$ } circle [radius=1.5pt];
            \fill[color=gray]  (B) node[anchor=west] { $B$ } circle [radius=1.5pt];
            \fill (C) node[anchor=north] { $C$ } circle [radius=1.5pt];
            \fill[color=gray]  (D) node[anchor=south] { $D$ } circle [radius=1.5pt];
            \fill[color=gray]  (E) node[anchor=north] { $E$ } circle [radius=1.5pt];
            \fill[color=gray]  (F) node[anchor=west] { $F$ } circle [radius=1.5pt];
            \fill (G) node[anchor=north] { $G$ } circle [radius=1.5pt];
            \fill[color=gray]  (H) node[anchor=west] { $H$ } circle [radius=1.5pt];
            \fill[color=gray]  (I) node[anchor=east] { $I$ } circle [radius=1.5pt];
            \fill (J) node[anchor=west] { $J$ } circle [radius=1.5pt];
            \fill (K) node[anchor=north] { $K$ } circle [radius=1.5pt];
            \fill[color=gray]  (L) node[anchor=east] { $L$ } circle [radius=1.5pt];
            \fill[color=gray]  (M) node[anchor=west] { $M$ } circle [radius=1.5pt];
            \fill[color=gray]  (N) node[anchor=south] { $N$ } circle [radius=1.5pt];

            \draw[dashed] ($(J) + (-1.414213, -1.414213)$) rectangle ($(J) + (0, 1.414213)$);

        \end{tikzpicture}
    \end{figure}

\end{frame}

\begin{frame}[fragile]{Visualização de identificação do par de pontos mais próximo}

    \begin{figure}
        \centering

        \begin{tikzpicture}
            \node[anchor=west] at (0, 7) { $\dist(J, M) = \dist(B, F) = \mathbf{1.414213}$ };
            %\node[anchor=west] at (0, 7) { Avaliação do ponto $J$ };

            \coordinate (A) at (2, 4);
            \coordinate (B) at (5, 3);
            \coordinate (C) at (8, 1);
            \coordinate (D) at (3, 6);
            \coordinate (E) at (1, 1);
            \coordinate (F) at (4, 4);
            \coordinate (G) at (8, 5);
            \coordinate (H) at (6, 2);
            \coordinate (I) at (1, 6);
            \coordinate (J) at (7, 4);
            \coordinate (K) at (8, 6);
            \coordinate (L) at (3, 2);
            \coordinate (M) at (6, 5);
            \coordinate (N) at (-1, 4);

            \fill[color=gray]  (A) node[anchor=east] { $A$ } circle [radius=1.5pt];
            \fill[color=gray]  (B) node[anchor=west] { $B$ } circle [radius=1.5pt];
            \fill (C) node[anchor=north] { $C$ } circle [radius=1.5pt];
            \fill[color=gray]  (D) node[anchor=south] { $D$ } circle [radius=1.5pt];
            \fill[color=gray]  (E) node[anchor=north] { $E$ } circle [radius=1.5pt];
            \fill[color=gray]  (F) node[anchor=west] { $F$ } circle [radius=1.5pt];
            \fill (G) node[anchor=north] { $G$ } circle [radius=1.5pt];
            \fill[color=gray]  (H) node[anchor=west] { $H$ } circle [radius=1.5pt];
            \fill[color=gray]  (I) node[anchor=east] { $I$ } circle [radius=1.5pt];
            \fill (J) node[anchor=west] { $J$ } circle [radius=1.5pt];
            \fill (K) node[anchor=north] { $K$ } circle [radius=1.5pt];
            \fill[color=gray]  (L) node[anchor=east] { $L$ } circle [radius=1.5pt];
            \fill[color=gray]  (M) node[anchor=west] { $M$ } circle [radius=1.5pt];
            \fill[color=gray]  (N) node[anchor=south] { $N$ } circle [radius=1.5pt];

            \draw[thick] (M) -- (J);
            \draw[dashed] ($(J) + (-1.414213, -1.414213)$) rectangle ($(J) + (0, 1.414213)$);

        \end{tikzpicture}
    \end{figure}

\end{frame}

\begin{frame}[fragile]{Visualização de identificação do par de pontos mais próximo}

    \begin{figure}
        \centering

        \begin{tikzpicture}
            %\node[anchor=west] at (0, 7) { $\dist(J, M) = \dist(B, F) = \mathbf{1.414213}$ };
            \node[anchor=west] at (0, 7) { Avaliação do ponto $C$ };

            \coordinate (A) at (2, 4);
            \coordinate (B) at (5, 3);
            \coordinate (C) at (8, 1);
            \coordinate (D) at (3, 6);
            \coordinate (E) at (1, 1);
            \coordinate (F) at (4, 4);
            \coordinate (G) at (8, 5);
            \coordinate (H) at (6, 2);
            \coordinate (I) at (1, 6);
            \coordinate (J) at (7, 4);
            \coordinate (K) at (8, 6);
            \coordinate (L) at (3, 2);
            \coordinate (M) at (6, 5);
            \coordinate (N) at (-1, 4);

            \fill[color=gray]  (A) node[anchor=east] { $A$ } circle [radius=1.5pt];
            \fill[color=gray]  (B) node[anchor=west] { $B$ } circle [radius=1.5pt];
            \fill (C) node[anchor=west] { $C$ } circle [radius=1.5pt];
            \fill[color=gray]  (D) node[anchor=south] { $D$ } circle [radius=1.5pt];
            \fill[color=gray]  (E) node[anchor=north] { $E$ } circle [radius=1.5pt];
            \fill[color=gray]  (F) node[anchor=west] { $F$ } circle [radius=1.5pt];
            \fill (G) node[anchor=north] { $G$ } circle [radius=1.5pt];
            \fill[color=gray]  (H) node[anchor=west] { $H$ } circle [radius=1.5pt];
            \fill[color=gray]  (I) node[anchor=east] { $I$ } circle [radius=1.5pt];
            \fill[color=gray]  (J) node[anchor=west] { $J$ } circle [radius=1.5pt];
            \fill (K) node[anchor=north] { $K$ } circle [radius=1.5pt];
            \fill[color=gray]  (L) node[anchor=east] { $L$ } circle [radius=1.5pt];
            \fill[color=gray]  (M) node[anchor=west] { $M$ } circle [radius=1.5pt];
            \fill[color=gray]  (N) node[anchor=south] { $N$ } circle [radius=1.5pt];

            \draw[dashed] ($(C) + (-1.414213, -1.414213)$) rectangle ($(C) + (0, 1.414213)$);

        \end{tikzpicture}
    \end{figure}

\end{frame}

\begin{frame}[fragile]{Visualização de identificação do par de pontos mais próximo}

    \begin{figure}
        \centering

        \begin{tikzpicture}
            %\node[anchor=west] at (0, 7) { $\dist(J, M) = \dist(B, F) = \mathbf{1.414213}$ };
            \node[anchor=west] at (0, 7) { Avaliação do ponto $G$ };

            \coordinate (A) at (2, 4);
            \coordinate (B) at (5, 3);
            \coordinate (C) at (8, 1);
            \coordinate (D) at (3, 6);
            \coordinate (E) at (1, 1);
            \coordinate (F) at (4, 4);
            \coordinate (G) at (8, 5);
            \coordinate (H) at (6, 2);
            \coordinate (I) at (1, 6);
            \coordinate (J) at (7, 4);
            \coordinate (K) at (8, 6);
            \coordinate (L) at (3, 2);
            \coordinate (M) at (6, 5);
            \coordinate (N) at (-1, 4);

            \fill[color=gray]  (A) node[anchor=east] { $A$ } circle [radius=1.5pt];
            \fill[color=gray]  (B) node[anchor=west] { $B$ } circle [radius=1.5pt];
            \fill[color=gray]  (C) node[anchor=west] { $C$ } circle [radius=1.5pt];
            \fill[color=gray]  (D) node[anchor=south] { $D$ } circle [radius=1.5pt];
            \fill[color=gray]  (E) node[anchor=north] { $E$ } circle [radius=1.5pt];
            \fill[color=gray]  (F) node[anchor=west] { $F$ } circle [radius=1.5pt];
            \fill (G) node[anchor=west] { $G$ } circle [radius=1.5pt];
            \fill[color=gray]  (H) node[anchor=west] { $H$ } circle [radius=1.5pt];
            \fill[color=gray]  (I) node[anchor=east] { $I$ } circle [radius=1.5pt];
            \fill[color=gray]  (J) node[anchor=west] { $J$ } circle [radius=1.5pt];
            \fill (K) node[anchor=west] { $K$ } circle [radius=1.5pt];
            \fill[color=gray]  (L) node[anchor=east] { $L$ } circle [radius=1.5pt];
            \fill[color=gray]  (M) node[anchor=west] { $M$ } circle [radius=1.5pt];
            \fill[color=gray]  (N) node[anchor=south] { $N$ } circle [radius=1.5pt];

            \draw[dashed] ($(G) + (-1.414213, -1.414213)$) rectangle ($(G) + (0, 1.414213)$);

        \end{tikzpicture}
    \end{figure}

\end{frame}

\begin{frame}[fragile]{Visualização de identificação do par de pontos mais próximo}

    \begin{figure}
        \centering

        \begin{tikzpicture}
            \node[anchor=west] at (0, 7) { $\dist(G, J) = \dist(B, F) = \mathbf{1.414213}$ };
            %\node[anchor=west] at (0, 7) { Avaliação do ponto $G$ };

            \coordinate (A) at (2, 4);
            \coordinate (B) at (5, 3);
            \coordinate (C) at (8, 1);
            \coordinate (D) at (3, 6);
            \coordinate (E) at (1, 1);
            \coordinate (F) at (4, 4);
            \coordinate (G) at (8, 5);
            \coordinate (H) at (6, 2);
            \coordinate (I) at (1, 6);
            \coordinate (J) at (7, 4);
            \coordinate (K) at (8, 6);
            \coordinate (L) at (3, 2);
            \coordinate (M) at (6, 5);
            \coordinate (N) at (-1, 4);

            \fill[color=gray]  (A) node[anchor=east] { $A$ } circle [radius=1.5pt];
            \fill[color=gray]  (B) node[anchor=west] { $B$ } circle [radius=1.5pt];
            \fill[color=gray]  (C) node[anchor=west] { $C$ } circle [radius=1.5pt];
            \fill[color=gray]  (D) node[anchor=south] { $D$ } circle [radius=1.5pt];
            \fill[color=gray]  (E) node[anchor=north] { $E$ } circle [radius=1.5pt];
            \fill[color=gray]  (F) node[anchor=west] { $F$ } circle [radius=1.5pt];
            \fill (G) node[anchor=west] { $G$ } circle [radius=1.5pt];
            \fill[color=gray]  (H) node[anchor=west] { $H$ } circle [radius=1.5pt];
            \fill[color=gray]  (I) node[anchor=east] { $I$ } circle [radius=1.5pt];
            \fill[color=gray]  (J) node[anchor=west] { $J$ } circle [radius=1.5pt];
            \fill (K) node[anchor=west] { $K$ } circle [radius=1.5pt];
            \fill[color=gray]  (L) node[anchor=east] { $L$ } circle [radius=1.5pt];
            \fill[color=gray]  (M) node[anchor=west] { $M$ } circle [radius=1.5pt];
            \fill[color=gray]  (N) node[anchor=south] { $N$ } circle [radius=1.5pt];

            \draw[thick] (G) -- (J);
            \draw[dashed] ($(G) + (-1.414213, -1.414213)$) rectangle ($(G) + (0, 1.414213)$);

        \end{tikzpicture}
    \end{figure}

\end{frame}

\begin{frame}[fragile]{Visualização de identificação do par de pontos mais próximo}

    \begin{figure}
        \centering

        \begin{tikzpicture}
            %\node[anchor=west] at (0, 7) { $\dist(G, J) = \dist(B, F) = \mathbf{1.414213}$ };
            \node[anchor=west] at (0, 7) { Avaliação do ponto $K$ };

            \coordinate (A) at (2, 4);
            \coordinate (B) at (5, 3);
            \coordinate (C) at (8, 1);
            \coordinate (D) at (3, 6);
            \coordinate (E) at (1, 1);
            \coordinate (F) at (4, 4);
            \coordinate (G) at (8, 5);
            \coordinate (H) at (6, 2);
            \coordinate (I) at (1, 6);
            \coordinate (J) at (7, 4);
            \coordinate (K) at (8, 6);
            \coordinate (L) at (3, 2);
            \coordinate (M) at (6, 5);
            \coordinate (N) at (-1, 4);

            \fill[color=gray]  (A) node[anchor=east] { $A$ } circle [radius=1.5pt];
            \fill[color=gray]  (B) node[anchor=west] { $B$ } circle [radius=1.5pt];
            \fill[color=gray]  (C) node[anchor=west] { $C$ } circle [radius=1.5pt];
            \fill[color=gray]  (D) node[anchor=south] { $D$ } circle [radius=1.5pt];
            \fill[color=gray]  (E) node[anchor=north] { $E$ } circle [radius=1.5pt];
            \fill[color=gray]  (F) node[anchor=west] { $F$ } circle [radius=1.5pt];
            \fill[color=gray]  (G) node[anchor=west] { $G$ } circle [radius=1.5pt];
            \fill[color=gray]  (H) node[anchor=west] { $H$ } circle [radius=1.5pt];
            \fill[color=gray]  (I) node[anchor=east] { $I$ } circle [radius=1.5pt];
            \fill[color=gray]  (J) node[anchor=west] { $J$ } circle [radius=1.5pt];
            \fill (K) node[anchor=west] { $K$ } circle [radius=1.5pt];
            \fill[color=gray]  (L) node[anchor=east] { $L$ } circle [radius=1.5pt];
            \fill[color=gray]  (M) node[anchor=west] { $M$ } circle [radius=1.5pt];
            \fill[color=gray]  (N) node[anchor=south] { $N$ } circle [radius=1.5pt];

            \draw[dashed] ($(K) + (-1.414213, -1.414213)$) rectangle ($(K) + (0, 1.414213)$);

        \end{tikzpicture}
    \end{figure}

\end{frame}

\begin{frame}[fragile]{Visualização de identificação do par de pontos mais próximo}

    \begin{figure}
        \centering

        \begin{tikzpicture}
            \node[anchor=west] at (0, 7) { $\dist(K, G) = \mathbf{1} < \dist(B, F) = {1.414213}$ };

            \coordinate (A) at (2, 4);
            \coordinate (B) at (5, 3);
            \coordinate (C) at (8, 1);
            \coordinate (D) at (3, 6);
            \coordinate (E) at (1, 1);
            \coordinate (F) at (4, 4);
            \coordinate (G) at (8, 5);
            \coordinate (H) at (6, 2);
            \coordinate (I) at (1, 6);
            \coordinate (J) at (7, 4);
            \coordinate (K) at (8, 6);
            \coordinate (L) at (3, 2);
            \coordinate (M) at (6, 5);
            \coordinate (N) at (-1, 4);

            \fill[color=gray]  (A) node[anchor=east] { $A$ } circle [radius=1.5pt];
            \fill[color=gray]  (B) node[anchor=west] { $B$ } circle [radius=1.5pt];
            \fill[color=gray]  (C) node[anchor=west] { $C$ } circle [radius=1.5pt];
            \fill[color=gray]  (D) node[anchor=south] { $D$ } circle [radius=1.5pt];
            \fill[color=gray]  (E) node[anchor=north] { $E$ } circle [radius=1.5pt];
            \fill[color=gray]  (F) node[anchor=west] { $F$ } circle [radius=1.5pt];
            \fill[color=gray]  (G) node[anchor=west] { $G$ } circle [radius=1.5pt];
            \fill[color=gray]  (H) node[anchor=west] { $H$ } circle [radius=1.5pt];
            \fill[color=gray]  (I) node[anchor=east] { $I$ } circle [radius=1.5pt];
            \fill[color=gray]  (J) node[anchor=west] { $J$ } circle [radius=1.5pt];
            \fill (K) node[anchor=west] { $K$ } circle [radius=1.5pt];
            \fill[color=gray]  (L) node[anchor=east] { $L$ } circle [radius=1.5pt];
            \fill[color=gray]  (M) node[anchor=west] { $M$ } circle [radius=1.5pt];
            \fill[color=gray]  (N) node[anchor=south] { $N$ } circle [radius=1.5pt];

            \draw[thick] (G) -- (K);
            \draw[dashed] ($(K) + (-1.414213, -1.414213)$) rectangle ($(K) + (0, 1.414213)$);

        \end{tikzpicture}
    \end{figure}

\end{frame}


\begin{frame}[fragile]{Implementação da identifacação do par mais próximo}
    \inputsnippet{cpp}{1}{19}{codes/closest.cpp}
\end{frame}

\begin{frame}[fragile]{Implementação da identifacação do par mais próximo}
    \inputsnippet{cpp}{21}{38}{codes/closest.cpp}
\end{frame}

\begin{frame}[fragile]{Implementação da identifacação do par mais próximo}
    \inputsnippet{cpp}{40}{60}{codes/closest.cpp}
\end{frame}
