\begin{frame}[fragile]{Exemplo de envoltório convexo}

    \begin{figure}
        \centering

        \begin{tikzpicture}
            \coordinate (A) at (0, 0);
            \coordinate (B) at (5, 3);
            \coordinate (C) at (8, -2);
            \coordinate (D) at (4, 4);
            \coordinate (E) at (2, 1);
            \coordinate (F) at (2, 5);
            \coordinate (G) at (3, -1);
            \coordinate (H) at (7, 2);
            \coordinate (I) at (5, 0);
            \coordinate (J) at (0, 4);
            \coordinate (K) at (1, -1);
            \coordinate (L) at (7, -2);
            \coordinate (M) at (6, 4);
            \coordinate (N) at (6, 0);
            \coordinate (O) at (1, 3);

            \fill (A) circle [radius=2pt];
            \fill (B) circle [radius=2pt];
            \fill (C) circle [radius=2pt];
            \fill (D) circle [radius=2pt];
            \fill (E) circle [radius=2pt];
            \fill (F) circle [radius=2pt];
            \fill (G) circle [radius=2pt];
            \fill (H) circle [radius=2pt];
            \fill (I) circle [radius=2pt];
            \fill (J) circle [radius=2pt];
            \fill (K) circle [radius=2pt];
            \fill (L) circle [radius=2pt];
            \fill (M) circle [radius=2pt];
            \fill (N) circle [radius=2pt];
            \fill (O) circle [radius=2pt];

            \draw (A) -- (J) -- (F) -- (M) -- (H) -- (C) -- (L) -- (K) -- (A);
        \end{tikzpicture}
    \end{figure}

\end{frame}
