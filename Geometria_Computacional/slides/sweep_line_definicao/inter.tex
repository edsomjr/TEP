\begin{frame}[fragile]{Visualização do algoritmo de  interseção de intervalos}

    \begin{figure}
        \begin{tikzpicture}
            \node[anchor=west] at (0, 7) { Intervalos: $(2, 5), (1, 6), (5, 8), (3, 7), (4, 5), (1, 4), (4, 10), (3, 4), (7, 9)$ };

            \coordinate (A1) at (2, 5);
            \coordinate (B1) at (5, 5);
            \coordinate (A2) at (1, 4.5);
            \coordinate (B2) at (6, 4.5);
            \coordinate (A3) at (5, 4);
            \coordinate (B3) at (8, 4);
            \coordinate (A4) at (3, 3.5);
            \coordinate (B4) at (7, 3.5);
            \coordinate (A5) at (4, 3);
            \coordinate (B5) at (5, 3);
            \coordinate (A6) at (1, 2.5);
            \coordinate (B6) at (4, 2.5);
            \coordinate (A7) at (4, 2);
            \coordinate (B7) at (10, 2);
            \coordinate (A8) at (3, 1.5);
            \coordinate (B8) at (4, 1.5);
            \coordinate (A9) at (7, 1);
            \coordinate (B9) at (9, 1);


            \draw[opacity=0] (0, 0);
            \draw (A1) -- (B1);
            \draw (A2) -- (B2);
            \draw (A3) -- (B3);
            \draw (A4) -- (B4);
            \draw (A5) -- (B5);
            \draw (A6) -- (B6);
            \draw (A7) -- (B7);
            \draw (A8) -- (B8);
            \draw (A9) -- (B9);

            \fill (A1) circle [radius=1.5pt];
            \fill (B1) circle [radius=1.5pt];
            \fill (A2) circle [radius=1.5pt];
            \fill (B2) circle [radius=1.5pt];
            \fill (A3) circle [radius=1.5pt];
            \fill (B3) circle [radius=1.5pt];
            \fill (A4) circle [radius=1.5pt];
            \fill (B4) circle [radius=1.5pt];
            \fill (A5) circle [radius=1.5pt];
            \fill (B5) circle [radius=1.5pt];
            \fill (A6) circle [radius=1.5pt];
            \fill (B6) circle [radius=1.5pt];
            \fill (A7) circle [radius=1.5pt];
            \fill (B7) circle [radius=1.5pt];
            \fill (A8) circle [radius=1.5pt];
            \fill (B8) circle [radius=1.5pt];
            \fill (A9) circle [radius=1.5pt];
            \fill (B9) circle [radius=1.5pt];

        \end{tikzpicture}
    \end{figure}

\end{frame}

\begin{frame}[fragile]{Visualização do algoritmo de  interseção de intervalos}

    \begin{figure}
        \begin{tikzpicture}
            \node[anchor=west] at (0, 7) { Intervalos: $(2, 5), (1, 6), (5, 8), (3, 7), (4, 5), (1, 4), (4, 10), (3, 4), (7, 9)$ };

            \coordinate (A1) at (2, 5);
            \coordinate (B1) at (5, 5);
            \coordinate (A2) at (1, 4.5);
            \coordinate (B2) at (6, 4.5);
            \coordinate (A3) at (5, 4);
            \coordinate (B3) at (8, 4);
            \coordinate (A4) at (3, 3.5);
            \coordinate (B4) at (7, 3.5);
            \coordinate (A5) at (4, 3);
            \coordinate (B5) at (5, 3);
            \coordinate (A6) at (1, 2.5);
            \coordinate (B6) at (4, 2.5);
            \coordinate (A7) at (4, 2);
            \coordinate (B7) at (10, 2);
            \coordinate (A8) at (3, 1.5);
            \coordinate (B8) at (4, 1.5);
            \coordinate (A9) at (7, 1);
            \coordinate (B9) at (9, 1);

            \draw[opacity=0] (0, 0);
            \draw (A1) -- (B1);
            \draw (A2) -- (B2);
            \draw (A3) -- (B3);
            \draw (A4) -- (B4);
            \draw (A5) -- (B5);
            \draw (A6) -- (B6);
            \draw (A7) -- (B7);
            \draw (A8) -- (B8);
            \draw (A9) -- (B9);

            \fill (A1) circle [radius=1.5pt];
            \fill (B1) circle [radius=1.5pt];
            \fill (A2) circle [radius=1.5pt];
            \fill (B2) circle [radius=1.5pt];
            \fill (A3) circle [radius=1.5pt];
            \fill (B3) circle [radius=1.5pt];
            \fill (A4) circle [radius=1.5pt];
            \fill (B4) circle [radius=1.5pt];
            \fill (A5) circle [radius=1.5pt];
            \fill (B5) circle [radius=1.5pt];
            \fill (A6) circle [radius=1.5pt];
            \fill (B6) circle [radius=1.5pt];
            \fill (A7) circle [radius=1.5pt];
            \fill (B7) circle [radius=1.5pt];
            \fill (A8) circle [radius=1.5pt];
            \fill (B8) circle [radius=1.5pt];
            \fill (A9) circle [radius=1.5pt];
            \fill (B9) circle [radius=1.5pt];

            \draw[dashed] (1, 0.5) node[anchor=east] { \tt \footnotesize \textcolor{blue}{+2} } 
                node[anchor=west] {\tt \footnotesize \textbf{2} } -- (1,5.5);
        \end{tikzpicture}
    \end{figure}

\end{frame}

\begin{frame}[fragile]{Visualização do algoritmo de  interseção de intervalos}

    \begin{figure}
        \begin{tikzpicture}
            \node[anchor=west] at (0, 7) { Intervalos: $(2, 5), (1, 6), (5, 8), (3, 7), (4, 5), (1, 4), (4, 10), (3, 4), (7, 9)$ };

            \coordinate (A1) at (2, 5);
            \coordinate (B1) at (5, 5);
            \coordinate (A2) at (1, 4.5);
            \coordinate (B2) at (6, 4.5);
            \coordinate (A3) at (5, 4);
            \coordinate (B3) at (8, 4);
            \coordinate (A4) at (3, 3.5);
            \coordinate (B4) at (7, 3.5);
            \coordinate (A5) at (4, 3);
            \coordinate (B5) at (5, 3);
            \coordinate (A6) at (1, 2.5);
            \coordinate (B6) at (4, 2.5);
            \coordinate (A7) at (4, 2);
            \coordinate (B7) at (10, 2);
            \coordinate (A8) at (3, 1.5);
            \coordinate (B8) at (4, 1.5);
            \coordinate (A9) at (7, 1);
            \coordinate (B9) at (9, 1);

            \draw[opacity=0] (0, 0);
            \draw (A1) -- (B1);
            \draw (A2) -- (B2);
            \draw (A3) -- (B3);
            \draw (A4) -- (B4);
            \draw (A5) -- (B5);
            \draw (A6) -- (B6);
            \draw (A7) -- (B7);
            \draw (A8) -- (B8);
            \draw (A9) -- (B9);

            \fill (A1) circle [radius=1.5pt];
            \fill (B1) circle [radius=1.5pt];
            \fill (A2) circle [radius=1.5pt];
            \fill (B2) circle [radius=1.5pt];
            \fill (A3) circle [radius=1.5pt];
            \fill (B3) circle [radius=1.5pt];
            \fill (A4) circle [radius=1.5pt];
            \fill (B4) circle [radius=1.5pt];
            \fill (A5) circle [radius=1.5pt];
            \fill (B5) circle [radius=1.5pt];
            \fill (A6) circle [radius=1.5pt];
            \fill (B6) circle [radius=1.5pt];
            \fill (A7) circle [radius=1.5pt];
            \fill (B7) circle [radius=1.5pt];
            \fill (A8) circle [radius=1.5pt];
            \fill (B8) circle [radius=1.5pt];
            \fill (A9) circle [radius=1.5pt];
            \fill (B9) circle [radius=1.5pt];

            \draw[dashed] (2, 0.5) node[anchor=east] { \tt \footnotesize \textcolor{blue}{+1} } 
                node[anchor=west] {\tt \footnotesize \textbf{3} } -- (2,5.5);
        \end{tikzpicture}
    \end{figure}

\end{frame}

\begin{frame}[fragile]{Visualização do algoritmo de  interseção de intervalos}

    \begin{figure}
        \begin{tikzpicture}
            \node[anchor=west] at (0, 7) { Intervalos: $(2, 5), (1, 6), (5, 8), (3, 7), (4, 5), (1, 4), (4, 10), (3, 4), (7, 9)$ };

            \coordinate (A1) at (2, 5);
            \coordinate (B1) at (5, 5);
            \coordinate (A2) at (1, 4.5);
            \coordinate (B2) at (6, 4.5);
            \coordinate (A3) at (5, 4);
            \coordinate (B3) at (8, 4);
            \coordinate (A4) at (3, 3.5);
            \coordinate (B4) at (7, 3.5);
            \coordinate (A5) at (4, 3);
            \coordinate (B5) at (5, 3);
            \coordinate (A6) at (1, 2.5);
            \coordinate (B6) at (4, 2.5);
            \coordinate (A7) at (4, 2);
            \coordinate (B7) at (10, 2);
            \coordinate (A8) at (3, 1.5);
            \coordinate (B8) at (4, 1.5);
            \coordinate (A9) at (7, 1);
            \coordinate (B9) at (9, 1);

            \draw[opacity=0] (0, 0);
            \draw (A1) -- (B1);
            \draw (A2) -- (B2);
            \draw (A3) -- (B3);
            \draw (A4) -- (B4);
            \draw (A5) -- (B5);
            \draw (A6) -- (B6);
            \draw (A7) -- (B7);
            \draw (A8) -- (B8);
            \draw (A9) -- (B9);

            \fill (A1) circle [radius=1.5pt];
            \fill (B1) circle [radius=1.5pt];
            \fill (A2) circle [radius=1.5pt];
            \fill (B2) circle [radius=1.5pt];
            \fill (A3) circle [radius=1.5pt];
            \fill (B3) circle [radius=1.5pt];
            \fill (A4) circle [radius=1.5pt];
            \fill (B4) circle [radius=1.5pt];
            \fill (A5) circle [radius=1.5pt];
            \fill (B5) circle [radius=1.5pt];
            \fill (A6) circle [radius=1.5pt];
            \fill (B6) circle [radius=1.5pt];
            \fill (A7) circle [radius=1.5pt];
            \fill (B7) circle [radius=1.5pt];
            \fill (A8) circle [radius=1.5pt];
            \fill (B8) circle [radius=1.5pt];
            \fill (A9) circle [radius=1.5pt];
            \fill (B9) circle [radius=1.5pt];

            \draw[dashed] (3, 0.5) node[anchor=east] { \tt \footnotesize \textcolor{blue}{+2} } 
                node[anchor=west] {\tt \footnotesize \textbf{5} } -- (3,5.5);
        \end{tikzpicture}
    \end{figure}

\end{frame}

\begin{frame}[fragile]{Visualização do algoritmo de  interseção de intervalos}

    \begin{figure}
        \begin{tikzpicture}
            \node[anchor=west] at (0, 7) { Intervalos: $(2, 5), (1, 6), (5, 8), (3, 7), (4, 5), (1, 4), (4, 10), (3, 4), (7, 9)$ };

            \coordinate (A1) at (2, 5);
            \coordinate (B1) at (5, 5);
            \coordinate (A2) at (1, 4.5);
            \coordinate (B2) at (6, 4.5);
            \coordinate (A3) at (5, 4);
            \coordinate (B3) at (8, 4);
            \coordinate (A4) at (3, 3.5);
            \coordinate (B4) at (7, 3.5);
            \coordinate (A5) at (4, 3);
            \coordinate (B5) at (5, 3);
            \coordinate (A6) at (1, 2.5);
            \coordinate (B6) at (4, 2.5);
            \coordinate (A7) at (4, 2);
            \coordinate (B7) at (10, 2);
            \coordinate (A8) at (3, 1.5);
            \coordinate (B8) at (4, 1.5);
            \coordinate (A9) at (7, 1);
            \coordinate (B9) at (9, 1);

            \draw[opacity=0] (0, 0);
            \draw (A1) -- (B1);
            \draw (A2) -- (B2);
            \draw (A3) -- (B3);
            \draw (A4) -- (B4);
            \draw (A5) -- (B5);
            \draw (A6) -- (B6);
            \draw (A7) -- (B7);
            \draw (A8) -- (B8);
            \draw (A9) -- (B9);

            \fill (A1) circle [radius=1.5pt];
            \fill (B1) circle [radius=1.5pt];
            \fill (A2) circle [radius=1.5pt];
            \fill (B2) circle [radius=1.5pt];
            \fill (A3) circle [radius=1.5pt];
            \fill (B3) circle [radius=1.5pt];
            \fill (A4) circle [radius=1.5pt];
            \fill (B4) circle [radius=1.5pt];
            \fill (A5) circle [radius=1.5pt];
            \fill (B5) circle [radius=1.5pt];
            \fill (A6) circle [radius=1.5pt];
            \fill (B6) circle [radius=1.5pt];
            \fill (A7) circle [radius=1.5pt];
            \fill (B7) circle [radius=1.5pt];
            \fill (A8) circle [radius=1.5pt];
            \fill (B8) circle [radius=1.5pt];
            \fill (A9) circle [radius=1.5pt];
            \fill (B9) circle [radius=1.5pt];

            \draw[dashed] (4, 0.5) node[anchor=east] { \tt \footnotesize \textcolor{red}{-2} } 
                node[anchor=west] {\tt \footnotesize \textbf{3} } -- (4,5.5);
        \end{tikzpicture}
    \end{figure}

\end{frame}

\begin{frame}[fragile]{Visualização do algoritmo de  interseção de intervalos}

    \begin{figure}
        \begin{tikzpicture}
            \node[anchor=west] at (0, 7) { Intervalos: $(2, 5), (1, 6), (5, 8), (3, 7), (4, 5), (1, 4), (4, 10), (3, 4), (7, 9)$ };

            \coordinate (A1) at (2, 5);
            \coordinate (B1) at (5, 5);
            \coordinate (A2) at (1, 4.5);
            \coordinate (B2) at (6, 4.5);
            \coordinate (A3) at (5, 4);
            \coordinate (B3) at (8, 4);
            \coordinate (A4) at (3, 3.5);
            \coordinate (B4) at (7, 3.5);
            \coordinate (A5) at (4, 3);
            \coordinate (B5) at (5, 3);
            \coordinate (A6) at (1, 2.5);
            \coordinate (B6) at (4, 2.5);
            \coordinate (A7) at (4, 2);
            \coordinate (B7) at (10, 2);
            \coordinate (A8) at (3, 1.5);
            \coordinate (B8) at (4, 1.5);
            \coordinate (A9) at (7, 1);
            \coordinate (B9) at (9, 1);

            \draw[opacity=0] (0, 0);
            \draw (A1) -- (B1);
            \draw (A2) -- (B2);
            \draw (A3) -- (B3);
            \draw (A4) -- (B4);
            \draw (A5) -- (B5);
            \draw (A6) -- (B6);
            \draw (A7) -- (B7);
            \draw (A8) -- (B8);
            \draw (A9) -- (B9);

            \fill (A1) circle [radius=1.5pt];
            \fill (B1) circle [radius=1.5pt];
            \fill (A2) circle [radius=1.5pt];
            \fill (B2) circle [radius=1.5pt];
            \fill (A3) circle [radius=1.5pt];
            \fill (B3) circle [radius=1.5pt];
            \fill (A4) circle [radius=1.5pt];
            \fill (B4) circle [radius=1.5pt];
            \fill (A5) circle [radius=1.5pt];
            \fill (B5) circle [radius=1.5pt];
            \fill (A6) circle [radius=1.5pt];
            \fill (B6) circle [radius=1.5pt];
            \fill (A7) circle [radius=1.5pt];
            \fill (B7) circle [radius=1.5pt];
            \fill (A8) circle [radius=1.5pt];
            \fill (B8) circle [radius=1.5pt];
            \fill (A9) circle [radius=1.5pt];
            \fill (B9) circle [radius=1.5pt];

            \draw[dashed] (4, 0.5) node[anchor=east] { \tt \footnotesize \textcolor{blue}{+2} } 
                node[anchor=west] {\tt \footnotesize \textbf{5} } -- (4,5.5);
        \end{tikzpicture}
    \end{figure}

\end{frame}

\begin{frame}[fragile]{Visualização do algoritmo de  interseção de intervalos}

    \begin{figure}
        \begin{tikzpicture}
            \node[anchor=west] at (0, 7) { Intervalos: $(2, 5), (1, 6), (5, 8), (3, 7), (4, 5), (1, 4), (4, 10), (3, 4), (7, 9)$ };

            \coordinate (A1) at (2, 5);
            \coordinate (B1) at (5, 5);
            \coordinate (A2) at (1, 4.5);
            \coordinate (B2) at (6, 4.5);
            \coordinate (A3) at (5, 4);
            \coordinate (B3) at (8, 4);
            \coordinate (A4) at (3, 3.5);
            \coordinate (B4) at (7, 3.5);
            \coordinate (A5) at (4, 3);
            \coordinate (B5) at (5, 3);
            \coordinate (A6) at (1, 2.5);
            \coordinate (B6) at (4, 2.5);
            \coordinate (A7) at (4, 2);
            \coordinate (B7) at (10, 2);
            \coordinate (A8) at (3, 1.5);
            \coordinate (B8) at (4, 1.5);
            \coordinate (A9) at (7, 1);
            \coordinate (B9) at (9, 1);

            \draw[opacity=0] (0, 0);
            \draw (A1) -- (B1);
            \draw (A2) -- (B2);
            \draw (A3) -- (B3);
            \draw (A4) -- (B4);
            \draw (A5) -- (B5);
            \draw (A6) -- (B6);
            \draw (A7) -- (B7);
            \draw (A8) -- (B8);
            \draw (A9) -- (B9);

            \fill (A1) circle [radius=1.5pt];
            \fill (B1) circle [radius=1.5pt];
            \fill (A2) circle [radius=1.5pt];
            \fill (B2) circle [radius=1.5pt];
            \fill (A3) circle [radius=1.5pt];
            \fill (B3) circle [radius=1.5pt];
            \fill (A4) circle [radius=1.5pt];
            \fill (B4) circle [radius=1.5pt];
            \fill (A5) circle [radius=1.5pt];
            \fill (B5) circle [radius=1.5pt];
            \fill (A6) circle [radius=1.5pt];
            \fill (B6) circle [radius=1.5pt];
            \fill (A7) circle [radius=1.5pt];
            \fill (B7) circle [radius=1.5pt];
            \fill (A8) circle [radius=1.5pt];
            \fill (B8) circle [radius=1.5pt];
            \fill (A9) circle [radius=1.5pt];
            \fill (B9) circle [radius=1.5pt];

            \draw[dashed] (5, 0.5) node[anchor=east] { \tt \footnotesize \textcolor{red}{-2} } 
                node[anchor=west] {\tt \footnotesize \textbf{3} } -- (5,5.5);
        \end{tikzpicture}
    \end{figure}

\end{frame}

\begin{frame}[fragile]{Visualização do algoritmo de  interseção de intervalos}

    \begin{figure}
        \begin{tikzpicture}
            \node[anchor=west] at (0, 7) { Intervalos: $(2, 5), (1, 6), (5, 8), (3, 7), (4, 5), (1, 4), (4, 10), (3, 4), (7, 9)$ };

            \coordinate (A1) at (2, 5);
            \coordinate (B1) at (5, 5);
            \coordinate (A2) at (1, 4.5);
            \coordinate (B2) at (6, 4.5);
            \coordinate (A3) at (5, 4);
            \coordinate (B3) at (8, 4);
            \coordinate (A4) at (3, 3.5);
            \coordinate (B4) at (7, 3.5);
            \coordinate (A5) at (4, 3);
            \coordinate (B5) at (5, 3);
            \coordinate (A6) at (1, 2.5);
            \coordinate (B6) at (4, 2.5);
            \coordinate (A7) at (4, 2);
            \coordinate (B7) at (10, 2);
            \coordinate (A8) at (3, 1.5);
            \coordinate (B8) at (4, 1.5);
            \coordinate (A9) at (7, 1);
            \coordinate (B9) at (9, 1);

            \draw[opacity=0] (0, 0);
            \draw (A1) -- (B1);
            \draw (A2) -- (B2);
            \draw (A3) -- (B3);
            \draw (A4) -- (B4);
            \draw (A5) -- (B5);
            \draw (A6) -- (B6);
            \draw (A7) -- (B7);
            \draw (A8) -- (B8);
            \draw (A9) -- (B9);

            \fill (A1) circle [radius=1.5pt];
            \fill (B1) circle [radius=1.5pt];
            \fill (A2) circle [radius=1.5pt];
            \fill (B2) circle [radius=1.5pt];
            \fill (A3) circle [radius=1.5pt];
            \fill (B3) circle [radius=1.5pt];
            \fill (A4) circle [radius=1.5pt];
            \fill (B4) circle [radius=1.5pt];
            \fill (A5) circle [radius=1.5pt];
            \fill (B5) circle [radius=1.5pt];
            \fill (A6) circle [radius=1.5pt];
            \fill (B6) circle [radius=1.5pt];
            \fill (A7) circle [radius=1.5pt];
            \fill (B7) circle [radius=1.5pt];
            \fill (A8) circle [radius=1.5pt];
            \fill (B8) circle [radius=1.5pt];
            \fill (A9) circle [radius=1.5pt];
            \fill (B9) circle [radius=1.5pt];

            \draw[dashed] (5, 0.5) node[anchor=east] { \tt \footnotesize \textcolor{blue}{+1} } 
                node[anchor=west] {\tt \footnotesize \textbf{4} } -- (5,5.5);
        \end{tikzpicture}
    \end{figure}

\end{frame}

\begin{frame}[fragile]{Visualização do algoritmo de  interseção de intervalos}

    \begin{figure}
        \begin{tikzpicture}
            \node[anchor=west] at (0, 7) { Intervalos: $(2, 5), (1, 6), (5, 8), (3, 7), (4, 5), (1, 4), (4, 10), (3, 4), (7, 9)$ };

            \coordinate (A1) at (2, 5);
            \coordinate (B1) at (5, 5);
            \coordinate (A2) at (1, 4.5);
            \coordinate (B2) at (6, 4.5);
            \coordinate (A3) at (5, 4);
            \coordinate (B3) at (8, 4);
            \coordinate (A4) at (3, 3.5);
            \coordinate (B4) at (7, 3.5);
            \coordinate (A5) at (4, 3);
            \coordinate (B5) at (5, 3);
            \coordinate (A6) at (1, 2.5);
            \coordinate (B6) at (4, 2.5);
            \coordinate (A7) at (4, 2);
            \coordinate (B7) at (10, 2);
            \coordinate (A8) at (3, 1.5);
            \coordinate (B8) at (4, 1.5);
            \coordinate (A9) at (7, 1);
            \coordinate (B9) at (9, 1);

            \draw[opacity=0] (0, 0);
            \draw (A1) -- (B1);
            \draw (A2) -- (B2);
            \draw (A3) -- (B3);
            \draw (A4) -- (B4);
            \draw (A5) -- (B5);
            \draw (A6) -- (B6);
            \draw (A7) -- (B7);
            \draw (A8) -- (B8);
            \draw (A9) -- (B9);

            \fill (A1) circle [radius=1.5pt];
            \fill (B1) circle [radius=1.5pt];
            \fill (A2) circle [radius=1.5pt];
            \fill (B2) circle [radius=1.5pt];
            \fill (A3) circle [radius=1.5pt];
            \fill (B3) circle [radius=1.5pt];
            \fill (A4) circle [radius=1.5pt];
            \fill (B4) circle [radius=1.5pt];
            \fill (A5) circle [radius=1.5pt];
            \fill (B5) circle [radius=1.5pt];
            \fill (A6) circle [radius=1.5pt];
            \fill (B6) circle [radius=1.5pt];
            \fill (A7) circle [radius=1.5pt];
            \fill (B7) circle [radius=1.5pt];
            \fill (A8) circle [radius=1.5pt];
            \fill (B8) circle [radius=1.5pt];
            \fill (A9) circle [radius=1.5pt];
            \fill (B9) circle [radius=1.5pt];

            \draw[dashed] (6, 0.5) node[anchor=east] { \tt \footnotesize \textcolor{red}{-1} } 
                node[anchor=west] {\tt \footnotesize \textbf{3} } -- (6, 5.5);
        \end{tikzpicture}
    \end{figure}

\end{frame}

\begin{frame}[fragile]{Visualização do algoritmo de  interseção de intervalos}

    \begin{figure}
        \begin{tikzpicture}
            \node[anchor=west] at (0, 7) { Intervalos: $(2, 5), (1, 6), (5, 8), (3, 7), (4, 5), (1, 4), (4, 10), (3, 4), (7, 9)$ };

            \coordinate (A1) at (2, 5);
            \coordinate (B1) at (5, 5);
            \coordinate (A2) at (1, 4.5);
            \coordinate (B2) at (6, 4.5);
            \coordinate (A3) at (5, 4);
            \coordinate (B3) at (8, 4);
            \coordinate (A4) at (3, 3.5);
            \coordinate (B4) at (7, 3.5);
            \coordinate (A5) at (4, 3);
            \coordinate (B5) at (5, 3);
            \coordinate (A6) at (1, 2.5);
            \coordinate (B6) at (4, 2.5);
            \coordinate (A7) at (4, 2);
            \coordinate (B7) at (10, 2);
            \coordinate (A8) at (3, 1.5);
            \coordinate (B8) at (4, 1.5);
            \coordinate (A9) at (7, 1);
            \coordinate (B9) at (9, 1);

            \draw[opacity=0] (0, 0);
            \draw (A1) -- (B1);
            \draw (A2) -- (B2);
            \draw (A3) -- (B3);
            \draw (A4) -- (B4);
            \draw (A5) -- (B5);
            \draw (A6) -- (B6);
            \draw (A7) -- (B7);
            \draw (A8) -- (B8);
            \draw (A9) -- (B9);

            \fill (A1) circle [radius=1.5pt];
            \fill (B1) circle [radius=1.5pt];
            \fill (A2) circle [radius=1.5pt];
            \fill (B2) circle [radius=1.5pt];
            \fill (A3) circle [radius=1.5pt];
            \fill (B3) circle [radius=1.5pt];
            \fill (A4) circle [radius=1.5pt];
            \fill (B4) circle [radius=1.5pt];
            \fill (A5) circle [radius=1.5pt];
            \fill (B5) circle [radius=1.5pt];
            \fill (A6) circle [radius=1.5pt];
            \fill (B6) circle [radius=1.5pt];
            \fill (A7) circle [radius=1.5pt];
            \fill (B7) circle [radius=1.5pt];
            \fill (A8) circle [radius=1.5pt];
            \fill (B8) circle [radius=1.5pt];
            \fill (A9) circle [radius=1.5pt];
            \fill (B9) circle [radius=1.5pt];

            \draw[dashed] (7, 0.5) node[anchor=east] { \tt \footnotesize \textcolor{red}{-1} } 
                node[anchor=west] {\tt \footnotesize \textbf{2} } -- (7, 5.5);
        \end{tikzpicture}
    \end{figure}

\end{frame}

\begin{frame}[fragile]{Visualização do algoritmo de  interseção de intervalos}

    \begin{figure}
        \begin{tikzpicture}
            \node[anchor=west] at (0, 7) { Intervalos: $(2, 5), (1, 6), (5, 8), (3, 7), (4, 5), (1, 4), (4, 10), (3, 4), (7, 9)$ };

            \coordinate (A1) at (2, 5);
            \coordinate (B1) at (5, 5);
            \coordinate (A2) at (1, 4.5);
            \coordinate (B2) at (6, 4.5);
            \coordinate (A3) at (5, 4);
            \coordinate (B3) at (8, 4);
            \coordinate (A4) at (3, 3.5);
            \coordinate (B4) at (7, 3.5);
            \coordinate (A5) at (4, 3);
            \coordinate (B5) at (5, 3);
            \coordinate (A6) at (1, 2.5);
            \coordinate (B6) at (4, 2.5);
            \coordinate (A7) at (4, 2);
            \coordinate (B7) at (10, 2);
            \coordinate (A8) at (3, 1.5);
            \coordinate (B8) at (4, 1.5);
            \coordinate (A9) at (7, 1);
            \coordinate (B9) at (9, 1);

            \draw[opacity=0] (0, 0);
            \draw (A1) -- (B1);
            \draw (A2) -- (B2);
            \draw (A3) -- (B3);
            \draw (A4) -- (B4);
            \draw (A5) -- (B5);
            \draw (A6) -- (B6);
            \draw (A7) -- (B7);
            \draw (A8) -- (B8);
            \draw (A9) -- (B9);

            \fill (A1) circle [radius=1.5pt];
            \fill (B1) circle [radius=1.5pt];
            \fill (A2) circle [radius=1.5pt];
            \fill (B2) circle [radius=1.5pt];
            \fill (A3) circle [radius=1.5pt];
            \fill (B3) circle [radius=1.5pt];
            \fill (A4) circle [radius=1.5pt];
            \fill (B4) circle [radius=1.5pt];
            \fill (A5) circle [radius=1.5pt];
            \fill (B5) circle [radius=1.5pt];
            \fill (A6) circle [radius=1.5pt];
            \fill (B6) circle [radius=1.5pt];
            \fill (A7) circle [radius=1.5pt];
            \fill (B7) circle [radius=1.5pt];
            \fill (A8) circle [radius=1.5pt];
            \fill (B8) circle [radius=1.5pt];
            \fill (A9) circle [radius=1.5pt];
            \fill (B9) circle [radius=1.5pt];

            \draw[dashed] (7, 0.5) node[anchor=east] { \tt \footnotesize \textcolor{blue}{+1} } 
                node[anchor=west] {\tt \footnotesize \textbf{3} } -- (7, 5.5);
        \end{tikzpicture}
    \end{figure}

\end{frame}

\begin{frame}[fragile]{Visualização do algoritmo de  interseção de intervalos}

    \begin{figure}
        \begin{tikzpicture}
            \node[anchor=west] at (0, 7) { Intervalos: $(2, 5), (1, 6), (5, 8), (3, 7), (4, 5), (1, 4), (4, 10), (3, 4), (7, 9)$ };

            \coordinate (A1) at (2, 5);
            \coordinate (B1) at (5, 5);
            \coordinate (A2) at (1, 4.5);
            \coordinate (B2) at (6, 4.5);
            \coordinate (A3) at (5, 4);
            \coordinate (B3) at (8, 4);
            \coordinate (A4) at (3, 3.5);
            \coordinate (B4) at (7, 3.5);
            \coordinate (A5) at (4, 3);
            \coordinate (B5) at (5, 3);
            \coordinate (A6) at (1, 2.5);
            \coordinate (B6) at (4, 2.5);
            \coordinate (A7) at (4, 2);
            \coordinate (B7) at (10, 2);
            \coordinate (A8) at (3, 1.5);
            \coordinate (B8) at (4, 1.5);
            \coordinate (A9) at (7, 1);
            \coordinate (B9) at (9, 1);

            \draw[opacity=0] (0, 0);
            \draw (A1) -- (B1);
            \draw (A2) -- (B2);
            \draw (A3) -- (B3);
            \draw (A4) -- (B4);
            \draw (A5) -- (B5);
            \draw (A6) -- (B6);
            \draw (A7) -- (B7);
            \draw (A8) -- (B8);
            \draw (A9) -- (B9);

            \fill (A1) circle [radius=1.5pt];
            \fill (B1) circle [radius=1.5pt];
            \fill (A2) circle [radius=1.5pt];
            \fill (B2) circle [radius=1.5pt];
            \fill (A3) circle [radius=1.5pt];
            \fill (B3) circle [radius=1.5pt];
            \fill (A4) circle [radius=1.5pt];
            \fill (B4) circle [radius=1.5pt];
            \fill (A5) circle [radius=1.5pt];
            \fill (B5) circle [radius=1.5pt];
            \fill (A6) circle [radius=1.5pt];
            \fill (B6) circle [radius=1.5pt];
            \fill (A7) circle [radius=1.5pt];
            \fill (B7) circle [radius=1.5pt];
            \fill (A8) circle [radius=1.5pt];
            \fill (B8) circle [radius=1.5pt];
            \fill (A9) circle [radius=1.5pt];
            \fill (B9) circle [radius=1.5pt];

            \draw[dashed] (8, 0.5) node[anchor=east] { \tt \footnotesize \textcolor{red}{-1} } 
                node[anchor=west] {\tt \footnotesize \textbf{2} } -- (8, 5.5);
        \end{tikzpicture}
    \end{figure}

\end{frame}

\begin{frame}[fragile]{Visualização do algoritmo de  interseção de intervalos}

    \begin{figure}
        \begin{tikzpicture}
            \node[anchor=west] at (0, 7) { Intervalos: $(2, 5), (1, 6), (5, 8), (3, 7), (4, 5), (1, 4), (4, 10), (3, 4), (7, 9)$ };

            \coordinate (A1) at (2, 5);
            \coordinate (B1) at (5, 5);
            \coordinate (A2) at (1, 4.5);
            \coordinate (B2) at (6, 4.5);
            \coordinate (A3) at (5, 4);
            \coordinate (B3) at (8, 4);
            \coordinate (A4) at (3, 3.5);
            \coordinate (B4) at (7, 3.5);
            \coordinate (A5) at (4, 3);
            \coordinate (B5) at (5, 3);
            \coordinate (A6) at (1, 2.5);
            \coordinate (B6) at (4, 2.5);
            \coordinate (A7) at (4, 2);
            \coordinate (B7) at (10, 2);
            \coordinate (A8) at (3, 1.5);
            \coordinate (B8) at (4, 1.5);
            \coordinate (A9) at (7, 1);
            \coordinate (B9) at (9, 1);

            \draw[opacity=0] (0, 0);
            \draw (A1) -- (B1);
            \draw (A2) -- (B2);
            \draw (A3) -- (B3);
            \draw (A4) -- (B4);
            \draw (A5) -- (B5);
            \draw (A6) -- (B6);
            \draw (A7) -- (B7);
            \draw (A8) -- (B8);
            \draw (A9) -- (B9);

            \fill (A1) circle [radius=1.5pt];
            \fill (B1) circle [radius=1.5pt];
            \fill (A2) circle [radius=1.5pt];
            \fill (B2) circle [radius=1.5pt];
            \fill (A3) circle [radius=1.5pt];
            \fill (B3) circle [radius=1.5pt];
            \fill (A4) circle [radius=1.5pt];
            \fill (B4) circle [radius=1.5pt];
            \fill (A5) circle [radius=1.5pt];
            \fill (B5) circle [radius=1.5pt];
            \fill (A6) circle [radius=1.5pt];
            \fill (B6) circle [radius=1.5pt];
            \fill (A7) circle [radius=1.5pt];
            \fill (B7) circle [radius=1.5pt];
            \fill (A8) circle [radius=1.5pt];
            \fill (B8) circle [radius=1.5pt];
            \fill (A9) circle [radius=1.5pt];
            \fill (B9) circle [radius=1.5pt];

            \draw[dashed] (9, 0.5) node[anchor=east] { \tt \footnotesize \textcolor{red}{-1} } 
                node[anchor=west] {\tt \footnotesize \textbf{1} } -- (9, 5.5);
        \end{tikzpicture}
    \end{figure}

\end{frame}

\begin{frame}[fragile]{Visualização do algoritmo de  interseção de intervalos}

    \begin{figure}
        \begin{tikzpicture}
            \node[anchor=west] at (0, 7) { Intervalos: $(2, 5), (1, 6), (5, 8), (3, 7), (4, 5), (1, 4), (4, 10), (3, 4), (7, 9)$ };

            \coordinate (A1) at (2, 5);
            \coordinate (B1) at (5, 5);
            \coordinate (A2) at (1, 4.5);
            \coordinate (B2) at (6, 4.5);
            \coordinate (A3) at (5, 4);
            \coordinate (B3) at (8, 4);
            \coordinate (A4) at (3, 3.5);
            \coordinate (B4) at (7, 3.5);
            \coordinate (A5) at (4, 3);
            \coordinate (B5) at (5, 3);
            \coordinate (A6) at (1, 2.5);
            \coordinate (B6) at (4, 2.5);
            \coordinate (A7) at (4, 2);
            \coordinate (B7) at (10, 2);
            \coordinate (A8) at (3, 1.5);
            \coordinate (B8) at (4, 1.5);
            \coordinate (A9) at (7, 1);
            \coordinate (B9) at (9, 1);

            \draw[opacity=0] (0, 0);
            \draw (A1) -- (B1);
            \draw (A2) -- (B2);
            \draw (A3) -- (B3);
            \draw (A4) -- (B4);
            \draw (A5) -- (B5);
            \draw (A6) -- (B6);
            \draw (A7) -- (B7);
            \draw (A8) -- (B8);
            \draw (A9) -- (B9);

            \fill (A1) circle [radius=1.5pt];
            \fill (B1) circle [radius=1.5pt];
            \fill (A2) circle [radius=1.5pt];
            \fill (B2) circle [radius=1.5pt];
            \fill (A3) circle [radius=1.5pt];
            \fill (B3) circle [radius=1.5pt];
            \fill (A4) circle [radius=1.5pt];
            \fill (B4) circle [radius=1.5pt];
            \fill (A5) circle [radius=1.5pt];
            \fill (B5) circle [radius=1.5pt];
            \fill (A6) circle [radius=1.5pt];
            \fill (B6) circle [radius=1.5pt];
            \fill (A7) circle [radius=1.5pt];
            \fill (B7) circle [radius=1.5pt];
            \fill (A8) circle [radius=1.5pt];
            \fill (B8) circle [radius=1.5pt];
            \fill (A9) circle [radius=1.5pt];
            \fill (B9) circle [radius=1.5pt];

            \draw[dashed] (10, 0.5) node[anchor=east] { \tt \footnotesize \textcolor{red}{-1} } 
                node[anchor=west] {\tt \footnotesize \textbf{0} } -- (10, 5.5);
        \end{tikzpicture}
    \end{figure}

\end{frame}
