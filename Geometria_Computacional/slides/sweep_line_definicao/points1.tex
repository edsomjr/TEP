\begin{frame}[fragile]{Visualização do problema de pontos de interseção}

    \begin{figure}
        \centering

        \begin{tikzpicture}
            \coordinate (A1) at (2, 6);
            \coordinate (B1) at (5, 6);
            \coordinate (A2) at (1, 5);
            \coordinate (B2) at (6, 5);
            \coordinate (A3) at (5, 4);
            \coordinate (B3) at (8, 4);
            \coordinate (A4) at (3, 3);
            \coordinate (B4) at (7, 3);
            \coordinate (A5) at (5, 2);
            \coordinate (B5) at (8, 2);
            \coordinate (A6) at (1, 1);
            \coordinate (B6) at (4, 1);
            \coordinate (A7) at (4, 7);
            \coordinate (B7) at (4, 2);
            \coordinate (A8) at (2, 3);
            \coordinate (B8) at (2, 0);
            \coordinate (A9) at (6, 3.5);
            \coordinate (B9) at (6, 1.5);


            \draw[opacity=0] (0, 0);
            \draw (A1) -- (B1);
            \draw (A2) -- (B2);
            \draw (A3) -- (B3);
            \draw (A4) -- (B4);
            \draw (A5) -- (B5);
            \draw (A6) -- (B6);
            \draw (A7) -- (B7);
            \draw (A8) -- (B8);
            \draw (A9) -- (B9);

            \fill (4, 6) circle [radius=1.5pt];
            \fill (4, 5) circle [radius=1.5pt];
            \fill (4, 3) circle [radius=1.5pt];
            \fill (2, 1) circle [radius=1.5pt];
            \fill (6, 3) circle [radius=1.5pt];
            \fill (6, 2) circle [radius=1.5pt];
            
        \end{tikzpicture}
    \end{figure}

\end{frame}



