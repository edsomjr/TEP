\section{Interseção de intervalos}

\begin{frame}[fragile]{Interseção de intervalos}

    \begin{itemize}
        \item Um problema que pode ser resolvido usando o paradigma \textit{sweep line} é o de
            contabilizar, dentre um conjunto de intervalos $I_i = [a_i, b_i)$, o tamanho do maior 
            subconjunto $S$ destes intervalos tal que $I_j \cap I_k \neq \emptyset$ para todo
            par $I_j, I_k\in S$ 
        \pause

        \item Uma aplicação prática deste problema seria: cada intervalo representa o início e o
            fim de um espetáculo que acontecerá em determinado dia. Qual é o número máximo de
            espetáculos que acontecerão simultaneamete?
        \pause

        \item A solução é criar, para cada intervalo, dois eventos: um evento de ínicio do 
            espetáculo $(a_i, 1)$ e um evento de final $(b_i, 0)$
        \pause

        \item Uma vez ordenados estes eventos em ordem lexicográfica (primeiro por coordenada $x$,
            depois por coordenada $y$), basta processar todos eles, um por vez

    \end{itemize}

\end{frame}

\begin{frame}[fragile]{Interseção de intervalos}

    \begin{itemize}
        \item Um evento de início incrementa o número de eventos em andamento, o evento de fim
            decrementa
        \pause

        \item Com a representação de eventos escolhida, os intervalos $[a, b)$ e $[b, c)$ não tem
            interseção
        \pause

        \item A complexidade deste algoritmo é $O(N\log N)$, por conta da ordenação, pois
            cada ponto será processado uma única vez
        \pause

        \item A representação dos eventos pode ser modificada, de tal modo que é possível 
            identificar quais são os intervalos simultâneos
        \pause

        \item Basta fazer $(a_i, i)$ e $(b_i, -i)$ e manter os índices dos intervalos em
            exibição em um conjuntos, removendo-os a cada evento de encerramento
        \pause

        \item Veja que, nesta represenção, os intervalos devem ser numerados a partir de 1,
            pois o zero geraria ambiguidade
    \end{itemize}

\end{frame}

\begin{frame}[fragile]{Visualização do algoritmo de  interseção de intervalos}

    \begin{figure}
        \begin{tikzpicture}
            \node[anchor=west] at (0, 7) { Intervalos: $(2, 5), (1, 6), (5, 8), (3, 7), (4, 5), (1, 4), (4, 10), (3, 4), (7, 9)$ };

            \coordinate (A1) at (2, 5);
            \coordinate (B1) at (5, 5);
            \coordinate (A2) at (1, 4.5);
            \coordinate (B2) at (6, 4.5);
            \coordinate (A3) at (5, 4);
            \coordinate (B3) at (8, 4);
            \coordinate (A4) at (3, 3.5);
            \coordinate (B4) at (7, 3.5);
            \coordinate (A5) at (4, 3);
            \coordinate (B5) at (5, 3);
            \coordinate (A6) at (1, 2.5);
            \coordinate (B6) at (4, 2.5);
            \coordinate (A7) at (4, 2);
            \coordinate (B7) at (10, 2);
            \coordinate (A8) at (3, 1.5);
            \coordinate (B8) at (4, 1.5);
            \coordinate (A9) at (7, 1);
            \coordinate (B9) at (9, 1);


            \draw[opacity=0] (0, 0);
            \draw (A1) -- (B1);
            \draw (A2) -- (B2);
            \draw (A3) -- (B3);
            \draw (A4) -- (B4);
            \draw (A5) -- (B5);
            \draw (A6) -- (B6);
            \draw (A7) -- (B7);
            \draw (A8) -- (B8);
            \draw (A9) -- (B9);

            \fill (A1) circle [radius=1.5pt];
            \fill (B1) circle [radius=1.5pt];
            \fill (A2) circle [radius=1.5pt];
            \fill (B2) circle [radius=1.5pt];
            \fill (A3) circle [radius=1.5pt];
            \fill (B3) circle [radius=1.5pt];
            \fill (A4) circle [radius=1.5pt];
            \fill (B4) circle [radius=1.5pt];
            \fill (A5) circle [radius=1.5pt];
            \fill (B5) circle [radius=1.5pt];
            \fill (A6) circle [radius=1.5pt];
            \fill (B6) circle [radius=1.5pt];
            \fill (A7) circle [radius=1.5pt];
            \fill (B7) circle [radius=1.5pt];
            \fill (A8) circle [radius=1.5pt];
            \fill (B8) circle [radius=1.5pt];
            \fill (A9) circle [radius=1.5pt];
            \fill (B9) circle [radius=1.5pt];

        \end{tikzpicture}
    \end{figure}

\end{frame}

\begin{frame}[fragile]{Visualização do algoritmo de  interseção de intervalos}

    \begin{figure}
        \begin{tikzpicture}
            \node[anchor=west] at (0, 7) { Intervalos: $(2, 5), (1, 6), (5, 8), (3, 7), (4, 5), (1, 4), (4, 10), (3, 4), (7, 9)$ };

            \coordinate (A1) at (2, 5);
            \coordinate (B1) at (5, 5);
            \coordinate (A2) at (1, 4.5);
            \coordinate (B2) at (6, 4.5);
            \coordinate (A3) at (5, 4);
            \coordinate (B3) at (8, 4);
            \coordinate (A4) at (3, 3.5);
            \coordinate (B4) at (7, 3.5);
            \coordinate (A5) at (4, 3);
            \coordinate (B5) at (5, 3);
            \coordinate (A6) at (1, 2.5);
            \coordinate (B6) at (4, 2.5);
            \coordinate (A7) at (4, 2);
            \coordinate (B7) at (10, 2);
            \coordinate (A8) at (3, 1.5);
            \coordinate (B8) at (4, 1.5);
            \coordinate (A9) at (7, 1);
            \coordinate (B9) at (9, 1);

            \draw[opacity=0] (0, 0);
            \draw (A1) -- (B1);
            \draw (A2) -- (B2);
            \draw (A3) -- (B3);
            \draw (A4) -- (B4);
            \draw (A5) -- (B5);
            \draw (A6) -- (B6);
            \draw (A7) -- (B7);
            \draw (A8) -- (B8);
            \draw (A9) -- (B9);

            \fill (A1) circle [radius=1.5pt];
            \fill (B1) circle [radius=1.5pt];
            \fill (A2) circle [radius=1.5pt];
            \fill (B2) circle [radius=1.5pt];
            \fill (A3) circle [radius=1.5pt];
            \fill (B3) circle [radius=1.5pt];
            \fill (A4) circle [radius=1.5pt];
            \fill (B4) circle [radius=1.5pt];
            \fill (A5) circle [radius=1.5pt];
            \fill (B5) circle [radius=1.5pt];
            \fill (A6) circle [radius=1.5pt];
            \fill (B6) circle [radius=1.5pt];
            \fill (A7) circle [radius=1.5pt];
            \fill (B7) circle [radius=1.5pt];
            \fill (A8) circle [radius=1.5pt];
            \fill (B8) circle [radius=1.5pt];
            \fill (A9) circle [radius=1.5pt];
            \fill (B9) circle [radius=1.5pt];

            \draw[dashed] (1, 0.5) node[anchor=east] { \tt \footnotesize \textcolor{blue}{+2} } 
                node[anchor=west] {\tt \footnotesize \textbf{2} } -- (1,5.5);
        \end{tikzpicture}
    \end{figure}

\end{frame}

\begin{frame}[fragile]{Visualização do algoritmo de  interseção de intervalos}

    \begin{figure}
        \begin{tikzpicture}
            \node[anchor=west] at (0, 7) { Intervalos: $(2, 5), (1, 6), (5, 8), (3, 7), (4, 5), (1, 4), (4, 10), (3, 4), (7, 9)$ };

            \coordinate (A1) at (2, 5);
            \coordinate (B1) at (5, 5);
            \coordinate (A2) at (1, 4.5);
            \coordinate (B2) at (6, 4.5);
            \coordinate (A3) at (5, 4);
            \coordinate (B3) at (8, 4);
            \coordinate (A4) at (3, 3.5);
            \coordinate (B4) at (7, 3.5);
            \coordinate (A5) at (4, 3);
            \coordinate (B5) at (5, 3);
            \coordinate (A6) at (1, 2.5);
            \coordinate (B6) at (4, 2.5);
            \coordinate (A7) at (4, 2);
            \coordinate (B7) at (10, 2);
            \coordinate (A8) at (3, 1.5);
            \coordinate (B8) at (4, 1.5);
            \coordinate (A9) at (7, 1);
            \coordinate (B9) at (9, 1);

            \draw[opacity=0] (0, 0);
            \draw (A1) -- (B1);
            \draw (A2) -- (B2);
            \draw (A3) -- (B3);
            \draw (A4) -- (B4);
            \draw (A5) -- (B5);
            \draw (A6) -- (B6);
            \draw (A7) -- (B7);
            \draw (A8) -- (B8);
            \draw (A9) -- (B9);

            \fill (A1) circle [radius=1.5pt];
            \fill (B1) circle [radius=1.5pt];
            \fill (A2) circle [radius=1.5pt];
            \fill (B2) circle [radius=1.5pt];
            \fill (A3) circle [radius=1.5pt];
            \fill (B3) circle [radius=1.5pt];
            \fill (A4) circle [radius=1.5pt];
            \fill (B4) circle [radius=1.5pt];
            \fill (A5) circle [radius=1.5pt];
            \fill (B5) circle [radius=1.5pt];
            \fill (A6) circle [radius=1.5pt];
            \fill (B6) circle [radius=1.5pt];
            \fill (A7) circle [radius=1.5pt];
            \fill (B7) circle [radius=1.5pt];
            \fill (A8) circle [radius=1.5pt];
            \fill (B8) circle [radius=1.5pt];
            \fill (A9) circle [radius=1.5pt];
            \fill (B9) circle [radius=1.5pt];

            \draw[dashed] (2, 0.5) node[anchor=east] { \tt \footnotesize \textcolor{blue}{+1} } 
                node[anchor=west] {\tt \footnotesize \textbf{3} } -- (2,5.5);
        \end{tikzpicture}
    \end{figure}

\end{frame}

\begin{frame}[fragile]{Visualização do algoritmo de  interseção de intervalos}

    \begin{figure}
        \begin{tikzpicture}
            \node[anchor=west] at (0, 7) { Intervalos: $(2, 5), (1, 6), (5, 8), (3, 7), (4, 5), (1, 4), (4, 10), (3, 4), (7, 9)$ };

            \coordinate (A1) at (2, 5);
            \coordinate (B1) at (5, 5);
            \coordinate (A2) at (1, 4.5);
            \coordinate (B2) at (6, 4.5);
            \coordinate (A3) at (5, 4);
            \coordinate (B3) at (8, 4);
            \coordinate (A4) at (3, 3.5);
            \coordinate (B4) at (7, 3.5);
            \coordinate (A5) at (4, 3);
            \coordinate (B5) at (5, 3);
            \coordinate (A6) at (1, 2.5);
            \coordinate (B6) at (4, 2.5);
            \coordinate (A7) at (4, 2);
            \coordinate (B7) at (10, 2);
            \coordinate (A8) at (3, 1.5);
            \coordinate (B8) at (4, 1.5);
            \coordinate (A9) at (7, 1);
            \coordinate (B9) at (9, 1);

            \draw[opacity=0] (0, 0);
            \draw (A1) -- (B1);
            \draw (A2) -- (B2);
            \draw (A3) -- (B3);
            \draw (A4) -- (B4);
            \draw (A5) -- (B5);
            \draw (A6) -- (B6);
            \draw (A7) -- (B7);
            \draw (A8) -- (B8);
            \draw (A9) -- (B9);

            \fill (A1) circle [radius=1.5pt];
            \fill (B1) circle [radius=1.5pt];
            \fill (A2) circle [radius=1.5pt];
            \fill (B2) circle [radius=1.5pt];
            \fill (A3) circle [radius=1.5pt];
            \fill (B3) circle [radius=1.5pt];
            \fill (A4) circle [radius=1.5pt];
            \fill (B4) circle [radius=1.5pt];
            \fill (A5) circle [radius=1.5pt];
            \fill (B5) circle [radius=1.5pt];
            \fill (A6) circle [radius=1.5pt];
            \fill (B6) circle [radius=1.5pt];
            \fill (A7) circle [radius=1.5pt];
            \fill (B7) circle [radius=1.5pt];
            \fill (A8) circle [radius=1.5pt];
            \fill (B8) circle [radius=1.5pt];
            \fill (A9) circle [radius=1.5pt];
            \fill (B9) circle [radius=1.5pt];

            \draw[dashed] (3, 0.5) node[anchor=east] { \tt \footnotesize \textcolor{blue}{+2} } 
                node[anchor=west] {\tt \footnotesize \textbf{5} } -- (3,5.5);
        \end{tikzpicture}
    \end{figure}

\end{frame}

\begin{frame}[fragile]{Visualização do algoritmo de  interseção de intervalos}

    \begin{figure}
        \begin{tikzpicture}
            \node[anchor=west] at (0, 7) { Intervalos: $(2, 5), (1, 6), (5, 8), (3, 7), (4, 5), (1, 4), (4, 10), (3, 4), (7, 9)$ };

            \coordinate (A1) at (2, 5);
            \coordinate (B1) at (5, 5);
            \coordinate (A2) at (1, 4.5);
            \coordinate (B2) at (6, 4.5);
            \coordinate (A3) at (5, 4);
            \coordinate (B3) at (8, 4);
            \coordinate (A4) at (3, 3.5);
            \coordinate (B4) at (7, 3.5);
            \coordinate (A5) at (4, 3);
            \coordinate (B5) at (5, 3);
            \coordinate (A6) at (1, 2.5);
            \coordinate (B6) at (4, 2.5);
            \coordinate (A7) at (4, 2);
            \coordinate (B7) at (10, 2);
            \coordinate (A8) at (3, 1.5);
            \coordinate (B8) at (4, 1.5);
            \coordinate (A9) at (7, 1);
            \coordinate (B9) at (9, 1);

            \draw[opacity=0] (0, 0);
            \draw (A1) -- (B1);
            \draw (A2) -- (B2);
            \draw (A3) -- (B3);
            \draw (A4) -- (B4);
            \draw (A5) -- (B5);
            \draw (A6) -- (B6);
            \draw (A7) -- (B7);
            \draw (A8) -- (B8);
            \draw (A9) -- (B9);

            \fill (A1) circle [radius=1.5pt];
            \fill (B1) circle [radius=1.5pt];
            \fill (A2) circle [radius=1.5pt];
            \fill (B2) circle [radius=1.5pt];
            \fill (A3) circle [radius=1.5pt];
            \fill (B3) circle [radius=1.5pt];
            \fill (A4) circle [radius=1.5pt];
            \fill (B4) circle [radius=1.5pt];
            \fill (A5) circle [radius=1.5pt];
            \fill (B5) circle [radius=1.5pt];
            \fill (A6) circle [radius=1.5pt];
            \fill (B6) circle [radius=1.5pt];
            \fill (A7) circle [radius=1.5pt];
            \fill (B7) circle [radius=1.5pt];
            \fill (A8) circle [radius=1.5pt];
            \fill (B8) circle [radius=1.5pt];
            \fill (A9) circle [radius=1.5pt];
            \fill (B9) circle [radius=1.5pt];

            \draw[dashed] (4, 0.5) node[anchor=east] { \tt \footnotesize \textcolor{red}{-2} } 
                node[anchor=west] {\tt \footnotesize \textbf{3} } -- (4,5.5);
        \end{tikzpicture}
    \end{figure}

\end{frame}

\begin{frame}[fragile]{Visualização do algoritmo de  interseção de intervalos}

    \begin{figure}
        \begin{tikzpicture}
            \node[anchor=west] at (0, 7) { Intervalos: $(2, 5), (1, 6), (5, 8), (3, 7), (4, 5), (1, 4), (4, 10), (3, 4), (7, 9)$ };

            \coordinate (A1) at (2, 5);
            \coordinate (B1) at (5, 5);
            \coordinate (A2) at (1, 4.5);
            \coordinate (B2) at (6, 4.5);
            \coordinate (A3) at (5, 4);
            \coordinate (B3) at (8, 4);
            \coordinate (A4) at (3, 3.5);
            \coordinate (B4) at (7, 3.5);
            \coordinate (A5) at (4, 3);
            \coordinate (B5) at (5, 3);
            \coordinate (A6) at (1, 2.5);
            \coordinate (B6) at (4, 2.5);
            \coordinate (A7) at (4, 2);
            \coordinate (B7) at (10, 2);
            \coordinate (A8) at (3, 1.5);
            \coordinate (B8) at (4, 1.5);
            \coordinate (A9) at (7, 1);
            \coordinate (B9) at (9, 1);

            \draw[opacity=0] (0, 0);
            \draw (A1) -- (B1);
            \draw (A2) -- (B2);
            \draw (A3) -- (B3);
            \draw (A4) -- (B4);
            \draw (A5) -- (B5);
            \draw (A6) -- (B6);
            \draw (A7) -- (B7);
            \draw (A8) -- (B8);
            \draw (A9) -- (B9);

            \fill (A1) circle [radius=1.5pt];
            \fill (B1) circle [radius=1.5pt];
            \fill (A2) circle [radius=1.5pt];
            \fill (B2) circle [radius=1.5pt];
            \fill (A3) circle [radius=1.5pt];
            \fill (B3) circle [radius=1.5pt];
            \fill (A4) circle [radius=1.5pt];
            \fill (B4) circle [radius=1.5pt];
            \fill (A5) circle [radius=1.5pt];
            \fill (B5) circle [radius=1.5pt];
            \fill (A6) circle [radius=1.5pt];
            \fill (B6) circle [radius=1.5pt];
            \fill (A7) circle [radius=1.5pt];
            \fill (B7) circle [radius=1.5pt];
            \fill (A8) circle [radius=1.5pt];
            \fill (B8) circle [radius=1.5pt];
            \fill (A9) circle [radius=1.5pt];
            \fill (B9) circle [radius=1.5pt];

            \draw[dashed] (4, 0.5) node[anchor=east] { \tt \footnotesize \textcolor{blue}{+2} } 
                node[anchor=west] {\tt \footnotesize \textbf{5} } -- (4,5.5);
        \end{tikzpicture}
    \end{figure}

\end{frame}

\begin{frame}[fragile]{Visualização do algoritmo de  interseção de intervalos}

    \begin{figure}
        \begin{tikzpicture}
            \node[anchor=west] at (0, 7) { Intervalos: $(2, 5), (1, 6), (5, 8), (3, 7), (4, 5), (1, 4), (4, 10), (3, 4), (7, 9)$ };

            \coordinate (A1) at (2, 5);
            \coordinate (B1) at (5, 5);
            \coordinate (A2) at (1, 4.5);
            \coordinate (B2) at (6, 4.5);
            \coordinate (A3) at (5, 4);
            \coordinate (B3) at (8, 4);
            \coordinate (A4) at (3, 3.5);
            \coordinate (B4) at (7, 3.5);
            \coordinate (A5) at (4, 3);
            \coordinate (B5) at (5, 3);
            \coordinate (A6) at (1, 2.5);
            \coordinate (B6) at (4, 2.5);
            \coordinate (A7) at (4, 2);
            \coordinate (B7) at (10, 2);
            \coordinate (A8) at (3, 1.5);
            \coordinate (B8) at (4, 1.5);
            \coordinate (A9) at (7, 1);
            \coordinate (B9) at (9, 1);

            \draw[opacity=0] (0, 0);
            \draw (A1) -- (B1);
            \draw (A2) -- (B2);
            \draw (A3) -- (B3);
            \draw (A4) -- (B4);
            \draw (A5) -- (B5);
            \draw (A6) -- (B6);
            \draw (A7) -- (B7);
            \draw (A8) -- (B8);
            \draw (A9) -- (B9);

            \fill (A1) circle [radius=1.5pt];
            \fill (B1) circle [radius=1.5pt];
            \fill (A2) circle [radius=1.5pt];
            \fill (B2) circle [radius=1.5pt];
            \fill (A3) circle [radius=1.5pt];
            \fill (B3) circle [radius=1.5pt];
            \fill (A4) circle [radius=1.5pt];
            \fill (B4) circle [radius=1.5pt];
            \fill (A5) circle [radius=1.5pt];
            \fill (B5) circle [radius=1.5pt];
            \fill (A6) circle [radius=1.5pt];
            \fill (B6) circle [radius=1.5pt];
            \fill (A7) circle [radius=1.5pt];
            \fill (B7) circle [radius=1.5pt];
            \fill (A8) circle [radius=1.5pt];
            \fill (B8) circle [radius=1.5pt];
            \fill (A9) circle [radius=1.5pt];
            \fill (B9) circle [radius=1.5pt];

            \draw[dashed] (5, 0.5) node[anchor=east] { \tt \footnotesize \textcolor{red}{-2} } 
                node[anchor=west] {\tt \footnotesize \textbf{3} } -- (5,5.5);
        \end{tikzpicture}
    \end{figure}

\end{frame}

\begin{frame}[fragile]{Visualização do algoritmo de  interseção de intervalos}

    \begin{figure}
        \begin{tikzpicture}
            \node[anchor=west] at (0, 7) { Intervalos: $(2, 5), (1, 6), (5, 8), (3, 7), (4, 5), (1, 4), (4, 10), (3, 4), (7, 9)$ };

            \coordinate (A1) at (2, 5);
            \coordinate (B1) at (5, 5);
            \coordinate (A2) at (1, 4.5);
            \coordinate (B2) at (6, 4.5);
            \coordinate (A3) at (5, 4);
            \coordinate (B3) at (8, 4);
            \coordinate (A4) at (3, 3.5);
            \coordinate (B4) at (7, 3.5);
            \coordinate (A5) at (4, 3);
            \coordinate (B5) at (5, 3);
            \coordinate (A6) at (1, 2.5);
            \coordinate (B6) at (4, 2.5);
            \coordinate (A7) at (4, 2);
            \coordinate (B7) at (10, 2);
            \coordinate (A8) at (3, 1.5);
            \coordinate (B8) at (4, 1.5);
            \coordinate (A9) at (7, 1);
            \coordinate (B9) at (9, 1);

            \draw[opacity=0] (0, 0);
            \draw (A1) -- (B1);
            \draw (A2) -- (B2);
            \draw (A3) -- (B3);
            \draw (A4) -- (B4);
            \draw (A5) -- (B5);
            \draw (A6) -- (B6);
            \draw (A7) -- (B7);
            \draw (A8) -- (B8);
            \draw (A9) -- (B9);

            \fill (A1) circle [radius=1.5pt];
            \fill (B1) circle [radius=1.5pt];
            \fill (A2) circle [radius=1.5pt];
            \fill (B2) circle [radius=1.5pt];
            \fill (A3) circle [radius=1.5pt];
            \fill (B3) circle [radius=1.5pt];
            \fill (A4) circle [radius=1.5pt];
            \fill (B4) circle [radius=1.5pt];
            \fill (A5) circle [radius=1.5pt];
            \fill (B5) circle [radius=1.5pt];
            \fill (A6) circle [radius=1.5pt];
            \fill (B6) circle [radius=1.5pt];
            \fill (A7) circle [radius=1.5pt];
            \fill (B7) circle [radius=1.5pt];
            \fill (A8) circle [radius=1.5pt];
            \fill (B8) circle [radius=1.5pt];
            \fill (A9) circle [radius=1.5pt];
            \fill (B9) circle [radius=1.5pt];

            \draw[dashed] (5, 0.5) node[anchor=east] { \tt \footnotesize \textcolor{blue}{+1} } 
                node[anchor=west] {\tt \footnotesize \textbf{4} } -- (5,5.5);
        \end{tikzpicture}
    \end{figure}

\end{frame}

\begin{frame}[fragile]{Visualização do algoritmo de  interseção de intervalos}

    \begin{figure}
        \begin{tikzpicture}
            \node[anchor=west] at (0, 7) { Intervalos: $(2, 5), (1, 6), (5, 8), (3, 7), (4, 5), (1, 4), (4, 10), (3, 4), (7, 9)$ };

            \coordinate (A1) at (2, 5);
            \coordinate (B1) at (5, 5);
            \coordinate (A2) at (1, 4.5);
            \coordinate (B2) at (6, 4.5);
            \coordinate (A3) at (5, 4);
            \coordinate (B3) at (8, 4);
            \coordinate (A4) at (3, 3.5);
            \coordinate (B4) at (7, 3.5);
            \coordinate (A5) at (4, 3);
            \coordinate (B5) at (5, 3);
            \coordinate (A6) at (1, 2.5);
            \coordinate (B6) at (4, 2.5);
            \coordinate (A7) at (4, 2);
            \coordinate (B7) at (10, 2);
            \coordinate (A8) at (3, 1.5);
            \coordinate (B8) at (4, 1.5);
            \coordinate (A9) at (7, 1);
            \coordinate (B9) at (9, 1);

            \draw[opacity=0] (0, 0);
            \draw (A1) -- (B1);
            \draw (A2) -- (B2);
            \draw (A3) -- (B3);
            \draw (A4) -- (B4);
            \draw (A5) -- (B5);
            \draw (A6) -- (B6);
            \draw (A7) -- (B7);
            \draw (A8) -- (B8);
            \draw (A9) -- (B9);

            \fill (A1) circle [radius=1.5pt];
            \fill (B1) circle [radius=1.5pt];
            \fill (A2) circle [radius=1.5pt];
            \fill (B2) circle [radius=1.5pt];
            \fill (A3) circle [radius=1.5pt];
            \fill (B3) circle [radius=1.5pt];
            \fill (A4) circle [radius=1.5pt];
            \fill (B4) circle [radius=1.5pt];
            \fill (A5) circle [radius=1.5pt];
            \fill (B5) circle [radius=1.5pt];
            \fill (A6) circle [radius=1.5pt];
            \fill (B6) circle [radius=1.5pt];
            \fill (A7) circle [radius=1.5pt];
            \fill (B7) circle [radius=1.5pt];
            \fill (A8) circle [radius=1.5pt];
            \fill (B8) circle [radius=1.5pt];
            \fill (A9) circle [radius=1.5pt];
            \fill (B9) circle [radius=1.5pt];

            \draw[dashed] (6, 0.5) node[anchor=east] { \tt \footnotesize \textcolor{red}{-1} } 
                node[anchor=west] {\tt \footnotesize \textbf{3} } -- (6, 5.5);
        \end{tikzpicture}
    \end{figure}

\end{frame}

\begin{frame}[fragile]{Visualização do algoritmo de  interseção de intervalos}

    \begin{figure}
        \begin{tikzpicture}
            \node[anchor=west] at (0, 7) { Intervalos: $(2, 5), (1, 6), (5, 8), (3, 7), (4, 5), (1, 4), (4, 10), (3, 4), (7, 9)$ };

            \coordinate (A1) at (2, 5);
            \coordinate (B1) at (5, 5);
            \coordinate (A2) at (1, 4.5);
            \coordinate (B2) at (6, 4.5);
            \coordinate (A3) at (5, 4);
            \coordinate (B3) at (8, 4);
            \coordinate (A4) at (3, 3.5);
            \coordinate (B4) at (7, 3.5);
            \coordinate (A5) at (4, 3);
            \coordinate (B5) at (5, 3);
            \coordinate (A6) at (1, 2.5);
            \coordinate (B6) at (4, 2.5);
            \coordinate (A7) at (4, 2);
            \coordinate (B7) at (10, 2);
            \coordinate (A8) at (3, 1.5);
            \coordinate (B8) at (4, 1.5);
            \coordinate (A9) at (7, 1);
            \coordinate (B9) at (9, 1);

            \draw[opacity=0] (0, 0);
            \draw (A1) -- (B1);
            \draw (A2) -- (B2);
            \draw (A3) -- (B3);
            \draw (A4) -- (B4);
            \draw (A5) -- (B5);
            \draw (A6) -- (B6);
            \draw (A7) -- (B7);
            \draw (A8) -- (B8);
            \draw (A9) -- (B9);

            \fill (A1) circle [radius=1.5pt];
            \fill (B1) circle [radius=1.5pt];
            \fill (A2) circle [radius=1.5pt];
            \fill (B2) circle [radius=1.5pt];
            \fill (A3) circle [radius=1.5pt];
            \fill (B3) circle [radius=1.5pt];
            \fill (A4) circle [radius=1.5pt];
            \fill (B4) circle [radius=1.5pt];
            \fill (A5) circle [radius=1.5pt];
            \fill (B5) circle [radius=1.5pt];
            \fill (A6) circle [radius=1.5pt];
            \fill (B6) circle [radius=1.5pt];
            \fill (A7) circle [radius=1.5pt];
            \fill (B7) circle [radius=1.5pt];
            \fill (A8) circle [radius=1.5pt];
            \fill (B8) circle [radius=1.5pt];
            \fill (A9) circle [radius=1.5pt];
            \fill (B9) circle [radius=1.5pt];

            \draw[dashed] (7, 0.5) node[anchor=east] { \tt \footnotesize \textcolor{red}{-1} } 
                node[anchor=west] {\tt \footnotesize \textbf{2} } -- (7, 5.5);
        \end{tikzpicture}
    \end{figure}

\end{frame}

\begin{frame}[fragile]{Visualização do algoritmo de  interseção de intervalos}

    \begin{figure}
        \begin{tikzpicture}
            \node[anchor=west] at (0, 7) { Intervalos: $(2, 5), (1, 6), (5, 8), (3, 7), (4, 5), (1, 4), (4, 10), (3, 4), (7, 9)$ };

            \coordinate (A1) at (2, 5);
            \coordinate (B1) at (5, 5);
            \coordinate (A2) at (1, 4.5);
            \coordinate (B2) at (6, 4.5);
            \coordinate (A3) at (5, 4);
            \coordinate (B3) at (8, 4);
            \coordinate (A4) at (3, 3.5);
            \coordinate (B4) at (7, 3.5);
            \coordinate (A5) at (4, 3);
            \coordinate (B5) at (5, 3);
            \coordinate (A6) at (1, 2.5);
            \coordinate (B6) at (4, 2.5);
            \coordinate (A7) at (4, 2);
            \coordinate (B7) at (10, 2);
            \coordinate (A8) at (3, 1.5);
            \coordinate (B8) at (4, 1.5);
            \coordinate (A9) at (7, 1);
            \coordinate (B9) at (9, 1);

            \draw[opacity=0] (0, 0);
            \draw (A1) -- (B1);
            \draw (A2) -- (B2);
            \draw (A3) -- (B3);
            \draw (A4) -- (B4);
            \draw (A5) -- (B5);
            \draw (A6) -- (B6);
            \draw (A7) -- (B7);
            \draw (A8) -- (B8);
            \draw (A9) -- (B9);

            \fill (A1) circle [radius=1.5pt];
            \fill (B1) circle [radius=1.5pt];
            \fill (A2) circle [radius=1.5pt];
            \fill (B2) circle [radius=1.5pt];
            \fill (A3) circle [radius=1.5pt];
            \fill (B3) circle [radius=1.5pt];
            \fill (A4) circle [radius=1.5pt];
            \fill (B4) circle [radius=1.5pt];
            \fill (A5) circle [radius=1.5pt];
            \fill (B5) circle [radius=1.5pt];
            \fill (A6) circle [radius=1.5pt];
            \fill (B6) circle [radius=1.5pt];
            \fill (A7) circle [radius=1.5pt];
            \fill (B7) circle [radius=1.5pt];
            \fill (A8) circle [radius=1.5pt];
            \fill (B8) circle [radius=1.5pt];
            \fill (A9) circle [radius=1.5pt];
            \fill (B9) circle [radius=1.5pt];

            \draw[dashed] (7, 0.5) node[anchor=east] { \tt \footnotesize \textcolor{blue}{+1} } 
                node[anchor=west] {\tt \footnotesize \textbf{3} } -- (7, 5.5);
        \end{tikzpicture}
    \end{figure}

\end{frame}

\begin{frame}[fragile]{Visualização do algoritmo de  interseção de intervalos}

    \begin{figure}
        \begin{tikzpicture}
            \node[anchor=west] at (0, 7) { Intervalos: $(2, 5), (1, 6), (5, 8), (3, 7), (4, 5), (1, 4), (4, 10), (3, 4), (7, 9)$ };

            \coordinate (A1) at (2, 5);
            \coordinate (B1) at (5, 5);
            \coordinate (A2) at (1, 4.5);
            \coordinate (B2) at (6, 4.5);
            \coordinate (A3) at (5, 4);
            \coordinate (B3) at (8, 4);
            \coordinate (A4) at (3, 3.5);
            \coordinate (B4) at (7, 3.5);
            \coordinate (A5) at (4, 3);
            \coordinate (B5) at (5, 3);
            \coordinate (A6) at (1, 2.5);
            \coordinate (B6) at (4, 2.5);
            \coordinate (A7) at (4, 2);
            \coordinate (B7) at (10, 2);
            \coordinate (A8) at (3, 1.5);
            \coordinate (B8) at (4, 1.5);
            \coordinate (A9) at (7, 1);
            \coordinate (B9) at (9, 1);

            \draw[opacity=0] (0, 0);
            \draw (A1) -- (B1);
            \draw (A2) -- (B2);
            \draw (A3) -- (B3);
            \draw (A4) -- (B4);
            \draw (A5) -- (B5);
            \draw (A6) -- (B6);
            \draw (A7) -- (B7);
            \draw (A8) -- (B8);
            \draw (A9) -- (B9);

            \fill (A1) circle [radius=1.5pt];
            \fill (B1) circle [radius=1.5pt];
            \fill (A2) circle [radius=1.5pt];
            \fill (B2) circle [radius=1.5pt];
            \fill (A3) circle [radius=1.5pt];
            \fill (B3) circle [radius=1.5pt];
            \fill (A4) circle [radius=1.5pt];
            \fill (B4) circle [radius=1.5pt];
            \fill (A5) circle [radius=1.5pt];
            \fill (B5) circle [radius=1.5pt];
            \fill (A6) circle [radius=1.5pt];
            \fill (B6) circle [radius=1.5pt];
            \fill (A7) circle [radius=1.5pt];
            \fill (B7) circle [radius=1.5pt];
            \fill (A8) circle [radius=1.5pt];
            \fill (B8) circle [radius=1.5pt];
            \fill (A9) circle [radius=1.5pt];
            \fill (B9) circle [radius=1.5pt];

            \draw[dashed] (8, 0.5) node[anchor=east] { \tt \footnotesize \textcolor{red}{-1} } 
                node[anchor=west] {\tt \footnotesize \textbf{2} } -- (8, 5.5);
        \end{tikzpicture}
    \end{figure}

\end{frame}

\begin{frame}[fragile]{Visualização do algoritmo de  interseção de intervalos}

    \begin{figure}
        \begin{tikzpicture}
            \node[anchor=west] at (0, 7) { Intervalos: $(2, 5), (1, 6), (5, 8), (3, 7), (4, 5), (1, 4), (4, 10), (3, 4), (7, 9)$ };

            \coordinate (A1) at (2, 5);
            \coordinate (B1) at (5, 5);
            \coordinate (A2) at (1, 4.5);
            \coordinate (B2) at (6, 4.5);
            \coordinate (A3) at (5, 4);
            \coordinate (B3) at (8, 4);
            \coordinate (A4) at (3, 3.5);
            \coordinate (B4) at (7, 3.5);
            \coordinate (A5) at (4, 3);
            \coordinate (B5) at (5, 3);
            \coordinate (A6) at (1, 2.5);
            \coordinate (B6) at (4, 2.5);
            \coordinate (A7) at (4, 2);
            \coordinate (B7) at (10, 2);
            \coordinate (A8) at (3, 1.5);
            \coordinate (B8) at (4, 1.5);
            \coordinate (A9) at (7, 1);
            \coordinate (B9) at (9, 1);

            \draw[opacity=0] (0, 0);
            \draw (A1) -- (B1);
            \draw (A2) -- (B2);
            \draw (A3) -- (B3);
            \draw (A4) -- (B4);
            \draw (A5) -- (B5);
            \draw (A6) -- (B6);
            \draw (A7) -- (B7);
            \draw (A8) -- (B8);
            \draw (A9) -- (B9);

            \fill (A1) circle [radius=1.5pt];
            \fill (B1) circle [radius=1.5pt];
            \fill (A2) circle [radius=1.5pt];
            \fill (B2) circle [radius=1.5pt];
            \fill (A3) circle [radius=1.5pt];
            \fill (B3) circle [radius=1.5pt];
            \fill (A4) circle [radius=1.5pt];
            \fill (B4) circle [radius=1.5pt];
            \fill (A5) circle [radius=1.5pt];
            \fill (B5) circle [radius=1.5pt];
            \fill (A6) circle [radius=1.5pt];
            \fill (B6) circle [radius=1.5pt];
            \fill (A7) circle [radius=1.5pt];
            \fill (B7) circle [radius=1.5pt];
            \fill (A8) circle [radius=1.5pt];
            \fill (B8) circle [radius=1.5pt];
            \fill (A9) circle [radius=1.5pt];
            \fill (B9) circle [radius=1.5pt];

            \draw[dashed] (9, 0.5) node[anchor=east] { \tt \footnotesize \textcolor{red}{-1} } 
                node[anchor=west] {\tt \footnotesize \textbf{1} } -- (9, 5.5);
        \end{tikzpicture}
    \end{figure}

\end{frame}

\begin{frame}[fragile]{Visualização do algoritmo de  interseção de intervalos}

    \begin{figure}
        \begin{tikzpicture}
            \node[anchor=west] at (0, 7) { Intervalos: $(2, 5), (1, 6), (5, 8), (3, 7), (4, 5), (1, 4), (4, 10), (3, 4), (7, 9)$ };

            \coordinate (A1) at (2, 5);
            \coordinate (B1) at (5, 5);
            \coordinate (A2) at (1, 4.5);
            \coordinate (B2) at (6, 4.5);
            \coordinate (A3) at (5, 4);
            \coordinate (B3) at (8, 4);
            \coordinate (A4) at (3, 3.5);
            \coordinate (B4) at (7, 3.5);
            \coordinate (A5) at (4, 3);
            \coordinate (B5) at (5, 3);
            \coordinate (A6) at (1, 2.5);
            \coordinate (B6) at (4, 2.5);
            \coordinate (A7) at (4, 2);
            \coordinate (B7) at (10, 2);
            \coordinate (A8) at (3, 1.5);
            \coordinate (B8) at (4, 1.5);
            \coordinate (A9) at (7, 1);
            \coordinate (B9) at (9, 1);

            \draw[opacity=0] (0, 0);
            \draw (A1) -- (B1);
            \draw (A2) -- (B2);
            \draw (A3) -- (B3);
            \draw (A4) -- (B4);
            \draw (A5) -- (B5);
            \draw (A6) -- (B6);
            \draw (A7) -- (B7);
            \draw (A8) -- (B8);
            \draw (A9) -- (B9);

            \fill (A1) circle [radius=1.5pt];
            \fill (B1) circle [radius=1.5pt];
            \fill (A2) circle [radius=1.5pt];
            \fill (B2) circle [radius=1.5pt];
            \fill (A3) circle [radius=1.5pt];
            \fill (B3) circle [radius=1.5pt];
            \fill (A4) circle [radius=1.5pt];
            \fill (B4) circle [radius=1.5pt];
            \fill (A5) circle [radius=1.5pt];
            \fill (B5) circle [radius=1.5pt];
            \fill (A6) circle [radius=1.5pt];
            \fill (B6) circle [radius=1.5pt];
            \fill (A7) circle [radius=1.5pt];
            \fill (B7) circle [radius=1.5pt];
            \fill (A8) circle [radius=1.5pt];
            \fill (B8) circle [radius=1.5pt];
            \fill (A9) circle [radius=1.5pt];
            \fill (B9) circle [radius=1.5pt];

            \draw[dashed] (10, 0.5) node[anchor=east] { \tt \footnotesize \textcolor{red}{-1} } 
                node[anchor=west] {\tt \footnotesize \textbf{0} } -- (10, 5.5);
        \end{tikzpicture}
    \end{figure}

\end{frame}


\begin{frame}[fragile]{Implementação da interseção de intervalos}
    \inputsnippet{cpp}{1}{18}{codes/intersection.cpp}
\end{frame}

\begin{frame}[fragile]{Implementação da interseção de intervalos}
    \inputsnippet{cpp}{20}{36}{codes/intersection.cpp}
\end{frame}
