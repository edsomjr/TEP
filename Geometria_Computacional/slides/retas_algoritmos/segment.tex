\section{Relação entre ponto e segmento}

\begin{frame}[fragile]{Interseção entre ponto e segmento}
    \begin{itemize}
        \item É possível verificar se um ponto $P$ pertence ao segmento de reta definido pelos pontos $A$ e $B$ em duas etapas
        \pause
        \begin{enumerate}
            \item Verificar se $P$ pertente à reta que passa pelos pontos $A$ e $B$
            \pause

            \item Verificar se a coordenada $x$ de $P$ está dentro do intervalo $[r, s]$, onde $r = \min(x_A, x_B)$ e $s = \max(x_A, x_B)$
        \end{enumerate}
            \pause

        \item Há uma abordagem alternativa, também em dois passos, que dispensa o conhecimento dos coeficientes da reta que passa por $A$ e $B$
        \pause


        \item O primeiro passo é verificar se $P$ está dentro da região retangular cujos vértices opostos são $A$ e $B$, respectivamente
        \pause

        \item O segundo passo é verificar se $P$ está no segmento $AB$, usando semelhança de triângulos
    \end{itemize} 

\end{frame}

\begin{frame}[fragile]{Rotina que verifica se um ponto $P$ pertence ao segmento $AB$}
    \inputcode{cpp}{codes/contains.cpp}
\end{frame}


\begin{frame}[fragile]{Ponto mais próximo de um segmento de reta}

    \begin{itemize}
        \item Para determinar o ponto do segmento $AB$ mais próximo de um ponto $P$ dado, é
            preciso, inicialmente, determinar o ponto $Q$ da reta $r$ que contém $A$ e $B$ mais
            próximo de $P$
        \pause

        \item Em seguida, é preciso avaliar também os extremos $A$ e $B$ do segmento, pois o ponto
            $Q$ pode estar fora do segmento
        \pause

        \item Assim, o ponto mais próximo (e a respectiva distância) será, dentre $A$, $B$ e $Q$,
            o mais próximo de $P$ que pertença ao intervalo

    \end{itemize}

\end{frame}

\begin{frame}[fragile]{Implementação do ponto mais próximo de $P$ em $AB$}
    \inputsnippet{cpp}{1}{18}{codes/closest_segment.cpp}
\end{frame}

\begin{frame}[fragile]{Implementação do ponto mais próximo de $P$ em $AB$}
    \inputsnippet{cpp}{20}{42}{codes/closest_segment.cpp}
\end{frame}

\section{Relação entre segmentos}

\begin{frame}[fragile]{Interseção entre segmentos de reta}

    \begin{itemize}
        \item Para determinar a interseção entre dois segmentos de reta é preciso resolver o
            problema para as duas retas que contém os respectivos segmentos e verificar se a interseção, se existir, pertence a ambos intervalos
        \pause

        \item Embora esta abordagem permita conhecer as coordenadas das possíveis interseções, ela              traz alguns problemas em potencial:
        \pause

        \begin{enumerate}
            \item mesmo que as retas sejam coincidentes, não há garantias que os segmentos
                tenham interseção
        \pause

            \item a concorrência também não garante interseção: ainda é preciso verificar se o ponto pertence a ambos intervalos
        \end{enumerate}
        \pause

        \item Para identificar apenas se há interseção entre ambos segmentos, sem determinar
            as coordenadas de tal interseção, o problema fica simplificado, e será abordado
            mais adiante
    \end{itemize}

\end{frame}

\begin{frame}[fragile]{Interseção entre segmentos}

    \begin{itemize}
        \item Para se determinar se dois segmentos $AB$ e $PQ$ se intersectam pode-se utilizar um
            algoritmo baseado no discriminante $D$
        \pause

        \item A ideia central é que dois segmentos se interceptam se a reta que passa por um
            dos segmento separa os dois pontos do outro segmento em semiplanos distintos
        \pause

        \item É preciso, contudo, tomar cuidado com o caso onde um dos pontos de um segmento
            (por exemplo, $A$) é colinear em relação aos pontos do outro segmento ($P$ e $Q$)
        \pause

        \item Neste caso especial, o discriminante será igual a zero, e será necessário verificar
            se o ponto $A$ pertence ou não a $PQ$
    \end{itemize}

\end{frame}

\begin{frame}[fragile]{Implementação da interseção de segmentos em C++}
    \inputsnippet{cpp}{1}{17}{codes/intersection.cpp}
\end{frame}

\begin{frame}[fragile]{Implementação da interseção de segmentos em C++}
    \inputsnippet{cpp}{18}{37}{codes/intersection.cpp}
\end{frame}
