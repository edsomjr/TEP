\section{Relação entre segmentos}

\begin{frame}[fragile]{Interseção entre segmentos}

    \begin{itemize}
        \item Para se determinar se dois segmentos $AB$ e $PQ$ se intersectam pode-se utilizar um 
            algoritmo baseado no discriminante $D$

        \item A ideia central é que dois segmentos se interceptam se a reta que passa por um 
            dos segmento separa os dois pontos do outro segmento em semiplanos distintos

        \item É preciso, contudo, tomar cuidado com o caso onde um dos pontos de um segmento 
            (por exemplo, $A$) é colinear em relação aos pontos do outro segmento ($P$ e $Q$)

        \item Neste caso especial, o discriminante será igual a zero, e será necessário verificar
            se o ponto $A$ pertence ou não a $PQ$
    \end{itemize}

\end{frame}

\begin{frame}[fragile]{Implementação da interseção de segmentos em C++}
    \inputsnippet{cpp}{1}{17}{intersection.cpp}
\end{frame}

\begin{frame}[fragile]{Implementação da interseção de segmentos em C++}
    \inputsnippet{cpp}{18}{37}{intersection.cpp}
\end{frame}
