\section{Relação entre retas}

\begin{frame}[fragile]{Interseção entre retas}

    \begin{itemize}
        \item Dado um par de retas $r$ e $s$, elas podem ser:
        \pause

        \begin{enumerate}
            \item coincidentes (infinitas interseções),
            \pause

            \item paralelas (nenhuma interseção), ou
            \pause

            \item concorrentes (um único ponto de interseção)
        \end{enumerate}
        \pause

        \item Para encontrar o ponto de interseção, no caso de retas concorrentes, basta resolver o sistema linear resultante das equações gerais das duas retas:
        \[
            \left\lbrace \begin{array}{l} a_rx + b_ry + c_r = 0 \\ a_sx + b_sy + c_s = 0
            \end{array} \right.
        \]
        \pause

        \item As soluções são
        \[
            x = (-c_r b_s + c_s b_r) / (a_rb_s - a_sb_r)
        \]
        \[
            y = (-c_sa_r + c_ra_s) / (a_rb_s - a_sb_r)
        \]

    \end{itemize}

\end{frame}

\begin{frame}[fragile]{Exemplo de implementação da interseção entre duas retas}
    \inputcode{cpp}{codes/inter.cpp}
\end{frame}

\begin{frame}[fragile]{Ângulo entre retas}

    \begin{itemize}
        \item Para mensurar o ângulo formado por duas retas (ou dois segmentos de reta), é preciso
            identificar os vetores $\vec{u}$ e $\vec{v}$ que estejam na mesma direção das duas retas e usar o produto interno
        \pause

        \item Dados dois pontos distintos $P = (x_p, y_p)$ e $Q = (x_q, y_q)$, o vetor direção da
            reta que passa por $P$ e $Q$ é dado por $\vec{u} = (x_q - x_p, y_q - y_p)$
        \pause

        \item De posse dos vetores de direção, o cosseno ângulo entre as retas é dado por
        \[
            \cos \theta = \frac{u \cdot v}{|u||v|} = \frac{u_xv_x + u_yv_y}{\sqrt{u_x^2 + u_y^2}\sqrt{v_x^2 + v_y^2}}
        \]
        \pause

        \item Para achar o ângulo, basta computar a função inversa do cosseno (\code{c}{acos()}, na
            biblioteca de matemática padrão do C/C++) no lado direito da expressão acima
    \end{itemize}

\end{frame}

\begin{frame}[fragile]{Exemplo de implementação do ângulo entre duas retas}
    \inputcode{cpp}{codes/angle.cpp}
\end{frame}

\begin{frame}[fragile]{Interseção entre segmentos de reta}

    \begin{itemize}
        \item Para determinar a interseção entre dois segmentos de reta é preciso resolver o
            problema para as duas retas que contém os respectivos segmentos e verificar se a interseção, se existir, pertence a ambos intervalos
        \pause

        \item Embora esta abordagem permita conhecer as coordenadas das possíveis interseções, ela              traz alguns problemas em potencial:
        \pause

        \begin{enumerate}
            \item mesmo que as retas sejam coincidentes, não há garantias que os segmentos
                tenham interseção
        \pause

            \item a concorrência também não garante interseção: ainda é preciso verificar se o ponto pertence a ambos intervalos
        \end{enumerate}
        \pause

        \item Para identificar apenas se há interseção entre ambos segmentos, sem determinar
            as coordenadas de tal interseção, o problema fica simplificado, e será abordado
            mais adiante
    \end{itemize}

\end{frame}

\begin{frame}[fragile]{Rotina que verifica se um ponto $P$ pertence ao segmento $AB$}
    \inputcode{cpp}{codes/contains.cpp}
\end{frame}
