\section{Relação entre retas e pontos}

\begin{frame}[fragile]{Distância entre ponto e reta}

    \begin{itemize}
        \item A distância de um ponto $P$ a uma reta $r$ é definida como a menor distância possível
            entre todos os pontos de $r$ e $P$:
        \[
            d(P, r) = \min \lbrace d(P, Q), Q\in r\rbrace
        \]
        \pause

        \item Não é necessário computar as infinitas distâncias possíveis: a menor
            distância será aquela entre $P$ e o ponto de interseção $Q$ de $r$ com a reta
            perpendicular a $r$ que passa por $P$
        \pause

        \item Seja usando álgebra, geometria ou álgebra linear, é possível mostrar que esta
            distância $d$ entre $P = (x_p, y_p)$ e a reta $ax + by + c = 0$ é dada por
        \[
            d(P, r) = \frac{|ax_p + by_p + c|}{\sqrt{a^2 + b^2}}
        \]
        \pause

        \item As coordenadas de $Q = (x_q, y_q)$ podem ser obtidas utilizando-se as expressões
        \[
            x_q = \frac{b(bx_p - ay_p) - ac}{a^2 + b^2}, \, \, \, \, y_q = \frac{a(-bx_p + ay_p) - bc}{a^2 + b^2}
        \]
    \end{itemize}

\end{frame}

\begin{frame}[fragile]{Implementação de distância entre ponto e reta em C++}
    \inputsnippet{cpp}{1}{16}{codes/dist.cpp}
\end{frame}

\begin{frame}[fragile]{Implementação de distância entre ponto e reta em C++}
    \inputsnippet{cpp}{18}{27}{codes/dist.cpp}
\end{frame}

\begin{frame}[fragile]{Reta mediatriz}

    \begin{itemize}
        \item Dado o segmento de reta $PQ$, a mediatriz é a reta perpendicular a $PQ$ que passa
            pelo ponto médio do segmento
        \pause

        \item Qualquer ponto da reta mediatriz é equidistante a $P$ e $Q$, e esta propriedade
            permite a dedução dos coeficientes $a, b, c$ da mediatriz
        \pause

        \item Seja $R = (x, y)$ um ponto qualquer da mediatriz. Então
        \[
            d^2(P, R) = d^2(Q, R),
        \]
        isto é,
        \[
            (x - x_p)^2 + (y - y_p)^2 = (x - x_q)^2 + (y - y_q)^2
        \]
        Logo os coeficientes são
        \[
            a = 2(x_q - x_p), \, \, b = 2(y_q - x_q), \, \, c = (x_p^2 + y_p^2) - (y_p^2 + y_q^2)
        \]
    \end{itemize}

\end{frame}

\begin{frame}[fragile]{Exemplo de implementação da reta mediatriz em C++}
    \inputcode{cpp}{codes/bisector.cpp}
\end{frame}

\begin{frame}[fragile]{Orientação entre ponto e reta}

    \begin{itemize}
        \item O determinante utilizado para o cálculo dos coeficientes da equação geral da reta
            também identifica a orientação de um ponto em relação a uma reta
        \pause

        \item Sejam $P, Q, R$ três pontos no plano e considere $r$ a reta que passa por $P$ e $Q$
        \pause

        \item Logo o vetor $\vec{v} = (x_p - x_q, y_p - y_q)$ tem mesma direção que $r$
        \pause

        \item Seja $\vec{u} = (x_r - x_q, y_r - y_q)$. O vetor $\vec{n} = (y_q - y_r, x_r - x_q)$
        é perpendicular ao vetor $\vec{u}$ (pois $\vec{u}\cdot \vec{n} = 0$)
        \pause

        \item Assim, $\vec{u}$ e $\vec{v}$ serão paralelos (e, como consequência, $P, Q$ e $R$
            serão colineares) se o produto interno entre $\vec{v}$ e $\vec{n}$ for igual a zero
        \pause

        \item Daí,
        \[
            0 = \vec{v}\cdot \vec{n} = (x_p - x_q)(y_q - y_r) + (y_p - y_q)(x_r - x_q),
        \]
        isto é,
        \[
            0 = (x_py_q - x_py_r - x_qy_p + x_qy_r) + (y_px_r - y_px_q - y_qx_r + y_qx_p)
        \]
    \end{itemize}

\end{frame}


\begin{frame}[fragile]{Orientação entre ponto e reta}
    \begin{itemize}
        \item Portanto,
        \[
            0 = (x_py_q + x_qy_r + x_ry_p) - (y_px_q + y_qx_r + y_rx_p),
        \]
        expressão que pode ser reescrita como
        \[
                \det \begin{bmatrix}
                x_p & y_p  & 1 \\
                x_q & y_q  & 1 \\
                x_r & y_r  & 1 \\
            \end{bmatrix} = 0
        \]
        \pause

        \item Deste modo, o discriminante $D(P, Q, R)$ é definido como o determinante acima
        \pause

        \item Se $r$ é uma reta que passa pelos pontos $P$ e $Q$, e $R$ é um ponto qualquer, então
        \pause

        \begin{enumerate}
            \item $R$ pertence a reta se $D = 0$,
            \pause

            \item $R$ está no semiplano à esquerda da reta, se $D > 0$, ou
            \pause

            \item $R$ no semiplano à direita da reta, se $D < 0$
        \end{enumerate}
        \pause

        \item A orientação (esquerda ou direita) diz respeito à direção que vai de $P$ a $Q$
    \end{itemize}

\end{frame}

\begin{frame}[fragile]{Exemplo de implementação do discriminante $D$ em C++}
    \inputcode{cpp}{codes/D.cpp}
\end{frame}

\begin{frame}[fragile]{Ponto mais próximo a um segmento de reta}

    \begin{itemize}
        \item Para determinar o ponto do segmento $AB$ mais próximo de um ponto $P$ dado, é
            preciso, inicialmente, determinar o ponto $Q$ da reta $r$ que contém $A$ e $B$ mais
            próximo de $P$
        \pause

        \item Em seguida, é preciso avaliar também os extremos $A$ e $B$ do segmento, pois o ponto
            $Q$ pode estar fora do segmento
        \pause

        \item Assim, o ponto mais próximo (e a respectiva distância) será, dentre $A$, $B$ e $Q$,
            o mais próximo de $P$ que pertença ao intervalo

    \end{itemize}

\end{frame}

\begin{frame}[fragile]{Implementação do ponto mais próximo de $P$ em $AB$}
    \inputsnippet{cpp}{1}{18}{codes/closest_segment.cpp}
\end{frame}

\begin{frame}[fragile]{Implementação do ponto mais próximo de $P$ em $AB$}
    \inputsnippet{cpp}{20}{42}{codes/closest_segment.cpp}
\end{frame}
