\section{Classificação de retas}

\begin{frame}[fragile]{Retas paralelas, concorrentes e coincidentes}

    \begin{itemize}
        \item Em relação às possíveis interseções entre duas retas, há três cenários possíveis:
        \pause

        \begin{enumerate}
            \item nenhum ponto em comum (retas paralelas)
        \pause
            \item um único ponto em comum  (retas concorrentes)
        \pause
            \item todos os pontos em comum (retas coincidentes)
        \end{enumerate}
        \pause

        \item O coeficiente angular é a chave para tal classificação: retas com coeficientes
            angulares distintos são concorrentes
        \pause

        \item Caso duas retas tenham coeficientes angulares iguais, é necessário verificar também
            o coeficiente linear: se iguais, as retas são coincidentes
        \pause

        \item Retas com coeficientes angulares iguais e coeficientes lineares distintos são paralelas
        \pause

        \item A implementação destas verificações é trivial na representação baseada na equação reduzida, sendo necessário apenas o cuidado no trato do caso das retas verticais
    \end{itemize}

\end{frame}

\begin{frame}[fragile]{Exemplo de implementação de classificação de retas em C++}
    \inputcode{cpp}{codes/classification.cpp}
\end{frame}

\begin{frame}[fragile]{Exemplo de implementação de classificação de retas em C++}
    \inputcode{cpp}{codes/classification2.cpp}
\end{frame}

\begin{frame}[fragile]{Retas perpendiculares}

    \begin{itemize}
        \item Duas retas são perpendiculares se o produto de seus coeficientes angulares for igual a -1
        \pause

        \item Outra maneira de checar se duas retas são perpendiculares é escolher dois pontos
        pertencentes a cada reta e montar dois vetores $\vec{u}$ e $\vec{v}$
        \pause

        \item Estes pontos podem ser escolhidos de forma eficiente, fazendo $x = 0$ e $y = 0$
        (caso a reta não passe na origem)
        \pause

        \item Se o produto interno dos dois vetores for igual a zero, as retas são perpendiculares
        \pause

        \item Importante notar, porém, é que os coeficientes $a$ e $b$ da equação geral de uma
            reta formam um vetor $\vec{v} = (a, b)$ perpendicular à reta
        \pause

        \item Tais vetores, denominados normais, podem ser utilizados na comparação descrita anteriormente
    \end{itemize}

\end{frame}

\begin{frame}[fragile]{Exemplo de verificação de retas perpendiculares em C++}
    \inputcode{cpp}{codes/ortho.cpp}
\end{frame}

\begin{frame}[fragile]{Exemplo de verificação de retas perpendiculares em C++}
    \inputcode{cpp}{codes/ortho2.cpp}
\end{frame}
