\section{Vetores}

\begin{frame}[fragile]{Definição de vetores}

    \begin{itemize}
        \item Vetores são segmentos de retas orientados

        \item Os vetores são caracterizados pela sua direção (inclinação da reta que contém o
            segmento), orientação (ponto de partida e de chegada) e tamanho (distância entre os
            dois pontos)

        \item Dois vetores são iguais apenas se coincidirem nestas três características

        \item Dados dois pontos $A$ e $B$, $\vv{u} = \overrightarrow{AB}$ é o vetor que parte do ponto $A$ em direção 
        ao ponto $B$ 

        \item Observe que $\overrightarrow{AB}$ e $\overrightarrow{BA}$ tem mesma direção e comprimento, mas orientações distintas)

    \end{itemize}

\end{frame}

\begin{frame}[fragile]{Vetor posição}

    \begin{itemize}
        \item O vetor posição de um ponto $P$ é o vetor que une a origem $O$ ao ponto $P$ 
            ($\overrightarrow{OP}$)

        \item Na prática, trabalha-se apenas com vetores-posição: o vetor posição que equivale ao
            vetor $\overrightarrow{AB}$ é o vetor $\vec{v} = (x_b - x_a, y_b - y_a)$

        \item Deste modo, embora seja possível definir um tipo de dado para representar vetores, é possível utilizar pontos para representar vetores

        \item Esta estratégia pode dificultar a leitura das rotinas, pois embora usem a mesma memória a semântica é diferente 

        \item Há porém a vantagem da velocidade de codificação, devido a eliminação de código redundante
    \end{itemize}

\end{frame}

\begin{frame}[fragile]{Exemplo de implementação de vetores em C++}
    \inputcode{cpp}{vector.cpp}
\end{frame}

\begin{frame}[fragile]{Direção de um vetor}

    \begin{itemize}
        \item A direção de um vetor pode ser caracterizada também pelo ângulo que vector posição
            equivalente faz com o eixo-$x$ positivo

        \item Este ângulo pode ser computado pela função \code{c}{atan2} da biblioteca 
            \code{c}{cmath}

        \item Esta função recebe dois parâmetros: a coordenada $y$ e a coordenada $x$ do
            vetor posição

        \item Esta função não lança exceções nem erros e tem retorno no intervalo $[-\pi, \pi]$

        \item A função \code{c}{atan} difere no número de argumentos e no intervalo do retorno
    \end{itemize}

\end{frame}

\begin{frame}[fragile]{Translações}

    \begin{itemize}
        \item Um ponto $P$ pode ser transladado no espaço, conhecidos os deslocamentos $dx$ e 
        $dy$ nas direções paralelas aos eixos $x$ e $y$, respectivamente

        \item Transladar ambos pontos que delimitam o vetor mantém o vetor inalterado

        \item Contudo, transladar apenas o ponto final $P$ de um vetor posição pode alterar todas
            as três características de um vetor

        \inputcode{cpp}{translate.cpp}
    \end{itemize}

\end{frame}

\begin{frame}[fragile]{Rotações}

    \begin{itemize}
        \item Um vetor posição pode ser rotacionado em $\theta$ graus no sentido anti-horário
            através da multiplicação da matriz de rotação $R_\theta$ e o vetor $\vec{v}$, onde
        \[
            R_\theta = \begin{bmatrix} \cos \theta & -\sin \theta \\ \sin \theta & \cos \theta
                \end{bmatrix}, \, \, \, \, \vec{v} = \begin{bmatrix} x \\ y \end{bmatrix}
        \]

        \item Esta matriz pode ser deduzida observando-se que as coordenadas do ponto $P$ do
            vetor posição $\vec{v}$ podem ser expressas como 
            \[
                x = r\cos \omega, \, \, \, \, y = r\sin \omega, 
            \]
            onde $r$ é o tamanho do vetor $\vec{v}$ e $\omega$ é o ângulo que $\vec{v}$ faz com
            o eixo-$x$ positivo

        \item Assim, as coordenadas do ponto resultante da rotação são 
        \[
            x' = r\cos (\omega + \theta), \, \, \, \, y' = r\sin (\omega + \theta)
        \]

    \end{itemize}

\end{frame}

\begin{frame}[fragile]{Rotações}

    \begin{itemize}
        \item Utilizando as fórmulas para soma de ângulos do seno e do cosseno obtemos
        \[
            x' = r\cos \omega\cos \theta - r\sin \omega\sin \theta
        \] e
        \[
            y' = r\sin \omega\cos \theta + r\cos \omega\sin \theta,
        \]
        o que corresponde ao resultado do produto matricial já citado 

        \inputcode{cpp}{rotation.cpp} 
    \end{itemize}

\end{frame}

\begin{frame}[fragile]{Rotação em torno de um ponto arbitrário}

    \begin{itemize}
        \item Caso se deseje rotacionar o ponto $P$ em torno de outro ponto $C$ que não seja a 
            origem (mais precisamente, outro eixo paralelo ao eixo-$z$ que passe pelo ponto 
            dado), basta seguir os três passos abaixo:

        \begin{enumerate}
            \item transladar o ponto com deslocamentos iguais aos opostos das coordenadas de $C$, 
                obtendo-se o ponto $P'$
            \item rotacionar o ponto transladado $P'$
            \item transladar $P'$, com deslocamentos iguais às coordenadas de $C$
        \end{enumerate}

        \item A translação inicial muda o sistema de coordenadas do problema, o levando a um novo 
            sistema onde $C$ é a origem

        \item Assim, pode-se utilizar a rotina de rotação em torno da origem e, ao final do processo, retornar ao sistema original, aplicando a translação inversa

        \item Importante notar que as três operações devem ser realizadas exatamente na ordem descrita
    \end{itemize}

\end{frame}

\begin{frame}[fragile]{Implementação da rotação em torno de um ponto arbitrário}
    \inputcode{cpp}{rotation2.cpp}
\end{frame}


\begin{frame}[fragile]{Rotações tridimensionais}

    \begin{itemize}
        \item A mesma ideia da rotação pode ser aplicada em pontos tridimensionais

        \item As matrizes $R_x, R_y$ e $R_z$ abaixo rotacionam o ponto tridimensional 
            $P = (x_p, y_p, z_p)$ em $\theta$ graus no sentido anti-horário
        \[
            R_x = \begin{bmatrix} 1 & 0 & 0 \\ 0 & \cos \theta & -\sin \theta \\
                0 & \sin \theta & \cos \theta \end{bmatrix}, \, \, \, \,
            R_y = \begin{bmatrix} \cos \theta & 0 & -\sin \theta \\ 0 & 1 & 0 \\
                \sin \theta & 0 & \cos \theta \end{bmatrix}
        \]
        \[
            R_z = \begin{bmatrix} \cos \theta & -\sin \theta & 0 \\
                \sin \theta & \cos \theta & 0 \\ 0 & 0 & 1\end{bmatrix}
        \]
    \end{itemize}

\end{frame}

\begin{frame}[fragile]{Observações sobre translações e rotações}

    \begin{itemize}
        \item Dado um conjunto de pontos $C$, se aplicadas a todos pontos $P\in C$, as operações de 
        translação e rotação não alteram as distâncias entre os pares de pontos

        \item Desta forma, se uma figura é descrita por um conjunto de pontos, todas as suas 
            características que são baseadas em distâncias (ângulos internos, perímetro, área, 
            volume, etc) são invariantes a estas duas transformações

        \item Este importante fato pode ser utilizado para simplificar problemas, 
            como exemplificado no caso da rotação em torno de um ponto arbitrário
    \end{itemize}

\end{frame}

\begin{frame}[fragile]{Escala}

    \begin{itemize}
        \item Outra transformação possível em um ponto (ou vetor posição) é a escala

        \item A escala consiste na multiplicação das coordenadas de um ponto por um escalar

        \item Se o mesmo escalar é utilizado em todos os produtos a escala é dita uniforme

        \item Ao contrário das transformações anteriores, a escala não preserva as distâncias

        \inputcode{cpp}{scale.cpp}
    \end{itemize}

\end{frame}

\begin{frame}[fragile]{Normalização de vetores}

    \begin{itemize}
        \item Uma aplicação comum da escala é a normalização de vetor

        \item Um vetor é dito unitário se o seu comprimento é igual a 1

        \item Dado um vetor $\vec{v}$ qualquer, é possível determinar um vetor unitário 
            $\vec{u}$, na mesma direção e sentido de $\vec{v}$, dividindo-se as coordenadas de 
            $\vec{v}$ pelo tamanho $|\vec{v}|$ de $\vec{v}$

        \item Observe que a escala com constantes positivas preserva a direção e o sentido do
            vetor

        \inputcode{cpp}{normalize.cpp}
    \end{itemize}

\end{frame}
