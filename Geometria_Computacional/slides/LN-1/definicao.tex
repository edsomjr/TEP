\section{Definição de reta}

\begin{frame}[fragile]{Definição}

    \begin{itemize}
        \item Reta também é um elemento primitivo da Geometria

        \item No primeiro livro dos elementos, Euclides define reta como 
            \lq\lq \textit{a line is breadthless length}\rq\rq, que numa tradução livre 
            diz que \lq\lq linha é comprimento sem largura\rq\rq

        \item As linhas são elementos unidimensionais

        \item Em C/C++, pontos podem ser representados através ou de sua equação geral, ou de
            sua equação reduzida

        \item  A equação reduzida de uma reta é a mais conhecida e utilizada nos cursos de ensino médio
        \begin{enumerate}
            \item tem a vantagem de facilitar comparações entre retas e identificar paralelismo
            \item não é capaz de representar retas verticais
        \end{enumerate}

        \item A equação geral, como o próprio nome diz, pode representar qualquer reta do plano
   \end{itemize}

\end{frame}

\begin{frame}[fragile]{Equação reduzida da reta}

    \begin{itemize}
        \item A equação reduzida da reta é dada por
        \[
            y = mx + b,
        \]
        onde $m$ é o coeficiente angular da reta e $b$ é o coeficiente linear da reta

        \item O primeiro coeficiente representa a taxa de variação da reta: consiste no número de 
        unidades que $y$ varia para cada unidade de variação de $x$ no sentido positivo do eixo horizontal

        \item O segundo coeficiente é o valor no qual a reta intercepta o eixo $y$

        \inputcode{cpp}{line.cpp}

    \end{itemize}

\end{frame}

\begin{frame}[fragile]{Equação reduzida a partir de dois pontos dados}

    \begin{itemize}
        \item Dados dois pontos $P = (x_p, y_p)$ e $Q = (x_q, y_q)$ tais que $x_p \neq x_q$, 
        a inclinação da reta é dada por
        \[
            m = \frac{y_q - y_p}{x_q - x_p}
        \]

        \item Deste modo, a equação reduzida da reta será dada por
        \[
            y = m(x - x_p) + y_p = mx + (y_p - mx_p)
        \]

        \item Se $x_p = x_q$, a reta é vertical


        \item Retas verticais podem ser tratadas por meio de uma variável booleana que indica
            se a reta é vertical ou não

        \item Caso seja, o coeficiente $b$ indica o ponto que a reta intercepta o eixo horizontal
    \end{itemize}

\end{frame}

\begin{frame}[fragile]{Implementação da reta através da equação reduzida}
    \inputcode{cpp}{line2.cpp}
\end{frame}

\begin{frame}[fragile]{Equação geral da reta}

    \begin{itemize}
        \item A equação geral da reta é dada por
        \[
            ax + by + c = 0
        \]

        \item Como dito, a equação geral pode representar retas verticais ($b = 0$)

        \item Nos demais casos, é possível obter a equação reduzida a partir da equação geral

        \inputcode{c++}{geral.cpp}

    \end{itemize}

\end{frame}

\begin{frame}[fragile]{Equação geral da reta a partir de dois pontos}

    \begin{itemize}
        \item Dados dois pontos distintos $P = (x_p, y_p)$ e $Q = (x_q, y_q)$, é possível obter os 
        coeficientes da equação geral da seguinte maneira:

        \begin{enumerate}
            \item Substitua as coordenadas de um dos dois pontos na equação geral
            \item Encontrado o valor de $c$, substitua as coordenadas de ambos pontos na equação
            geral, obtendo um sistema linear
            \item Os valores de $a$ e $b$ são a solução deste sistema linear
        \end{enumerate}

        \item Este processo pode ser simplificado através do uso de Álgebra Linear: se três pontos 
        $P = (x_p, y_p), Q = (x_q, y_q)$ e $R = (x, y)$ são colineares 
        (isto é, pertencem a uma mesma reta), então
        \[
            \det \begin{bmatrix}
                x_p & y_p  & 1 \\
                x_q & y_q  & 1 \\
                x & y  & 1 \\
            \end{bmatrix} = 0
        \]

        \item A soluçaõ da equação acima é
        \[
            a = y_p - y_q, b = x_p - x_q, c = x_py_q - x_qy_p
        \]
    \end{itemize}
\end{frame}

\begin{frame}[fragile]{Implementação da reta através da equação geral}
    \inputcode{cpp}{geral2.cpp}
\end{frame}

\begin{frame}[fragile]{Observações sobre a equação geral da reta}

    \begin{itemize}
        \item Um mesma reta pode ter infinitas equações gerais associadas: basta multiplicar toda a equação por uma número real diferente de zero

        \item Para normalizar a representação, associando uma única equação a cada reta, é necessário dividir toda a equação pelo coeficiente $a$ (ou por $b$, caso $a$ seja igual a zero)

        \item Esta estratégia permite a simplificação de algoritmos de comparação entre retas

        \item Por outro lado, não uniformizar a representação permite manter os coeficientes 
            inteiros, caso as coordenadas dos pontos sejam inteiras

        \item Importante notar que, em ambas representações, pode acontecer da reta resultante ser degenerada

        \item Isto ocorre quando os pontos $P$ e $Q$ são idênticos: neste caso, a reta se reduz a um único ponto 

        \item O tratamento deste caso especiais nos demais algoritmos aumenta o tamanho e a sofisticação dos códigos

        \item Porém o não tratamento de casos especiais pode levar ao WA
    \end{itemize}

\end{frame}

\begin{frame}[fragile]{Relação entre ponto e reta}

    \begin{itemize}
        \item Seja $r$ uma reta com equação geral $ax + by + c = 0$ e $P = (x_p, y_p)$ um ponto qualquer
        \item $P\in r$ se, e somente se, $ax_p + by_p + c = 0$

        \item Esta relação pode ser verificada diretamente a partir da equação geral da reta, ou através do determinante apresentado anteriormente, conhecidos dois pontos $Q$ e $R$ da reta

        \inputcode{cpp}{contains.cpp}
    \end{itemize}

\end{frame}

\begin{frame}[fragile]{Segmentos de reta}

    \begin{itemize}
        \item Sejam $A$ e $B$ dois pontos pertencentes à reta $r$. O segmento de reta $AB$ é o conjunto de pontos de $r$ que estão entre os pontos $A$ e $B$

        \item O comprimento de um segmento de reta é a distância entre os pontos $A$ e $B$
    \end{itemize}

    \begin{center}
    \begin{tikzpicture}
        \draw[thick] (2, 0) -- (9, 3);
    
        \node[anchor=north] at (2, 0) { $A$ };
        \node[anchor=south] at (9, 3) { $B$ };

    \end{tikzpicture}
    \end{center}

\end{frame}

\begin{frame}[fragile]{Distância entre dois pontos}

    \begin{itemize}
        \item A definição de distância depende da norma utilizada

        \item A distância euclidiana entre dois pontos $A = (x_a, y_a)$ e $B = (x_b, y_b)$ é dada por
        \[
            d(A, B) = \sqrt{(x_a - y_a)^2 + (x_b - y_b)^2}
        \]

        \item A distância do motorista de táxi é dada por
        \[
            d(A, B) = |x_a - x_b| + |y_a - y_b|
        \]

        \item Segunda a norma do máximo, a distância entre $A$ e $B$ é dada por
        \[
            d(A, B) = \max\lbrace |x_a - x_b|, |y_a - y_b|\rbrace
        \]
    \end{itemize}

\end{frame}

\begin{frame}[fragile]{Observações sobre distância entre dois pontos}

    \begin{itemize}
        \item Embora a distãncia euclidiana seja a mais comum, a raiz quadrada que aparece na
            sua definição leva a resultados em ponto flutuante

        \item Por este motivo, em geral é implementada a função que computa o quadrado da
            distância, o que elimina a raiz quadrada e permite a aritmética de inteiros

        \item O quadrado da distância pode ser usado para comparar os pontos por distância, pois
            ela preserva esta relação

        \item A distância do motorista de táxi considera que os movimentos no plano só podem ser
            feitos na horizontal e vertical

        \item Um exemplo prático da norma do máximo acontece no tabuleiro do xadrez: ela define o
            raio de ação do rei (todas as casas que estão a uma unidade de distância dele)
    \end{itemize}

\end{frame}

\begin{frame}[fragile]{Exemplo de implementação das distâncias}
    \inputsnippet{cpp}{1}{21}{dist.cpp}
\end{frame}

\begin{frame}[fragile]{Exemplo de implementação das distâncias}
    \inputsnippet{cpp}{22}{40}{dist.cpp}
\end{frame}

\begin{frame}[fragile]{Exemplo de implementação das distâncias}
    \inputsnippet{cpp}{41}{61}{dist.cpp}
\end{frame}
