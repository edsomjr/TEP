\section{Produto interno e produto vetorial}

\begin{frame}[fragile]{Produto interno}

    \begin{itemize}
        \item O produto interno (ou escalar) entre dois vetores $\vec{u}$ e $\vec{v}$ é dado pela 
        soma dos produtos das coordenadas correspondentes dos dois vetores

        \item O resultado deste produto não é um vetor, e sim um escalar

        \item É possível mostrar que este produto coincide com o produto do tamanho dos dois 
            vetores e do cosseno do ângulo entre estes vetores, conforme mostra a expressão abaixo:
        \[
            <\vec{u}, \vec{v}> = \vec{u} \cdot \vec{v} = u_xv_x + u_yv_y = |\vec{u}||\vec{v}|\cos \theta
        \]

        \item A relação acima permite computar o ângulo entre os vetores $\vec{u}$ e $\vec{v}$

        \item O sinal do produto interno $d = \vec{u}\cdot\vec{v}$ pode ser utilizado para interpretar a natureza do ângulo entre os dois vetores:

        \begin{enumerate}
            \item se $d = 0$, os vetores são ortogonais (formam um ângulo de 90\textdegree)
            \item se $d < 0$, os vetores foram um ângulo agudo (menor que 90\textdegree)
            \item se $d > 0$, os vetores formam um ângulo obtuso (maior que 90\textdegree)
        \end{enumerate}
    \end{itemize}

\end{frame}

\begin{frame}[fragile]{Implementação do produto interno e do ângulo entre vetores em C++}

    \inputcode{cpp}{dot.cpp}

\end{frame}

\begin{frame}[fragile]{Produto vetorial}

    \begin{itemize}
        \item O produto vetorial entre dois vetores $\vec{u}$ e $\vec{v}$ é $\vec{w} = \vec{u}
        \times \vec{v}$ cujas coordenadas são obtidas através do determinante 
        \[
            u\times v = \begin{vmatrix}
                \vec{i} & \vec{j} & \vec{k} \\
                u_x & u_y & u_z \\
                v_x & v_y & v_z \\
            \end{vmatrix},
        \]
        onde $\vec{i}, \vec{j}, \vec{k}$ são vetores unitários com mesma direção e sentido que os 
        eixos $x, y, z$, respectivamente

        \item Sendo definido por um determinante, o produto vetorial não é comutativo: a troca da 
            ordem dos vetores altera o sentido do resultado

        \item Para computar o produto vetorial entre vetores bidimensionais, basta fazer a 
            coordenada $z$ de ambos vetores iguais a zero

        \item O vetor resultante é perpendicular tanto a $\vec{u}$ quanto a $\vec{v}$ 

    \end{itemize}

\end{frame}

\begin{frame}[fragile]{Produto vetorial}

    \begin{itemize}
        \item A magnitude deste vetor é igual ao produto dos comprimentos dos vetores $\vec{u}$ e 
            $\vec{v}$ pelo seno do ângulo $\theta$ formado por estes dois vetores, isto é,
        \[
            |\vec{u}\times \vec{v}| = |\vec{u}||\vec{v}|\sin \theta
        \]

        \item Este valor coincide com a área do paralelogramo formado por $\vec{u}$ e $\vec{v}$

        \item Se os vetores tiverem mesma direção, o produto vetorial terá comprimento zero (como não definirão um plano, não há um vetor normal)

        \item Vetores normais podem ser utilizados para definir a orientação de uma figura tridimensional (o lado interno e externo da figura)

    \end{itemize}

\end{frame}

\begin{frame}[fragile]{Exemplo de implementação de produto vetorial em C++}
    \inputcode{cpp}{cross.cpp}
\end{frame}
