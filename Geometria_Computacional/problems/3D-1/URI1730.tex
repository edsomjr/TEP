\section{URI 1730 -- Global Roaming}

\begin{frame}[fragile]{Problema}

Hoje em dia vários dispositivos móveis de comunicação dependem de uma vista direta para um satélite. Portanto, para os provedores de comunicação é crucial saber onde os seus serviços estão disponíveis. Sua tarefa é identificar os locais que têm uma vista direta para um satélite particular, ou seja, este satélite deve estar acima do horizonte. Para facilitar as coisas, você pode assumir que a Terra é uma esfera perfeita com um raio de 6378km (montanhas serão adicionadas no próximo ano...). O satélite é um objeto \textit{pointlike} acima da superfície terrestre.

\end{frame}

\begin{frame}[fragile]{Entrada e saída}

\textbf{Entrada}

A entrada consiste de vários casos de teste. Em cada caso de teste, a primeira linha contém o número de localizações $N$ a serem verificados, seguido pela a posição do satélite: a sua latitude, a longitude (ambas em grau) e sua altura (em km) acima da superfície terrestre.

Cada uma das seguintes linhas $N$ contém um local na superfície terrestre: o nome da localidade (uma string com menos de 60 caracteres ASCII que não contém espaços em branco), seguido por sua latitude e longitude (ambos em graus). A entrada é terminada por $N$ = 0.

\end{frame}

\begin{frame}[fragile]{Entrada e saída}
\textbf{Saída}

Para cada caso de teste o seu programa deverá imprimir uma linha dizendo "Test case $K$:", onde $K$ é o número da instância atual. Então nas seguintes linhas, imprimir em linhas separadas, os nomes das localidades onde o satélite é visível na mesma ordem em que aparecem no arquivo de entrada. Imprima uma linha em branco após cada instância.

\end{frame}

\begin{frame}[fragile]{Exemplo de entradas e saídas}

\begin{minipage}[t]{0.6\textwidth}
\textbf{Exemplo de Entrada}
\begin{verbatim}
3 20.0 -60.0 150000000.0
Ulm 48.406 10.002
Jakarta -6.13 106.75
Honolulu 21.32 -157.83
2 48.4 10 0.5
Ulm 48.406 10.002
Honolulu 21.32 -157.83
0 0.0 0.0 0.0
\end{verbatim}
\end{minipage}
\begin{minipage}[t]{0.35\textwidth}
\textbf{Exemplo de Saída}
\begin{verbatim}
Test case 1:
Ulm
Honolulu

Test case 2:
Ulm
\end{verbatim}
\end{minipage}
\end{frame}

\begin{frame}[fragile]{Solução}

    \begin{itemize}
        \item Para computar o ângulo que o satélite faz com a linha do horizonte, é preciso
            converter as coordenadas geográficas (latitude, longitue e altura) em coordenadas
            esféricas

        \item É preciso atentar ao fato de que a latitude $\lambda$ se relaciona com colatitude
            $\varphi$ das coordenadas esféricas através da relação
            $ \lambda = 90^o - \varphi$.

        \item Assim, sendo $\theta$ a longitude, a mudança de variáveis é dada por
        \[
            \begin{array}{l}
                x = r\cos \theta\cos \lambda \\
                y = r\sin \theta\cos \lambda \\
                z = r\sin \lambda, \\
            \end{array}
        \]
        pois $\sin (90^o - \lambda) = \cos \lambda$ e $\cos (90^o - \lambda) = \sin \lambda$

        \item Se $\vec{P}$ é o vetor posição da cidade e $\vec{Q}$ o vetor posição do satélite,
            o ângulo em relação ao horizonte é o ângulo entre $\vec{P}$ e $\vec{u} = 
            \vec{Q} - \vec{P}$

        \item É preciso tomar cuidado com possíveis erros de precisão no argumento da função
            \code{c}{acos()}, cujo domínio é o intervalo $[-1, 1]$
    \end{itemize}

\end{frame}

\begin{frame}[fragile]{Solução AC}
    \inputsnippet{cpp}{1}{20}{1730.cpp}
\end{frame}

\begin{frame}[fragile]{Solução AC}
    \inputsnippet{cpp}{21}{39}{1730.cpp}
\end{frame}

\begin{frame}[fragile]{Solução AC}
    \inputsnippet{cpp}{40}{60}{1730.cpp}
\end{frame}

\begin{frame}[fragile]{Solução AC}
    \inputsnippet{cpp}{61}{81}{1730.cpp}
\end{frame}

\begin{frame}[fragile]{Solução AC}
    \inputsnippet{cpp}{82}{102}{1730.cpp}
\end{frame}

\begin{frame}[fragile]{Solução AC}
    \inputsnippet{cpp}{103}{123}{1730.cpp}
\end{frame}
