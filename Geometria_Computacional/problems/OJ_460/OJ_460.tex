\section{OJ 460 -- Overlapping Rectangles}

\begin{frame}[fragile]{Problema}

When displaying a collection of rectangular windows on a SUN screen, a critical step is 
determining whether two windows overlap, and, if so, where on the screen the overlapping region 
lies.

Write a program to perform this function. Your program will accept as input the coordinates of
two rectangular windows. If the windows do not overlap, your program should produce a message to
that effect. If they do overlap, you should compute the coordinates of the overlapping region 
(which must itself be a rectangle).

All coordinates are expressed in ``pixel numbers”, integer values ranging from 0 to 9999. A 
rectangle will be described by two pairs of $(X, Y)$ coordinates. The first pair gives the 
coordinates of the lower left-hand corner ($X_{LL}, Y_{LL}$). The second pair gives the 
coordinates of the upper right-hand coordinates
($X_{UR}, Y_{UR}$). You are guaranteed that $X_{LL} < X_{UR}$ and $Y_{LL} < Y_{UR}$.
\end{frame}

\begin{frame}[fragile]{Entrada e saída}

\textbf{Input}

Input will contain several test case. It begins with a single positive integer on a line by itself 
indicating the number of the cases following, each of them as described below. This line is 
followed by a blank line, and there is also a blank line between two consecutive inputs.

Each test case consists of two lines. The first contains the integer numbers $X_{LL}, Y_{LL}, X_{UR}$ and
$Y_{UR}$ for the first window. The second contains the same numbers for the second window.
\end{frame}

\begin{frame}[fragile]{Entrada e saída}
\textbf{Output}

For each test case, the output must follow the description below. The outputs of two consecutive 
cases will be separated by a blank line.

For each set of input if the two windows do not overlap, print the message `\texttt{No Overlap}’. 
If the two windows do overlap, print 4 integer numbers giving the $X_{LL}, Y_{LL}, X_{UR}$ and 
$Y_{UR}$ for the region of overlap.

Note that two windows that share a common edge but have no other points in common are considered
to have `\texttt{No Overlap}’.

\end{frame}

\begin{frame}[fragile]{Exemplo de entradas e saídas}

\begin{minipage}[t]{0.5\textwidth}
\textbf{Sample Input}
\begin{verbatim}
1
0 20 100 120
80 0 500 60
\end{verbatim}
\end{minipage}
\begin{minipage}[t]{0.45\textwidth}
\textbf{Sample Output}
\begin{verbatim}
80 20 100 60
\end{verbatim}
\end{minipage}
\end{frame}

\begin{frame}[fragile]{Solução $O(T)$}

    \begin{itemize}
        \item As restrições na entrada simplificam a rotina de interseção entre retângulos
        \pause

        \item Não é necessário codificar estruturas para representar pontos nem intervalos
        \pause

        \item Todas as expressões envolvem apenas aritmética inteira
        \pause

        \item Um retângulo com coordenadas negativas será utilizado para indicar o caso onde
            não há interseção
        \pause

        \item Cada caso de teste pode ser resolvido em $O(1)$, de modo que a solução tem 
            complexidade $O(T)$, onde $T$ é o número de casos de teste
    \end{itemize}

\end{frame}

\begin{frame}[fragile]{Solução com complexidade $O(T)$}
    \inputsnippet{cpp}{3}{21}{460.cpp}
\end{frame}

\begin{frame}[fragile]{Solução com complexidade $O(T)$}
    \inputsnippet{cpp}{22}{38}{460.cpp}
\end{frame}

\begin{frame}[fragile]{Solução com complexidade $O(T)$}
    \inputsnippet{cpp}{40}{63}{460.cpp}
\end{frame}
