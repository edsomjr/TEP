\section{Codeforces Round \#329 (Div. 2) -- Problem B: Anton and Lines}

\begin{frame}[fragile]{Problema}


The teacher gave Anton a large geometry homework, but he didn't do it (as usual) as he 
participated in a regular round on Codeforces. In the task he was given a set of n lines defined 
by the equations $y = k_ix + b_i$. It was necessary to determine whether there is at least one 
point of intersection of two of these lines, that lays strictly inside the strip between 
$x_1 < x_2$. In other words, is it true that there are $1\leq i < j\leq n$ and $x', y'$, such that:

\begin{itemize}
    \item $y' = k_ix' + b_i$, that is, point ($x', y'$) belongs to the line number $i$;
    \item $y' = k_jx' + b_j$, that is, point ($x', y'$) belongs to the line number $j$;
    \item $x_1 < x' < x_2$, that is, point ($x', y'$) lies inside the strip bounded by $x_1 < x_2$.
\end{itemize}

You can't leave Anton in trouble, can you? Write a program that solves the given task.

\end{frame}

\begin{frame}[fragile]{Entrada e saída}

\textbf{Input}

The first line of the input contains an integer $n$ $(2\leq n\leq 100 000)$ --  the number of lines 
in the task given to Anton. The second line contains integers $x_1$ and $x_2$ 
$(-1 000 000\leq x_1 < x_2\leq 1 000 000)$ defining the strip inside which you need to find a point 
of intersection of at least two lines.

The following $n$ lines contain integers $k_i, b_i$ $(-1 000 000\leq k_i, b_i\leq 1 000 000)$ -- the 
descriptions of the lines. It is guaranteed that all lines are pairwise distinct, that is, for any 
two $i\neq j$ it is true that either $k_i\neq k_j$, or $b_i\neq b_j$.

\textbf{Output}

Print "\texttt{Yes}" (without quotes), if there is at least one intersection of two distinct lines, located strictly inside the strip. Otherwise print "\texttt{No}" (without quotes).

\end{frame}

\begin{frame}[fragile]{Exemplo de entradas e saídas}

\begin{minipage}[t]{0.5\textwidth}
\textbf{Sample Input}
\begin{verbatim}
4
1 2
1 2
1 0
0 1
0 2

2
1 3
1 0
-1 3
\end{verbatim}
\end{minipage}
\begin{minipage}[t]{0.45\textwidth}
\textbf{Sample Output}
\begin{verbatim}
YES






NO
\end{verbatim}
\end{minipage}
\end{frame}

\begin{frame}[fragile]{Solução com complexidade $O(N\log N)$}

    \begin{itemize}
        \item A busca completa, que verifica todos os pares de segmentos de reta, tem
            complexidade $O(N^2)$, o que leva ao TLE, pois $N\leq 10^5$
        \pause

        \item Contudo, é possível determinar as possíveis interseções no intervalo
            $(x_1, x_2)$ com um algoritmo \textit{sweep line} 
        \pause

        \item Os segmentos devem ser ordenados, em ordem crescente, pelo valor da coordenada
            $y$ do segmento no ponto $x_1$
        \pause

        \item Em caso de empate, deve-se ordenar pela coordena $y$ no ponto $x_2$
        \pause

        \item Cada segmento deve ser processado uma única vez, nesta ordem
        \pause

        \item Deve-se manter o registro da maior coordenada $y$ em $x_2$ já encontrada 
            (inicialmente, este valor deve ser igual a $-\infty$)
        \pause

        \item Se a coordenada $y$ em $x_2$ do segmento a ser processado for menor do que
            a maior já encontrada, significa que houve uma interseção com algum dos segmentos
            já processados 
    \end{itemize}

\end{frame}

\begin{frame}[fragile]{Solução com complexidade $O(N \log N)$}
    \inputsnippet{cpp}{1}{19}{593B.cpp}
\end{frame}

\begin{frame}[fragile]{Solução com complexidade $O(N \log N)$}
    \inputsnippet{cpp}{21}{37}{593B.cpp}
\end{frame}
