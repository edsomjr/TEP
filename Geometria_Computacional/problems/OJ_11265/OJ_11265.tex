\section{OJ 11265 -- The Sultan's Problem}

\begin{frame}[fragile]{Problema}

Once upon a time, there lived a great sultan, who was very much fond of his wife. He wanted to build
a Tajmahal for his wife (ya, our sultan idolized Mughal emperor Shahjahan). But alas, due to budget
cuts, loans, dues and many many things, he had no fund to build something so big. So, he decided to
build a beautiful garden, inside his palace.

The garden is a rectangular piece of land. All the palaces water lines run through the garden, thus,
dividing it into pieces of varying shapes. He wanted to cover each piece of the land with flowers of same
kind.

The figure above shows a sample flower garden. All the lines are straight lines, with two end points
on two different edge of the rectangle. Each piece of the garden is covered with the same kind of flowers.

The garden has a small fountain, located at position $(x, y)$. You can assume that, it is not on any
of the lines. He wants to fill that piece with her favourite flower. So, he asks you to find the area of
that piece.

\end{frame}

\begin{frame}[fragile]{Problema}

\begin{center}
\includegraphics[scale=0.7]{figure.png}
\end{center}

\end{frame}

\begin{frame}[fragile]{Entrada e saída}

\textbf{Input}

Input contains around 500 test cases. Each case starts with 5 integers, $N, W, H, x, y$, the number 
of lines, width and height of the garden and the location of the fountain. Next $N$ lines each 
contain 4 integers, $x_1$ $y_1$ $x_2$ $y_2$, the two end points of the line. The end points are always 
on the boundary of the garden. You can assume that, the fountain is not located on any line.

\textbf{Constraints}

\begin{itemize}
    \item $1\leq W, H\leq 200,000$
    \item $1\leq N\leq 500$
\end{itemize}

\textbf{Output}

For each test case, output the case number, with the area of the piece covered with her favourite flower.
Print 3 digits after the decimal point.

\end{frame}

\begin{frame}[fragile]{Exemplo de entradas e saídas}

\begin{minipage}[t]{0.5\textwidth}
\textbf{Sample Input}
\begin{verbatim}
3 100 100 20 30
0 0 100 100
100 0 0 100
0 40 40 100
\end{verbatim}
\end{minipage}
\begin{minipage}[t]{0.45\textwidth}
\textbf{Sample Output}
\begin{verbatim}
Case #1: 1780.000
\end{verbatim}
\end{minipage}
\end{frame}

\begin{frame}[fragile]{Solução}

    \begin{itemize}
        \item A área que contém a fonte $F$ deve ser obtida através de sucessivos cortes do
            polígono $P$ que representa o jardim
        \pause

        \item Inicialmente $P$ é um retângulo
        \pause

        \item Cada nova reta fará um corte em $P$, separando a área que contém $F$ do restante
        \pause

        \item A orientação do corte é importante: se $F$ está à esquerda dos pontos $A$ e $B$ da
            reta, o corte é feito na orientação $AB$
        \pause

        \item Se estiver à direita, a orientação é feita no sentido oposto $BA$ 
        \pause

        \item Cada corte tem complexidade $O(n)$, onde $n$ é o número de vértices do polígono
        \pause

        \item No pior caso, a solução tem complexidade $O(N^2)$, onde $N$ é o número retas
    \end{itemize}

\end{frame}

\begin{frame}[fragile]{Solução AC}
    \inputsnippet{cpp}{1}{17}{11265.cpp}
\end{frame}

\begin{frame}[fragile]{Solução AC}
    \inputsnippet{cpp}{19}{38}{11265.cpp}
\end{frame}

\begin{frame}[fragile]{Solução AC}
    \inputsnippet{cpp}{40}{59}{11265.cpp}
\end{frame}

\begin{frame}[fragile]{Solução AC}
    \inputsnippet{cpp}{61}{78}{11265.cpp}
\end{frame}

\begin{frame}[fragile]{Solução AC}
    \inputsnippet{cpp}{80}{101}{11265.cpp}
\end{frame}
