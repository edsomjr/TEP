\section{UVA 11343 -- Isolated Segments}

\begin{frame}[fragile]{Problema}

You’re given $n$ segments in the rectangular coordinate system. The segments are defined by start 
and end points ($X_i , Y_i$) and ($X_j , Y_j$) ($1 \leq i, j \leq n$). Coordinates of these points 
are integer numbers with real value smaller then 1000. Length of each segment is positive.

When 2 segments don’t have a common point then it is said that segments don’t collide. In any
other case segments collide. Be aware that segments collide even if they have only one point in 
common.

Segment is said to be isolated if it doesn’t collide with all the other segments that are given, 
i.e.  segment $i$ is isolated when for each $1 \leq j \leq n$, ($ i\neq j$), segments $i$ and $j$ 
don’t collide. You are asked to find number $T$ -- how many segments are isolated.

\end{frame}

\begin{frame}[fragile]{Entrada e saída}

\textbf{Input}
First line of input contains number $N$ $(N \leq 50)$, then tests follow. First line of each test 
case contains number $M$ ($M \leq 100$) — the number of segments for this test case to be 
considered. For this particular test case $M$ lines follow each containing a description of one 
segment. Segment is described by 2 points: start point ($X_{pi}, Y_{pi}$) and end point 
($X_{ei}, Y_{ei}$). They are given in such order: $X_{pi}$ $Y_{pi}$ $X_{ei}$ $Y_{ei}$
\end{frame}

\begin{frame}[fragile]{Entrada e saída}
\textbf{Output}

For each test case output one line containing number $T$.
\end{frame}

\begin{frame}[fragile]{Exemplo de entradas e saídas}

\begin{footnotesize}
\begin{minipage}[t]{0.5\textwidth}
\textbf{Sample Input}
\begin{verbatim}
6
3
0 0 2 0
1 -1 1 1
2 2 3 3
2
0 0 1 1
1 0 0 1
2
0 0 0 1
0 2 0 3
2
0 0 1 0
1 0 2 0
2
0 0 2 2
1 0 1 1
2
1 3 1 5
1 0 1 6
\end{verbatim}
\end{minipage}
\begin{minipage}[t]{0.45\textwidth}
\textbf{Sample Output}
\begin{verbatim}
1
0
2
0
0
0
\end{verbatim}
\end{minipage}
\end{footnotesize}

\end{frame}

\begin{frame}[fragile]{Observações sobre o problema}

    \begin{itemize}
        \item O problema consiste na identificação da interseção entre segmentos

        \item A verificação de interseção entre segmentos consiste em duas etapas

        \item A primeira etapa é checar se as retas $r$ e $s$ que contém os segmentos se
            interceptam

        \item Em caso positivo é preciso ainda ver se o ponto de interseção está contido nos 
            segmentos ou não

        \item É preciso atentar aos \textit{corner cases}

        \item Como é preciso, para cada segmento $i$,  testar todos os segmentos $j$, a
            solução tem complexidade $O(M^2)$

        \item Atenção ao produto dos discriminantes, que pode exceder a capacidade de um
            inteiro, de modo que é preciso usar o tipo \code{c}{long long} em seu retorno
    \end{itemize}

\end{frame}

\begin{frame}[fragile]{Solução AC com complexidade $O(N^2)$}
    \inputsnippet{cpp}{1}{21}{ac2.cpp}
\end{frame}

\begin{frame}[fragile]{Solução AC com complexidade $O(N^2)$}
    \inputsnippet{cpp}{22}{42}{ac2.cpp}
\end{frame}

\begin{frame}[fragile]{Solução AC com complexidade $O(N^2)$}
    \inputsnippet{cpp}{44}{64}{ac2.cpp}
\end{frame}

\begin{frame}[fragile]{Solução AC com complexidade $O(N^2)$}
    \inputsnippet{cpp}{66}{86}{ac2.cpp}
\end{frame}

\begin{frame}[fragile]{Solução AC com complexidade $O(N^2)$}
    \inputsnippet{cpp}{87}{107}{ac2.cpp}
\end{frame}
