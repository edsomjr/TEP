\section{UVA 10245 -- The Closest Pair Problem}

\begin{frame}[fragile]{Problema}

Given a set of points in a two dimensional space, you will have to find the distance between the closest two points.

\end{frame}

\begin{frame}[fragile]{Entrada e saída}

\textbf{Input}

The input file contains several sets of input. Each set of input starts with an integer 
$N$ $(0\leq N\leq 10000)$, which denotes the number of points in this set. The next $N$ line 
contains the coordinates of $N$ twodimensional points. The first of the two numbers denotes the 
$X$-coordinate and the latter denotes the $Y$-coordinate. The input is terminated by a set whose 
$N = 0$. This set should not be processed. The value of the coordinates will be less than 40000 
and non-negative.

\textbf{Output}

For each set of input produce a single line of output containing a floating point number 
(with four digits after the decimal point) which denotes the distance between the closest two 
points. If there is no such two points in the input whose distance is less than 10000, print the 
line ‘\texttt{INFINITY}’.

\end{frame}

\begin{frame}[fragile]{Exemplo de entradas e saídas}

\begin{minipage}[t]{0.5\textwidth}
\textbf{Sample Input}
\begin{verbatim}
3
0 0
10000 10000
20000 20000
5
0 2
6 67
43 71
39 107
189 140
0
\end{verbatim}
\end{minipage}
\begin{minipage}[t]{0.45\textwidth}
\textbf{Sample Output}
\begin{verbatim}
INFINITY
36.2215
\end{verbatim}
\end{minipage}
\end{frame}

\begin{frame}[fragile]{Solução $O(TN\log N)$}

    \begin{itemize}
        \item A solução consiste em três etapas

        \item A primeira é determinar o área total ocupada pelas placas

        \item Basta somar a área individual de cada placa, que é o produto da base pela altura

        \item Em seguida, é preciso determinar os vértices de cada placa

        \item Pode-se assumir que eles estão inicialmente com o centro no origem, fazer a 
            rotação em sentido horário e, em seguida, transladar os pontos para a posição
            correta

        \item Determinados estes pontos, os limites do polígono corresponde ao envoltório
            convexo

        \item A área do polígono pode ser determinada através da expressão que computa esta
            área por meio dos vértices do polígono

        \item A resposta será a diferença entre ambas áreas, em porcentagem
    \end{itemize}

\end{frame}

\begin{frame}[fragile]{Solução com complexidade $O(TN\log N)$}
    \inputsnippet{cpp}{1}{21}{10245.cpp}
\end{frame}
