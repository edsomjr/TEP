\section{Codechef -- Ant Colony}

\begin{frame}[fragile]{Problema}

In an ants' colony spread over a flat surface, the ants can reside at integral coordinates as close 
as unit distance apart from each other (but no closer). The colony has the shape of a quadrilateral
$ABCD$ where one ant must reside at each of $A, B, C$, and $D$, and all other ants can be on any 
integral coordinate on its perimeter and its interior. Given the integral positions of $A, B, C$, 
and $D$, print the maximum number of ants that can reside in the colony (including the ants that 
can reside on the four corners and the perimeter).

\end{frame}

\begin{frame}[fragile]{Entrada e saída}

\textbf{Input}

The first line contains the number of test cases, $N$.

For each test case, a single line contains the $x$ and $y$ coordinates of the four corners of the 
quadrilateral in clockwise order starting with the left-most, bottom-most point.

\textbf{Output}

For each test case, print the case number, followed by a colon, followed by a single space, followed by a single integer indicating the maximum number of ants.

\textbf{Constraints}

$O < N \leq 3$

$|x|, |y| \leq 10,000$

\end{frame}

\begin{frame}[fragile]{Exemplo de entradas e saídas}

\begin{minipage}[t]{0.5\textwidth}
\textbf{Sample Input}
\begin{verbatim}
2
1 1 3 4 5 3 6 1
1 3 5 6 6 3 3 1
\end{verbatim}
\end{minipage}
\begin{minipage}[t]{0.45\textwidth}
\textbf{Sample Output}
\begin{verbatim}
Case 1: 14
Case 2: 16
\end{verbatim}
\end{minipage}
\end{frame}

\begin{frame}[fragile]{Solução com complexidade $O(N)$}

    \begin{itemize}
        \item Este problema é semelhante ao anterior, com pequenas alterações

        \item A primeira delas é que o polígono a ser avaliado é sempre um quadrilátero

        \item Outro ponto é a orientação do polígono, que passa a ser fixa 

        \item Além disso, a resposta consiste na soma dos pontos interiores $I$ e dos pontos
            da borda $B$

        \item A saída contém a descrição de cada caso de teste

        \item Como o número de operações é constante para cada um dos $N$ casos de teste,
            a solução é $O(N)$
   \end{itemize}

\end{frame}

\begin{frame}[fragile]{Solução $O(N)$}
    \inputsnippet{cpp}{1}{21}{KGP13F.cpp}
\end{frame}

\begin{frame}[fragile]{Solução $O(N)$}
    \inputsnippet{cpp}{22}{42}{KGP13F.cpp}
\end{frame}

\begin{frame}[fragile]{Solução $O(N)$}
    \inputsnippet{cpp}{43}{63}{KGP13F.cpp}
\end{frame}

\begin{frame}[fragile]{Solução $O(N)$}
    \inputsnippet{cpp}{64}{84}{KGP13F.cpp}
\end{frame}
