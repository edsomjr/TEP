\section{UVA 10652 -- Board Wrapping}

\begin{frame}[fragile]{Problema}

\begin{minipage}{0.5\textwidth}
The small sawmill in Mission, British Columbia, has
developed a brand new way of packaging boards for
drying. By fixating the boards in special moulds, the
board can dry efficiently in a drying room.

Space is an issue though. The boards cannot be
too close, because then the drying will be too slow.
On the other hand, one wants to use the drying room
efficiently.
\end{minipage}
\begin{minipage}{0.45\textwidth}
\begin{figure}
    \centering
    \includegraphics[scale=0.6]{10652.png}
\end{figure}
\end{minipage}

\end{frame}

\begin{frame}[fragile]{Problema}

Looking at it from a 2-D perspective, your task is to
calculate the fraction between the space occupied by the
boards to the total space occupied by the mould. Now, the
mould is surrounded by an aluminium frame of negligible
thickness, following the hull of the boards’ corners tightly.
The space occupied by the mould would thus be the interior
of the frame.

\end{frame}

\begin{frame}[fragile]{Entrada e saída}

\textbf{Input}

On the first line of input there is one integer, $N\leq 50$, giving
the number of test cases (moulds) in the input. After this line, $N$ test cases follow. 
Each test case starts with a line containing one integer $n$, $1 < n\leq 600$, which is the number 
of boards in the mould.  Then $n$ lines follow, each with five floating point numbers 
$x, y, w, h, \phi$ where $0\leq x, y, w, h\leq 10000$ and $-90^o < \phi\leq 90^o$. 
The $x$ and $y$ are the coordinates of the center of the board and $w$ and $h$ are the
width and height of the board, respectively. $\phi$ is the angle between the height axis of the 
board to the $y$-axis in degrees, positive clockwise. That is, if $\phi = 0$, the projection of 
the board on the $x$-axis would be $w$. Of course, the boards cannot intersect.

\end{frame}

\begin{frame}[fragile]{Entrada e saída}

\textbf{Output}

For every test case, output one line containing the fraction of the space occupied by the boards to the total space in percent. Your output should have one decimal digit and be followed by a space 
and a percent sign (‘\texttt{\%}’).

Note: The Sample Input and Sample Output corresponds to the given picture

\end{frame}

\begin{frame}[fragile]{Exemplo de entradas e saídas}

\begin{minipage}[t]{0.5\textwidth}
\textbf{Sample Input}
\begin{verbatim}
1
4
4 7.5 6 3 0
8 11.5 6 3 0
9.5 6 6 3 90
4.5 3 4.4721 2.2361 26.565
\end{verbatim}
\end{minipage}
\begin{minipage}[t]{0.45\textwidth}
\textbf{Sample Output}
\begin{verbatim}
64.3 %
\end{verbatim}
\end{minipage}
\end{frame}

\begin{frame}[fragile]{Solução $O(TN\log N)$}

    \begin{itemize}
        \item A solução consiste em três etapas

        \item A primeira é determinar o área total ocupada pelas placas

        \item Basta somar a área individual de cada placa, que é o produto da base pela altura

        \item Em seguida, é preciso determinar os vértices de cada placa

        \item Pode-se assumir que eles estão inicialmente com o centro no origem, fazer a 
            rotação em sentido horário e, em seguida, transladar os pontos para a posição
            correta

        \item Determinados estes pontos, os limites do polígono corresponde ao envoltório
            convexo

        \item A área do polígono pode ser determinada através da expressão que computa esta
            área por meio dos vértices do polígono

        \item A resposta será a diferença entre ambas áreas, em porcentagem
    \end{itemize}

\end{frame}

\begin{frame}[fragile]{Solução com complexidade $O(TN\log N)$}
    \inputsnippet{cpp}{1}{21}{10652.cpp}
\end{frame}

\begin{frame}[fragile]{Solução com complexidade $O(TN\log N)$}
    \inputsnippet{cpp}{22}{42}{10652.cpp}
\end{frame}

\begin{frame}[fragile]{Solução com complexidade $O(TN\log N)$}
    \inputsnippet{cpp}{43}{62}{10652.cpp}
\end{frame}

\begin{frame}[fragile]{Solução com complexidade $O(TN\log N)$}
    \inputsnippet{cpp}{63}{80}{10652.cpp}
\end{frame}

\begin{frame}[fragile]{Solução com complexidade $O(TN\log N)$}
    \inputsnippet{cpp}{81}{101}{10652.cpp}
\end{frame}

\begin{frame}[fragile]{Solução com complexidade $O(TN\log N)$}
    \inputsnippet{cpp}{102}{122}{10652.cpp}
\end{frame}

\begin{frame}[fragile]{Solução com complexidade $O(TN\log N)$}
    \inputsnippet{cpp}{123}{143}{10652.cpp}
\end{frame}

\begin{frame}[fragile]{Solução com complexidade $O(TN\log N)$}
    \inputsnippet{cpp}{144}{164}{10652.cpp}
\end{frame}

\begin{frame}[fragile]{Solução com complexidade $O(TN\log N)$}
    \inputsnippet{cpp}{165}{185}{10652.cpp}
\end{frame}
