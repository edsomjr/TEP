\section{UVA 10991 -- Region}

\begin{frame}[fragile]{Problema}

\begin{minipage}{0.55\textwidth}
From the figure on the right, it is clear that C1,
C2 and C3 circles are touching each other.

Consider,
\begin{itemize}
\item C1 circle have R1 radius.
\item C2 circle have R2 radius.
\item C3 circle have R3 radius.
\end{itemize}

Write a program that will calculate the area
of shaded region G.
\end{minipage}
\begin{minipage}{0.4\textwidth}
\begin{center}
\includegraphics[scale=0.5]{figure.png}
\end{center}
\end{minipage}

\end{frame}

\begin{frame}[fragile]{Entrada e saída}

\textbf{Input}

The first line will contain an integer $k$ ($1 \leq k \leq 1000$) which is the number of cases to 
solve.  Each of the following $k$ lines will contain three floating point number 
$R1$ ($1 \leq R1 \leq 1000$), $R2$ ($1 \leq R2 \leq 1000$) and $R3$ ($1 \leq R3 \leq 1000$).

\textbf{Output}

For each line of input, generate one line of output containing the area of G rounded to six
decimal digits after the decimal point. Floating-point errors will be ignored by special judge program.

\end{frame}

\begin{frame}[fragile]{Exemplo de entradas e saídas}

\begin{minipage}[t]{0.5\textwidth}
\textbf{Sample Input}
\begin{verbatim}
2
5.70 1.00 7.89
478.61 759.84 28.36
\end{verbatim}
\end{minipage}
\begin{minipage}[t]{0.45\textwidth}
\textbf{Sample Output}
\begin{verbatim}
1.2243
2361.0058
\end{verbatim}
\end{minipage}
\end{frame}

\begin{frame}[fragile]{Solução $O(k)$}

    \begin{itemize}
        \item A solução parte da observação que os centros dos círculos formam um triângulo
            $T$ cujos lados são dados por
        \[
            a = R2 + R3,\ \ \ \ b = R1 + R3,\ \ \ \ c = R1 + R2
        \]

        \item A área do triângulo $T$ contém a região G e mais três setores gerados pelos
            ângulos opostos a cada um destes lados

        \item O ângulo oposto a um lado pode ser obtido através da Lei dos Cossenos

        \item Por exemplo, o ângulo $\alpha$ oposto ao lado $a$ é dado por
        \[
            \alpha = \cos^{-1} \left(\frac{b^2 + c^2 - a^2}{2bc}\right)
        \]

        \item A área de cada setor é igual a metade do produto do ângulo pelo quadrado do
            raio

        \item Logo a área G é igual a área de $T$ menos a área dos três setores

    \end{itemize}

\end{frame}

\begin{frame}[fragile]{Solução com complexidade $O(k)$}
    \inputsnippet{cpp}{1}{21}{10991.cpp}
\end{frame}

\begin{frame}[fragile]{Solução com complexidade $O(k)$}
    \inputsnippet{cpp}{22}{42}{10991.cpp}
\end{frame}
