\section{Codeforces Beta Round \#1 -- Problem C: Ancient Berland Circus}

\begin{frame}[fragile]{Problema}

Nowadays all circuses in Berland have a round arena with diameter 13 meters, but in the past things were different.

In Ancient Berland arenas in circuses were shaped as a regular (equiangular) polygon, the size and the number of angles could vary from one circus to another. In each corner of the arena there was a special pillar, and the rope strung between the pillars marked the arena edges.

Recently the scientists from Berland have discovered the remains of the ancient circus arena. They found only three pillars, the others were destroyed by the time.

You are given the coordinates of these three pillars. Find out what is the smallest area that the arena could have.

\end{frame}

\begin{frame}[fragile]{Entrada e saída}

\textbf{Input}

The input file consists of three lines, each of them contains a pair of numbers -- coordinates of the pillar. Any coordinate doesn't exceed 1000 by absolute value, and is given with at most six digits after decimal point.

\vspace{0.2in}

\textbf{Output}

Output the smallest possible area of the ancient arena. This number should be accurate to at least 6 digits after the decimal point. It's guaranteed that the number of angles in the optimal polygon is not larger than 100.

\end{frame}

\begin{frame}[fragile]{Exemplo de entradas e saídas}

\begin{minipage}[t]{0.5\textwidth}
\textbf{Sample Input}
\begin{verbatim}
0.000000 0.000000
1.000000 1.000000
0.000000 1.000000
\end{verbatim}
\end{minipage}
\begin{minipage}[t]{0.45\textwidth}
\textbf{Sample Output}
\begin{verbatim}
1.00000000
\end{verbatim}
\end{minipage}
\end{frame}

\begin{frame}[fragile]{Solução}

    \begin{itemize}
        \item Para encontrar o polígono regular que contém os três pontos dados, é preciso
            determinar o círculo circunscrito
        \pause

        \item Observe que este círculo será o mesmo que circunscreve o triângulo formado pelos
            pontos dados
        \pause

        \item O número mínimo de lados pode ser determinado por força bruta, uma vez que o
            número máximo de lados é igual a 100
        \pause

        \item Para facilitar o processo de rotação, os três pontos devem ser transladados de
            modo que o centro do círculo circunscrito fique na origem
        \pause

        \item Para um número de lados $n$, rotacione um ponto $P$ escolhido em todos os ângulos
            possíveis e veja se os outros dois pontos foram encontrados
        \pause

        \item A rotina de comparação de pontos flutuantes não deve usar um valor $\varepsilon$ muito
            agressivo, pois pode levar ao WA ($\varepsilon = 10^{-5}$ é suficiente, 
            $\varepsilon = 10^{-6}$ gera WA no oitavo caso de teste)

    \end{itemize}

\end{frame}

\begin{frame}[fragile]{Solução AC}
    \inputsnippet{cpp}{1}{15}{1C.cpp}
\end{frame}

\begin{frame}[fragile]{Solução AC}
    \inputsnippet{cpp}{17}{34}{1C.cpp}
\end{frame}

\begin{frame}[fragile]{Solução AC}
    \inputsnippet{cpp}{36}{53}{1C.cpp}
\end{frame}

\begin{frame}[fragile]{Solução AC}
    \inputsnippet{cpp}{55}{74}{1C.cpp}
\end{frame}

\begin{frame}[fragile]{Solução AC}
    \inputsnippet{cpp}{76}{93}{1C.cpp}
\end{frame}

\begin{frame}[fragile]{Solução AC}
    \inputsnippet{cpp}{95}{113}{1C.cpp}
\end{frame}

\begin{frame}[fragile]{Solução AC}
    \inputsnippet{cpp}{115}{144}{1C.cpp}
\end{frame}
