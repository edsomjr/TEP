\section{OJ 10585 -- Center of symmetry}

\begin{frame}[fragile]{Problema}

\begin{minipage}{0.6\textwidth}
É dado um conjunto de $n$ pontos com coordenadas inteiras. Sua tarefa é decidir se o conjunto tem 
ou não um centro de simetria. 

Um conjunto de pontos $S$ tem um centro de simetria se existe um ponto $s$ (não necessariamente em
$S$) tal que para todo ponto $p$ em $S$ existe um ponto $q$ em $S$ tal que $p - s = s - q$.
\end{minipage}
\begin{minipage}{0.35\textwidth}
\begin{center}
\includegraphics[scale=0.5]{figure.png}
\end{center}
\end{minipage}

\end{frame}

\begin{frame}[fragile]{Entrada e saída}

\textbf{Entrada}

A primeira linha da entrada contém um número $c$ que indica o número de casos de teste que se
seguem. A primeira linha de um caso de teste contém um número $1 \leq n \leq 10000$. As $n$ linhas
subsequentes contém dois números inteiros cada, os quais são as coordenadas $x$ e $y$ de um ponto.
Cada ponto é único e temos que $-10000000 \leq x, y \leq 10000000$.

\textbf{Saída}

Para cada caso de teste imprima \lq\texttt{yes}\rq\ se o conjunto de pontos tem um centro de 
simetria, e \lq\texttt{no}\rq, caso contrário.

\end{frame}

\begin{frame}[fragile]{Exemplo de entradas e saídas}

\begin{minipage}[t]{0.5\textwidth}
\textbf{Entrada}
\begin{verbatim}
1
8
1 10
3 6
6 8
6 2
3 -4
1 0
-2 -2
-2 4
\end{verbatim}
\end{minipage}
\begin{minipage}[t]{0.45\textwidth}
\textbf{Saída}
\begin{verbatim}
yes
\end{verbatim}
\end{minipage}
\end{frame}

\begin{frame}[fragile]{Solução por força bruta}

    \begin{itemize}
        \item Manipulando a expressão $p - s = s - q$ obtemos
        \[
            s = \frac{p + q}{2},
        \]
        ou seja, o centro de simetria é o ponto médio entre $p$ e $q$
        \pause

        \item O uso de ponto flutuante pode ser evitado se usarmos a expressão
        \[
            2s = p + q
        \]
        \pause
        \item A solução de força bruta consiste em fixar um ponto $A$ em $S$ e, para todos os
        pontos $B$ em $S$, computar $2s$ 
        \pause

        \item Agora, para todos os pontos $p$ em $S$, calculamos $q = 2s - p$ e verificamos se
            $q$ pertence a $S$ ou não
        \pause

        \item Usando uma estrutura \code{c}{set}, que permite verificar se $q$ pertence a $S$ em
            $O(\log n)$, temos uma solução $O(n^2\log n)$
        \pause

        \item Como $n\leq 10^4$, esta solução pode dar TLE ou AC, a depender da velocidade do
            servidor
    \end{itemize}

\end{frame}

\begin{frame}[fragile]{Solução AC/TLE com complexidade $O(n^2\log n)$}
    \inputsnippet{cpp}{1}{19}{codes/ac_tle.cpp}
\end{frame}

\begin{frame}[fragile]{Solução AC/TLE com complexidade $O(n^2\log n)$}
    \inputsnippet{cpp}{21}{38}{codes/ac_tle.cpp}
\end{frame}

\begin{frame}[fragile]{Solução AC/TLE com complexidade $O(n^2\log n)$}
    \inputsnippet{cpp}{39}{58}{codes/ac_tle.cpp}
\end{frame}

\begin{frame}[fragile]{Solução mais eficiente}

    \begin{itemize}
        \item Para reduzir a complexidade assintótica da solução, é preciso investigar as
            propriedades do problema
        \pause

        \item Suponha que o centro de simetria $s$ pertença ao conjunto $S$. Então fazendo $p = s$
            na expressão $p - s = s - q$ temos que $q = s$, ou seja, o ponto de simetria fica
            pareado consigo mesmo
        \pause

        \item Como todos os pontos são distintos, então se $p\neq s$ então $q\neq s$, isto é,
            os pontos são pareados dois a dois
        \pause

        \item Deste modo, se existir, $s$ só estará em $S$ se $n$ for ímpar
        \pause

        \item Por um breve momento, vamos pensar no caso especial onde todos os pontos tem
            coordenada $y$ igual a zero
        \pause

        \item Considere agora $p$ o ponto com menor coordenada $x$
        \pause

        \item Neste cenário, podemos observar que $q$ deve ser, obrigatoriamente, o ponto com
            maior coordenada $x$
    \end{itemize}

\end{frame}

\begin{frame}[fragile]{Solução mais eficiente}

    \begin{itemize}
        \item De fato, seja $r \neq q$ o ponto de maior coordenada $x$. Então $r$ deve parear
            com um ponto $t$ com coordenada maior do que $x$ (pois os pontos são todos
            distintos), de modo que teremos
        \[
            \frac{x_r + x_t}{2} > \frac{x_p + x_q}{2},
        \]
        pois $x_p < x_t$ e $x_q < x_r$, o que impossibilita a existência de um centro de simetria
        \pause

        \item Assim, se os pontos estiverem ordenados, o primeiro deve parear com o último, de modo
            que é necessário verificar apenas um único candidato
        \pause

        \item Porém o fato acima foi deduzido para pontos sobre o eixo-$x$
        \pause

        \item Contudo, é fácil estender o resultado para duas dimensões: uma vez ordenados por
            coordenada $x$, o ponto com menor coordenada $x$ e menor coordenada $y$ deve parear
            com o ponto com maior coordenada $x$ e maior coordenada $y$, pelo mesmo motivo
            já apresentado
        \pause

        \item Assim, a solução passa a ter complexidade $O(n\log  n)$
    \end{itemize}

\end{frame}

\begin{frame}[fragile]{Solução AC com complexidade $O(n\log n)$}
    \inputsnippet{cpp}{1}{19}{codes/ac.cpp}
\end{frame}

\begin{frame}[fragile]{Solução AC com complexidade $O(n\log n)$}
    \inputsnippet{cpp}{21}{39}{codes/ac.cpp}
\end{frame}

\begin{frame}[fragile]{Solução AC com complexidade $O(n\log n)$}
    \inputsnippet{cpp}{41}{58}{codes/ac.cpp}
\end{frame}






