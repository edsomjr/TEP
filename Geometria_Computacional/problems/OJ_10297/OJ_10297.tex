\section{OJ 10297 -- Beavergnaw}

\begin{frame}[fragile]{Problema}

When chomping a tree the beaver cuts a very specific shape out of the tree trunk. What is left in
the tree trunk looks like two frustums of a cone joined by a cylinder with the diameter the same as
its height. A very curious beaver tries not to demolish a tree but rather sort out what should be the
diameter of the cylinder joining the frustums such that he chomped out certain amount of wood. You
are to help him to do the calculations.

\begin{figure}
    \centering

    \includegraphics[scale=0.5]{10297.png}
\end{figure}

\end{frame}

\begin{frame}[fragile]{Problema}

We will consider an idealized beaver chomping an idealized tree. Let us assume that the tree trunk
is a cylinder of diameter $D$ and that the beaver chomps on a segment of the trunk also of height $D$.
What should be the diameter $d$ of the inner cylinder such that the beaver chmped out $V$ cubic units of wood?

\end{frame}

\begin{frame}[fragile]{Entrada e saída}

\textbf{Input}

Input contains multiple cases each presented on a separate line. Each line contains two integer 
numbers $D$ and $V$ separated by whitespace. $D$ is the linear units and $V$ is in cubic units. 
$V$ will not exceed the maximum volume of wood that the beaver can chomp. A line with $D = 0$ and 
$V = 0$ follows the last case.

\textbf{Output}

For each case, one line of output should be produced containing one number rounded to three 
fractional digits giving the value of $d$ measured in linear units.  

\end{frame}

\begin{frame}[fragile]{Exemplo de entradas e saídas}

\begin{minipage}[t]{0.5\textwidth}
\textbf{Sample Input}
\begin{verbatim}
10 250
20 2500
25 7000
50 50000
0 0
\end{verbatim}
\end{minipage}
\begin{minipage}[t]{0.45\textwidth}
\textbf{Sample Output}
\begin{verbatim}
8.054
14.775
13.115
30.901
\end{verbatim}
\end{minipage}
\end{frame}

\begin{frame}[fragile]{Solução $O(T\log D)$}

    \begin{itemize}
        \item Um cone reto com diâmetro $D$ tem volume igual a
        \[
            V_C = D\pi\left(\frac{D}{2}\right)^2 = \frac{\pi D^3}{4}
        \]
        \pause

        \item Após o castor roer um volume igual a $V$, restará um volume $L = V_c - V$
        \pause

        \item Este volume deve ser composto por dois troncos de cone com altura $(D - d)/2$ e 
            bases com diâmetro $D$ e $d$, mais um cone reto de altura e diâmetros iguais a
            $d$
        \pause

        \item Em notação matemática
        \[
            L = 2\left[\frac{1}{3}\pi\left(\frac{D - d}{2}\right)\left(\frac{D^2}{4} + \frac{Dd}{4} + \frac{d^2}{4}\right)\right] + \frac{\pi d^3}{4}
        \]
    \end{itemize}

\end{frame}

\begin{frame}[fragile]{Solução $O(T\log D)$}

    \begin{itemize}
        \item Caso o castor não coma nada, $d$ deve ter valor máximo igual a $D$
        \pause

        \item Caso o castor coma toda a região, $d$ assume o valor mínimo igual a zero
        \pause

        \item Assim, dado o valor $V$, é possível determinar o valor de $d$ por meio de uma
            busca binária
        \pause

        \item A complexidade desta solução é $O(T\log D)$, onde $T$ é o número de casos de teste

    \end{itemize}

\end{frame}

\begin{frame}[fragile]{Solução com complexidade $O(T\log D)$}
    \inputsnippet{cpp}{1}{15}{10297_bs.cpp}
\end{frame}

\begin{frame}[fragile]{Solução com complexidade $O(T\log D)$}
    \inputsnippet{cpp}{17}{26}{10297_bs.cpp}
\end{frame}


\begin{frame}[fragile]{Solução $O(T)$}

    \begin{itemize}
        \item Para resolver cada caso de teste com complexidade $O(1)$, é preciso 
            interpretar o volume restante no tronco de forma ligeiramente diferente
        \pause

        \item Observe que este volume pode ser visto como a união de dois cones retos de 
            altura $D/2$ e bases com
            diâmetro $D$ e um sólido $S$, composto por um cone reto de altura e diâmetros iguais a
            $d$, subtraído de dois cones retos de altura e raio iguais a $d/2$.
        \pause

        \item Em notação matemática
        \[
            L = 2\left[\frac{1}{3}\pi\left(\frac{D}{2}\right)\left(\frac{D}{2}\right)^2\right]
            + \left\lbrace \pi d\left(\frac{d}{2}\right)^2 
            -2\left[\frac{1}{3}\pi\left(\frac{d}{2}\right)\left(\frac{d}{2}\right)^2\right] \right\rbrace
        \]
    \end{itemize}

\end{frame}

\begin{frame}[fragile]{Solução $O(T)$}

    \begin{itemize}
        \item Esta expressão pode ser reescrita como
        \begin{align*}
            L &= \frac{1}{12}\pi D^3 + \frac{1}{4}\pi d^3 - \frac{1}{12}\pi d^3 \\
            \frac{1}{4}\pi D^3 - V &= \frac{1}{12}\pi D^3 + \frac{1}{6}\pi d^3 \\
            \frac{1}{6}\pi D^3 - V &= \frac{1}{6}\pi d^3 \\
        \end{align*}
        \pause
 
        \item Assim,
        \[
            d = \sqrt[3]{\frac{6}{\pi}\left(\frac{1}{6}\pi D^3 - V\right)} =
            \sqrt[3]{D^3 - \frac{6V}{\pi}}
        \]
    \end{itemize}

\end{frame}

\begin{frame}[fragile]{Solução com complexidade $O(T)$}
    \inputsnippet{cpp}{5}{8}{10297.cpp}
\end{frame}
