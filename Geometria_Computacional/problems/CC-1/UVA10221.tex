\section{UVA 10221 -- Satellites}

\begin{frame}[fragile]{Problema}

\begin{minipage}{0.45\textwidth}
The radius of earth is 6440 Kilometer. There are many Satellites
and Asteroids moving around the earth. If two Satellites create
an angle with the center of earth, can you find out the \textit{distance}
between them? By \textit{distance} we mean both the \textit{arc} and \textit{chord}
distances. Both satellites are on the same orbit (However, please
consider that they are revolving on a circular path rather than
an elliptical path).
\end{minipage}
\begin{minipage}{0.5\textwidth}
\begin{center}
\includegraphics[scale=0.75]{figure.png}
\end{center}
\end{minipage}

\end{frame}

\begin{frame}[fragile]{Entrada e saída}

\textbf{Input}

The input file will contain one or more test cases.

Each test case consists of one line containing two-integer $s$
and $a$, and a string ‘\texttt{min}’ or ‘\texttt{deg}’. Here $s$ is the distance of the
satellite from the surface of the earth and $a$ is the angle that the
satellites make with the center of earth. It may be in minutes ($'$) or in degrees (\textdegree). Remember that the
same line will never contain minute and degree at a time.

\textbf{Output}

For each test case, print one line containing the required distances i.e. both arc \textit{distance}
 and \textit{chord distance} respectively between two satellites in Kilometer. The distance will be a floating-point value
with six digits after decimal point.

\end{frame}

\begin{frame}[fragile]{Exemplo de entradas e saídas}

\begin{minipage}[t]{0.5\textwidth}
\textbf{Sample Input}
\begin{verbatim}
500 30 deg
700 60 min
200 45 deg
\end{verbatim}
\end{minipage}
\begin{minipage}[t]{0.45\textwidth}
\textbf{Sample Output}
\begin{verbatim}
3633.775503 3592.408346
124.616509 124.614927
5215.043805 5082.035982
\end{verbatim}
\end{minipage}
\end{frame}

\begin{frame}[fragile]{Solução}

    \begin{itemize}
        \item A primeira etapa da solução é uniformizar as unidades de medida

        \item Sabendo que um grau tem 60 minutos, a conversão de $x$ minutos para $y$ graus é
        \[
            y = \frac{x}{60}
        \]

        \item Sabendo que $\pi$ radianos correspondem a 180\textdegree, segue que o ângulo $\alpha$,
            em radianos, será de
        \[
            \alpha = \frac{a\pi}{180}
        \]

        \item O comprimento $c$ do arco será dado por
        \[
            c = \alpha R,
        \]
        onde $R$ é o raio até o satélite, isto é, $r + s$

        \item A corda terá comprimento $L$ igual a
        \[
            L = 2R\sin(\frac{\alpha}{2})
        \]

        \item Com estas informações, a solução terá complexidade $O(T)$, onde $T$ é o número de
            casos de teste
    \end{itemize}

\end{frame}

\begin{frame}[fragile]{Solução AC com complexidade $O(T)$}
    \inputsnippet{cpp}{1}{21}{10221.cpp}
\end{frame}

\begin{frame}[fragile]{Solução AC com complexidade $O(T)$}
    \inputsnippet{cpp}{23}{43}{10221.cpp}
\end{frame}
