\section{UVA 10209 -- Is This Integration?}

\begin{frame}[fragile]{Problema}

\begin{minipage}{0.45\textwidth}
In the image below you can see a square $ABCD$,
where $AB = BC = CD = DA = a$. Four arcs
are drawn taking the four vertexes $A, B, C, D$
as centers and $a$ as the radius. The arc that is
drawn taking $A$ as center, starts at neighboring
vertex $B$ and ends at neighboring vertex $D$. All
other arcs are drawn in a similar fashion. Regions of three different shapes are created in
this fashion. You will have to determine the
total area these different shaped regions.
\end{minipage}
\begin{minipage}{0.5\textwidth}
\begin{center}
\includegraphics[scale=0.75]{figure2.png}
\end{center}
\end{minipage}

\end{frame}

\begin{frame}[fragile]{Entrada e saída}

\textbf{Input}

The input file contains a floating-point number
$a (0 \leq a \leq 10000)$ in each line which indicates
the length of one side of the square. Input is
terminated by end of file.

\textbf{Output}

For each line of input, output in a single line
the total area of the three types of region (filled with different patterns in the image above).

These three numbers will of course be floating point numbers with three digits after the decimal
point. First number will denote the area of the striped region, the second number will denote the total
area of the dotted regions and the third number will denote the area of the rest of the regions.
\end{frame}

\begin{frame}[fragile]{Exemplo de entradas e saídas}

\begin{minipage}[t]{0.5\textwidth}
\textbf{Sample Input}
\begin{verbatim}
0.1
0.2
0.3
\end{verbatim}
\end{minipage}
\begin{minipage}[t]{0.45\textwidth}
\textbf{Sample Output}
\begin{verbatim}
0.003 0.005 0.002
0.013 0.020 0.007
0.028 0.046 0.016
\end{verbatim}
\end{minipage}
\end{frame}

\begin{frame}[fragile]{Solução}

    \begin{itemize}
        \item Observe, inicialmente, que a área tracejada é composta por um quadrado cujo lado 
            é o comprimento da corda $c$ gerada por um ângulo igual a um terço de $\pi/2$ , somado 
            a quatro segmentos $g$

        \item Assim, $A_T = c^2 - 4g$, onde
        \begin{align*}
            c &= 2a\sin\left(\frac{\pi}{6}\right) \\
            s &= \frac{a + a + c}{2} \\
            T &= \sqrt{(s - a)(s - a)(s - c)} \\
            S &= \frac{\pi a^2}{12}  \\
            g &= S - T
        \end{align*}

        \item Já a área pontilhada é a metade do que resta quando subtraída, da área de um 
            semi-círculo de raio $a$, a área de um quadrado de lado $a$ e a área tracejada
    \end{itemize}

\end{frame}


\begin{frame}[fragile]{Solução}

    \begin{itemize}
        \item Portanto,
        \[
            A_P = \frac{\pi a^2}{2} - a^2 - A_T
        \]

        \item Por fim, a área restante $A_R$ é o que sobra da diferença entre a área de um
            quadrado de lado $a$ e as duas áreas já computadas

        \item Logo,
        \[
            A_R = a^2 - A_T - A_P
        \]

        \item As expressões apresentadas permitem a implementação de uma solução $O(T)$, onde
            $T$ é o número de casos de teste
    \end{itemize}

\end{frame}
\begin{frame}[fragile]{Solução AC com complexidade $O(T)$}
    \inputsnippet{cpp}{1}{21}{10209.cpp}
\end{frame}

\begin{frame}[fragile]{Solução AC com complexidade $O(T)$}
    \inputsnippet{cpp}{22}{36}{10209.cpp}
\end{frame}

\begin{frame}[fragile]{Solução AC com complexidade $O(T)$}
    \inputsnippet{cpp}{37}{57}{10209.cpp}
\end{frame}
