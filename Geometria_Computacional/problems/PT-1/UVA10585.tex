\section{UVA 10585 -- Center of symmetry}

\begin{frame}[fragile]{Problema}

\begin{minipage}{0.45\textwidth}
Given is a set of $n$ points with integer coordinates. Your task is to decide whether the set
has a center of symmetry.

A set of points $S$ has the center of symmetry if there exists a point $s$ (not necessarily in $S$)
such that for every point $p$ in $S$ there exists a point $q$ in $S$ such that $p - s = s - q$.
\end{minipage}
\begin{minipage}{0.5\textwidth}
\begin{center}
\includegraphics[scale=0.5]{figure.png}
\end{center}
\end{minipage}

\end{frame}

\begin{frame}[fragile]{Entrada e saída}

\textbf{Input}

The first line of input contains a number $c$ giving the
number of cases that follow. The first line of data for
a single case contains number $1 \leq n \leq 10000$. The
subsequent $n$ lines contain two integer numbers each
which are the $x$ and $y$ coordinates of a point. Every
point is unique and we have that $-10000000 \leq x, y \leq
10000000$.

\textbf{Output}

For each set of input data print \lq\texttt{yes}\rq\ if the set of
points has a center of symmetry and \lq\texttt{no}\rq\ otherwise.

\end{frame}

\begin{frame}[fragile]{Exemplo de entradas e saídas}

\begin{minipage}[t]{0.5\textwidth}
\textbf{Sample Input}
\begin{verbatim}
1
8
1 10
3 6
6 8
6 2
3 -4
1 0
-2 -2
-2 4
\end{verbatim}
\end{minipage}
\begin{minipage}[t]{0.45\textwidth}
\textbf{Sample Output}
\begin{verbatim}
yes
\end{verbatim}
\end{minipage}
\end{frame}

\begin{frame}[fragile]{Solução por força bruta}

    \begin{itemize}
        \item Manipulando a expressão $p - s = s - q$ obtemos
        \[
            s = \frac{p + q}{2},
        \]
        ou seja, o centro de simetria é o ponto médio de $p$ e $q$

        \item O uso de ponto flutuante pode ser evitado se usarmos a expressão
        \[
            2s = p + q,
        \]
        e trabalharmos com o dobro do centro de simetria

        \item A solução de força bruta consiste em fixar um ponto $A$ em $S$ e, para todos os
        pontos $B$ em $S$, computar $2s$ 

        \item Agora, para todos os pontos $p$ em $S$, calculamos $q = 2s - p$ e verificamos se
            $q$ pertence a $S$ ou não

        \item Usando uma estrutura \code{c}{set}, que permite verificar se $q$ pertence a $S$ em
            $O(\log n)$, temos uma solução $O(n^2\log n)$

        \item Como $n\leq 10^4$, esta solução pode dar TLE ou AC, a depender da velocidade do
            servidor
    \end{itemize}

\end{frame}

\begin{frame}[fragile]{Solução AC/TLE com complexidade $O(n^2\log n)$}
    \inputsnippet{cpp}{1}{21}{ac_tle.cpp}
\end{frame}

\begin{frame}[fragile]{Solução AC/TLE com complexidade $O(n^2\log n)$}
    \inputsnippet{cpp}{22}{41}{ac_tle.cpp}
\end{frame}

\begin{frame}[fragile]{Solução AC/TLE com complexidade $O(n^2\log n)$}
    \inputsnippet{cpp}{42}{62}{ac_tle.cpp}
\end{frame}

\begin{frame}[fragile]{Solução mais eficiente}

    \begin{itemize}
        \item Para reduzir a complexidade assintótica da solução, é preciso investigar as
            propriedades do problema

        \item Suponha que o centro de simetria $s$ pertença ao conjunto $S$. Então fazendo $p = s$
            na expressão $p - s = s - q$ temos que $q = s$, ou seja, o ponto de simetria fica
            pareado consigo mesmo

        \item Como todos os pontos são distintos, então se $p\neq s$ então $q\neq s$, isto é,
            os pontos são pareados dois a dois

        \item Deste modo, se existir, $s$ só estará em $S$ se $n$ for ímpar

        \item Por um breve momento, vamos pensar no caso especial onde todos os pontos tem
            coordenada $y$ igual a zero

        \item Considere agora $p$ o ponto com menor coordenada $x$

        \item Neste cenário, podemos observar que $q$ deve ser, obrigatoriamente, o ponto com
            maior coordenada $x$
    \end{itemize}

\end{frame}

\begin{frame}[fragile]{Solução mais eficiente}

    \begin{itemize}
        \item De fato, seja $r \neq q$ o ponto de maior coordenada $x$. Então $r$ deve parear
            com um ponto $t$ com coordenada maior do que $x$ (pois os pontos são todos
            distintos), de modo que teremos
        \[
            \frac{x_r + x_t}{2} > \frac{x_p + x_q}{2},
        \]
        pois $x_p < x_t$ e $x_q < x_r$, o que impossibilita a existência de um centro de simetria

        \item Assim, se os pontos estiverem ordenados, o primeiro deve parear com o último, de modo
            que é necessário verificar apenas um único candidato

        \item Porém o fato acima foi deduzido para pontos sobre o eixo-$x$

        \item Contudo, é fácil estender o resultado para duas dimensões: uma vez ordenados por
            coordenada $x$, o ponto com menor coordenada $x$ e menor coordenada $y$ deve parear
            com o ponto com maior coordenada $x$ e maior coordenada $y$, pelo mesmo motivo
            já apresentado

        \item Assim, a solução passa a ter complexidade $O(n\log  n)$
    \end{itemize}

\end{frame}

\begin{frame}[fragile]{Solução AC com complexidade $O(n\log n)$}
    \inputsnippet{cpp}{1}{12}{ac.cpp}
\end{frame}

\begin{frame}[fragile]{Solução AC com complexidade $O(n\log n)$}
    \inputsnippet{cpp}{13}{32}{ac.cpp}
\end{frame}

\begin{frame}[fragile]{Solução AC com complexidade $O(n\log n)$}
    \inputsnippet{cpp}{33}{53}{ac.cpp}
\end{frame}






