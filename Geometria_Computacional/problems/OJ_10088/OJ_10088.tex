\section{OJ 10088 -- Trees on My Island}

\begin{frame}[fragile]{Problema}

I have bought an island where I want to plant trees in rows and columns. So, the trees will
form a rectangular grid and each of them can be thought of having integer coordinates by taking
a suitable grid point as the origin.

But, the problem is that the island itself is not rectangular. So, I have identified a simple
polygonal area inside the island with vertices on the grid points and have decided to plant trees
on grid points lying strictly inside the polygon.

Now, I seek your help for calculating the number of trees that can be planted on my island.

\end{frame}

\begin{frame}[fragile]{Problema}

\begin{center}
\includegraphics[scale=0.8]{figure.png}
\end{center}

\end{frame}

\begin{frame}[fragile]{Entrada e saída}

\textbf{Input}

The input file may contain multiple test cases. Each test case begins with a line containing an
integer $N$ $(3\leq N\leq 1,000)$ identifying the number of vertices of the polygon. The next
$N$ lines contain the vertices of the polygon either in clockwise or in anti-clockwise direction. 
Each of these $N$ lines contains two integers identifying the $x$ and $y$-coordinates of a vertex. 
You may assume that none of the coordinates will be larger than $1,000,000$ in absolute values.

A test case containing a zero for $N$ in the first line terminates the input.

\textbf{Output}

For each test case in the input print a line containing the number of trees that can be planted inside the polygon.

\end{frame}

\begin{frame}[fragile]{Exemplo de entradas e saídas}

\begin{tiny}
\begin{minipage}[t]{0.5\textwidth}
\textbf{Sample Input}
\begin{verbatim}
12
3 1
6 3
9 2
8 4
9 6
9 9
8 9
6 5
5 8
4 4
3 5
1 3
12
1000 1000
2000 1000
4000 2000
6000 1000
8000 3000
8000 8000
7000 8000
5000 4000
4000 5000
3000 4000
3000 5000
1000 3000
0
\end{verbatim}
\end{minipage}
\begin{minipage}[t]{0.45\textwidth}
\textbf{Sample Output}
\begin{verbatim}
21
25990001
\end{verbatim}
\end{minipage}
\end{tiny}
\end{frame}

\begin{frame}[fragile]{Solução $O(TN)$}

    \begin{itemize}
        \item A tentativa de testar todos os pontos dentro do retângulo que delimita o
            polígono gera um algoritmo $O(N^2)$ que leva ao TLE
        \pause

        \item Contudo, este problema pode ser resolvido com complexidade $O(N)$ para cada um dos
            $T$ casos de teste
        \pause

        \item Basta utilizar o Teorema de Pick, que relaciona o número de pontos com coordenadas
            inteiras que estão no interior ($I$) e na borda ($B$) do polígono $P$ com sua
            área $A$
        \pause

        \item A área $A$ pode ser computada em $O(N)$ a partir das coordenadas de seus vértices,
            utilizando a expressão
        \[
            A = \frac{1}{2}\left|\sum_{i = 1}^{N} x_iy_{i + 1} -
            \sum_{j = 1}^{N} y_jx_{j + 1}\right|,
        \]
        com $(x_N, y_N) = (x_0, y_0)$

   \end{itemize}

\end{frame}

\begin{frame}[fragile]{Solução $O(TN)$}

    \begin{itemize}
        \item Cada aresta $QR$ de $P$ intercepta $d + 1$ pontos com coordenadas inteiras,
            onde $d = (b, h)$ é o maior divisor comum entre $b = |Q_x - R_x|$ e $h = |Q_y - R_y|$
        \pause

        \item Como os vértices são contados duas vezes cada, segue que
        \[
            B = -N + \sum_i^N \gcd(b_i, h_i),
        \]
        \pause
        
        \item Assim, pelo Teorema de Pick,
        \[
            2I = 2A - B + 2,
        \]
        o que permite computar o valor de $I$ a partir de $A$ e $B$
    \end{itemize}

\end{frame}

\begin{frame}[fragile]{Solução $O(TN)$}
    \inputsnippet{cpp}{6}{21}{10088.cpp}
\end{frame}

\begin{frame}[fragile]{Solução $O(TN)$}
    \inputsnippet{cpp}{23}{37}{10088.cpp}
\end{frame}

\begin{frame}[fragile]{Solução $O(TN)$}
    \inputsnippet{cpp}{39}{46}{10088.cpp}
\end{frame}
